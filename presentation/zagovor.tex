%LTeX: enabled=false
% \documentclass[handout, slovene]{beamer}
\documentclass[slovene]{beamer}

% \usetheme{boxes} % see http://www.deic.uab.es/~iblanes/beamer_gallery/ for lots of examples
\usetheme{metropolis}
\usecolortheme{rose}
% \useinnertheme{circles}
% \useoutertheme{split}
% \setbeamertemplate{blocks}[rounded][shadow=true]

\setbeamertemplate{navigation symbols}{} % remove navigation symbols
\setbeamertemplate{footline}{} % remove title, too long

% %next set colors - not needed
% \setbeamercolor{title}{fg=black!70!black}
% \setbeamercolor{frametitle}{fg=blue!70!black}
% \setbeamercolor{framesubtitle}{fg=green!30!black}
% \setbeamercolor{author}{fg=red!70!black}
% \setbeamercolor{institute}{fg=green!30!black}
% \setbeamercolor{date}{fg=blue!50!black}

% \usepackage[T1]{fontenc}        % kodiranje znakov v .pdf
% \usepackage[utf8]{inputenc}     % kodiranje znakov v .tex
% \usepackage[slovene]{babel}     % nastavimo slovenščino
% \usepackage{stmaryrd}

\usepackage{fontspec}
\usepackage{unicode-math}

% \setmainfont{Latin Modern Sans}
\setmathfont{Asana Math}
\setmathfont{Latin Modern Math}
\setmathfont{Asana Math}[range={scr}]
\setmathfont{STIX Two Math-Regular}[range={bb,"1D538-"1D56B,"0211D}]
\setmathfont{STIX Two Math-Regular}[range={"2102,"210D,"2115,"2119,"211A,"211D,"2124}]
% \setmathfont{Asana Math}[range={"007B,"007D}]  % {}
\setmathfont{Asana Math}[range={8709,"29B6,"2987,"2988}]  % U+2205, emptyset, ⦶, ⦇, ⦈

\usepackage{qcircuit}
\usepackage{tikz}
\usetikzlibrary{babel}
\usetikzlibrary{angles,quotes}
\usetikzlibrary{cd}
\usepackage{mathpartir}
\usepackage{quiver}

\usepackage{minted}

\usepackage{ulem}
\renewcommand{\ULdepth}{1.8pt}
\newcommand{\ul}[1]{\uline{#1}}
\usepackage[style=german]{csquotes}
\usepackage{stmaryrd}

\newtheorem{izrek}[theorem]{Izrek}
\newtheorem{trditev}[theorem]{Trditev}
\newtheorem{posledica}[theorem]{Posledica}
\newtheorem{vprasanje}[theorem]{Vprašanje}
\newtheorem{domneva}[theorem]{Domneva}
\newtheorem{lema}{Lema}
\newtheorem{definicija}{Definicija}
{\theoremstyle{example}
    \newtheorem{zgled}{Zgled}
    \newtheorem{primer}{Primer}
    \newtheorem{primeri}{Primeri}
}
% \newtheorem{theorem}{Izrek}
% \newtheorem{statement}{Trditev}
% \newtheorem{lemma}{Lema}
% \newtheorem*{corrolary}{Posledica}

% \theoremstyle{definicija}
% \newtheorem{definicija}{Definicija}

% \newtheorem*{example}{Primer}
% \newtheorem*{example*}{Primer}
% \newtheorem*{examples}{Primeri}

% \theoremstyle{remark}
% \newtheorem*{remark}{Opomba}

% \newcounter{axiom}
% \setcounter{axiom}{0}
% \renewcommand\theaxiom{\Alph{axiom}}
% \newenvironment{axiom}[2][-1]{
%     \ifnum#1>0
%         \refsetcounter{axiom}{#1}
%     \fi
%     \refstepcounter{axiom}
%     \setlength\parindent{0pt}
%     \ifblank{#2}{}{\vskip.5em #2\par}
%     \textbf{Aksiom \theaxiom.}
% }{
% }

% \beamertemplatetransparentcovereddynamic

\title{Kvantni algebrajski učinki}
\author{Luna Strah}
\institute{mentor: doc. dr. Matija Pretnar}
\date{9.~9.~2022}

% LTeX: enabled=false
\renewcommand{\d}{\;\mathrm d}
\renewcommand{\hat}{\widehat}
\renewcommand{\tilde}{\widetilde}
\let\oldbar\bar
\renewcommand{\bar}{\overline}
\newcommand{\subs}{\subseteq}
\newcommand{\nin}{\not\in}
\newcommand{\contradiction}{\,\,\lightning}

\newcommand{\p}[1]{\left( {#1} \right)}
\newcommand{\set}[2]{\left\{ #1 \mid #2 \right\}}
\newcommand{\newfrac}[2]{{}^{#1}\!/_{\!#2}}
\newcommand{\smallfrac}[2]{{\textstyle\frac{#1}{#2}}}
\newcommand{\im}[1]{\mathrm{im}{\p{#1}}}
\newcommand{\mb}[1]{\mathbf{#1}}
\newcommand{\mf}[1]{\mathfrak{#1}}
\newcommand{\mc}[1]{\mathcal{#1}}
\newcommand{\id}{\mathrm{id}}
\newcommand{\cat}[1]{\mathbf{#1}}
\newcommand{\tr}[1]{\textrm{tr}{#1}}
\newcommand{\rang}[1]{\textrm{rang}{\p{#1}}}

\usepackage{xstring}
\newcommand{\mat}[1]{\begin{matrix} #1 \end{matrix}}
\newcommand{\pmat}[1]{\left(\mat{#1}\right)}
\newcommand{\bmat}[1]{\left[\mat{#1}\right]}
\let\oldvec\vec
\renewcommand{\vec}[1]{\bmat{\StrSubstitute[0]{#1}{,}{\\}}}
\newcommand{\state}[1]{\left\{ {#1} \right\}}
% \newcommand{\mixstate}[2]{%
%   \StrSubstitute[0]{#1}{…}{⋯}[\A]%
%   \StrSubstitute[0]{#2}{…}{⋯}[\B]%
%   \pmat{\StrSubstitute[0]{\A}{,}{&}\\\StrSubstitute[0]{\B}{,}{&}}%
% }
\usepackage{pgffor}
\newcommand{\mixstate}[2]{% cf. https://tex.stackexchange.com/a/87423/64454
  \foreach \A [count=\ai] in {#1} {%
  \foreach \B [count=\bi] in {#2} {
    \ifnum\ai=\bi%
      \ifnum\ai>1 {+} \fi
      \if\A… {⋯} \else {\A \state{\B}} \fi
    \fi%
  }}
}
\newcommand{\bra}[1]{\left\langle #1 \right\rvert}
\newcommand{\ket}[1]{\left\lvert #1 \right\rangle}
\newcommand{\braket}[2]{\left\langle #1 \middle\vert \mathopen{}#2 \right\rangle}
\newcommand{\density}[1]{\left\lvert #1 \middle\rangle\middle\langle \mathopen{}#1 \right\rvert}
\newcommand{\ketbra}[2]{\left\lvert #1 \middle\rangle\middle\langle \mathopen{}#2 \right\rvert}
% \newcommand{\expval}[2]{\left\langle #2 \middle\vert #1 \middle\vert #2 \right\rangle}
\newcommand{\expval}[2]{{\left\langle #1 \right\rangle}_{\!#2}}
\newcommand{\matelt}[3]{\left\langle #1 \middle\vert #2 \middle\vert #3 \right\rangle}

\newcommand{\had}{\mathtt{Had}}
\newcommand{\swap}{\mathtt{swap}}
\newcommand{\hh}[1][]{\mathbf{h}_{#1}}
\newcommand{\B}[1][]{\mathbf{B}_{#1}}
\newcommand{\M}[1][]{\mathbf{M}_{#1}}
\renewcommand{\H}[1][]{\mathbf{H}_{#1}}
\renewcommand{\L}[2][]{\mathcal{L}_{#1}{\p{#2}}}
\newcommand{\g}[1]{\mathtt{#1}}
\newcommand{\D}[1]{D{\p{\g{#1}}}}
\newcommand{\U}[1][]{\mathbf{U}_{#1}}
\newcommand{\ctl}[1]{\g{c#1}}
\newcommand{\ctlo}[1]{\bar{\g{c}} \g{#1}}
\newcommand{\cstarcat}{\cat{Cstar}}
\newcommand{\cstarcpucat}{\cat{Cstar}_{\textrm{CPU}}}
\newcommand{\cstar}[2]{\cat{Cstar}{\arity{#1}{#2}}}
\newcommand{\brat}[2]{\cat{Brat}{\arity{#1}{#2}}}

\newcommand{\op}[1]{\textnormal{\ttfamily#1}}
\newcommand{\mor}[1]{\textnormal{\sffamily#1}}
\newcommand{\eff}[1]{\textnormal{\sffamily\ul{#1}}}
\newcommand{\enew}{\eff{new}{\p{}}}
% \newcommand{\enew}[1]{\eff{new}{\p{#1}}}
\newcommand{\eapply}[2]{\eff{apply}_{\g{#1}}{\p{#2}}}
\newcommand{\emeasure}[1]{\eff{measure}{\p{#1}}}
\newcommand{\ediscard}[1]{\eff{discard}{\p{#1}}}
\newcommand{\tnew}[2]{\op{new}{\p{#1.\,#2}}}
\newcommand{\tapply}[3]{\op{apply}_{\g{#1}}{\p{#2;#2.\,#3}}}
\newcommand{\tapplyd}[3]{\op{apply}_{\D{#1}}{\p{#2;#2.\,#3}}}
\newcommand{\tmeasure}[3]{\op{measure}{\p{#1;\,#2,#3}}}
\newcommand{\tdiscard}[2]{\op{discard}{\p{#1;\,#2}}}
\newcommand{\new}[2]{ν{#1}.\,#2}
\newcommand{\apply}[3]{{\g{#1}}_{#2}{\p{#3}}}
\newcommand{\applyd}[3]{{\D{#1}}_{#2}{\p{#3}}}
\renewcommand{\measure}[3]{#2 {\;?_{#1}\,} #3}
\newcommand{\measurebr}[3]{\measure{#1}{\p{#2}}{\p{#3}}}
\newcommand{\discard}[2]{\op{disc}_{#1}{\p{#2}}}
\newcommand{\lam}[2]{λ{#1}.\,#2}
\newcommand{\plet}[3]{\qpl{let |\(#1 ≔ #2\)| in |\(#3\)|}}
\newcommand{\pseq}[2]{\qpl{|\(#1\)|;|\(#2\)|}}
\newcommand{\pret}[1]{\qpl{return |\(#1\)|}}
% \newcommand{\pret}[1]{#1}
\newcommand{\snew}[2]{\mor{new}#1{\p{#2}}}
\newcommand{\sapply}[2]{\mor{apply}_{\g{#1}}{\p{#2}}}
\newcommand{\smeasure}[3]{\mor{measure}#1{\p{#2,#3}}}

\newcommand{\type}[1]{\text{\ttfamily#1}}
\newcommand{\unit}{\type{I}}
\newcommand{\bit}{\type{bit}}
\newcommand{\qbit}{\type{qbit}}

\newcommand{\arity}[2]{\p{#1 \mid #2}}
% \newcommand{\arity}[2]{\p{#1 \mid \StrSubstitute[0]{#2}{,}{\ }}}

\newcommand{\sequent}[3]{#1 \mid #2 ⊢ #3}
\newcommand{\sseq}[2]{#1 ⊢ #2}
\newcommand{\seq}[1]{\sequent{x₁:m₁,…,xₖ:mₖ}{a₁,…,aₚ}{#1}}
\newcommand{\absseq}[1]{\sequent{\Gamma}{\Delta}{#1}}
\newcommand{\sem}[1]{\left⟦ #1 \right⟧}
% \newcommand{\brsem}[1]{⟬ #1 ⟭}
% \newcommand{\brsem}[1]{⦅ #1 ⦆}
\newcommand{\brsem}[1]{\left⦇ #1 \right⦈}
\usepackage[nomessages]{fp}
\newcommand{\mormap}[2]{% cf. https://tex.stackexchange.com/a/87423/64454
  \def\nextitem{\def\nextitem{⊕}}% Separator
  \forcsvlist\mormapitem{#2} → M_{\IfInteger{#1}{\FPeval\result{round(2^(#1):0)}\result}{2^{#1}}}{\p X}
}
\newcommand{\mormapitem}[1]{%
  \nextitem
  \def\param{\detokenize{#1}}%
  \if\param…%
    ⋯%
  \else%
    M_{\IfInteger{\param}{\FPeval\result{round(2^(\param):0)}\result}{2^{\param}}}{\p X}\!%
  \fi
}
\newcommand{\semmap}[2]{% cf. https://tex.stackexchange.com/a/87423/64454
  \def\nextitem{\def\nextitem{⊕}}% Separator
  \forcsvlist\semmapitem{#2} → \M[\IfInteger{#1}{\FPeval\result{round(2^(#1):0)}\result}{2^{#1}}]
}
\newcommand{\semmapitem}[1]{%
  \nextitem
  \def\param{\detokenize{#1}}%
  \if\param…%
    ⋯%
  \else%
    \M[\IfInteger{\param}{\FPeval\result{round(2^(\param):0)}\result}{2^{\param}}]\!%
  \fi
}
\newcommand{\abssemmap}[2]{% cf. https://tex.stackexchange.com/a/87423/64454
  \def\nextitem{\def\nextitem{×}}% Separator
  \forcsvlist\abssemmapitem{#2} → #1 • X
}
\newcommand{\abssemmapitem}[1]{%
  \nextitem
  \def\param{\detokenize{#1}}%
  \if\param…%
    ⋯%
  \else%
    (\param • X)\!%
  \fi
}
\newcommand{\semhom}[2]{
  \def\nextitem{\def\nextitem{⊕}}% Separator
  \cstar{\p{\forcsvlist\semhomitem{#2}, \M[\IfInteger{#1}{\FPeval\result{round(2^(#1):0)}\result}{2^{#1}}]}}
}
\newcommand{\semhomitem}[1]{%
  \nextitem
  \def\param{\detokenize{#1}}%
  \if\param…%
    ⋯%
  \else%
    \M[\IfInteger{\param}{\FPeval\result{round(2^(\param):0)}\result}{2^{\param}}]\!%
  \fi
}

\newcommand{\airquotes}[1]{\enquote{#1}}

\newminted[qplcode]{ocaml}{escapeinside=||, autogobble}
\newmintinline[qpl]{ocaml}{escapeinside=||, autogobble}

\makeatletter
\newcommand{\oset}[3][0ex]{%
  \mathrel{\mathop{#3}\limits^{
    \vbox to#1{\kern-2\ex@
    \hbox{$\scriptstyle#2$}\vss}}}}
\makeatother

\makeatletter
\newcommand{\listintertext}{\@ifstar\listintertext@\listintertext@@}
\newcommand{\listintertext@}[1]{% \listintertext*{#1}
  \hspace*{-\@totalleftmargin}#1}
\newcommand{\listintertext@@}[1]{% \listintertext{#1}
  \hspace{-\leftmargin}#1}
\makeatother

\newcommand{\for}[2]{\forall#1.\;#2}
\newcommand{\exist}[2]{\exists#1.\;#2}
\newcommand{\existi}[2]{\exists!#1.\;#2}

\setbeameroption{hide notes} % Only slides
% \setbeameroption{show only notes} % Only notes
% \setbeameroption{show notes} % Both
% \setbeameroption{show notes on second screen=right} % Both
%LTeX: enabled=true


\begin{document}
%%%%%
\frame{\titlepage}
%%%%%

%%%%%%%%%%%%%%%%%%%%%%%%%%%%%%%%%%%%%%%%%%%%%%%%%%%%%%%%%%%%%%%%%%%%%%%%%%%%%%%
% Intro
%%%%%%%%%%%%%%%%%%%%%%%%%%%%%%%%%%%%%%%%%%%%%%%%%%%%%%%%%%%%%%%%%%%%%%%%%%%%%%%
\begin{frame}
    \frametitle{Motivacija}

    \pause
    Kvantno programiranje
    \begin{itemize}
        \item Enakost programov
        \pause
        \item Novi problemi za teorijo programskih jezikov
        % \begin{itemize}
        %     \item Linearnost
        %     \item Učinki
        % \end{itemize}
    \end{itemize}

    \note[item]{
        Dva izziva:
        \begin{itemize}
            \item Strojna oprema
            \item Programska oprema
        \end{itemize}
    }
    \note[item]{
        Kloniranje
        \begin{itemize}
            \item Linearnost
        \end{itemize}
    }
\end{frame}
%%%%%
\begin{frame}
    \frametitle{Pregled}

    \begin{enumerate}
        \item Kvantno računalništvo
        \item Interpretacije operacij
        \item Polnost
    \end{enumerate}
\end{frame}
%%%%%%%%%%%%%%%%%%%%%%%%%%%%%%%%%%%%%%%%%%%%%%%%%%%%%%%%%%%%%%%%%%%%%%%%%%%%%%%
% Quantum mechanics
%%%%%%%%%%%%%%%%%%%%%%%%%%%%%%%%%%%%%%%%%%%%%%%%%%%%%%%%%%%%%%%%%%%%%%%%%%%%%%%
\begin{frame}
    \frametitle{Kvantni vektorji}

    \begin{definicija}[Binarni vektorji]
        Binarni vektorji so elementi množice \( \B[n] ≔ \{0, 1\}ⁿ \) in jih pišemo kot nize.
    \end{definicija}
    Primer: \( \B[2] = \{\mb{00}, \mb{01}, \mb{10}, \mb{11}\} \).

    \pause
    \begin{definicija}[Kvantni prostor]
        Kvantni vektorji (nadaljnje vektorji) so elementi prostora \( \H[n] ≔ ℂ^{2ⁿ} \).
        Kubiti so elementi \( \H ≔ \H[1] \).\\
        Če je \( \{eⱼ\} \) standardna baza \( \H[n] \) pišemo \( \ket j ≔ eⱼ \).
    \end{definicija}
    Očitno je \( \H[n] = \L[ℂ]{\set{ \ket j }{ j∈\B[n]} } \).
    % \pause
    % \begin{definicija}[Kvantna vrata]
    %     Kvantna vrata reda \(n\) so unitarne matrike dimenzije \(2ⁿ\).
    %     % Tenzorski produkt \( U \otimes V = [u_{jk}V]_{j,k} \) uporabljen na \(a\otimes b\) je enak \( Ua \otimes Vb \).
    % \end{definicija}

\end{frame}
%%%%%
\begin{frame}
    \frametitle{Primeri}

    \begin{primer}[\( n = 1 \)]
        \[ a = \vec{a₀, a₁}
             = a₀\vec{1,0} + a₁\vec{0,1}
             = a₀\ket{\mb0} + a₁\ket{\mb1}.
        \]
    \end{primer}

    \pause
    \begin{primer}[\( n = 2 \)]
        \( a = \vec{a₀,  a₁,  a₂,  a₃}
             = \vec{a\mb{₀₀}, a\mb{₀₁}, a\mb{₁₀}, a\mb{₁₁}}
             = a₀₀\ket{\mb{00}} + a₀₁\ket{\mb{01}} + a₁₀\ket{\mb{10}} + a₁₁\ket{\mb{11}}.
        \)
    \end{primer}

    \pause
    \begin{primer}
        \vspace{-1pt}
        \begin{center}
            \( \hh ≔ ρ\vec{1,1} = ρ\p{\ket{\mb0} + \ket{\mb1}},\quad
                \hh[n] ≔ ρⁿ{\displaystyle∑_{j ∈ \B[n]} \ket j,\quad
                ρ ≔ \frac{1}{\sqrt2}}.
            \)
        \end{center}
        \vspace{-1pt}
    \end{primer}

\end{frame}
%%%%%
\begin{frame}
    \frametitle[tenzorji]{Tenzorski produkt}

    \begin{definicija}[Tenzorski produkt]
        Tenzorski produkt prostorov \(\H[m]\) in \(\H[n]\) je enak \( \H[m+n] \).\\
        Če sta \( a ∈ \H[m] \) in \( b ∈ \H[n] \) je \( a⊗b ∈ \H[m]⊗\H[n] = \H[m+n] \).
    \end{definicija}

    \pause
    \begin{primer}[\(n=m=1\)]
        \[ \vec{a₀, a₁}⊗\vec{b₀, b₁} = \vec{a₀b₀, a₀b₁, a₁b₀, a₁b₁} \]
    \end{primer}
    \begin{posledica}
        \[ \ket j⊗\ket k = \ket j \ket k = \ket{jk},\quad
           a⊗b = ∑_{\substack{j∈\B[n],\\k∈\B[m]}} aⱼbₖ\ket{jk}
        \]
    \end{posledica}

\end{frame}
%%%%%
% \begin{frame}
%     \begin{definicija}
%         Če lahko vektor \(a∈\H[n]\) zapišemo kot \(⨂_{j=1}^{n} aⱼ\) z \(aⱼ∈\H\) pravimo, da je enostaven ali separabilen, sicer je pa sestavljen ali kvantno prepleten.
%     \end{definicija}
% \end{frame}
%%%%%
\begin{frame}
    \frametitle{Unitarna vrata}

    \begin{definicija}[Unitarna vrata]
        Unitarna vrata reda \(n\) so unitarne matrike dimenzije \(2ⁿ\).
        \pause
        Tenzorski produkt \( U ⊗ V = [uⱼₖV]_{j,k} \) uporabljen na \(a⊗b\) je enak \(Ua⊗Vb\).
    \end{definicija}
    % \begin{definicija}[Bločno-diagonalna matrika] % lahko spustim?
    %     Za unitarne matrike \(U₁,… ,Uₙ\) označimo njihovo bločno-diagnoalno matriko z \(D{\p{Uₙ,…, Uₙ}}\).
    % \end{definicija}

    \begin{primer}[Tenzorski produkt unitarnih vrat]
        \[
            \begin{bmatrix}
                a₀₀ & a₀₁ \\ a₁₀ & a₁₁
            \end{bmatrix} ⊗ B
            =
            \begin{bmatrix}
                a₀₀ B & a₀₁ B \\ a₁₀ B & a₁₁ B
            \end{bmatrix}.
        \]
    \end{primer}

    % \pause
    % \begin{izrek}[No cloning]
    %     Ne obstajajo vrata (reda \(2\)), ki vsak \(a⊗\ket0 ∈ \H[2]\) slika v \(a⊗a\).
    % \end{izrek}

\end{frame}
%%%%%
\begin{frame}
    \frametitle{Blochova sfera}

    \parbox{.4\textwidth}{
        Kubit \(a\) predstavimo kot točko v \(𝕊²\) z identifikacijo:
        \[a = \cos\smallfrac{θ}{2}\ket{\mb0} + e^{iφ}\sin\smallfrac{θ}{2}\ket{\mb1}\]
    }\hfill\parbox{.5\textwidth}{
        \begin{tikzpicture}
            % Define radius
            \def\r{9}
            \def\q{\r/4}
            
            % Bloch vector
            \draw (0,0) node[circle,fill,inner sep=1] (orig) {} -- (\q/3,\q/2) node[circle,fill,inner sep=0.7,label=above:\(a\)] (a) {};
            \draw[dashed] (orig) -- (\q/3,-\q/5) node (phi) {} -- (a);
            
            % Sphere
            \draw               (orig) circle  (\q);
            \draw[dashed, gray] (orig) ellipse ({\q} and {\q/3});
            
            % Axes
            \draw[->           ] (orig) -- ++(-\q/5, -\q/3) node[below] (x) {\(x\)};
            \draw[->           ] (orig) -- ++( \q  ,   0  ) node[right] (y) {\(y\)};
            \draw[->           ] (orig) -- ++(  0  ,  \q  ) node[above] (z) {\(z=\ket{\mb0}\)};
            \draw[->, draw=gray] (orig) -- ++(  0  , -\q  ) node[below] (s) {\(\ket{\mb1}\)};
            
            %Angles
            \pic[draw=gray,->,"\(φ\)",angle eccentricity=0.6] {angle=x--orig--phi};
            \pic[draw=gray,<-,"\(θ\)",angle eccentricity=1.6] {angle=a--orig--z};
        \end{tikzpicture}
    }

\end{frame}
%%%%%
\begin{frame}
    \frametitle{Primeri}

    \begin{primer}[Paulijeve matrike]
        To so matrike rotacije okrog osi na Blochovi sferi:
        \[
            I₂ = \begin{bmatrix} 1 &  0 \\ 0 &  1 \end{bmatrix}\text{, }
            X  = \begin{bmatrix} 0 &  1 \\ 1 &  0 \end{bmatrix}\text{, }
            Y  = \begin{bmatrix} 0 & -i \\ i &  0 \end{bmatrix}\text{, }
            Z  = \begin{bmatrix} 1 &  0 \\ 0 & -1 \end{bmatrix}\text.
        \]
        Velja \(X² = Y² = Z² = I₂\).
    \end{primer}

    \pause
    \begin{primer}[Paulijeve matrike]
        \[ \Qcircuit @C=1em @R=.7em {
                & \gate{\g X} & \qw && \gate{\g Z} & \qw && \gate{\g{I₂}} & \qw
            }
        \]
        \[ \Qcircuit @C=1em @R=.7em {
                & \gate{\g Z} & \gate{\g X} & \qw
                &        =    &
                & \gate{\g{XZ}} & \qw
                &        =    &
                & \gate{-i\g Y} & \qw
            }
        \]
        % \[ \Qcircuit @C=1em @R=.7em {
        %     & \gate{\had} & \gate{\g Z} & \gate{\had} & \qw \\
        %     &             &      =      &             &     \\
        %     & \qw         & \gate{\g X} & \qw         & \qw
        %     }
        % \]
        % \[ \Qcircuit @C=1em @R=.7em {
        %     & \gate{\had} & \gate{\g Z} & \gate{\had} & \qw
        %     &                      =    &
        %     &               \gate{\g X} & \qw
        %     }
        % \]
    \end{primer}

\end{frame}
%%%%%
% \begin{frame}
%     \frametitle{Primeri}

%     % \begin{primer}[Fazni zamik]
%     %     \begin{gather*}
%     %         S_α = \begin{bmatrix} 1 & 0 \\ 0 & e^{iα}\end{bmatrix}
%     %         \text{, posebej označimo } S = S_{\frac{π}{2}}; S² = X,\\
%     %         S_α(a₀\ket 0 + a₁\ket 1) = a₀\ket 0 + a₁e^{iα}\ket 1
%     %     \end{gather*}
%     % \end{primer}

%     \begin{primer}[Hadamardova matrika]
%         Predstavlja rotacijo okrog \(x = z, y=0\) premice.
%         \[
%             \had = \rho\begin{bmatrix}1&1\\1&-1\end{bmatrix}\text{, }
%             % \had^{\otimes n}(\ket{0^n}) = \hh[n]
%             \had(\ket{\mb0}) = \hh
%         \]
%     \end{primer}

%     \pause
%     \begin{primer}[Hadamardova matrika]
%         \[ \Qcircuit @C=1em @R=.7em {
%                 & \gate{\had} & \gate{\g Z} & \gate{\had} & \qw \\
%                 &             &      =      &             &     \\
%                 & \qw         & \gate{\g X} & \qw         & \qw
%             }
%         \]
%         % \[ \Qcircuit @C=1em @R=.7em {
%         %     & \gate{\had} & \gate{\g Z} & \gate{\had} & \qw
%         %     &                      =    &
%         %     &               \gate{\g X} & \qw
%         %     }
%         % \]
%     \end{primer}

% \end{frame}
%%%%%
\begin{frame}[fragile]
    \frametitle{Kvantna meritev}

    \begin{definicija}[Kvantna meritev]
        Meritev kubita \(a = a_0\ket{\mb0} + a_1\ket{\mb1}\) označimo \(M(a)\) in je \(0\) z verjetnostjo \(|a_0|^2\) in \(1\) z verjetnostjo \(|a_1|^2\). To \airquotes{uniči} kubit \(a\).
    \end{definicija}
    \pause
    \begin{primer}[Pogojna uporaba vrat]
        \[ \Qcircuit @C=1em @R=.7em {
                \lstick{a} & \meter & \cctrlo{1}  & \cctrl{1}\\
                \lstick{b} & \qw    & \gate{\g V} & \gate{\g U} & \rstick{b'} \qw
            }
        \]
        \center\qpl{if |\(\emeasure{a} = 0\)| then |\( Ub \)| else |\( Vb \)|}
    \end{primer}

    % \begin{primer}[Projekcija na Z os in naključna rotacija faze]
    %     \[ \Qcircuit @C=1em @R=.7em {
    %             \lstick{a} & \meter & \cctrl{1} &&&&
    %             &&& \lstick{\ket0} & \gate{\had} & \meter & \cctrl{1}\\
    %             & \lstick{\ket0} & \gate{\g X} & \rstick{a'} \qw &&&&
    %             \lstick{a} & \qw & \qw & \qw & \qw & \gate{\g Z} & \rstick{a'}\qw
    %         }
    %     \]
    % \end{primer}

\end{frame}
%%%%%
\begin{frame}[fragile]
    \frametitle{Kvantna kontrola}

    \begin{definicija}
        Kontrola \airquotes{na ena}.
        \begin{align*}
            C_{r,s}{\p U}\ket j &= \begin{cases}
                \ket j &;\quad jᵣ = 0\\
                \ket{j₁\dots}\ket{U{jₛ}}\ket{\dots jₙ} &;\quad jᵣ = 1
            % \end{cases}\\
            % \overline{C}_{r,s}{\p U}\ket j &= \begin{cases}
            %     \ket{j₁\dots}\ket{U{jₛ}}\ket{\dots jₙ} &;\quad jᵣ = 0\\
            %     \ket j &;\quad jᵣ = 1
            \end{cases}
        \end{align*}
    \end{definicija}
    Posebej za \( U ∈ \mathrm{U}₂ \) označimo \(\ctl{}U ≔ C_{1,2}{\p U} = \D{I₂, U}.\)
    % \[ \ctl{}  U ≔ C_{1,2}{\p U} = \D{I₂, U},\quad
    %    \ctlo{} U ≔ \overline{C}_{1,2}{\p U} = \D{U, I₂}. \]
    \begin{primer}[Pogojna uporaba vrat]
        \[ \Qcircuit @C=1em @R=.7em {
                \lstick{a} & \ctrlo{1}   & \ctrl{1}    & \rstick{a}  \qw\\
                \lstick{b} & \gate{\g V} & \gate{\g U} & \rstick{b'} \qw
            }
        \]
        \center\qpl{if |\(\emeasure{a} = 0\)| then |\( (a,Ub) \)| else |\( (a,Vb) \)|}
    \end{primer}
\end{frame}
%%%%%
% \begin{frame}[fragile]
%     \frametitle{Primeri}

%     \begin{primer}
%         \[ \Qcircuit @C=1em @R=.7em {
%                 \lstick{a} & \meter & \cctrlo{1}  & \cctrl{1}                    \\
%                 \lstick{b} & \qw    & \gate{\g U} & \gate{\g V} & \rstick{b'} \qw\\
%                 \lstick{a} & \qw    & \ctrlo{1}   & \ctrl{1}    & \rstick{a}  \qw\\
%                 \lstick{b} & \qw    & \gate{\g U} & \gate{\g V} & \rstick{b'} \qw
%             }
%         \]
%         \center\qpl{if |\(\emeasure{a} = 0\)| then |\( \p{a, Ub} \)| else |\( \p{a, Vb} \)|}
%     \end{primer}

%     \pause
%     \begin{primer}[Prepleteni pari kubitov]
%         \[ \ctl{X}(a⊗\ket 0) = \text{\airquotes{\(a⊗b\)}}
%            \onslide<3->{= a₀\ket{00} + a₁\ket{11}} \]
%         \vspace{-2em}
%         \[ \Qcircuit @C=1em @R=.7em {
%                 \lstick{a}          & \ctrl{1}    & \rstick{a} \qw \\
%                 \lstick{\ket0}      & \targ       & \rstick{b} \qw
%             }
%         \]
%     \end{primer}
%     % \begin{primer}[Projekcija na Z os]
%     %     \[ \Qcircuit @C=1em @R=.7em {
%     %             \lstick{a} & \meter & \cctrl{1}   &                &&&&
%     %             \lstick{a}          & \ctrl{1}    & \rstick{a} \qw &&&&
%     %             \lstick{a}          & \ctrl{1}    & \rstick{a} \qw \\
%     %             &\lstick{\ket0}     & \gate{\g X} & \rstick{b} \qw &&&&
%     %             \lstick{\ket0}      & \gate{\g X} & \rstick{b} \qw &&&&
%     %             \lstick{\ket0}      & \targ       & \rstick{b} \qw
%     %         }
%     %     \]
%     % \end{primer}

% \end{frame}
%%%%%
\begin{frame}
    \frametitle{Opazljivke}

    \begin{definicija}
        Opazljivka je sebi-adjungiran operator na prostoru \(\H[n]\) oziroma \(2ⁿ×2ⁿ\) hermitska matrika.
    \end{definicija}

    \pause
    \begin{definicija}
        Rezultati meritve opazljivke v stanju \(u\) je ena od lastnih vrednosti \(λⱼ\) z verjetnostjo \(|P_{λⱼ}u|²\), stanje \(u\) se pa po meritvi spremeni v izmerjeno stanje, torej \(P_{λⱼ}u\).

        % \emph{Pričakovana vrednost opazljivke v stanju \(u\)} je definirana kot \(\expval{A}{u} ≔ \matelt{u}{A}{u} = ∑ⱼ |P_{λⱼ}u|²λⱼ\).
    \end{definicija}
\end{frame}
%%%%%%%%%%%%%%%%%%%%%%%%%%%%%%%%%%%%%%%%%%%%%%%%%%%%%%%%%%%%%%%%%%%%%%%%%%%%%%%
% Effects
%%%%%%%%%%%%%%%%%%%%%%%%%%%%%%%%%%%%%%%%%%%%%%%%%%%%%%%%%%%%%%%%%%%%%%%%%%%%%%%
% \begin{frame}
%     \frametitle{Algebrajski učinki}

%     \begin{definicija}[Računski učinki]
%         Če ima funkcija ali operacija še kak navzven viden učinek poleg vrnjene vrednosti temu pravimo računski učinek (učinek računanja).
%     \end{definicija}
%     \pause
%     \begin{definicija}[Algebrajski učinki]
%         So računski učinki, ki jih lahko predstavimo z algebrajsko teorijo.
%     \end{definicija}

%     \begin{primeri}
%         Vhod in izhod, izjeme, nedeterminizem, spomin, %TODO: More examples
%     \end{primeri}

% \end{frame}
%%%%%
% \begin{frame}[fragile]
%     \frametitle{Programski jezik}

%     \begin{itemize}
%         \item Tip kubitov \texttt{qubit}.
%             \pause Funkcije dostopanja:
%         \begin{itemize}
%             \item \(\eff{new}\): Dodeli nov kubit, z začetno vrednostjo \(\ket 0\)
%             \item \(\eff{apply}_{\g V}\): Uporabi vrata \(U\) na danem vektorju
%             \item \(\eff{measure}\): Izvede meritev na kubitu, vrne element tipa \texttt{bit}
%         \end{itemize}
%         \pause
%         \item Za tipa \(A\) in \(B\) obstaja tip \(A⊗B\) prepletenih parov.
%     \end{itemize}
%     \begin{primer}[Prepleteni pari kubitov]
%         \[ \ctl{X}(a⊗\ket 0) = \text{\airquotes{\(a⊗b\)}} = a₀\ket{00} + a₁\ket{11} \]
%         \[ \Qcircuit @C=1em @R=.7em {
%                 \lstick{a}          & \ctrl{1}    & \rstick{a} \qw \\
%                 \lstick{\ket0}      & \targ       & \rstick{b} \qw
%             }
%         \]
%     \end{primer}

% \end{frame}
%%%%%%%%%%%%%%%%%%%%%%%%%%%%%%%%%%%%%%%%%%%%%%%%%%%%%%%%%%%%%%%%%%%%%%%%%%%%%%%
% Algebraic theory
%%%%%%%%%%%%%%%%%%%%%%%%%%%%%%%%%%%%%%%%%%%%%%%%%%%%%%%%%%%%%%%%%%%%%%%%%%%%%%%
% \begin{frame}[fragile]
%     \frametitle{Algebrajski jezik}

%     \begin{itemize}
%         \item Tip kubitov \texttt{qubit}.
%             \pause Funkcije dostopanja:
%         \begin{itemize}
%             \item \(\tnew{a}{t}\): Dodeli nov kubit, z začetno vrednostjo \(\ket 0\)
%             \item \(\tapply{U}{a}{b.t}\): Uporabi vrata \(U\) na danem vektorju
%             \item \(\tmeasure{a}{t}{u}\): Izmeri kubit, nadaljuje v \(t\) ali \(u\)
%             \item \(\tdiscard{a}{t} ≔ \tmeasure{a}{t}{t}\)
%         \end{itemize}
%         \pause
%         \item Za tipa \(A\) in \(B\) obstaja tip \(A⊗B\) prepletenih parov.
%     \end{itemize}

% \end{frame}
%%%%%
% \begin{frame}[fragile]
%     \frametitle{Pretvorba v algebrajske izraze}

%     \begin{primer}[Projekcija na Z os in naključna rotacija faze]
%         \begin{enumerate}
%             \item \begin{minted}[escapeinside=||,autogobble]{ocaml}
%                 if |\(\eff{measure}(a) = 0\)| then |$\eff{new}()$| else |\(\eff{apply}_{\g{X}}(\eff{new}())\)|
%             \end{minted}
%             \item \begin{minted}[escapeinside=||,autogobble]{ocaml}
%                 if |\(\eff{measure}(\eff{apply}_{\had}(\eff{new}())) = 0\)| then |$a$| else |\(\eff{apply}_{\g{Z}}(a)\)|
%             \end{minted}
%         \end{enumerate}
%     \end{primer}
%     \pause
%     \begin{itemize}
%         \item \(\tnew{a}{t} ≔\) \qpl{let |\(a ← \enew\)| in |\(t\)|}
%         \item \(\tapply{U}{a}{a.t} ≔\) \qpl{|\(\eapply{\g V}{a}\)|; |\(t\)|}
%         \item \(\tmeasure{a}{u}{v} ≔\) \qpl{if |\(\emeasure{a} = 0\)| then |\(u\)| else |\(v\)|}
%         \item \(\tdiscard{a}{u} ≔ \tmeasure{a}{u}{u}\)
%     \end{itemize}
%     \note{\(a\) je vektor različnih kubitov}
%     \pause
%     \begin{primer}[Projekcija na Z os in naključna rotacija faze]
%         \begin{enumerate}
%             \item \(\tmeasure{a}{\tnew{b}{t}}{\tnew{b}{\tapply{X}{b}{t}}}\)
%             \item \(\tnew{b}{\tapply{\had}{b}{\tmeasure{b}{t}{\tapply{Z}{a}{t}}}}\)
%         \end{enumerate}
%     \end{primer}
% \end{frame}
%%%%%
% \begin{frame}
%     \frametitle{Aksiomi}

%     \begin{enumerate}[(A)]
%         \item Kvantna negacija pred meritvijo je negacija po meritvi.
%             \begin{mathpar}
%                 %LTeX: enabled=false
%                 \inferrule{}{\Qcircuit @C=1em @R=.7em {
%                         & \gate{\g X} & \meter & \cw
%                 }}
%                 \and=\and
%                 \inferrule{}{\Qcircuit @C=1em @R=.7em {
%                     & \meter & \gate{\g X} & \cw
%                 }}
%                 %LTeX: enabled=true
%             \end{mathpar}
%         \item Kvantna kontrola je po meritvi kot klasična kontrola.
%             \begin{mathpar}
%                 %LTeX: enabled=false
%                 \inferrule{}{\Qcircuit @C=1em @R=.7em {
%                     & \meter & \cctrlo{1}  & \cctrl{1}   & \cw\\
%                     & \qw    & \gate{\g U} & \gate{\g V} & \qw
%                 }}
%                 \and=\and
%                 \inferrule{}{\Qcircuit @C=1em @R=.7em {
%                     & \ctrlo{1}   & \ctrl{1}    & \meter & \cw\\
%                     & \gate{\g U} & \gate{\g V} & \qw    & \qw
%                 }}
%                 %LTeX: enabled=true
%             \end{mathpar}
%         \item Kvantna vrata uporabljena na zavrženih kubitih so odveč.
%         \item Novi kubiti so \( \ket{\mb0} \) glede na meritev.
%             \begin{mathpar}
%                 %LTeX: enabled=false
%                 \inferrule{}{\Qcircuit @C=1em @R=.7em {
%                     \lstick{\ket{\mb0}} & \meter & \cw
%                 }}
%                 \and=\and
%                 \inferrule{}{\Qcircuit @C=1em @R=.7em {
%                     \lstick{0} & \cw & \cw & \cw
%                 }}
%                 %LTeX: enabled=true
%             \end{mathpar}
%         \item Novi kubiti so \( \ket{\mb0} \) glede na kontrolo.
%             \begin{mathpar}
%                 %LTeX: enabled=false
%                 \inferrule{}{\Qcircuit @C=1em @R=.7em {
%                     & \lstick{\ket{\mb0}} & \ctrlo{1}   & \ctrl{1}    & \qw\\
%                     & \qw                 & \gate{\g U} & \gate{\g V} & \qw
%                 }}
%                 \and=\and
%                 \inferrule{}{\Qcircuit @C=1em @R=.7em {
%                     &             &     & \lstick{\ket{\mb0}} & \qw\\
%                     & \gate{\g U} & \qw & \qw                 & \qw
%                 }}
%                 %LTeX: enabled=true
%             \end{mathpar}
%         \item[(…)] Plus še sedem manj zanimivih akisomov.
%     \end{enumerate}

%     % \note{Lahko prestavim malo kasneje}
%     % \note[item]{Kvantna negacija in kontrola se obnašata kot klasični verziji.}
%     % \note[item]{\(\op{discard}\) dela kot pričakujemo.}
%     % \note[item]{Novi kubiti so vedno \(\ket0\) glede na meritev in kontrolo.}
%     % \note[item]{Sklopa sta
%     %     \begin{enumerate}
%     %         \item \(\op{apply}\) se razume z matrikami.
%     %         \item Stvari komutirajo, kolikor lahko, do vezave spremenljivk.
%     %     \end{enumerate}
%     % }

% \end{frame}
%%%%%
% \begin{frame}
%     \frametitle{Formalna pravila algebrajskega jezika}

%     \note{Kaj s tem?}  %TODO: figure out

%     %TODO: add judgement rules
%     %TODO: add examples

% \end{frame}
%%%%%
% \begin{frame}
%     \frametitle{Aksiomi}
    
%     Osnovni aksiomi na kratko:
%     \begin{enumerate}
%         \item Kvantna negacija pred meritvijo je negacija po meritvi.
%         \item Kvantna kontrola je po meritvi kot klasična kontrola.
%         \item Kvantna vrata uporabljena na zavrženih kubitih so odveč.
%         \item Meritve novih kubitov so vedno \(0\).
%         \item Vrata kontrolirana z novimi kubiti se nikoli ne uporabijo.
%     \end{enumerate}
%     … Plus še sedem manj zanimivih akisomov.
% \end{frame}
%%%%%
\begin{frame}
    \frametitle{Jezik}

    % \begin{itemize}
    %     \item \(\tnew{a}{t}\): Dodeli nov kubit, z začetno vrednostjo \(\ket 0\)
    %     \item \(\tapply{U}{a}{t}\): Uporabi vrata \(U\) na danem vektorju
    %     \item \(\tmeasure{a}{t}{u}\): Izmeri kubit, nadaljuje v \(t\) ali \(u\)
    %     \item \(\tdiscard{a}{t} ≔ \tmeasure{a}{t}{t}\)
    % \end{itemize}
    
    % \pause
    \begin{definicija} %NOTE: na tablo?
        Členost je oblike \(\arity{p}{m₁, …, mₖ}\), kjer so \(p, mᵢ ∈ ℕ\).

        % Signatura z linearnimi parametri je množica operacij s členostmi.
        Neformalno členost pove, da operacija \(\op O\) sprejme \(p\) parametrov in \(k\) računskih spremenljivk, kjer \(i\)-ta sprejme \(mᵢ\) parametrov.
        Pišemo \(\op O : \arity{p}{m₁, …, mₖ}\).
    \end{definicija}

    \pause
    \begin{definicija}
        Interpretacija operacije s členostjo \(\arity{p}{m₁, …, mₖ}\) je preslikava oblike \(\semmap{p}{m₁, …, mₖ}\).
    \end{definicija}

    % \pause
    % \vspace{-2em}
    \note{\begin{gather*}
        % \op{O}            : \arity{p}{m₁, …, mₖ}\\
        \op{new}          : \arity{0}{1}\qquad
        \op{measure}      : \arity{1}{0, 0}\qquad
        \op{apply}_{\g U} : \arity{n}{n}
    \end{gather*}}

    % \pause
    % \vspace{-2em}
    % \[\begin{prooftree}
    %       \hypo{\sequent{Γ}{Δ, a}{         t } }
    %     \infer1{\sequent{Γ}{Δ   }{\tnew{a}{t}} }
    % \end{prooftree}\quad
    % \begin{prooftree}
    %       \hypo{\sequent{Γ}{Δ   }{             t    } }
    %       \hypo{\sequent{Γ}{Δ   }{                u } }
    %     \infer2{\sequent{Γ}{Δ, a}{\tmeasure{a}{t}{u}} }
    % \end{prooftree}\]
    % \[\begin{prooftree}
    %       \hypo{\sequent{Γ}{Δ, a₁, …, aₙ}{                      t }}
    %     \infer1{\sequent{Γ}{Δ, a₁, …, aₙ}{\tapply{U}{a₁, …, aₙ}{t}}}
    % \end{prooftree}\]
  
\end{frame}
%%%%%
% \begin{frame}
%     \frametitle{Sequents?}

%     \begin{definicija}
%         Za sezname računskih spremenljivk \(Γ\), sezname parametrov \(Δ\), in izraze \(t\) definiramo relacijo \(\seq{t}\) kot najmanjšo relacijo zaprto za naslednja pravila.
%     \end{definicija}

%     \[\begin{prooftree}
%           \hypo{ - }
%         \infer1{\sequent{Γ, x:p, Γ'}{a₁, …, aₚ}{x{\p{a₁, …, aₚ}}}}
%     \end{prooftree}\]
%     \[\begin{prooftree}
%           \hypo                         {\sequent{Γ}{a₁      , …, aₚ      }{t}}
%         \infer1[(σ \text{ permutacija})]{\sequent{Γ}{a_{σ(1)}, …, a_{σ(p)}}{t}}
%     \end{prooftree}\]
%     \[\begin{prooftree}
%           \hypo{\sequent{Γ}{Δ, bᵢ₁, …, b_{im₁}}{tᵢ}}\delims{\left(}{\right)_{i∈\{1,…,k\}}}
%           \hypo{\op O : \arity{p}{m₁, …, mₖ}}
%         \infer2{\sequent{Γ}{Δ, a₁, …, aₚ}{\op O{\p{a₁, …, aₚ; b₁₁…b_{1m₁}.t₁,…, bₖ₁…b_{kmₖ}.tₖ}}}}
%     \end{prooftree}\]

% \end{frame}
%%%%%%%%%%%%%%%%%%%%%%%%%%%%%%%%%%%%%%%%%%%%%%%%%%%%%%%%%%%%%%%%%%%%%%%%%%%%%%%
% C*-algebras and unitary matrices
%%%%%%%%%%%%%%%%%%%%%%%%%%%%%%%%%%%%%%%%%%%%%%%%%%%%%%%%%%%%%%%%%%%%%%%%%%%%%%%
\begin{frame}
    \frametitle{\(C^*\)-algebre}

    \begin{definicija}
        Množica \(A\) je \(C^*\)-algebra, če:
        \begin{enumerate}[(a)]
            \item je Banachova \(ℂ\)-algebra z enoto,
            \item ima involucijo \((-)^*\),
            \item za vsak \(a ∈ A\) velja \(\|a\|² = \|a^*a\|\).
        \end{enumerate}
    \end{definicija}

    \begin{primer}
        Množice \(\M[n] ≔ M_{n}(ℂ)\) so \(C^*\)-algebre.
    \end{primer}
    % Za nas so \(\M[n] ≔ M_{n}(ℂ)\) unitarne \(n×n\) matrike.

    % \note[item]{Involucija je hermitsko transponiranje}

    \begin{definicija}
        Preslikava \(f : A → B\) je \(*\)-homomorfizem, če je linearna in ohranja množenje, enoto, ter involucijo.
    \end{definicija}

\end{frame}
%%%%%
% \begin{frame}
%     \frametitle{Matrike}

%     \begin{trditev}
%         % Velja \(Mₙ(\M[p]) = \M[np]\) \note{\(Mₙ(X)\) predstavlja \(n\) prepletenih parov \(X\)}

%         Vsaka linearna preslikava \(f : X → Y\) se naravno razširi do \(Mₙ(f) : Mₙ(X) → Mₙ(Y)\).

%         Velja \(Mₙ(X⊕Y) ≅ Mₙ(X)⊕Mₙ(Y)\)
%     \end{trditev}

%     \pause
%     \begin{trditev}
%         Izraze algebrajske teorije interpretiramo z unitarnimi matrikami:
%         Izraz \(\sequent{x₁ : m₁, …, xₖ : mₖ}{a₁, …, aₚ}{t}\) interpretiramo kot linearno preslikavo \(\sem{t} : \semmap{p}{m₁, …, mₖ}\).
%     \end{trditev}

% \end{frame}
%%%%%%%%%%%%%%%%%%%%%%%%%%%%%%%%%%%%%%%%%%%%%%%%%%%%%%%%%%%%%%%%%%%%%%%%%%%%%%%
% Our model
%%%%%%%%%%%%%%%%%%%%%%%%%%%%%%%%%%%%%%%%%%%%%%%%%%%%%%%%%%%%%%%%%%%%%%%%%%%%%%%
\begin{frame}
    \frametitle{Osnovne operacije}

    \begin{definicija}
        Meritev in uporabo vrat interpretiramo z \(*\)-homomorfizmoma
        \( \mor{measure} : \semmap{1}{0, 0} \) in \( \mor{apply}_{\g U} : \semmap{p}{p}\),
        s predpisoma \[\smeasure{}{α}{β} = \p{\mat{α & 0 \\ 0 & β}}\qquad \sapply{U}{A} = U^*AU.\]
    \end{definicija}

    % \pause
    % \vspace{-2em}
    % \begin{gather*}
    %     % \op{O}            : \arity{p}{m₁, …, mₖ}\\
    %     % \op{new}          : \arity{0}{1}\qquad
    %     \op{measure}      : \arity{1}{0, 0}\qquad
    %     \op{apply}_{\g U} : \arity{p}{p}
    % \end{gather*}


\end{frame}
%%%%%
\begin{frame}[fragile]
    \frametitle{Polnost}

    \begin{izrek}[Polnost v posebnem]
        \begin{enumerate}
            \item Za vsak \(*\)-homomorfizem \(f : \semmap{p}{m₁,…,mₖ}\) obstaja izraz v algebrajski teoriji, ki ne vsebuje operacije \(\op{new}\), tako da je
            \(\seq{t}\) in \(\sem{t} = f\).
            % \(t : \arity{p}{m₁, …, mₖ}\) in \(\sem{t} = f\).
            \item Če \(\absseq{t, u}\) ne vsebujeta \(\op{new}\) in \(\sem{t} = \sem{u}\) lahko izpeljemo \(\absseq{t = u}\).
        \end{enumerate}
    \end{izrek}

\end{frame}
%%%%%
% \begin{frame}
%     \frametitle{Dokaz}

%     \begin{definicija}
%         Množica Bratelijevih diagram za signaturo \(\arity{p}{m₁,…,mₖ}\) je množica \(k\)-teric \((sᵢ)ᵢ\), tako da velja \( ∑_{i=1}^k sₖmₖ = p \).
%     \end{definicija}

%     \pause
%     \begin{izrek}
%         \(*\)-homomorfizmi \(\M[k] → \M[n]\) so oblike \(A ↦ U^*D(A,…,A,0)U\) za neko unitarno matriko \(U\). \note[item]{\(k ≤ n\), \(n = mk + r\)}
%     \end{izrek}

%     \begin{izrek}
%         \( μ : \brat{p}{m₁, …, mₖ} ↔ \semhom{p}{m₁, …, mₖ} : ρ \)
%         obstajata in \(ρμ = \id\).
%         \note[item]{\( μ(s₁,…,sₖ)(A₁,…,Aₖ) = D(A₁,…,A₁,A₂,…,Aₖ) \).}

%         Enakost \(ρ(f) = ρ(g)\) velja natanko tedaj, ko obstaja unitarna matrika \(U\), tako da za vsak \(\underline{A}\) velja \(f(\underline A) = U^*g(\underline A)U\).
%     \end{izrek}

%     % \note[item]{Dokaz: znamo prehajat med \(\bf{Brat}\) in \(\cstarcat\), želimo iz \(T\) v \(\cstarcat\), gremo prek \(\bf{Brat}\), znamo v \(\bf{Brat}\), iz \(\bf{Brat}\) gremo z \(\op{measure}\)}

%     % \note[item]{Tako dobimo točno, kar želimo (obliko \(\op{apply}(\op{measure}(…))\))}

%     % \note[item]{Ostane pokazati, da je to surjekcija, sledi, ker lahko uporabimo aksiome, da preuredimo vsak izraz v tako obliko, te pa dobimo vse}

% \end{frame}
%%%%%%
\begin{frame}
    \frametitle{Dodeljevanje novih kubitov}

    \begin{definicija}
        Dodeljevanje novih kubitov interpretiramo kot linearno preslikavo
        \(\mor{new} : \semmap{0}{1}\), s predpisom \[\snew{}{\mat{α₁₁ & α₁₂ \\ α₂₁ & α₂₂}} = α₁₁.\]
    \end{definicija}

    % \note{Dokaz, da ni \(*\)-homomorfizem:
    % \(\bmat{0&1\\0&0}\bmat{0&0\\1&0} = \bmat{1&0\\0&0}\)}

    \pause
    \begin{definicija}
        Element \(x\) \(C^*\)-algebre je pozitiven, če obstaja kak element \(y\), da je \(x = y^*y\).
    \end{definicija}

    \begin{definicija}
        Preslikava \(f\) je popolnoma pozitivna, če za vsak \(k ∈ ℕ\) preslikava \(Mₖ(f)\) ohranja pozitivnost elementov.
    \end{definicija}
\end{frame}
%%%%%
\begin{frame}
    \frametitle{Polnost 2: Electric boogaloo}

    \begin{izrek}[Polnost v splošnem]
        \begin{enumerate}
            \item Za vsako linearno preslikavo \(f : \semmap{p}{m₁,…,mₖ}\), ki je popolnoma pozitivna in enotska, obstaja izraz v algebrajski teoriji, tako da je
            % \(\sequent{x₁ : m₁, …, xₖ : mₖ}{a₁, …, aₚ}{t}\) in \(\sem{t} = f\).
            \(t : \arity{p}{m₁, …, mₖ}\) in \(\sem{t} = f\).
            \item Če \(\absseq{t, u}\) in \(\sem{t} = \sem{u}\) lahko izpeljemo \(\absseq{t=u}\).
        \end{enumerate}
    \end{izrek}

\end{frame}
%%%%%
\begin{frame}[fragile]
    \frametitle{Dokaz}

    \begin{izrek}[Stinespringov izrek o dilaciji]
        Naj bo \(f : \mc A → \M[p]\) popolnoma pozitivna in naj ohranja enoto. Tedaj obstaja \(q ≥ p\) in \(*\)-homomorfizem \(g : \mc A → \M[q]\), tako da je \(f(A) = g(A)|ₚ\).
        \[\begin{tikzcd}
            A \\
            {\M[q]} & {\M[p]}
            \arrow[from=2-1, to=2-2]
            \arrow["g"', from=1-1, to=2-1]
            \arrow["f", from=1-1, to=2-2]
        \end{tikzcd}\]
    \end{izrek}

    % \pause
    % \begin{izrek}[o minimalnosti dilacije]
    %     Lahko izberemo minimalno dilacijo; če je \(r ≥ p\) in \( h : \mc A → \M[p] \) \(*\)-homomorfizem tak, da je \( h(-)|ₚ = f(-) \) je \(r ≥ q\) in \(g(-) = Uh(-)U^*|_q\).
    % \end{izrek}

\end{frame}
%%%%%

% \begin{frame}[fragile]
%     \frametitle{Pretvorba v algebrajske izraze}

%     \begin{primer}[Projekcija na Z os in naključna rotacija faze]
%         \begin{enumerate}
%             \item \(\tmeasure{a}{\tnew{b}{x(b)}}{\tnew{b}{\tapply{X}{b}{x(b)}}}\)
%             \item \(\tnew{b}{\tapply{\had}{b}{\tmeasure{b}{x(a)}{\tapply{Z}{a}{x(a)}}}}\)
%         \end{enumerate}
%     \end{primer}
%     \pause
%     \begin{itemize}
%         \item \(\new{a}{t} \coloneq \tnew{a}{t}\)
%         \item \(\apply{U}{a}{t} \coloneq \tapply{U}{a}{t}\)
%         \item \(\measure{a}{t}{u} \coloneq \tmeasure{a}{t}{u}\)
%         \item \(\discard{a}{t} \coloneq \tdiscard{a}{t}\)
%     \end{itemize}
%     \pause
%     \begin{primer}[Projekcija na Z os in naključna rotacija faze]
%         \begin{enumerate}
%             \item \(\measure{b}{\p{\new{a}{x(a)}}}{\p{\new{a}{\apply{X}{a}{x(a)}}}}\)
%             \item \(\new{a}{\apply{\had}{a}{\measure{a}{x(b)}{\apply{Z}{b}{x(b)}}}}\)
%         \end{enumerate}
%     \end{primer}
% \end{frame}
%%%%%
\begin{frame}
    \frametitle{Aksiomi}

    \begin{enumerate}[(A)]
        \item Kvantna negacija pred meritvijo je negacija po meritvi.
            \begin{mathpar}
                %LTeX: enabled=false
                \inferrule{}{\Qcircuit @C=1em @R=.7em {
                        & \gate{\g X} & \meter & \cw
                }}
                \and=\and
                \inferrule{}{\Qcircuit @C=1em @R=.7em {
                    & \meter & \gate{\g X} & \cw
                }}
                %LTeX: enabled=true
            \end{mathpar}
        \item Kvantna kontrola je po meritvi kot klasična kontrola.
            \begin{mathpar}
                %LTeX: enabled=false
                \inferrule{}{\Qcircuit @C=1em @R=.7em {
                    & \meter & \cctrlo{1}  & \cctrl{1}   & \cw\\
                    & \qw    & \gate{\g U} & \gate{\g V} & \qw
                }}
                \and=\and
                \inferrule{}{\Qcircuit @C=1em @R=.7em {
                    & \ctrlo{1}   & \ctrl{1}    & \meter & \cw\\
                    & \gate{\g U} & \gate{\g V} & \qw    & \qw
                }}
                %LTeX: enabled=true
            \end{mathpar}
        \item Kvantna vrata uporabljena na zavrženih kubitih so odveč.
        \item Novi kubiti so \( \ket{\mb0} \) glede na meritev.
            \begin{mathpar}
                %LTeX: enabled=false
                \inferrule{}{\Qcircuit @C=1em @R=.7em {
                    \lstick{\ket{\mb0}} & \meter & \cw
                }}
                \and=\and
                \inferrule{}{\Qcircuit @C=1em @R=.7em {
                    \lstick{0} & \cw & \cw & \cw
                }}
                %LTeX: enabled=true
            \end{mathpar}
        \item Novi kubiti so \( \ket{\mb0} \) glede na kontrolo.
            \begin{mathpar}
                %LTeX: enabled=false
                \inferrule{}{\Qcircuit @C=1em @R=.7em {
                    & \lstick{\ket{\mb0}} & \ctrlo{1}   & \ctrl{1}    & \qw\\
                    & \qw                 & \gate{\g U} & \gate{\g V} & \qw
                }}
                \and=\and
                \inferrule{}{\Qcircuit @C=1em @R=.7em {
                    &             &     & \lstick{\ket{\mb0}} & \qw\\
                    & \gate{\g U} & \qw & \qw                 & \qw
                }}
                %LTeX: enabled=true
            \end{mathpar}
        \item[(…)] Plus še sedem manj zanimivih akisomov.
    \end{enumerate}

    % \note{Lahko prestavim malo kasneje}
    % \note[item]{Kvantna negacija in kontrola se obnašata kot klasični verziji.}
    % \note[item]{\(\op{discard}\) dela kot pričakujemo.}
    % \note[item]{Novi kubiti so vedno \(\ket0\) glede na meritev in kontrolo.}
    % \note[item]{Sklopa sta
    %     \begin{enumerate}
    %         \item \(\op{apply}\) se razume z matrikami.
    %         \item Stvari komutirajo, kolikor lahko, do vezave spremenljivk.
    %     \end{enumerate}
    % }

\end{frame}
%%%%%
\frame{
    \frametitle{Uporaba}

    \begin{align*}
        &\hspace{-3em}\measure{b}{\p{\new{a}{x(a)}}}{\p{\new{a}{\apply{X}{a}{x(a)}}}}&\\
        =&\new{a}{\measure{b}{x(a)}{\apply{X}{a}{x(a)}}}&\text{komutativnost}\\
        =&\new{a}{\apply{\ctl X}{b,a}{\measure{b}{x(a)}{x(a)}}}&{(2)}\\
        =&\new{a}{\apply{\ctl X}{b,a}{\discard{b}{x(a)}}}&\\
        =&\new{a}{\apply{\ctl X}{a,b}{\apply{\ctl X}{b,a}{\discard{a}{x(b)}}}}&\\
        =&\new{a}{\apply{\ctl X}{b,a}{\discard{a}{x(b)}}}&{(5)}\\
        =&\new{a}{\apply{\had}{a}{\apply{\ctl Z}{a,b}{\apply{\had}{a}{\discard{a}{x(b)}}}}}&\\
        =&\new{a}{\apply{\had}{a}{\apply{\ctl Z}{a,b}{\discard{a}{x(b)}}}}&{(3)}\\
        =&\new{a}{\apply{\had}{a}{\measure{a}{x(b)}{\apply{Z}{b}{x(b)}}}}&{(2)}
    \end{align*}
}
%%%%%

\end{document}
