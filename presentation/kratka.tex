% \documentclass[handout, slovene]{beamer}
\documentclass[slovene]{beamer}

\usetheme{boxes} % see http://www.deic.uab.es/~iblanes/beamer_gallery/ for lots of examples
% \usetheme{metropolis}
\usecolortheme{rose}
% \useinnertheme{circles}
\useoutertheme{split}
\setbeamertemplate{blocks}[rounded][shadow=true]

\setbeamertemplate{navigation symbols}{} % remove navigation symbols
\setbeamertemplate{footline}{} % remove title, too long

%next set colors - not needed
\setbeamercolor{title}{fg=black!70!black}
\setbeamercolor{frametitle}{fg=red!70!black}
\setbeamercolor{framesubtitle}{fg=green!30!black}
\setbeamercolor{author}{fg=red!70!black}
\setbeamercolor{institute}{fg=green!30!black}
\setbeamercolor{date}{fg=blue!50!black}

\usepackage[T1]{fontenc}        % kodiranje znakov v .pdf
% \usepackage[utf8]{inputenc}     % kodiranje znakov v .tex
% \usepackage[slovene]{babel}     % nastavimo slovenščino

\usepackage{fontspec}
\usepackage{unicode-math}

% \setmainfont{TeX Gyre Termes}
% \setmathfont{TeX Gyre Termes Math}

\usepackage[braket]{qcircuit}
\usepackage{tikz}
\usetikzlibrary{babel}
\usetikzlibrary{angles,quotes}

\usepackage{minted}

\usepackage{ulem}
\renewcommand{\ULdepth}{1.8pt}
\newcommand{\ul}[1]{\uline{#1}}

\newtheorem{izrek}[theorem]{Izrek}
\newtheorem{trditev}[theorem]{Trditev}
\newtheorem{posledica}[theorem]{Posledica}
\newtheorem{vprasanje}[theorem]{Vpra\v sanje}
\newtheorem{domneva}[theorem]{Domneva}
\newtheorem{definicija}{Definicija}
\newtheorem{zgled}{Zgled}
\newtheorem{lema}{Lema}
\newtheorem{primer}{Primer}

% \beamertemplatetransparentcovereddynamic

\title{Kvantni algebrajski učinki}
\author{Rok Strah}
\institute{mentor: doc. dr. Matija Pretnar}
\date{22.~11.~2021}

\newcommand{\N}{\mathbb N} % naravna števila
\newcommand{\Z}{\mathbb Z} % cela števila
\newcommand{\Q}{\mathbb Q} % racionalna števila
\newcommand{\R}{\mathbb R} % realna števila
\newcommand{\C}{\mathbb C} % kompleksna števila
\renewcommand{\d}{\mathrm d}
% \newcommand{\vphi}{\phi}
\renewcommand{\phi}{\varphi}
\newcommand{\eps}{\varepsilon}
\renewcommand{\hat}{\widehat}
\renewcommand{\tilde}{\widetilde}
\renewcommand{\bar}{\overline}

\newcommand{\p}[1]{\left( {#1} \right)}
\newcommand{\set}[2]{\left\{ #1 \mid #2 \right\}}
\newcommand{\newfrac}[2]{{}^{#1}\!/_{\!#2}}
\newcommand{\im}[1]{\mathrm{im}{\p{#1}}}

\newcommand{\had}{\mathtt{Had}}
\newcommand{\hh}[1][]{\mathbf{h}_{#1}}
\newcommand{\B}[1][]{\mathbf{B}_{#1}}
\renewcommand{\H}[1][]{\mathbf{H}_{#1}}
\renewcommand{\S}{\mathbb{S}}
\renewcommand{\L}[2][]{\mathcal{L}_{#1}{\p{#2}}}
\newcommand{\g}[1]{\mathtt{#1}}
\newcommand{\D}[1]{D{\p{\g{#1}}}}
\newcommand{\ctl}[1]{\g{c#1}}
\newcommand{\ctlo}[1]{\g{\bar{c}#1}}

\newcommand{\cmd}[1]{\textnormal{\sffamily#1}}
\newcommand{\eff}[1]{\textnormal{\sffamily\ul{#1}}}
\newcommand{\tnew}[2]{\cmd{new}{\p{#1.#2}}}
\newcommand{\tapply}[3]{\cmd{apply}_{\g #1}{\p{#2.#3}}}
\newcommand{\tmeasure}[3]{\cmd{measure}{\p{#1; #2,#3}}}
\newcommand{\tdiscard}[2]{\cmd{discard}{\p{#1.#2}}}
\newcommand{\new}[2]{\nu{#1}.\,#2}
\newcommand{\apply}[3]{{\g #1}_{#2}{\p{#3}}}
\renewcommand{\measure}[3]{#2 {\;?_{#1}\,} #3}
\newcommand{\discard}[2]{\cmd{disc}_{#1}{\p{#2}}}

\begin{document}
%%%%%
\frame{\titlepage}
%%%%%
% Preliminaries
%%%%%
% \frame{
%     \frametitle{Kvantni simboli}
%     \begin{align*}
%         \hh = \rho \p{\ket 0 + \ket 1}, \rho = \frac{1}{\sqrt{2}}\\
%         \H, \H[n], \B, \B[n], \H[n] = \L[\C]{\B[n]}\\
%         \D{\g U, \g V}, \ctl U = \D{I_n, U}, \ctlo U = \D{U, I_n}
%     \end{align*}
%     % \[ \Qcircuit @C=1em @R=.7em {
%     %         & \gate U & \meter & \qw \\
%     %         & \ctrlo{1} & \ctrl{1} & \qw & & \ctrl{1} & \qw \\
%     %         & \gate U & \gate V & \qw & & \targ & \qw \\
%     %         & \qswap & \qw & & & \ctrl{1} & \targ & \ctrl{1} & \qw \\
%     %         & \qswap \qwx & \qw & & & \targ & \ctrl{-1} & \targ & \qw
%     %     }
%     % \]
% }

\frame{
    \frametitle[q-Vektorji]{Kvantni vektorji}

    \begin{definicija}[Binarni vektorji]
        Binarni vektorji so elementi prostora \( \B[n] \coloneq 2^n \) in jih pišemo kot nize.
    \end{definicija}
    Primer: \(\B[2] = \{00, 01, 10, 11\}\).

    \begin{definicija}[Kvantni prostor]
        Kvantni vektorji ali q-vektorji so elementi prostora \( \H[n] \coloneq \C^{2^n} \).
        Kubiti so elementi \( \H \coloneq \H[1] \).\\
        Če je \( \{e_j\} \) standardna baza \( \H[n] \) pišemo \( \ket j \coloneq e_j \).
    \end{definicija}
    Očitno je \(\H[n] = \L[\C]{\set{\ket j}{j\in\B[n]}}\).
    \pause
    \begin{definicija}[Kvantna vrata]
        Kvantna vrata reda \( n \) so unitarne matrike dimenzije \( 2^n \).
        % Tenzorski produkt \( U \otimes V = [u_{jk}V]_{j,k} \) uporabljen na \(a\otimes b\) je enak \( Ua \otimes Vb \).
    \end{definicija}
}

\frame{
    \frametitle{Primeri}

    \begin{primer}[\(n = 1\)]
        \[ 
            a = \begin{bmatrix}a_0 \\ a_1\end{bmatrix}
              = a_0\ket 0 + a_1\ket 1
              = a_0\begin{bmatrix}1 \\ 0\end{bmatrix} + a_1\begin{bmatrix}0 \\ 1\end{bmatrix}\text.
        \]
    \end{primer}

    \begin{primer}[\(n = 2\)]
        \[
            a = \begin{bmatrix}a_0\\ a_1\\ a_2\\ a_3\end{bmatrix}
              = \begin{bmatrix}a_{00}\\ a_{01}\\ a_{10}\\ a_{11}\end{bmatrix}
              = a_{00}\ket{00} + a_{01}\ket{01} + a_{10}\ket{10} + a_{11}\ket{11}\text.
        \]
    \end{primer}

    \begin{primer}[Hadamardov vektor] % maybe skip this one?
        \[\hh \coloneq \rho \left(\ket 0 + \ket 1\right),\quad \hh[n] \coloneq \rho^n\textstyle\sum_{j\in\B[n]} \ket j\text. \]
    \end{primer}
}

\frame{
    \frametitle[Bloch]{Blochova sfera}

    \parbox{.4\textwidth}{
        Kubit \(a\) predstavimo kot točko v \(\S^2\) z identifikacijo:
        \[a = \cos\frac{\theta}{2}\ket0 + e^{i\varphi}\sin\frac{\theta}{2}\ket1\]
    }\hfill\parbox{.5\textwidth}{
        \begin{tikzpicture}
            % Define radius
            \def\r{5}
            
            % Bloch vector
            \draw (0,0) node[circle,fill,inner sep=1] (orig) {} -- (\r/6,\r/4) node[circle,fill,inner sep=0.7,label=above:\(a\)] (a) {};
            \draw[dashed] (orig) -- (\r/6,-\r/10) node (phi) {} -- (a);
            
            % Sphere
            \draw (orig) circle (\r/2);
            \draw[dashed, gray] (orig) ellipse (\r/2 and \r/6);
            
            % Axes
            \draw[->] (orig) -- ++(-\r/10,-\r/6) node[below] (x) {\(x\)};
            \draw[->] (orig) -- ++(\r/2,0) node[right] (y) {\(y\)};
            \draw[->] (orig) -- ++(0,\r/2) node[above] (z) {\(z=\ket 0\)};
            \draw[->, draw=gray] (orig) -- ++(0,-\r/2) node[below] (s) {\(\ket 1\)};
            
            %Angles
            \pic[draw=gray,->,"\(\varphi\)",angle eccentricity=0.6] {angle=x--orig--phi};
            \pic[draw=gray,<-,"\(\theta\)",angle eccentricity=1.6] {angle=a--orig--z};
        \end{tikzpicture}
    }
}

% \frame{
%     \frametitle[tenzorji]{Tenzorski produkt}

%     \begin{definicija}[Tenzorski produkt]
%         Tenzorski produkt prostorov \( \H[n] \) in \( \H[m] \) je enak \( \H[n+m] \).\\
%         Če sta \( a\in\H[n] \) in \( b\in\H[m] \) je \( a\otimes b \in \H[n]\otimes\H[m] \).
%     \end{definicija}

%     \begin{primer}[\(n=m=1\)]
%         \[
%             \begin{bmatrix}a_0 \\ a_1\end{bmatrix}\otimes \begin{bmatrix}b_0 \\ b_1\end{bmatrix}
%             = \begin{bmatrix}a_0b_0\\ a_0b_1\\ a_1b_0\\ a_1b_1\end{bmatrix}
%         \]
%     \end{primer}
%     \begin{posledica}
%         \[\ket j \otimes \ket k = \ket j \ket k = \ket{j\#k}, a\otimes b = \sum_{\substack{j\in\B[n],\\k\in\B[m]}} a_jb_k\ket{jk}\]
%     \end{posledica}
% }

% \frame{
%     \frametitle{Primeri}
    
%     \begin{primer}
%         \[
%             \hh[n] = \hh^{\otimes n}
%             = \rho^n\underbrace{(\ket 0 + \ket 1)\otimes\dots\otimes(\ket 0 + \ket 1)}_n\text.
%         \]
%     \end{primer}
%     \begin{primer}
%         \[\H[n] = \H^{\otimes n}\text.\]
%     \end{primer}

%     \begin{definicija}
%         Če lahko q-vektor \(a\in\H[n]\) zapišemo kot \(\bigotimes_{j=1}^{n} a_j; a_j\in\H\) pravimo, da je enostaven ali separabilen, sicer je pa sestavljen ali kvantno prepleten.
%     \end{definicija}
% }

% \frame{
%     \frametitle{Unitarna vrata}

%     \begin{definicija}[Unitarna vrata]
%         Unitarna vrata reda \( n \) so unitarna matrika dimenzije \( 2^n \).
%         % Tenzorski produkt \( U \otimes V = [u_{jk}V]_{j,k} \) uporabljen na \(a\otimes b\) je enak \( Ua \otimes Vb \).
%     \end{definicija}
%     % \begin{definicija}[Bločno-diagonalna matrika] % lahko spustim?
%     %     Za unitarne matrike \(U_0,\dots,U_n\) označimo njihovo bločno-diagnoalno matriko z \(D\p{U_0,\dots,U_n}\).
%     % \end{definicija}
%     % TODO: QPL, vezje

%     % \begin{izrek}[No cloning]
%     %     Ne obstaja unitarna matrika (vrata reda \(2\)), ki vsak \(a\otimes\ket{0}\in \H\otimes\H\) slika v \(a\otimes a\).
%     % \end{izrek}
% }

\frame{
    \frametitle{Primeri}

    \begin{primer}[Paulijeve matrike]
        To so matrike zrcaljenja okrog osi na Blochovi sferi:
        \[
            I_2 = \begin{bmatrix}1&0\\0&1\end{bmatrix}\text{, }
            X   = \begin{bmatrix}0&1\\1&0\end{bmatrix}\text{, }
            Y   = \begin{bmatrix}0&-i\\i&0\end{bmatrix}\text{, }
            Z   = \begin{bmatrix}1&0\\0&-1\end{bmatrix}\text.
        \]
        Velja \(X^2 = Y^2 = Z^2 = I_2\).
    \end{primer}

    \begin{primer}[Hadamardova matrika]
        \[
            \had = \rho\begin{bmatrix}1&1\\1&-1\end{bmatrix}\text{, }
            % \had^{\otimes n}(\ket{0^n}) = \hh[n]
            \had(\ket{0}) = \hh
        \]
    \end{primer}

    \[ \Qcircuit @C=1em @R=.7em {
            & \gate{\had} & \gate{\g Z} & \gate{\had} & \qw\\
            & \qw         & \gate{\g X} & \qw         & \qw
        }
    \]
    % \begin{primer}[Fazni zamik]
    %     \[
    %         S_\alpha = \begin{bmatrix}1&0\\0&e^{i\alpha}\end{bmatrix}%\text{, posebej označimo }
    %         %S = S_{\frac{\pi}{2}}; S^2 = X
    %         \text{, } S_\alpha(a_0\ket 0 + a_1\ket 1) = a_0\ket 0 + a_1e^{i\alpha}\ket 1
    %     \]
    % \end{primer}
}

\frame{
    \frametitle[Meritev]{Kvantna meritev}

    \begin{definicija}[Kvantna meritev]
        Meritev kubita \(a = a_0\ket0 + a_1\ket1\) označimo \(M(a)\) in je \(0\) z verjetnostjo \(|a_0|^2\) in \(1\) z verjetnostjo \(|a_1|^2\). To ``uniči'' kubit \(a\).
        \[ \Qcircuit @C=1em @R=.7em {
                & \meter & \cw %& & \ctrlo{1} & \ctrl{1} & \qw & & \ctrl{1} & \qw \\
                % &        &     & & \gate U & \gate V & \qw & & \targ & \qw
            }
        \]
    \end{definicija}
    \pause
    % \begin{definicija}[Kontrola]
    %     Kontrolirana vrata so …
    % \end{definicija}

    % TODO: QPL, vezje
    \begin{primer}[Projekcija na Z os]
    \[ \Qcircuit @C=1em @R=.7em {
            & \lstick{\ket0} & \gate{\g X} & \rstick{b} \qw\\
            \lstick{a} & \meter & \cctrl{-1}
        }
    \]
    \end{primer}

    \begin{primer}[Naključna rotacija faze]
    \[ \Qcircuit @C=1em @R=.7em {
            \lstick{a} & \qw & \qw & \qw & \qw & \gate{\g Z} & \rstick{a}\qw\\
            && \lstick{\ket0} & \gate{\had} & \meter & \cctrl{-1}
        }
    \]
        
    \end{primer}
    % \[ \Qcircuit @C=1em @R=.7em {
    %         & \lstick{\ket0} & \gate{\g X} & \rstick{b} \qw &&&&
    %         \lstick{a} & \qw & \qw & \qw & \qw & \gate{\g Z} & \rstick{a}\qw\\
    %         \lstick{a} & \meter & \cctrl{-1} &&&&
    %         &&& \lstick{\ket0} & \gate{\had} & \meter & \cctrl{-1}
    %     }
    % \]

}

% \frame{
%     \frametitle[Vezja]{Grafični prikaz}

%     Grafično predstavimo kot kvantna vezja:
%     \[ \Qcircuit @C=1em @R=.7em {
%             & \gate{\g U} & \qw & & \meter & \cw \\
%             & \ctrl{1} & \qw & & \ctrlo{1} & \ctrl{1} & \qw \\
%             & \targ & \qw & & \gate{\g U} & \gate{\g V} & \qw
%         }
%     \]
% }

\frame{
    \frametitle[Učinki]{Algebrajski učinki}

    \begin{definicija}[Računski učinki]
        Če ima funkcija ali operacija še kak navzven viden učinek poleg vrnjene vrednosti temu pravimo računski učinek (učinek računanja).
    \end{definicija}
    \pause
    \begin{definicija}[Algebrajski učinki]
        So računski učinki, ki jih lahko predstavimo z algebrajsko teorijo.
    \end{definicija}
}
%%%%%
% Actual stuff
%%%%%
\frame{
    \frametitle{Motivacija}

    \begin{itemize}
        \item Kvantni programski jeziki predstavljajo nove probleme za teorijo programskih jezikov.
        \begin{itemize}
            \item Linearnost
            \item Učinki
        \end{itemize}
        \pause
        \item Dober način razumevanja je, da razumemo enakost programov.
    \end{itemize}
}

\frame{
    \frametitle{Pregled}

    \begin{itemize}
        \item Predstavimo algebrajsko teorijo za kvantne izračune
        \begin{itemize}
            \item Zgrajena na unitarnih vratih in meritvah
            \item Ima linearne parametre
        \end{itemize}
        \item Dokažemo polnost v tej teoriji
        \item Iz algebre izpeljemo pravila za enakost kvantnih programov
    \end{itemize}
}

\begin{frame}[fragile]
    \frametitle{Kvantno računalništvo}

    Kaj sploh imamo?
    \begin{itemize}
        \item Tip kubitov \texttt{qubit}. Funkcije dostopanja:
        \begin{itemize}
            \item \eff{new}: dodeli nov kubit, z začetno vrednostjo \(\ket 0\).
            \item \(\eff{apply}_{\g U}\): Uporabi vrata \(U\) na danem kubitu.
            \item \eff{measure}: izvede meritev na kubitu, vrne element tipa \texttt{bit}.
        \end{itemize}
        \pause
        \item Za tipa \(A\) in \(B\) obstaja tip \(A\otimes B\) prepletenih parov.
    \end{itemize}
    \begin{primer}[Prepleteni pari kubitov]
        Obstajajo t.i. kontrolirana vrata, na primer \(\ctl X\).
        \(\eff{apply}_{\ctl X}(a, b)\) se obnaša kot
        \mintinline[escapeinside=||]{ocaml}{if |\(\eff{measure}(a) = 0\)| then |$(a, b)$| else |$(a, ¬b)$|}.
    \end{primer}
\end{frame}

\begin{frame}[fragile]
    \frametitle{Pretvorba v algebrajske izraze}

    \begin{primer}[Projekcija na Z os in naključna rotacija faze]
        \begin{enumerate}
            \item \begin{minted}[escapeinside=||,autogobble]{ocaml}
                if |\(\eff{measure}(a) = 0\)| then |$\eff{new}()$| else |\(\eff{apply}_{\g{X}}(\eff{new}())\)|
            \end{minted}
            \item \begin{minted}[escapeinside=||,autogobble]{ocaml}
                if |\(\eff{measure}(\eff{apply}_{\had}(\eff{new}())) = 0\)| then |$a$| else |\(\eff{apply}_{\g{Z}}(a)\)|
            \end{minted}
        \end{enumerate}
    \end{primer}
    \pause
    \begin{itemize}
        \item \(\tnew{a}{x(a)} \coloneq\) 
            \mintinline[escapeinside=||]{ocaml}{let |\(a \leftarrow\eff{new}()\)| in |$x(a)$|}
        \item \(\tapply{U}{a}{x(a)} \coloneq\) 
            \mintinline[escapeinside=||]{ocaml}{|\(\eff{apply}_{\g{U}}(a)\)|; |$x(a)$|}
        \item \(\tmeasure{a}{t}{u} \coloneq\) 
            \mintinline[escapeinside=||]{ocaml}{if |\(\eff{measure}(a) = 0\)| then |$t$| else |$u$|}
        \item \(\tdiscard{a}{t} \coloneq \tmeasure{a}{t}{t}\)
    \end{itemize}
    \pause
    \begin{primer}[Projekcija na Z os in naključna rotacija faze]
        \begin{enumerate}
            \item \(\tmeasure{a}{\tnew{b}{x(b)}}{\tnew{b}{\tapply{X}{b}{x(b)}}}\)
            \item \(\tnew{b}{\tapply{\had}{b}{\tmeasure{b}{x(a)}{\tapply{Z}{a}{x(a)}}}}\)
        \end{enumerate}
    \end{primer}
\end{frame}

\begin{frame}[fragile]
    \frametitle{Pretvorba v algebrajske izraze}

    \begin{primer}[Projekcija na Z os in naključna rotacija faze]
        \begin{enumerate}
            \item \(\tmeasure{a}{\tnew{b}{x(b)}}{\tnew{b}{\tapply{X}{b}{x(b)}}}\)
            \item \(\tnew{b}{\tapply{\had}{b}{\tmeasure{b}{x(a)}{\tapply{Z}{a}{x(a)}}}}\)
        \end{enumerate}
    \end{primer}
    \pause
    \begin{itemize}
        \item \(\new{a}{t} \coloneq \tnew{a}{t}\)
        \item \(\apply{U}{a}{t} \coloneq \tapply{U}{a}{t}\)
        \item \(\measure{a}{t}{u} \coloneq \tmeasure{a}{t}{u}\)
        \item \(\discard{a}{t} \coloneq \tdiscard{a}{t}\)
    \end{itemize}
    \pause
    \begin{primer}[Projekcija na Z os in naključna rotacija faze]
        \begin{enumerate}
            \item \(\measure{b}{\p{\new{a}{x(a)}}}{\p{\new{a}{\apply{X}{a}{x(a)}}}}\)
            \item \(\new{a}{\apply{\had}{a}{\measure{a}{x(b)}{\apply{Z}{b}{x(b)}}}}\)
        \end{enumerate}
    \end{primer}
\end{frame}

\frame{
    \frametitle{Aksiomi}
    
    Osnovni aksiomi na kratko:
    \begin{enumerate}
        \item Kvantna negacija pred meritvijo je negacija po meritvi.
        \item Kvantna kontrola je po meritvi kot klasična kontrola.
        \item Kvantna vrata uporabljena na zavrženih kubitih so odveč.
        \item Meritve novih kubitov so vedno \(0\).
        \item Vrata kontrolirana z novimi kubiti se nikoli ne uporabijo.
        \item Plus še sedem manj zanimivih akisomov.
    \end{enumerate}
}

\frame{
    \frametitle{Uporaba}

    \begin{align*}
        &\hspace{-3em}\measure{b}{\p{\new{a}{x(a)}}}{\p{\new{a}{\apply{X}{a}{x(a)}}}}&\\
        =&\new{a}{\measure{b}{x(a)}{\apply{X}{a}{x(a)}}}&\text{komutativnost}\\
        =&\new{a}{\apply{\ctl X}{b,a}{\measure{b}{x(a)}{x(a)}}}&{(2)}\\
        =&\new{a}{\apply{\ctl X}{b,a}{\discard{b}{x(a)}}}&\\
        =&\new{a}{\apply{\ctl X}{a,b}{\apply{\ctl X}{b,a}{\discard{a}{x(b)}}}}&\\
        =&\new{a}{\apply{\ctl X}{b,a}{\discard{a}{x(b)}}}&{(5)}\\
        =&\new{a}{\apply{\had}{a}{\apply{\ctl Z}{a,b}{\apply{\had}{a}{\discard{a}{x(b)}}}}}&\\
        =&\new{a}{\apply{\had}{a}{\apply{\ctl Z}{a,b}{\discard{a}{x(b)}}}}&{(3)}\\
        =&\new{a}{\apply{\had}{a}{\measure{a}{x(b)}{\apply{Z}{b}{x(b)}}}}&{(2)}
    \end{align*}
}
%%%%%

\end{document}
