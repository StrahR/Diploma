\documentclass[a4paper]{article}
\usepackage[british]{babel}

% \title{Kvantni algebrajski učinki}
\title{Quantum Algebraic Effects}
\author{Strah,\quad mentor: doc.~dr.~Matija~Pretnar}
% \mentor{doc.~dr.~Matija~Pretnar}
\date{\today}

\usepackage{cmap}
\usepackage[T1]{fontenc}
\usepackage[utf8]{luainputenc}
\usepackage{fontspec}
\usepackage{lualatex-math}
\usepackage{unicode-math}
\usepackage[backend=biber,bibencoding=utf8,backref=true]{biblatex}
% \usepackage{hyperxmp}
\usepackage[pdfa]{hyperref}
% \hypersetup{%
%     unicode,
%     pdfapart=1,
%     pdfaconformance=B,
%     pdfauthor={Strah},
%     pdfdate={\today},
%     pdftitle={Kvantni algebrajski učinki},
% }


% \setmainfont{TeX Gyre Pagella}
% \setmathfont{TeX Gyre Pagella Math}
\setmathfont{Latin Modern Math}
\setmathfont{Asana Math}[range={scr}]
\setmathfont{STIX Two Math-Regular}[range={bb}]
% \setmathfont{Asana Math}[range={"007B,"007D}]  % {}
\setmathfont{Asana Math}[range={8709}]  % U+2205, emptyset

% \usepackage{mathtools}
\usepackage{amsthm}

\usepackage[braket]{qcircuit}
\usepackage{tikz}
\usetikzlibrary{babel}
\usetikzlibrary{angles,quotes}

\usepackage{minted}

\usepackage[style=english]{csquotes}
\usepackage{stmaryrd}

\usepackage[normalem]{ulem}
\renewcommand{\ULdepth}{1.8pt}
\newcommand{\ul}[1]{\uline{#1}}


\newtheorem{theorem}{Theorem}
\newtheorem{statement}{Statement}
\newtheorem{lemma}{Lemma}
\newtheorem*{corrolary}{Corrolary}

\theoremstyle{definition}
\newtheorem{definition}{Definicija}

\newtheorem*{example}{Example}
\newtheorem*{example*}{Example}
\newtheorem*{examples}{Examples}

\theoremstyle{remark}
\newtheorem*{remark}{Remark}

\newcounter{axiom}
\setcounter{axiom}{0}
\renewcommand\theaxiom{\Alph{axiom}}
\newenvironment{axiom}[2][-1]{
    \ifnum#1>0
        \refsetcounter{axiom}{#1}
    \fi
    \refstepcounter{axiom}
    \setlength\parindent{0pt}
    \ifblank{#2}{}{\vskip.5em #2\par}
    \textbf{Axiom \theaxiom.}
}{
}

% \topmargin=-0.45in
% \evensidemargin=0in
% \oddsidemargin=0in
% \textwidth=6.5in
% \textheight=9.0in
% \headsep=0.25in

% \RequirePackage{array}
% \renewcommand{\arraystretch}{1.4}
% \renewcommand{\arraystretch}{2}
% \linespread{0.5}
% \linespread{1.111}

\textwidth 15cm
\textheight 245mm
\oddsidemargin.5cm
\evensidemargin.5cm
\topmargin-10mm
\addtolength{\footskip}{0pt}
\pagestyle{plain}
\overfullrule=15pt


% \renewcommand\headrulewidth{0.4pt}
% \renewcommand\footrulewidth{0.4pt}

% \setlength\parindent{0pt}

\newcommand{\N}{\mathbb N}
\newcommand{\Z}{\mathbb Z}
\newcommand{\Q}{\mathbb Q}
\newcommand{\R}{\mathbb R}
\newcommand{\C}{\mathbb C}
\renewcommand{\d}{\;\mathrm d}
% \newcommand{\vphi}{\phi}
\renewcommand{\phi}{\varphi}
\newcommand{\eps}{\varepsilon}
\renewcommand{\hat}{\widehat}
\renewcommand{\tilde}{\widetilde}
% \newcommand{\oldbar}{\bar}
% \renewcommand{\bar}{\overline}
\newcommand{\subs}{\subseteq}
\newcommand{\nin}{\not\in}

\newcommand{\p}[1]{\left( {#1} \right)}
\newcommand{\set}[2]{\left\{ #1 \mid #2 \right\}}
\newcommand{\newfrac}[2]{{}^{#1}\!/_{\!#2}}
\newcommand{\im}[1]{\mathrm{im}{\p{#1}}}
\newcommand{\mb}[1]{\mathbf{#1}}
\newcommand{\mf}[1]{\mathfrak{#1}}
\newcommand{\mc}[1]{\mathcal{#1}}

\newcommand{\had}{\mathtt{Had}}
\newcommand{\swap}{\mathtt{swap}}
\newcommand{\hh}[1][]{\mathbf{h}_{#1}}
\newcommand{\B}[1][]{\mathbf{B}_{#1}}
\newcommand{\HH}[1][]{\mathbf{H}_{#1}}
\renewcommand{\S}{\mathbb{S}}
\newcommand{\LL}[2][]{\mathcal{L}_{#1}{\p{#2}}}
\newcommand{\g}[1]{\mathtt{#1}}
\newcommand{\D}[1]{D{\p{\g{#1}}}}
\newcommand{\U}[1][]{\mathbf{U}_{#1}}
\newcommand{\ctl}[1]{\g{c#1}}
\newcommand{\ctlo}[1]{\g{\bar c} \g #1}

\newcommand{\cmd}[1]{\text{\sffamily#1}}
\newcommand{\eff}[1]{\text{\sffamily\ul{#1}}}
\newcommand{\enew}{\eff{new}{\p{}}}
\newcommand{\eapply}[2]{\eff{apply}_{\g{#1}}{\p{#2}}}
\newcommand{\emeasure}[1]{\eff{measure}{\p{#1}}}
\newcommand{\ediscard}[1]{\eff{discard}{\p{#1}}}
\newcommand{\tnew}[2]{\cmd{new}{\p{#1.\,#2}}}
\newcommand{\tapply}[3]{\cmd{apply}_{\g{#1}}{\p{#2.\,#3}}}
\newcommand{\tmeasure}[3]{\cmd{measure}{\p{#1.\,#2;#3}}}
\newcommand{\tdiscard}[2]{\cmd{discard}{\p{#1.\,#2}}}
\newcommand{\new}[2]{\nu{#1}.\,#2}
\newcommand{\apply}[3]{{\g{#1}}_{#2}{\p{#3}}}
\newcommand{\applyd}[3]{{\D{#1}}_{#2}{\p{#3}}}
\renewcommand{\measure}[3]{\p{#2} {\;?_{#1}\,} \p{#3}}
\newcommand{\measureraw}[3]{#2 {\;?_{#1}\,} #3}
\newcommand{\discard}[2]{\cmd{disc}_{#1}{\p{#2}}}

\newcommand{\sequent}[3]{#1 \mid #2 \vdash #3}
\newcommand{\seq}[1]{\sequent{\Gamma}{\Delta}{#1}}
\newcommand{\sem}[2][]{{\llbracket#2\rrbracket}_{#1}}

\newcommand{\airquotes}[1]{"`#1"'}

% \newcommand{\for}[2]{\forall#1.\;#2}
% \newcommand{\exist}[2]{\exists#1\smallni:\;#2}
% \newcommand{\existi}[2]{\exists!#1\smallni:\;#2}

\makeatletter
\newcommand{\oset}[3][0ex]{%
  \mathrel{\mathop{#3}\limits^{
    \vbox to#1{\kern-2\ex@
    \hbox{$\scriptstyle#2$}\vss}}}}
\makeatother

\newmintinline[qpl]{ocaml}{escapeinside=||, autogobble}

\makeatletter
\newcommand{\listintertext}{\@ifstar\listintertext@\listintertext@@}
\newcommand{\listintertext@}[1]{% \listintertext*{#1}
  \hspace*{-\@totalleftmargin}#1}
\newcommand{\listintertext@@}[1]{% \listintertext{#1}
  \hspace{-\leftmargin}#1}
\makeatother


\addbibresource{draft.bib}

% \renewcommand{\thesection}{\Roman{section}}

\begin{document}
\maketitle

\begin{abstract}
    % \subsubsection*{Motivacija} % TODO: Reword

    Quantum computing is based on many modern programming language theory concepts,
    % Kvantno računalništvo temelji na veliko modernih konceptih v teoriji programskih jezikov,
    % Kvantni programski jeziki predstavljajo nove probleme za teorijo programskih jezikov,
    such as linear types, (quantum) physics, and many others.
    % kot na primer linearnost tipov, (kvantnimi) fizikalnimi pojavi, in še mnogo drugimi.
    In my thesis I will be focusing on these two, but in this article I will only present the latter.
    % V diplomski nalogi se bomo posvetili tema dvema, v tem članku pa zgolj drugemu.
    % V tej nalogi se bomo posvetili tema dvema.
    % Dober način razumevanja je, da razumemo enakost programov.
    Our goal is understanding quantum programs,
    % Naš cilj je razumeti, kako se kvantni programi obnašajo,
    and a good way of understanding them is understanding program equality.
    % , that is, when are two programs equal.
    % in dober način je razumevanje enakosti programov.
    % tj. kdaj sta dva programa enaka.

    % \subsubsection*{Pregled}

    % Najprej bomo predstavili algebrajsko teorijo za kvantne programe;
    % ta je zgrajena na unitarnih vratih in meritvah ter ima linearne parametre.
    % Nato bomo dokazali, da lahko s to teorijo predstavimo vse programe (polnost)
    % in nato iz nje izpeljali pravila za enakost kvantnih programov.
\end{abstract}


\section{Quantum Mechanics}

This part is mostly summarised from \cite{ess-qc}.
% Ta del je povzet po \cite{ess-qc}.
It contains basic definitions (and examples) of mathematical basis of quantum mechanics
% Vsebuje osnovne definicije (in primere) matematičnih osnov kvantne mehanike,
that we need to define our desired qubit operations.
% ki jih potrebujemo za definicije želenih operacij nad kubiti.
% Te se nahajajo na prvih nekaj straneh.

\subsubsection*{Notation}
Throughout this part I will be using the following symbols:
\begin{itemize}
    \item \( ℕ = \{ 0, \dots \} \), \( ℕ_+ = \{ 1, \dots \} \), \( ℕₙ = \{ 0, \dots, 2ⁿ-1 \} \),
    \item \( n,m \in ℕ_+ \), which we call the number of qubits,
    \item \( j, k, \dots \in ℕₙ \),
    % \item \( j = jₙ \dots j₀ \) binarni zapis števila \( j \),
    \item \( aⱼ \) \( j \)-th component of vector \( a \),
    \item \( j = j₁ \dots jₙ \) the binary representation of \( j \).
    % \item \( \mb 0ⁿ = 0\dots0, \mb 1ⁿ = 1\dots1 \).
    % \item \( \mb n = \{ 0, \dots, n-1 \} \), na primer \( \mb 2 = \{ 0, 1 \} \).
    % \item \( \mathbb n = \{ 0, \dots, n-1 \} \), na primer \( \mathbb 2 = \{ 0, 1 \} \).
\end{itemize}

\subsection{Quantum Vectors}

\begin{definition}\label{binv}
    Binary vectors are elements of the space \( \B[n] ≔ \{ 0, 1 \}ⁿ \) and I will be writing them as stings of \(1\)s and \(0\)s.  For us they represent classical computation.
\end{definition}

\begin{example}
    \(\B[2] = \{00, 01, 10, 11\}\).
\end{example}
\begin{remark}
    \(1\) and \(01\) represent different vectors.
\end{remark}

\begin{definition}[Hilbert space]\label{hilb-sp}
    Elements of the space \( \HH[n] ≔ \C^{2ⁿ} \) are called quantum vectors, while elements of  \( \HH ≔ \HH[1] \) are qubits.  The space \( \HH[n] \) is thus called the space of quantum vectors of order \( n \), and its standard basis is denoted with \( \{eⱼ\} \).  This is where quantum processes take place.
\end{definition}

\begin{definition}[Braket notation]\label{braket}
    Let \( j \in ℕₙ \) and \( \hat{\jmath} \in \B[n] \) be the corresponding binary vector.  Then \( \ket j = \ket{\hat{\jmath}} ≔ eⱼ \).
\end{definition}
\begin{remark}
    By definition, it follows that \( \HH[n] = \LL[\C]{\set{\ket j}{j\in\B[n]}} \).
\end{remark}

\begin{example}[\( n = 1 \) and \( n = 2 \)]
    \begin{align*}
        a &= \begin{bmatrix}a₀\\a₁\end{bmatrix}
        = a₀\begin{bmatrix}1\\0\end{bmatrix} + a₁\begin{bmatrix}0\\1\end{bmatrix}
        = a₀\ket 0 + a₁\ket 1,\\
        a &= \begin{bmatrix}a₀\\a₁\\a₂\\a₃\end{bmatrix}
        = \begin{bmatrix}a₀₀\\ a₀₁\\ a₁₀\\ a₁₁\end{bmatrix}
        = a₀₀\ket{00} + a₀₁\ket{01} + a₁₀\ket{10} + a₁₁\ket{11}.
    \end{align*}
\end{example}

\begin{example}[Hadamard vector]\label{had}\(\)
    \vspace{-1em}
    \[ \hh ≔ ρ \left(\ket 0 + \ket 1\right),\quad
        \hh[n] ≔ ρⁿ\sum_{j\in\B[n]} \ket j,\quad
        ρ ≔ \frac{1}{\sqrt2}.
    \]
\end{example}

\subsection{Bloch Sphere}

\vspace{-1.5em}
\parbox[t]{.7\textwidth}{
    \begin{statement}
        In the physical world two qubits that differ by a (complex) factor represent the same state.
        Mathematically this means states of qubits (also qubits) live in \( \mathbf{P}\C¹ \cong \S² \):
        \[ a = \cos\frac{θ}{2}\ket0 + e^{iφ}\sin\frac{θ}{2}\ket1,\quad
            φ \in [0, 2π), θ \in [0, π]. \]
    \end{statement}
    \begin{proof}
        Let \( a = a₀\ket0 + a₁\ket1 = r₀e^{iφ₀}\ket0 + r₁e^{iφ₁}\ket1 \).\\
        Denote \( r ≔ \sqrt{r₀² + r₁²}, φ ≔ φ₁ - φ₀, θ ≔ 2\arccos{\frac{r₀}{r}} \).\\
        Then \( a = \hat a ≔ \frac{a}{re^{iφ₀}} = \frac{r₀}{r}\ket0 + \frac{r₁}{r}e^{iφ}\ket1 = \cos\frac{θ}{2}\ket0 + e^{iφ}\sin\frac{θ}{2}\ket1 \).
    \end{proof}
}\hfill
\begin{tikzpicture}[baseline=(current bounding box.north)]
    % Define radius
    \def\r{7/2}
    
    % Bloch vector
    \draw (0,0) node[circle,fill,inner sep=1] (orig) {} -- (\r/6,\r/4) node[circle,fill,inner sep=0.7,label=above:\(a\)] (a) {};
    \draw[dashed] (orig) -- (\r/6,-\r/10) node (phi) {} -- (a);
    
    % Sphere
    \draw (orig) circle (\r/2);
    \draw[dashed, gray] (orig) ellipse (\r/2 and \r/6);
    
    % Axes
    \draw[->] (orig) -- ++(-\r/10,-\r/6) node[below] (x) {\(x\)};
    \draw[->] (orig) -- ++(\r/2,0) node[right] (y) {\(y\)};
    \draw[->] (orig) -- ++(0,\r/2) node[above] (z) {\(z=\ket 0\)};
    \draw[->, draw=gray] (orig) -- ++(0,-\r/2) node[below] (s) {\(\ket 1\)};
    
    %Angles
    % \draw[->, draw=gray] 
    % \draw[draw=gray,->] (orig) ellipse (\r/6 and \r/12) node {\(φ\)};
    \pic[draw=gray,->,"\(φ\)",angle eccentricity=0.6] {angle=x--orig--phi};
    \pic[draw=gray,<-,"\(θ\)",angle eccentricity=1.4] {angle=a--orig--z};
\end{tikzpicture}

\subsection{Tensor Product}

\begin{definition}[Tensor product]\label{tensorprod}
    The tensor product of spaces \( \HH[n] \) and \( \HH[m] \) is equal to \( \HH[n+m] \).
    We write \( \HH[n]⊗\HH[m] \).
    If \( a\in\HH[n] \) and \( b\in\HH[m] \) then \( a⊗b \in \HH[n]⊗\HH[m] \).
\end{definition}
\begin{remark}
    The operator \(⊗\) is indeed a tensor product.
\end{remark}
\begin{example}[Tensor product of basis vectors]
    \[ \ket j ⊗ \ket k = \ket{j₁\dots jₙk₁\dots kₘ} = \ket j \ket k = \ket{jk},
    \]
\end{example}
\begin{example}[General tensor product]\(\)
    \vspace{-1em}
    \[ 
        \begin{bmatrix}a₀ \\ a₁\end{bmatrix}⊗\begin{bmatrix}b₀ \\ b₁\end{bmatrix}
        = \begin{bmatrix}a₀b₀\\ a₀b₁\\ a₁b₀\\ a₁b₁\end{bmatrix},\quad
        a⊗b = \sum_{\substack{j\in\B[n],\\k\in\B[m]}} aⱼbₖ\ket{jk}.
    \]
\end{example}
\begin{examples}[Tensor exponent]
    \begin{align*}
        &\hh[n] = \hh^{⊗n}
        = ρⁿ\underbrace{(\ket 0 + \ket 1) ⊗ \dots ⊗ \p{\ket 0 + \ket 1}}_n,\\
        &\HH[n] = \HH^{⊗n} = \underbrace{\HH ⊗ \dots ⊗ \HH}_n.
    \end{align*}
\end{examples}

\begin{definition}
    If \( a\in\HH[n] \) can be written as \( \bigotimes_{j=1}^{n} aⱼ \) for some \( aⱼ\in\HH \) we say it's simple or separable, otherwise it's composite or quantum entangled. % NOTE: simple? elementary?
\end{definition}

\subsection{Quantum Maps}

\begin{definition}%[Unitarna vrata]
    The space of unitary gates of order \( n \) is \( \U[n] ≔ \U{\p{2ⁿ}} \), the space of unitary \( 2ⁿ \times 2ⁿ \) matrices.
    The tensor product of gates \( U⊗V ≔ [u_{jk}V]_{j,k} \) used on \( a⊗b \) is \( Ua⊗Vb \).
\end{definition}

\begin{example}[Tensor product of unitary gates]
    \[
        \begin{bmatrix}
            a₀₀ & a₀₁ \\ a₁₀ & a₁₁
        \end{bmatrix} ⊗ B
        % \begin{bmatrix}
        %     b₀₀ & b₀₁ \\ b₁₀ & b₁₁
        % \end{bmatrix}
        =
        \begin{bmatrix}
            a₀₀ B & a₀₁ B \\ a₁₀ B & a₁₁ B
        \end{bmatrix}.
    \]
        % \begin{bmatrix}
        %     a₀₀b₀₀ & a₀₁b₀₀ & a₀₀b₀₁ & a₀₁b₀₁ \\
        %     a₁₀b₀₀ & a₁₁b₀₀ & a₁₀b₀₁ & a₁₁b₀₁ \\
        %     a₀₀b₁₀ & a₀₁b₁₀ & a₀₀b₁₁ & a₀₁b₁₁ \\
        %     a₁₀b₁₀ & a₁₁b₁₀ & a₁₀b₁₁ & a₁₁b₁₁
        % \end{bmatrix}
    % \]
\end{example}

\begin{definition}%[Bločno-diagonalna matrika]
    For gates \( U₀,\dots,Uₛ \) we denote their block-diagonal matrix with \( D{\p{U₀,\dots,Uₛ}} \).
\end{definition}

\begin{theorem}[No cloning]\label{no-cloning}
    There doesn't exist a unitary gate of odrer \( 2 \), which maps every vector of the form \(a ⊗\ket{0}\in \HH ⊗\HH\) to \(a⊗a\).
\end{theorem}

\begin{proof}
    Let \(U\) be such that for all \( a \in \HH \) we have \( U{\p{a⊗\ket0}} = a⊗a \).\\
    Then for \( \hh ⊗\ket0 = ρ(\ket{00} + \ket{10}) \) the following holds:
    \[
        U{\p{ρ(\ket{00} + \ket{10})}} =
        \begin{cases}
            ρ²(\ket{00} + \ket{01} + \ket{10} + \ket{11}),\\
            ρ U\ket{00} + U\ket{10} = ρ(\ket{00} + \ket{11}),
        \end{cases}
    \]
    which is a contradiction.
\end{proof}

\begin{example}[Pauli matrices]
    These are the matrices of half-rotation around the axes on the bloch sphere:
    \[
        I₂ = \begin{bmatrix} 1 &  0 \\ 0 &  1 \end{bmatrix},\quad
        X  = \begin{bmatrix} 0 &  1 \\ 1 &  0 \end{bmatrix},\quad
        Y  = \begin{bmatrix} 0 & -i \\ i &  0 \end{bmatrix},\quad
        Z  = \begin{bmatrix} 1 &  0 \\ 0 & -1 \end{bmatrix}.
    \]
    It holds that \(X² = Y² = Z² = I₂\).
    The map \( X \) is also called negation,
    since \( X{\ket0} = \ket1 \) and \( X{\ket1} = \ket0 \).
\end{example}

\begin{example}[Hadamard matrix]
    \[
        \had = ρ\begin{bmatrix}1&1\\1&-1\end{bmatrix},\quad
        \had\ket{0} = \hh,\quad
        \had^{⊗n}\ket{\mathbf{0}ⁿ} = \hh[n].
    \]
\end{example}

% \[ \Qcircuit @C=1em @R=.7em {
%         & \gate{\had} & \gate{\g Z} & \gate{\had} & \qw\\
%         &             &      =      &             &    \\
%         & \qw         & \gate{\g X} & \qw         & \qw
%     }
% \]
\begin{example}[Phase shift]
    \begin{align*}
        &S_α = \begin{bmatrix}1&0\\0&e^{iα}\end{bmatrix}
        \text{, posebej označimo } S ≔ S_{\newfrac{π}{2}}, T ≔ S_{\newfrac{π}{4}}, \\
        &S_α\p{a₀\ket 0 + a₁\ket1} = a₀\ket0 + a₁e^{iα}\ket1.
    \end{align*}
\end{example}

\subsection{Quantum Measurement}

In classical computation we have \texttt{if} statements.
This can be generalised in two ways;
first, with direct measurement (and classical \texttt{if} statements),
and second by using quantum entanglement.
It turns out that after measurement the two ways are equivalent.

% V klasičnem računalništvu poznamo pogojne stavke. To lahko na kubite posplošimo na dva načina,
% prvi z direktno meritvijo kubita (in uporabo klasičnih pogojnih stavkov),
% drugi pa z uporabo kvantne prepletenosti.
% Izkaže se, da če na koncu zmerimo kubite, se drugi način obnaša enako kot prvi.

\begin{definition}[Quantum Measurement]
    Denote the measurement of a qubit \( a = a_0\ket0 + a_1\ket1 \) with \( M{\p a} \) and it is \(0\) with probability \(|a_0|^2\) and \(1\) with probability \(|a_1|^2\). This will \airquotes{destroy} the qubit \(a\).
    % Meritev kubita \(a = a_0\ket0 + a_1\ket1\) označimo \(M\p{a}\) in je \(0\) z verjetnostjo \(|a_0|^2\) in \(1\) z verjetnostjo \(|a_1|^2\). To \airquotes{uniči} kubit \(a\).
    % \[ \Qcircuit @C=1em @R=.7em {
    %         & \meter & \cw %& & \ctrlo{1} & \ctrl{1} & \qw & & \ctrl{1} & \qw \\
    %         % &        &     & & \gate U & \gate V & \qw & & \targ & \qw
    %     }
    % \]
\end{definition}

\begin{definition}[Quantum Control]
    For \( r,s\in\N \) and \( U \in \U[1] \) define  \( C_{r,s}{\p U} \) and \( \overline{C}_{r,s}{\p U} \) by
    % Za \( r,s\in\N \) in \( U \in \U[1] \) definiramo \( C_{r,s}{\p U} \) in \( \overline{C}_{r,s}{\p U} \) s predpisoma
    \[ C_{r,s}{\p U}\ket j = \begin{cases}
        \ket j &; jᵣ = 0\\
        \ket{j₁\dots}\ket{U{jₛ}}\ket{\dots jₙ} &; jᵣ = 1
    \end{cases}\qquad
    \overline{C}_{r,s}{\p U}\ket j = \begin{cases}
        \ket j &; jᵣ = 1\\
        \ket{j₁\dots}\ket{U{jₛ}}\ket{\dots jₙ} &; jᵣ = 0
    \end{cases}
    \]
    We call such gates controlled (\airquotes{on one} and \airquotes{on zero}).
    % Takim vratom pravimo kontrolirana (\airquotes{na ena} in \airquotes{na nič}).
    Specially, for \( U \in \U[1] \) we denote
    % Posebej za \( U \in \U[1] \) označimo
    \[ \ctl U ≔ C_{1,2}{\p U} = \D{I₂, U},\quad
        \ctlo U ≔ \overline{C}_{1,2}{\p U} = \D{U, I₂}. \]
\end{definition}

\begin{example}[Entangled qubits]
    Controlled gates entangle pairs of qubits. For example \( \eapply{\ctl X}{a, b} \) behaves like 
    \qpl{if |\(\emeasure{a} = 0\)| then |\( \p{a, b} \)| else |\( \p{a, \lnot b} \)|}.
    % Kontrolirana vrata prepletejo pare kubitov. Na primer\\
    % \( \eapply{\ctl X}{a, b} \) se obnaša kot
    % \qpl{if |\(\emeasure{a} = 0\)| then |\( \p{a, b} \)| else |\( \p{a, \lnot b} \)|}.
    Of course, we know the second expression is not valid (since measurement destroys \(a\)),
    but it is for this exact reason that quantum control is the tool we want to replace \texttt{if} statements with!
    % Seveda vemo, da drugi izraz ni veljaven (ker meritev uniči kubit a),
    % ampak je zato kontrola ravno tisto orodje, s katerim želimo nadomestiti pogojne stavke.
\end{example}


\section{Quantum Computation, Algebraic Effects and Diagrams}

\subsection{Quantum Circuits}

Quantum programs can be represented as circuit diagrams.
Boxes represent quantum gates and lines between them wires; single for qubits, double for classical bits.
Circles on the wire (and then a vertical wire from them)
represent control; empty circles control \airquotes{on zero} and full control \airquotes{on one}.
We read such circuits left to right.
A more thorough account can be found in \cite{ess-qc}.

% Kvantne programe lahko predstavimo kot diagrame vezja.
% Škatle predstavljajo unitarna vrata, črte med njimi pa žice;
% po enojnih žicah tečejo kubiti, po dvojnih pa klasični biti (\(0\) ali \(1\)).
% Pike na žici (in potem navpična žica ven) pomenijo kontrolo;
% prazna pika kontrolira \airquotes{na nič}, polna pa \airquotes{na ena}.
% Taka vezja beremo od leve proti desni.
% Natančnejši opis lahko najdete v \cite{ess-qc}.

Below are two examples of quantum programs, described both with words and diagrams. We will also see these examples used later.

% Spodaj sta dva primera kvantnih programov, opisana z besedami in diagrami, ki ju bomo srečali tudi še kasneje.

\begin{example}[Projection to the \(z\)-axis]\label{ex-proj-z}
    First measure \(a\) and then depending on the result either flip or leave alone a fresh qubit.
    On the Bloch sphere this looks approximately like a projection to the \(z\)-axis (as the only qubits on it are \(\ket 0\) and \(\ket 1\)).
    % Najprej zmerimo \( a \) in nato glede na rezultat svež kubit bodisi negiramo bodisi ne.
    % Na Blochovi sferi to zgleda približno kot projekcija na \(z\)-os (edina kubita na \(z\)-osi sta \( \ket0 \) in \( \ket1 = X \ket0 \)).
    \[ \Qcircuit @C=1em @R=.7em {
            & \lstick{\ket0} & \gate{\g X} & \rstick{b} \qw\\
            \lstick{a} & \meter & \cctrl{-1}
        }
    \]
\end{example}

\begin{example}[Random phase shift]\label{ex-rand-ph-shift}
    Measurement of the Hadamard vector simulates a fair coin toss and the \(Z\) gate is a phase shift by \(\pi\) radians. The algorithm randomly either shifts the phase or not.
    % Meritev Hadamardovega vektorja simulira pravičen met kovanca,
    % vrata \( Z \) pa rotirajo fazo, torej bomo v polovici primerov kubitu \( a \) rotirali fazo.
    \[ \Qcircuit @C=1em @R=.7em {
            \lstick{a} & \qw & \qw & \qw & \qw & \gate{\g Z} & \rstick{a}\qw\\
            && \lstick{\ket0} & \gate{\had} & \meter & \cctrl{-1}
        }
    \]
\end{example}
% \[ \Qcircuit @C=1em @R=.7em {
%         & \lstick{\ket0} & \gate{\g X} & \rstick{b} \qw &&&&
%         \lstick{a} & \qw & \qw & \qw & \qw & \gate{\g Z} & \rstick{a}\qw\\
%         \lstick{a} & \meter & \cctrl{-1} &&&&
%         &&& \lstick{\ket0} & \gate{\had} & \meter & \cctrl{-1}
%     }
% \]

% \frame{
%     \frametitle[Vezja]{Grafični prikaz}

%     Grafično predstavimo kot kvantna vezja:
%     \[ \Qcircuit @C=1em @R=.7em {
%             & \gate{\g U} & \qw & & \meter & \cw \\
%             & \ctrl{1} & \qw & & \ctrlo{1} & \ctrl{1} & \qw \\
%             & \targ & \qw & & \gate{\g U} & \gate{\g V} & \qw
%         }
%     \]
% }

\subsection{Algebraic Effects}

We often use computational effects while programming:
global variable state, IO devices, randomness, exceptions, non-determinism, etc.

% Z računskimi učinki se med programiranjem pogosto srečamo: globalno stanje spremenljivk, vhodno/izhodne naprave, naključnost, izjeme, nedeterminizem, ipd.

\begin{definition}[Computational Effects]
    If a function or operation has some outward detectable effect other than the returned value we call it a computational effect (effect of the computation).
    % Če ima funkcija ali operacija še kak navzven viden učinek poleg vrnjene vrednosti, slednjemu pravimo računski učinek (učinek računanja).
    % Programe lahko ločimo na dve vrsti:
    % \begin{itemize}
    %     \item na čiste funkcije, ki zgolj vrnejo vrednost,
    %     \item in operacije, ki imajo poleg vrnjene vrednosti še kak stranski učinek; temu pravimo računski učinek (učinek računanja).
    % \end{itemize}
\end{definition}

\begin{definition}[Algebraic Effects]
    Computational effects that can be represented with some algebraic theory are called algebraic effects.
    % Računskim učinkom, ki jih lahko predstavimo s kašno algebrajsko teorijo, pravimo algebrajski učinki.
\end{definition}


\section{Quantum Programming Language}

In our language\cite{algeff-lin-qpl} we have the usual basic constructs, e.g. types, \texttt{let} and \texttt{if} statements, etc.
Other than that we also have quantum elements:
the type of qubits \texttt{qubit} and for every two types \(A\) and \(B\) we have the type of entangled pairs \( A⊗B \).
Due to the nature of qubits we cannot access their inner state directly, but we do have the following accessor functions:
\begin{itemize}
    \item \(\eff{new}\): assigns a new qubit with initial value \(\ket 0\),
    \item \(\eff{apply}_{\g U}\): uses gate \(U\) on a given vector,
    \item \(\eff{measure}\): measures a given qubit and returns an element of type \texttt{bit}.
\end{itemize}


% V našem jeziku\cite{algeff-lin-qpl} imamo navadne osnovne konstrukte,
% npr. tipe, \texttt{let} ter \texttt{if} stavke, itd.
% Poleg tega imamo pa še elemente kvantnega računalništva:
% tip kubitov \texttt{qubit} in tip prepletenih parov \( A⊗B \) za vsaka dva tipa \(A\) in \(B\).
% Zaradi narave kubitov ne moremo neposredno dostopati do notranjega stanja pomnilnika,
% imamo pa naslednje funkcije dostopanja:
% \begin{itemize}
%     \item \(\eff{new}\): dodeli nov kubit, z začetno vrednostjo \(\ket 0\),
%     \item \(\eff{apply}_{\g U}\): uporabi vrata \(U\) na danem vektorju,
%     \item \(\eff{measure}\): izvede meritev na kubitu, vrne element tipa \texttt{bit}.
% \end{itemize}

\subsection{Conversion into Algebraic Expressions}

The above constructs are assigned the following algebraic expressions. We also introduce a concise notation for pen and paper manipulation.

% Konstruktom v programskem jeziku priredimo naslednje algebrajske izraze ter uvedemo še strnjeno obliko, za lažjo manipulacijo na papirju.

\begin{table}[ht]
    \centering
    \begin{tabular}{|l|l|l|}
        \hline
        Quantum programming language 
            & Algebraic expressions              & Mathematical symbols          \\
        \hline
        \qpl{let |\( a \leftarrow \enew \)| in |\( x{\p{a}} \)|}
            & \( \tnew{a}{x{\p a}} \)         & \( \new{a}{x{\p a}} \)       \\
        \qpl{|\( \eapply{\g{U}}{a} \)|; |\( x{\p{a}} \)|}
            & \( \tapply{U}{a}{x{\p a}} \)    & \(\apply{U}{a}{x{\p a}}\)    \\
        \qpl{if |\( \emeasure{a} = 0 \)| then |\( t \)| else |\( u \)|}
            & \( \tmeasure{a}{t}{u} \)        & \( \measureraw{a}{t}{u} \)   \\
        % \qpl{if |\( \emeasure{a} = 0 \)| then |\( t \)| else |\( t \)|}
        \qpl{|\( \ediscard{a} \)|; |\( t \)|}
            & \( \tdiscard{a}{t} \)           & \( \discard{a}{t} \)         \\
        \hline
    \end{tabular}
\end{table}

% \pagebreak

\begin{example}[Projection to the \( z \)-axis]\(\)
    \begin{enumerate}
        \item \qpl{if |\(\emeasure{a} = 0\)| then |\( \enew \)| else |\( \eapply{\g{X}}{\enew} \)|}
        \item \( \tmeasure{a}{\tnew{b}{x{\p b}}}{\tnew{b}{\tapply{X}{b}{x{\p b}}}} \)
        \item \(
            \measure{a}
                {\new{b}              {x{\p b}}}
                {\new{b} {\apply{X}{b}{x{\p b}}}
            }\)
    \end{enumerate}
\end{example}

\begin{example}[Random phase shift]\(\)
    \begin{enumerate}
        \item \qpl{if |\( \emeasure{\eapply{\had}{\enew}} = 0 \)| then |\( a \)| else |\( \eapply{\g{Z}}{a} \)|}
        \item \( \tnew{b}{\tapply{\had}{b}{\tmeasure{b}{x{\p a}}{\tapply{Z}{a}{x{\p a}}}}} \)
        \item \(
            \new{b}{ \apply{\had}{b}{
                \measureraw{b}
                    {x{\p a}}
                    {\apply{Z}{a}{x{\p a}}}
            }}\)
    \end{enumerate}
\end{example}

\subsection{Axioms}

We can separate the axioms for program equality into two groups: the first five are conceptual, while the other seven are administrative.
The latter tell us \(\eff{apply}\) agrees with the algebraic structure of unitary matrices and that things commute as far as variable binding (and matrix order) allows. A more detailed description (with proofs) can be found in \cite{algeff-lin-qpl}.

% Aksiomi za enakost programov so (brez dokaza) sledeči:
% Aksiome za enakost programov lahko delimo na dva dela; prvih pet je glavnih, ostalih sedem pa bolj \airquotes{administrativnih} oziroma pomožnih. Slednji nam povejo zgolj, da se \( \eff{apply} \)
% strinja s strukturo unitarnih matrik, ter da stvari komutirajo, kolikor vezanje spremenljivk (in vrstni red uporabe matrik) dopušča. Podrobnejši opis (z dokazom) najdete v \cite{algeff-lin-qpl}.

% \setlength\parindent{0pt}

\begin{axiom}{Quantum negation before measurement is regular negation after measurement:}\label{ax-1}
    \( \apply{X}{a}{\measureraw{a}{x}{y}} = \measureraw{a}{y}{x} \).
\end{axiom}

\begin{axiom}{Quantum control is like classical control after measurement:}\label{ax-2}
    \( \applyd{U, V}{a,b}{\measureraw{a}{x(b)}{y(b)}}
        = \measureraw{a}{\apply{U}{b}{x(b)}}{\apply{V}{b}{y(b)}} \).
\end{axiom}

\begin{axiom}{Quantum gates used on discarded qubits are unnecessary:}\label{ax-3}
    \( \apply{U}{a}{\discard{a}{t}} = \discard{a}{t} \).
\end{axiom}

% \begin{axiom}{Meritve novih kubitov so vedno \(0\):}\label{ax-4}
\begin{axiom}{Fresh qubits are \(\ket 0\) wrt. measurement:}\label{ax-4}
    \( \new{a}{\measureraw{a}{x}{y}} = x \).
\end{axiom}

% \begin{axiom}{Vrata kontrolirana z novimi kubiti se nikoli ne uporabijo:}\label{ax-5}
\begin{axiom}{Fresh qubits are \(\ket 0\) wrt. control:}\label{ax-5}
    \( \new{a}{\applyd{U,V}{a,b}{x{\p{a,b}}}} = \apply{U}{b}{\new{a}{x{\p{a,b}}}} \).
\end{axiom}

\begin{axiom}{Symmetric group \(\U[n]\) structure is respected:}\label{ax-6}
    \( \apply{swap}{a,b}{x{\p{a,b}}} = x{\p{b,a}} \),
\end{axiom}

\begin{axiom}{}\label{ax-7}
    \( \apply{I}{a}{x{\p a}} = x{\p a} \),
\end{axiom}

\begin{axiom}{}\label{ax-8}
    \( \apply{UV}{a}{x{\p a}} = \apply{V}{a}{\apply{U}{a}{x{\p a}}} \),
\end{axiom}

\begin{axiom}{}\label{ax-9}
    \( \apply{U⊗V}{a,b}{x{\p{a,b}}} = \apply{U}{a}{\apply{V}{b}{x{\p{a,b}}}} \).
\end{axiom}

\begin{axiom}{Commutativity:}\label{ax-10}
    \( \measure{a}{\measureraw{b}{u}{v}}{\measureraw{b}{x}{y}}
        = \measure{b}{\measureraw{a}{u}{x}}{\measureraw{a}{v}{y}} \),
\end{axiom}

\begin{axiom}{}\label{ax-11}
    \( \new{a}{\new{b}{x{\p{a,b}}}} = \new{b}{\new{a}{x{\p{a,b}}}} \),
\end{axiom}

\begin{axiom}{}\label{ax-12}
    \( \new{a}{\measureraw{b}{x{\p a}}{y{\p a}}}
        = \measure{b}{\new{a}{x{\p a}}}{\new{a}{y{\p a}}} \).
\end{axiom}

% \begin{gather}
% \intertext{Kvantna negacija pred meritvijo je negacija po meritvi.}
%     \apply{X}{a}{\measureraw{a}{x}{y}} = \measureraw{a}{y}{x}\label{ax-1}
% \intertext{Kvantna kontrola je po meritvi kot klasična kontrola.}
%     \measureraw{a}{\apply{U}{b}{x(b)}}{\apply{V}{b}{y(b)}}
%         = \applyd{U, V}{a,b}{\measureraw{a}{x(b)}{y(b)}}\label{ax-2}
% \intertext{Kvantna vrata uporabljena na zavrženih kubitih so odveč.}
%     \apply{U}{a}{\discard{a}{t}} = \discard{a}{t}\label{ax-3}
% \intertext{Meritve novih kubitov so vedno \(0\).}
%     \new{a}{\measureraw{a}{x}{y}} = x\label{ax-4}
% \intertext{Vrata kontrolirana z novimi kubiti se nikoli ne uporabijo.}
%     \new{a}{\applyd{U,V}{a,b}{x{\p{a,b}}}}
%         = \apply{U}{b}{\new{a}{x{\p{a,b}}}}\label{ax-5}
% \intertext{Ostanejo še bolj \airquotes{administrativni} aksiomi:
% Spoštovanje simetrične grupe \( \U[n] \).}
%     \apply{swap}{a,b}{x{\p{a,b}}} = x(b,a)\label{ax-6}\\
%     \apply{I}{a}{x{\p a}} = x{\p a}\label{ax-7}\\
%     \apply{UV}{a}{x{\p a}} = \apply{U}{a}{\apply{V}{a}{x{\p a}}}\label{ax-8}\\
%     \apply{U⊗V}{a,b}{x{\p{a,b}}} = \apply{U}{a}{\apply{V}{b}{x{\p{a,b}}}}\label{ax-9}
% \intertext{Komutativnost.}
%     % \apply{U}{a}{\apply{V}{b}{x{\p{a,b}}}} = \apply{V}{b}{\apply{U}{a}{x{\p{a,b}}}}\\
%     \measure{a}{\measureraw{b}{u}{v}}{\measureraw{b}{x}{y}}
%         = \measure{b}{\measureraw{a}{u}{x}}{\measureraw{a}{v}{y}}\label{ax-10}\\
%     \new{a}{\new{b}{x{\p{a,b}}}} = \new{b}{\new{a}{x{\p{a,b}}}}\label{ax-11}\\
%     \new{a}{\measureraw{b}{x{\p a}}{y{\p a}}}
%         = \measure{b}{\new{a}{x{\p a}}}{\new{a}{y{\p a}}}\label{ax-12}
% \end{gather}

% \apply{X}{a}{\apply{\HH}{b}{x{\p{a,b}}}} = \apply{\HH}{b}{\apply{X}{a}{x{\p{a,b}}}}
\begin{example}[Derivation of equality between \(z\)-axis projection and random phase shift]
    The derivation relies on the identities
    % Izpeljava se zanaša na identiteti 
    \( \ctl X . \swap . \ctl X \overset{\dagger}{=} \swap . \ctl X . \swap \) and
    \( \swap . \ctl X . \swap \overset{\ddagger}{=} (\had⊗\g{I₂}) . \ctl Z . (\had⊗\g{I₂}) \).
    \begin{NoHyper}
    \begin{align*}
        &\hspace{-3em}\measure{a}{\new{b}{x{\p b}}}{\new{b}{\apply{X}{b}{x{\p b}}}}\\
        =&\new{b}{\measureraw{a}{x{\p b}}{\apply{X}{b}{x{\p b}}}}
            &{(\ref{ax-12})}\\
        =&\new{b}{\apply{\ctl X}{a,b}{\measureraw{a}{x{\p b}}{x{\p b}}}}
            &{(\ref{ax-2})}\\
        =&\new{b}{\apply{\ctl X}{a,b}{\discard{a}{x{\p b}}}}
            &{(\text{def.})}\\
        =&\new{b}{\apply{\ctl X}{b,a}{\apply{\ctl X}{a,b}{\discard{b}{x{\p a}}}}}
            &{(\dagger)}\\
        =&\new{b}{\apply{\ctl X}{a,b}{\discard{b}{x{\p a}}}}
            &{(\ref{ax-5})}\\
        =&\new{b}{\apply{\had}{b}{\apply{\ctl Z}{b,a}{\apply{\had}{b}{\discard{b}{x{\p a}}}}}}
            &{(\ddagger)}\\
        =&\new{b}{\apply{\had}{b}{\apply{\ctl Z}{b,a}{\discard{b}{x{\p a}}}}}
            &{(\ref{ax-3})}\\
        =&\new{b}{\apply{\had}{b}{\measureraw{b}{x{\p a}}{\apply{Z}{a}{x{\p a}}}}}.
            &(\ref{ax-2})
    \end{align*}
    \end{NoHyper}
    % Kot vidimo, smo definirali jezik (in orodja) s katerimi lahko relativno enostavno dokazujemo enakost med programi.
    % Opazimo, da ta dokaz ni bil popolnoma mehanski;
    % uporabiti smo morali dve identiteti v \( \U[2] \).
    % To nam namiguje, da enakost programov ni preprost problem,
    % ki bi ga lahko računalniki sami rešili.
\end{example}

\printbibliography
\end{document}
