\documentclass[mat1]{fmfdelo}

\usepackage[slovene]{babel}

\usepackage{mathtools}
\usepackage{amsthm}

\usepackage[braket]{qcircuit}
\usepackage{tikz}
\usetikzlibrary{babel}
\usetikzlibrary{angles,quotes}

\usepackage{minted}

\usepackage{stmaryrd}

% naslednje ukaze ustrezno napolnite
\avtor{Ime Priimek}

\naslov{Kvantni algebrajski učinki}
\title{Quantum Algebraic Effects}

% navedite ime mentorja s polnim nazivom: doc.~dr.~Ime Priimek,
% izr.~prof.~dr.~Ime Priimek, prof.~dr.~Ime Priimek
% uporabite le tisti ukaz/ukaze, ki je/so za vas ustrezni
\mentor{doc.~dr.~Matija~Pretnar}

\letnica{2022?} % leto diplome

%  V povzetku na kratko opišite vsebinske rezultate dela. Sem ne sodi razlaga organizacije dela --
%  v katerem poglavju/razdelku je kaj, pač pa le opis vsebine.
\povzetek{}

%  Prevod slovenskega povzetka v angleščino.
\abstract{}

% navedite vsaj eno klasifikacijsko oznako --
% dostopne so na www.ams.org/mathscinet/msc/msc2020.html
\klasifikacija{81P68, 81P70}
\kljucnebesede{kvantno računalništvo, algebrajski učinki} % navedite nekaj ključnih pojmov, ki nastopajo v delu
\keywords{quantum computing, algebraic effects} % angleški prevod ključnih besed

\zapisiMetaPodatke  % poskrbi za metapodatke in veljaven PDF/A-1b standard


\newcommand{\N}{\mathbb N}
\newcommand{\Z}{\mathbb Z}
\newcommand{\Q}{\mathbb Q}
\newcommand{\R}{\mathbb R}
\newcommand{\C}{\mathbb C}
\renewcommand{\d}{\mathrm d}
\renewcommand{\phi}{\varphi}
\newcommand{\eps}{\varepsilon}
\renewcommand{\hat}{\widehat}
\renewcommand{\tilde}{\widetilde}
\renewcommand{\bar}{\overline}

\newcommand{\p}[1]{\left( {#1} \right)}
\newcommand{\set}[2]{\left\{ #1 \mid #2 \right\}}
\newcommand{\newfrac}[2]{{}^{#1}\!/_{\!#2}}
\newcommand{\im}[1]{\mathrm{im}{\p{#1}}}

\newcommand{\had}{\mathtt{Had}}
\newcommand{\hh}[1][]{\mathbf{h}_{#1}}
\newcommand{\B}[1][]{\mathbf{B}_{#1}}
\renewcommand{\H}[1][]{\mathbf{H}_{#1}}
\renewcommand{\S}{\mathbb{S}}
\renewcommand{\L}[2][]{\mathcal{L}_{#1}{\p{#2}}}
\newcommand{\g}[1]{\mathtt{#1}}
\newcommand{\D}[1]{D{\p{\g{#1}}}}
\newcommand{\ctl}[1]{\g{c#1}}
\newcommand{\ctlo}[1]{\g{\bar{c}#1}}

\newcommand{\cmd}[1]{\textnormal{\sffamily#1}}
\newcommand{\eff}[1]{\textnormal{\sffamily\ul{#1}}}
\newcommand{\tnew}[2]{\cmd{new}{\p{#1.#2}}}
\newcommand{\tapply}[3]{\cmd{apply}_{\g #1}{\p{#2.#3}}}
\newcommand{\tmeasure}[3]{\cmd{measure}{\p{#1; #2,#3}}}
\newcommand{\tdiscard}[2]{\cmd{discard}{\p{#1.#2}}}
\newcommand{\new}[2]{\nu{#1}.\,#2}
\newcommand{\apply}[3]{{\g #1}_{#2}{\p{#3}}}
\renewcommand{\measure}[3]{#2 {\;?_{#1}\,} #3}
\newcommand{\discard}[2]{\cmd{disc}_{#1}{\p{#2}}}

\newcommand{\sequent}[3]{#1 \mid #2 \vdash #3}
\newcommand{\seq}[1]{\sequent{\Gamma}{\Delta}{#1}}
\newcommand{\sem}[2][]{{\llbracket#2\rrbracket}_{#1}}

\begin{document}

\section{Uvod}
\subsection{Motivacija}
\subsection{Oris dela}  % ?

\section{Kvantno računalništvo ter algebrajski učinki in diagrami}
\subsection{Kvantna mehanika}

\subsubsection{Kvantni vektorji}

\begin{definicija}[Oznake]
    V tem delu bomo uporabljali naslednje oznake:
    \begin{itemize}
        \item \( \N = \{0, \dots \}, \N_+ = \{1, \dots\} \),
        \item \( n\in\N_+ \), ki mu bomo pravili število kubitov,
        \item \( j, k, \dots \in \{0, \dots, 2^n-1 \),
        \item \( j = j_1\dots j_n \) binarni zapis števila \( j \).
    \end{itemize}
\end{definicija}

\begin{definicija}
    Binarni vektorji so elementi prostora \( \B[n] \coloneqq 2^n \) in jih pišemo kot nize v binarnem zapisu.
\end{definicija}

\begin{primer}
    \(\B[2] = \{00, 01, 10, 11\}\).
\end{primer}

\begin{definicija}
    Elementom prostora \( \H[n] \coloneqq \C^{2^n} \) pravimo kvantni vektorji, elementom \( \H \coloneqq \H[1] \) pa kubiti.  Prostoru \( \H[n] \) pravimo prostor kvantnih vektorjev reda \( n \), njegovo standardno bazo pa označimo z \( \{e_j\} \).
\end{definicija}

\begin{definicija}%[Braket notacija]
    Naj bo \( j \in \{0, \dots, 2^n-1\} \), ter \( \hat{\jmath} \in \B[n] \) pripadajoč vektor v binarnem zapisu. Potem je \( \ket j = \ket{\hat{\jmath}} \coloneqq e_j \).
\end{definicija}
\begin{opomba}
    Po definiciji je torej \( \H[n] = \L[\C]{\set{\ket j}{j\in\B[n]}} \).
\end{opomba}

\begin{primer}[\(n = 1\)]
    \[ 
        a = \begin{bmatrix}a_0 \\ a_1\end{bmatrix}
            = a_0\ket 0 + a_1\ket 1
            = a_0\begin{bmatrix}1 \\ 0\end{bmatrix} + a_1\begin{bmatrix}0 \\ 1\end{bmatrix}\text.\qedhere
    \]
\end{primer}

\begin{primer}[\(n = 2\)]
    \[
        a = \begin{bmatrix}a_0\\ a_1\\ a_2\\ a_3\end{bmatrix}
            = \begin{bmatrix}a_{00}\\ a_{01}\\ a_{10}\\ a_{11}\end{bmatrix}
            = a_{00}\ket{00} + a_{01}\ket{01} + a_{10}\ket{10} + a_{11}\ket{11}\text.\qedhere
    \]
\end{primer}

\begin{primer}%[Hadamardov vektor] % maybe skip this one?
    \[
        \hh \coloneqq \rho \left(\ket 0 + \ket 1\right),\quad
        \hh[n] \coloneqq \rho^n\sum_{j\in\B[n]} \ket j,\quad
        \rho \coloneqq \frac{1}{\sqrt2}\text.\qedhere
        \]
\end{primer}

\subsubsection{Tenzorski produkt}

\begin{definicija}[Tenzorski produkt]
    Tenzorski produkt prostorov \( \H[n] \) in \( \H[m] \) je enak \( \H[n+m] \).
    Pišemo \( \H[n]\otimes\H[m] \). Če sta \( a\in\H[n] \) in \( b\in\H[m] \) je \( a\otimes b \in \H[n]\otimes\H[m] \).
\end{definicija}
\begin{opomba}
    Operator \(\otimes\) je res tenzorski produkt.
\end{opomba}

\begin{primer}[\(n=m=1\)]
    \[
        \begin{bmatrix}a_0 \\ a_1\end{bmatrix}\otimes \begin{bmatrix}b_0 \\ b_1\end{bmatrix}
        = \begin{bmatrix}a_0b_0\\ a_0b_1\\ a_1b_0\\ a_1b_1\end{bmatrix}\text.\qedhere
    \]
\end{primer}

\begin{primer}
    \[
        \ket j \otimes \ket k = \ket{j\#k} \eqqcolon \ket{jk},\quad
        a\otimes b = \sum_{\substack{j\in\B[n],\\k\in\B[m]}} a_jb_k\ket{jk}.
    \]
\end{primer}

\begin{primer}
    \[
        \hh[n] = \hh^{\otimes n} = \rho^n\underbrace{(\ket 0 + \ket 1)\otimes\dots\otimes(\ket 0 + \ket 1)}_n\text.\qedhere
    \]
\end{primer}
\begin{primer}
    \[\H[n] = \H^{\otimes n}\text.\qedhere\]
\end{primer}

\begin{definicija}
    Če lahko \(a\in\H[n]\) zapišemo kot \(\bigotimes_{j=1}^{n} a_j; a_j\in\H\) pravimo, da je enostaven ali separabilen, sicer je pa sestavljen oziroma kvantno prepleten.
\end{definicija}


\begin{definicija}%[Unitarna vrata]
    Unitarna vrata reda \( n \) so unitarna matrika dimenzije \( 2^n \).
    Tenzorski produkt vrat \( U \otimes V = [u_{jk}V]_{j,k} \) uporabljen na \(a\otimes b\) je enak \( Ua \otimes Vb \).
\end{definicija}
\begin{primer}
    \[
        \begin{bmatrix}
            a_{00} & a_{01} \\ a_{10} & a_{11}
        \end{bmatrix}
        \otimes B
        % \begin{bmatrix}
        %     b_{00} & b_{01} \\ b_{10} & b_{11}
        % \end{bmatrix}
        =
        \begin{bmatrix}
            a_{00} B & a_{01} B \\ a_{10} B & a_{11} B
        \end{bmatrix}\text.\qedhere\]
        % \begin{bmatrix}
        %     a_{00}b_{00} & a_{01}b_{00} & a_{00}b_{01} & a_{01}b_{01} \\
        %     a_{10}b_{00} & a_{11}b_{00} & a_{10}b_{01} & a_{11}b_{01} \\
        %     a_{00}b_{10} & a_{01}b_{10} & a_{00}b_{11} & a_{01}b_{11} \\
        %     a_{10}b_{10} & a_{11}b_{10} & a_{10}b_{11} & a_{11}b_{11}
        % \end{bmatrix}
    % \]
\end{primer}
\begin{definicija}%[Bločno-diagonalna matrika]
    Za vrata \( U_0,\dots,U_n \) označimo njihovo bločno-diagnoalno matriko z \( D{\p{U_0,\dots,U_n}} \).
\end{definicija}
% TODO: QPL, vezje

\begin{izrek}[No cloning]
    Ne obstaja unitarna matrika (vrata reda \(2\)), ki vsak vektor \(a\otimes\ket{0}\in \H\otimes\H\) slika v \(a\otimes a\).
\end{izrek}

\begin{dokaz}
    Naj bo \(U\) tak, da \(\forall a \in \H \) velja \( U{\p{a\otimes\ket0}} = a\otimes a \).
    Potem za \( \hh\otimes\ket0 = \rho(\ket{00} + \ket{10}) \) velja, da je \( U{\p{\rho(\ket{00} + \ket{10})}} = \rho^2(\ket{00} + \ket{01} + \ket{10} + \ket{11}) \), oziroma \(U\ket{00} + U\ket{10} = \rho(\ket{00} + \ket{01} + \ket{10} + \ket{11})\)
\end{dokaz}



\subsection{Algebrajski učinki}  % Naši
% \subsection{Algebrajska struktura}
\subsection{Diagrami}

\section{Programski jezik}
\subsection{Signatura}
\subsection{Sintaksa}
\subsection{Teorija tipov}  % ?
% \subsection{Kvantni izračuni kot unitarne matrike}  % ?
% \subsection{Signatura naše teorije}  % signature, prevod iz Taslak mag delo.
\subsection{Aksiomi}

\section{Semantika}
\subsection{Kvantni izračuni kot unitarne matrike}  % ?
\subsection{Semantika jezika}  % modeli, kategorije
\subsection{C*--Algebre}
\subsection{Polnost}
\begin{izrek}[Polnost]
    Če je \(\seq{t, u}\) in \(\sem t = \sem u\) je \(\seq{t = u}\).
\end{izrek}

% \section{Kvantni programski jezik}
% % Tu gre celotna definicija nastalega programskega jezika z \(\cmd{new}\), \(\cmd{apply}\), in \(\cmd{measure}\).

% \section{Razširitve}
% \subsection{Operacija \cmd{discard}}
% % Aksiom (c) uporabi operacijo \cmd{discard}.
% \subsection{Nekompatibilnost s klasičnim računalništvom}

\section*{Slovar strokovnih izrazov}

\geslo{}{}
\geslo{}{}

% seznam uporabljene literature
\begin{thebibliography}{99}

%\bibitem{}

\end{thebibliography}

\end{document}

