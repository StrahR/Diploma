\section{Uvod}

Leta 1981 je Richard Feynman predlagal uporabo računalnikov za namene simulacije fizikalnih pojavov.\cite{feynman-1981}
Ker je naš svet kvanten, je tudi predlagal kvanten računalnik.
Izkazalo se pa je, da kvantni računalniki niso uporabni zgolj za simulacije kvantne fizike,
vendar so bolj učinkoviti tudi za mnoge druge klasično eksponentne probleme, kot so faktorizacija na praštevila (Shorov algoritem), iskanje praslike funkcije (Groverjev algoritem), itd.

Štirideset let kasneje njegove ideje tudi že v neki meri izvajamo v praksi;
čeprav niso prav zmogljivi, kvantni računalniki vseeno obstajajo.
Naša naloga zdaj je najti dober način, kako jih programirati,
poleg tega, bi pa kvantno računalništvo tudi radi formalizirali.

V tem delu si bomo ogledali nov pristop do formalizacije enakosti programov z uporabo operatorskih algeber, ki pa jo lahko v splošnem uporabimo za formalizacijo poljubnih algebrajskih učinkov z linearnimi parametri.\cite{algeff-lin-qpl}
Vredno je še omeniti še en pristopa: ZX-račun\cite{zx-calculus,tdj}, ki se je razvil iz programa kategorične kvantne mehanike.
Prav tako kot naš pristop je ZX-račun poln, ampak čeprav je zelo močno orodje za poenostavljanje, teh poenostavljenih ZX-diagramov ne more vedno pretvoriti nazaj v kvantno vezje.
Ima pa pristop drug cilj od našega, želi definirati način, kako en izraz prepisati v drugega,
medtem ko se naš pristop osredotoči na polno aksiomatizacijo enakosti v večji splošnosti, kot zgolj za kvantno računalništvo (in kot bomo videli na koncu je naš model zelo preprosto razširiti z dodatnimi koncepti).
