\section{Kvantna mehanika in kvantno računalništvo}
V tem razdelku povzamemo nekaj standardnih definicij in rezultatov kvantne mehanike~\cite{ess-qc,ramšak-qm,selinger-qpl}.
% Vsebujejo osnovne definicije (in primere) matematičnih osnov kvantne mehanike,
% ki jih potrebujemo za definicije želenih operacij nad kubiti.
% Te se nahajajo na prvih nekaj straneh.

\addtocontents{toc}{\protect\setcounter{tocdepth}{0}}  % skrij oznake
\subsection*{Oznake:}
Skozi delo bomo uporabljali naslednje oznake:
\begin{itemize}
    \item \( ℕ = \{ 0, … \} \), \( ℕ⁺ = \{ 1, … \} \), \( ℕₙ = \{ 0, …, 2ⁿ-1 \} \),
    \item \(n, m, p, q ∈ ℕ⁺\), ki jih bomo imenovali število kubitov,
    \item \(j, k, … ∈ ℕₙ\),
    \item \(aⱼ\) \(j\)-ta komponenta vektorja \(a\),
    \item \(j = j₁ ⋯ jₙ\) binarni zapis števila \(j\).
    % \item \( \mb 0ⁿ = 0\dots0, \mb 1ⁿ = 1\dots1 \).
    % \item \( \mb n = \{ 0, \dots, n-1 \} \), na primer \( \mb 2 = \{ 0, 1 \} \).
    % \item \( \mathbb n = \{ 0, \dots, n-1 \} \), na primer \( \mathbb 2 = \{ 0, 1 \} \).
\end{itemize}
\addtocontents{toc}{\protect\setcounter{tocdepth}{2}}

\subsection{Kvantni biti}
Klasični biti zavzamejo dve vrednosti: \(\mb{0}\) in \(\mb{1}\).
Te lahko posplošimo na \emph{kvantne bite} (oziroma \emph{kubite}) kot formalne kompleksne linearne kombinacije klasičnih stanj, t.j. vsak kubit \(q\) je oblike \(α\mb{0} + β\mb{1}\), za \(α,β ∈ ℂ\), kjer nista oba 0. Števili \(α\) in \(β\) imenujemo \emph{amplitudi}.
Tu \(\mb0\) in \(\mb1\) razumemo kot formalna simbola, ki predstavljata klasični stanji.
Stanje kubita na splošno ni odvisno od skalarnega večkratnika,
tako da kubita \(α\mb0 + β\mb1\) in \(α'\mb0 + β'\mb1\) predstavljata enako \emph{kvantno stanje} čim obstaja kak neničelen \(γ ∈ ℂ\), tako da je \(α = γα'\) in \(β = γβ'\).
Iz praktičnih razlogov se amplitude pogosto normira, tako da velja \(|α|² + |β|² = 1\).
To normiranje pa nam amplitud še vedno ne definira enolično;
\(α\) in \(β\) sta določeni zgolj do množenja s kompleksno enoto natančno.

Klasični stanji \(\mb0\) in \(\mb1\) lahko torej identificiramo z baznima vektorjema prostora \(ℂ²\), \(e₀\) in \(e₁\), ta pa pogosto označujemo v t.i. \emph{\airquotes{ket} zapisu} kot \(\ket{\mb0} ≔ e₀\) in \(\ket{\mb1} ≔ e₁\).
Kvantna stanja, ki niso klasična, so \emph{kvantna superpozicija} klasičnih stanj \(\ket{\mb0}\) in \(\ket{\mb1}\).

\subsection{Kvantni vektorji}
Ko želimo kubite posplošiti na kvantne vektorje, je naš prvi instinkt, da vzamemo kar \(n\)-terice kubitov, vendar se izkaže, da to ne ustreza fizikalnim pojavom.
Kljub temu pa nismo daleč od resnice, le da moramo namesto \(n\)-kratnega kartezičnega produkta
vzeti \(n\)-kratni tenzorski produkt.

Začnimo s točnimi definicijami vseh gornjih pojmov.

\begin{definition}\label{binv}
    \emph{Binarni vektorji} so elementi množice \( \B[n] ≔ \{ 0, 1 \}ⁿ \) in jih pišemo kot nize v binarnem zapisu.
\end{definition}

\begin{example}
    Nize urejamo leksikografsko. V dveh dimenzijah so binarni vektorji naslednji: 
    \( \B[2] = \{\mb{00}, \mb{01}, \mb{10}, \mb{11}\} \).
    Binarne vektorje pišemo odebeljeno, da jih ločimo od števil (npr. število \(11\) in vektor \(\mb{11}\), ki ga dejansko želimo identificirati s številom \(3\)).
\end{example}
\begin{remark}
    Niza \(\mb1\) in \(\mb{01}\) predstavljata različna vektorja.
\end{remark}

\begin{definition}
    Elementom prostora \( \H[n] ≔ ℂ^{2ⁿ} \) pravimo \emph{kvantni vektorji} (v nadaljevanju kar vektorji), elementom \(\H ≔ \H[1]\) pa \emph{kubiti}.  Prostor \(\H[n]\) imenujemo torej \emph{prostor kvantnih vektorjev reda \(n\)}, njegovo standardno bazo pa označimo z \(\{eⱼ\}ⱼ\).
\end{definition}

\begin{definition}
    Izkaže se, da lahko več kvantnih vektorjev predstavlja enako fizično stanje.
    Definiramo ekvivalenčno relacijo z \(a \sim b \iff \exist{γ ∈ ℂ}{a = γb}\),
    ekvivalenčnim razredom pa pravimo \emph{stanja vektorjev}.
\end{definition}

\begin{definition}[Ket zapis]
    Naj bo \(j ∈ ℕₙ\), ter \( \hat{\jmath} ∈ \B[n] \) pripadajoč vektor v binarnem zapisu. Potem je \( \ket j = \ket{\hat{\jmath}} ≔ eⱼ \).
\end{definition}
\begin{remark}
    \airquotes{Ket} zapis je del \airquotes{braket} zapisa (izvira iz angl. \foreignlanguage{english}{\emph{bracket}}), kjer \airquotes{bra} igra vlogo dualnega vektorja in ga pišemo \(\bra{φ}\).
\end{remark}
\begin{remark}
    Po definiciji je torej \( \H[n] = \L[ℂ]{\set{\ket j}{j ∈ \B[n]}} \).
\end{remark}

Ker so matematično kvantni vektorji zgolj navadni vektorji, jih lahko tako tudi zapišemo.
Ekvivalenca med zapisoma je prikazana na naslednjih primerih:
\begin{example}[\( n = 1 \)]
    \[ a = \vec{a₀, a₁}
         = a₀\vec{1,0} + a₁\vec{0,1}
         = a₀\ket{\mb0} + a₁\ket{\mb1}.\qedhere
    \]
\end{example}
\begin{example}[\( n = 2 \)]
    \[ a = \vec{a₀,  a₁,  a₂,  a₃}
         = \vec{a\mb{₀₀}, a\mb{₀₁}, a\mb{₁₀}, a\mb{₁₁}}
         = a₀₀\ket{\mb{00}} + a₀₁\ket{\mb{01}} + a₁₀\ket{\mb{10}} + a₁₁\ket{\mb{11}}.\qedhere
    \]
\end{example}

\begin{example}[Hadamardov vektor]\label{had}
    Vektor, ki ima na vseh mestih enico (modulo skalarni večkratnik),
    se izkaže za zelo pomembnega. Kasneje bomo videli, kako raznolika je njegova uporaba, od vira naključnosti do bistva dokaza izreka o ne-kloniranju.
    \[ \hh ≔ ρ\vec{1,1} = ρ\p{\ket{\mb0} + \ket{\mb1}},\quad
       \hh[n] ≔ ρⁿ∑_{j ∈ \B[n]} \ket j,\quad
       ρ ≔ \frac{1}{\sqrt2}.\qedhere
    \]
\end{example}

\subsection{Blochova sfera}
Spomnimo se, da so stanja kubitov določena zgolj do skalarja natančno. Dejansko torej živijo v projektivnem prostoru. Formalno to zajamemo z naslednjo trditvijo:

\begin{proposition}
    % Stanji dveh kubitov, ki se razlikujeta zgolj za (kompleksen) faktor, sta enaki.
    % V fizičnem svetu sta dva kubita, ki se razlikujeta zgolj za (kompleksen) faktor, enaka.
    % Matematično to pomeni, da stanja kubitov živijo v \( \mathbf{P}ℂ¹ ≅ 𝕊² \) in jih lahko zapišemo v naslednji obliki:
    Za vsako stanje kubita lahko izberemo predstavnika ekvivalenčnega razreda oblike
    % To pomeni, da lahko vsak kubit zapišemo kot točko v \( \S² \):
    \[ \textstyle\cos\frac{θ}{2}\ket{\mb0} + e^{iφ}\sin\frac{θ}{2}\ket{\mb1},\quad
        φ ∈ [0, 2π), θ ∈ [0, π]. \]
\end{proposition}
\begin{proof}
    Naj bo \( a = a₀\ket{\mb0} + a₁\ket{\mb1} = r₀e^{iφ₀}\ket{\mb0} + r₁e^{iφ₁}\ket{\mb1} \).
    Če označimo
    \[ r ≔ \sqrt{r₀² + r₁²}\text{, }φ ≔ φ₁ - φ₀\text{, }θ ≔ 2\arccos{\frac{r₀}{r}}, \]
    je potem \[ \hat a ≔ \frac{a}{re^{iφ₀}} = \frac{r₀}{r}\ket{\mb0} + \frac{r₁}{r}e^{iφ}\ket{\mb1} = \cos\frac{θ}{2}\ket{\mb0} + e^{iφ}\sin\frac{θ}{2}\ket{\mb1}\]
    in predstavlja enako stanje kot \(a\).
    Ker je \(a\) poljuben, lahko za vsako stanje izberemo predstavnika ustrezne oblike.
\end{proof}

Če te predstavnike zberemo v prostor, dobimo tako imenovano Blochovo sfero,
tako da si lahko predstavljamo, da stanja kubitov dejansko živijo v prostoru \(𝕊²\).

\begin{figure}[ht]
\centering
\begin{tikzpicture}[baseline=(current bounding box.north)]
    % Define radius
    \def\r{7}
    
    % Bloch vector
    \draw  (   0,   0) node[circle, fill, inner sep=1                     ] (orig) {}
        -- (\r/6,\r/4) node[circle, fill, inner sep=0.7, label=above:\(a\)]    (a) {};
    \draw[dashed] (orig) -- (\r/6,-\r/10) node (phi) {} -- (a);
    
    % Sphere
    \draw                (orig) circle  (\r/2);
    \draw[dashed, gray]  (orig) ellipse (\r/2 and \r/6);
    
    % Axes
    \draw[->]            (orig) -- ++(-\r/10,-\r/6) node[below] (x) {\(x\)};
    \draw[->]            (orig) -- ++( \r/ 2,    0) node[right] (y) {\(y\)};
    \draw[->]            (orig) -- ++(     0, \r/2) node[above] (z) {\(z=\ket 0\)};
    \draw[->, draw=gray] (orig) -- ++(     0,-\r/2) node[below] (s) {\(\ket 1\)};
    
    %Angles
    % \draw[->, draw=gray] 
    % \draw[draw=gray,->] (orig) ellipse (\r/6 and \r/12) node {\(φ\)};
    \pic[draw=gray, ->, "\(φ\)", angle eccentricity=0.6] {angle=x--orig--phi};
    \pic[draw=gray, <-, "\(θ\)", angle eccentricity=1.4] {angle=a--orig--z};
\end{tikzpicture}
\caption{Blochova sfera}
\label{fig:bloch-sphere}
\end{figure}

\subsection{Tenzorski produkt}

V splošnem je \emph{tenzorski produkt prostorov} \(A\) in \(B\) enak množici elementov oblike \(∑ⱼaⱼ⊗bⱼ\) za \(aⱼ∈A\) in \(bⱼ∈B\).

\emph{Tenzorski produkt vektorjev} \(a ∈ \H[m]\) in \(b ∈ \H[n]\) je enak \(a⊗b = \displaystyle∑_{\substack{j ∈ \B[m],\\k ∈ \B[n]}} aⱼbₖ\ket{j}⊗\ket{k}.\)

V našem primeru je prostor \(\H[m]⊗\H[n]\) naravno izomorfen \(\H[m+n]\).
Izomorfizem slika vektorje \(\ket{j}⊗\ket{k}\) v \(\ket{jk} = \ket{j₁\dots jₘk₁\dots kₙ}\), kjer \(jk\) razumemo kot konkatenacijo binarnih vektorjev.

\begin{example}
    \(\ket{\mb0}⊗\ket{\mb1} = \ket{\mb{01}}\).
\end{example}

V nadaljevanju raje kot definicijo tenzorskega produkta vzamemo kar gornji izomorfizem, tako da pišemo \(\ket{j}⊗\ket{k} ≔ \ket{jk}\) in \(\H[m]⊗\H[n] ≔ \H[m+n]\).

% \begin{definition}
%     % \emph{Tenzorski produkt prostorov} \(\H[n]\) in \(\H[m]\) je enak \(\H[n+m]\).
%     \emph{Tenzorski produkt prostorov} \(\H[m]\) in \(\H[n]\) je množica elementov oblike \(∑ⱼ aⱼ⊗bⱼ\), kjer so \(aⱼ∈\H[m]\) in \(bⱼ∈\H[n]\).
%     Pišemo ga \(\H[m]⊗\H[n]\).
% \end{definition}
% \begin{remark}
%     Prostor \(\H[m]⊗\H[n]\) je izomorfen \(\H[m+n]\), torej lahko tenzorski produkt baznih vektorjev pišemo kot \( \ket j ⊗ \ket k = \ket{jk} = \ket{j₁\dots jₘk₁\dots kₙ}\).
% \end{remark}

\begin{example}[\(n = m = 1\)]
    Naj bosta \(a, b ∈ \H\). Tedaj je \(a⊗b = a₀b₀\ket{\mb{00}} + a₀b₁\ket{\mb{01}} + a₁b₀\ket{\mb{10}} + a₁b₁\ket{\mb{11}}\). Kot vektor to napišemo tako:
    \[ \vec{a₀, a₁}⊗\vec{b₀, b₁} = \vec{a₀b₀, a₀b₁, a₁b₀, a₁b₁}.\qedhere \]
\end{example}

Tenzorski produkt \(n\)-tih enakih vektorjev ali prostorov zapišemo kot eksponent \((-)^{⊗n}\).
Hadamardove vektorje in prostor vektorjev reda \(n\) lahko torej zapišemo z eksponenti.
\begin{align*}
    &\hh[n] = \hh^{⊗n}
    = ρⁿ\underbrace{(\ket{\mb0} + \ket{\mb1}) ⊗ ⋯ ⊗ \p{\ket{\mb0} + \ket{\mb1}}}_n,\\
    &\H[n] = \H^{⊗n} = \underbrace{\H ⊗ ⋯ ⊗ \H}_n.\qedhere
\end{align*}

\begin{definition}
    Če lahko \(a ∈ \H[n]\) zapišemo kot \( ⨂_{j=1}^{n} aⱼ \) za neke \(aⱼ ∈ \H\) pravimo, da je \emph{enostaven} ali \emph{separabilen}, sicer je pa \emph{sestavljen} oziroma \emph{kvantno prepleten}.
\end{definition}

\begin{example}
    Vektor \(\ket{\mb{00}} + \ket{\mb{10}}\) je separabilen, saj je enak \(\p{\ket{\mb0} + \ket{\mb1}}⊗\ket{\mb0}\) in \(\ket{\mb{00}} + \ket{\mb{11}}\) je prepleten, saj ga ne moremo zapisati kot tenzorski produkt dveh kubitov. Res, če je \(a₀b₁ = 0\) mora biti vsaj en izmed \(a₀\) in \(b₁\) enak \(0\), vendar je to v nasprotju z enakostmi \(a₀b₀ = a₁b₁ = 1\).
\end{example}

\begin{example}
    Stanju vektorja \(\ket{\mb{00}} + \ket{\mb{11}}\) (in v splošnem tudi \(\ket{\mb0…\mb0} + \ket{\mb1…\mb1}\)) pravimo \emph{Bellovo stanje} (\(n\)-tih kubitov).
    Predstavlja maksimalno prepleteno stanje.
\end{example}

\subsection{Kvantne preslikave}

\begin{definition}%[Unitarna vrata]
    \emph{Prostor unitarnih vrat reda \(n\)} je \( \U[n] ≔ \U{\p{2ⁿ}} \), prostor unitarnih \(2ⁿ×2ⁿ\) matrik.
    V literaturi jih tudi imenujejo \emph{unitarne transformacije}~\ref{ramšak-qc}.
\end{definition}
\begin{definition}
    \emph{Tenzorski produkt vrat} \( U⊗V ≔ [u_{jk}V]_{j,k} \) je običajen Kroneckerjev produkt matrik.
\end{definition}

\begin{example}[Tenzorski produkt vrat]
    Če je \(A\) reda \(1\) (torej \(2×2\) matrika), je tenzorski produkt \(A⊗B\) tak:
    \[ \bmat{a₀₀ & a₀₁ \\ a₁₀ & a₁₁}⊗B = \bmat{a₀₀ B & a₀₁ B \\ a₁₀ B & a₁₁ B}.\qedhere \]
\end{example}

\begin{example}
    Naj bodo \(a ∈ \H[m]\), \(b ∈ \H[n]\), \(U ∈ \U[m]\), in \(V ∈ \U[n]\).
    Potem je \(U⊗V(a⊗b) = Ua⊗Vb\).
\end{example}

\begin{theorem}[o nekloniranju]\label{th:no-cloning}
    Ne obstajajo vrata reda \(2\), ki vsak vektor oblike \(a⊗\ket{\mb0} ∈ \H⊗\H\) slikajo v \(a⊗a\).
\end{theorem}

\begin{proof}
    Naj bodo \(U\) taka vrata, da za vsak \(a ∈ \H\) velja \(U{\p{a⊗\ket{\mb0}}} = a⊗a\).
    Potem za \( \hh⊗\ket{\mb0} = ρ(\ket{\mb{00}} + \ket{\mb{10}}) \) po predpostavki velja
    \( U{\p{\hh⊗\ket{\mb0}}} = \hh[2]. \)
    Če pa upoštevamo, da je \(U\) linearna, lahko zapišemo tudi
    \[ U{\p{ρ\p{\ket{\mb{00}} + \ket{\mb{10}}}}}
       = ρ\p{U\ket{\mb{00}} + U\ket{\mb{10}}} = ρ\p{\ket{\mb{00}} + \ket{\mb{11}}},\]
    % \[
    %     U{\p{ρ\p{\ket{\mb{00}} + \ket{\mb{10}}}}} =
    %     \begin{cases}
    %         ρ²\p{\ket{\mb{00}} + \ket{\mb{01}} + \ket{\mb{10}} + \ket{\mb{11}}},\\
    %         ρ\p{U\ket{\mb{00}} + U\ket{\mb{10}}} = ρ\p{\ket{\mb{00}} + \ket{\mb{11}}},
    %     \end{cases}
    % \]
    kar je protislovje, saj \(ρ\p{\ket{\mb{00}} + \ket{\mb{11}}}\) niti ni separabilen (za razliko od \(\hh[2]\)).
    Sledi, da taka vrata ne morejo obstajati.
\end{proof}
\begin{remark}
    Podoben izrek velja tudi v obratni smeri: ne obstajajo vrata reda \(2\), ki vsak vektor oblike \(a⊗a ∈ \H⊗\H\) slikajo v \(a⊗\ket{\mb0}\).
    Imenuje se izrek o neizbrisu in sledi neposredno iz gornjega izreka, saj za taka vrata njihov inverz ravno klonira kubite. 
\end{remark}

\begin{definition}
    \emph{Paulijeve matrike} so matrike rotacije za pol kroga okrog osi na Blochovi sferi:
    \[ I₂ = \bmat{1 &  0 \\ 0 &  1},\quad
       X  = \bmat{0 &  1 \\ 1 &  0},\quad
       Y  = \bmat{0 & -i \\ i &  0},\quad
       Z  = \bmat{1 &  0 \\ 0 & -1}.
    \]
    Velja \(X² = Y² = Z² = I₂\), in ker so matrike hermitske so tudi unitarne.

    Preslikav \(X\) imenujemo kvantna negacija,
    saj je \( X{\ket{\mb0}} = \ket{\mb1} \) in \( X{\ket{\mb1}} = \ket{\mb0} \).
\end{definition}
\begin{remark}
    Velja tudi \(Y = iXZ\).
\end{remark}

\begin{definition}
    Namesto kanonične baze \(\ket{\mb0}\) in \(\ket{\mb1}\) bi lahko vzeli tudi kakšno drugo bazo.
    Definirajmo \emph{Hadamardovo bazo} \(\ket{+} ≔ ρ(\ket{\mb0} + \ket{\mb1})\) in \(\ket{-} ≔ ρ(\ket{\mb0} - \ket{\mb1})\).
    Potem prehodno matriko iz kanonične v Hadamardovo bazo imenujemo \emph{Hadamardova vrata}:
    \[ \had = ρ\bmat{1&1\\1&-1},\quad
       \had\ket{\mb{0}} = \hh = \ket{+},\quad
       \had\ket{\mb{1}} = \ket{-}.\]
    Hadamardova matrika predstavlja tudi rotacijo okrog \(z=x\) osi na Blochovi sferi, ki zamenja osi \(x\) in \(z\).
\end{definition}

\subsection{Meritev}
V klasičnem računalništvu poznamo pogojne stavke. To lahko na kubite posplošimo na dva načina:
prvič, z direktno meritvijo kubita (in uporabo klasičnih pogojnih stavkov),
drugič pa z uporabo kvantne prepletenosti.
Če po vseh operacijah zmerimo kubite, se izkaže, da se je drugi način obnašal enako kot prvi.

\begin{definition}
    \emph{Meritev kubita} \(a = a_0\ket{\mb0} + a_1\ket{\mb1}\) označimo \(M{\p{a}}\) in je \(0\) z verjetnostjo \(|a₀|^2\) ter \(1\) z verjetnostjo \(|a₁|^2\).
    Po izvedeni meritvi se \(a\) spremeni v \(a₀\ket{\mb0}\) ali \(a₁\ket{\mb1}\), odvisno od rezultata meritve.  To imenujemo \emph{kolaps prepletenega stanja}.
\end{definition}

V praksi to pomeni, da je vsa informacija o izmerjenem kubitu zajeta že v rezultatu meritve, tako da izmerjenega kubita sploh ne potrebujemo več.
To pomeni, da bi meritev lahko definirali tudi tako, da se kubit \(a\) po meritvi kar uniči in da do njega ne bomo mogli več dostopati.

Ti definiciji sta ekvivalentni, kar bomo bolje videli v primeru~\ref{ex:proj-z}.
V fizikalnih in matematičnih delih bomo uporabljali prvotno definicijo, v računalniških delih pa se izkaže, da je alternativna definicija bolj praktična.

\begin{definition}
    \emph{Meritev vektorja} \(a\) na \(i\)-tem mestu označimo \(Mᵢ{\p{a}}\) in je \(0\) z verjetnostjo \(∑_{jᵢ = \mb0}|aⱼ|²\) ter \(1\) z verjetnostjo \(∑_{jᵢ = \mb1} |aⱼ|²\).
    Po meritvi bo \(a\) enak bodisi \(∑_{jᵢ = \mb0} aⱼ\ket{j}\) bodisi \(∑_{jᵢ = \mb1} aⱼ\ket{j}\) (glede na rezultat meritve).
\end{definition}

\begin{example}
    Naj bo \(a = α\ket{\mb{00}} + β\ket{\mb{01}} + γ\ket{\mb{10}} + δ\ket{\mb{11}}\), kjer so amplitude normirane.
    Potem je rezultat \(M₁{\p a}\) lahko naslednji:
    \begin{itemize}
        \item \(0\) z verjetnostjo \(|α|² + |β|²\) in \(a = α\ket{\mb{00}} + β\ket{\mb{01}}\),
        \item \(1\) z verjetnostjo \(|γ|² + |δ|²\) in \(a = γ\ket{\mb{10}} + δ\ket{\mb{11}}\).
    \end{itemize}
    Če namesto po prvem kubitu merimo po drugem, pa je rezultat \(M₂{\p a}\) tak:
    \begin{itemize}
        \item \(0\) z verjetnostjo \(|α|² + |γ|²\) in \(a = α\ket{\mb{00}} + γ\ket{\mb{10}}\),
        \item \(1\) z verjetnostjo \(|β|² + |δ|²\) in \(a = β\ket{\mb{01}} + δ\ket{\mb{11}}\).
    \end{itemize}
    Če po meritvi po prvem kubitu izvedemo še meritev po drugem, je rezultat \(M₂{\p a}\) naslednji:
    \begin{itemize}
        \item Če je bil \(M₁{\p a} = 0\) označimo \(p₀ = |α|² + |β|²\)
        \begin{itemize}
            \item z verjetnostjo \(p₀₀ = \frac{|α|²}{|α|² + |β|²}\) je \(M₂{\p a} = 0\) in \(a = α\ket{\mb{00}}\),
            \item z verjetnostjo \(p₀₁ = \frac{|β|²}{|α|² + |β|²}\) je \(M₂{\p a} = 1\) in \(a = β\ket{\mb{01}}\).
        \end{itemize}
        \item Če je bil \(M₁{\p a} = 1\) označimo \(p₁ = |γ|² + |δ|²\)
        \begin{itemize}
            \item z verjetnostjo \(p₁₀ = \frac{|γ|²}{|γ|² + |δ|²}\) je \(M₂{\p a} = 0\) in \(a = γ\ket{\mb{10}}\),
            \item z verjetnostjo \(p₁₁ = \frac{|δ|²}{|γ|² + |δ|²}\) je \(M₂{\p a} = 1\) in \(a = δ\ket{\mb{11}}\).
        \end{itemize}
    \end{itemize}
    Skupaj je verjetnost, da na koncu izmerimo \(\ket{\mb{00}}\) enaka \(p₀p₀₀ = |α|²\), verjetnost, da izmerimo \(\ket{\mb{01}}\) je \(p₀p₀₁ = |β|²\), itd.
\end{example}

\begin{remark}
    Če meritev na vektorjih interpretiramo kot \airquotes{uničujočo}, potem preostanek vektorja po meritvi postane enak kot zgoraj, le da odstranimo izmerjeni \airquotes{kubit}.
    Na prvem od zgornjih primerov bi potem \(a\) bil enak \(α\ket{\mb{0}} + β\ket{\mb{1}}\) namesto \(α\ket{\mb{00}} + β\ket{\mb{01}}\), pri čemer \(a\) zdaj razumemo kot preostanek vektorja, ko mu odstranimo prvo komponento.
\end{remark}

\begin{definition}[Kontrola]
    Za \( r,s ∈ ℕ \) in \( U ∈ \U[1] \) definiramo \( C_{r,s}{\p U} \) in \( \overline{C}_{r,s}{\p U} \) s predpisoma
    \begin{align*}
        C_{r,s}{\p U}\ket j &= \begin{cases}
            \ket j &;\quad jᵣ = \mb0\\
            \ket{j₁\dots}\ket{U{jₛ}}\ket{\dots jₙ} &;\quad jᵣ = \mb1
        \end{cases}\\
        \overline{C}_{r,s}{\p U}\ket j &= \begin{cases}
            \ket{j₁\dots}\ket{U{jₛ}}\ket{\dots jₙ} &;\quad jᵣ = \mb0\\
            \ket j &;\quad jᵣ = \mb1
        \end{cases}
    \end{align*}
    Taka vratom imenujemo \emph{kontrolirana} (\airquotes{na ena} ali \airquotes{na nič}).
    Posebej označimo
    \[ \ctl {}U ≔ C_{1,2}{\p U} = D\p{I₂, U},\quad
       \ctlo{}U ≔ \overline{C}_{1,2}{\p U} = D\p{U, I₂}, \]
    kjer \( D{\p{U₁,…,Uₛ}} \) predstavlja bločno-diagonalno matriko matrik \( U₁, …, Uₛ \).
\end{definition}

\begin{example}
    Poglejmo si, kako \(\ctl X\) slika bazne vektorje:
    \begin{align*}
        \ctl{X}\ket{\mb{00}} &= \ket{\mb{00}}\\
        \ctl{X}\ket{\mb{01}} &= \ket{\mb{01}}\\
        \ctl{X}\ket{\mb{10}} &= \ket{\mb{11}}\\
        \ctl{X}\ket{\mb{11}} &= \ket{\mb{10}}
    \end{align*}
    Na prvi pogled je videti, kot da vrata delujejo le na drugi kubit, vendar ni nujno tako!
    Oglejmo si isto preslikavo v Hadamardovi bazi.

    \begin{align*}
        \ctl{X}\ket{\mb{++}} &= \ctl{X}⋅\had^{⊗2}\ket{\mb{00}} = \ket{\mb{++}}\\
        \ctl{X}\ket{\mb{+-}} &= \ctl{X}⋅\had^{⊗2}\ket{\mb{01}} = \ket{\mb{--}}\\
        \ctl{X}\ket{\mb{-+}} &= \ctl{X}⋅\had^{⊗2}\ket{\mb{10}} = \ket{\mb{-+}}\\
        \ctl{X}\ket{\mb{--}} &= \ctl{X}⋅\had^{⊗2}\ket{\mb{11}} = \ket{\mb{+-}}
    \end{align*}
    Tu pa je videti, kot da je kontrolni kubit na drugem mestu, spremeni pa se zgolj prvi!
    Kar se dejansko zgodi je, da se oba kubita spremenita in končata v prepletenem stanju.
\end{example}

Kot smo omenili, se kvantna kontrola in meritev obnašata enako\footnote{po meritvi}, zakaj bi torej sploh želeli uporabljati kontrolirana vrata?
Meritev nas prisili v simulacijo kvantnih operacij s klasičnimi biti, kar je računsko neučinkovito, medtem ko nam kvantna kontrola omogoča, da kvantno mehaniko simuliramo neposredno s kubiti, kar je bistveno učinkovitejše.

Za učinkovitost želimo torej meritve premakniti čim bolj proti koncu naših programov.
Kot bomo videli kasneje, veljajo določeni zakoni, ki nam to dopuščajo.

\begin{remark}
    Obstaja tudi relativno moderen pristop k formalizaciji kvantnega računalništva iz 2007, račun meritev~\cite{measurement-calculus}, ki kot glavno orodje kvantnega računalništva uporablja zgolj meritve namesto kombinacije unitarnih vrat in meritev.
\end{remark}
    
\subsection{Kvantna vezja}
Kvantne programe lahko predstavimo kot diagrame vezja, ki jih beremo od leve proti desni.
Po enojnih žicah \airquotes{tečejo} kubiti, po dvojnih pa klasični biti.
Škatle (z imenom) predstavljajo uporabo unitarnih vrat na vhodnih kubitih (povezave na levi).
Kontrolirana vrata imajo še navpično žico, ki se s piko priklopi na druge žice;
če je pika polna, so vrata kontrolirana \airquotes{na ena}, če je prazna, pa \airquotes{na nič}.
Meritev bomo predstavili s škatlo, ki vsebuje merilnik.~\cite{ess-qc}. %TODO: maybe expand

Spodaj sta dva primera kvantnih programov, opisana z besedami in diagrami, ki ju bomo srečali tudi še kasneje.

\begin{example*}[Projekcija na \(z\)-os]\label{ex:proj-z}
    Najprej izmerimo \(a\) in nato glede na rezultat sveži kubit bodisi negiramo bodisi ne.
    Na Blochovi sferi je to videti približno kot projekcija na \(z\)-os (edina kubita na \(z\)-osi sta \( \ket{\mb0} \) in \( \ket{\mb1} = X \ket{\mb0} \)).
    \[ \Qcircuit @C=1em @R=.7em {
            & \lstick{\ket{\mb0}} & \gate{\g X} & \rstick{b} \qw\\
            \lstick{a} & \meter & \cctrl{-1}
        }
    \]
    Kubit \(b\) je na koncu v prav takem stanju, kot je \(a\) po meritvi, tako da kubita \(a\) res ne potrebujemo več, zato lahko \(a\) po meritvi zavržemo.
\end{example*}

\begin{example*}[Naključna rotacija faze]\label{ex:c-rot}
    Meritev Hadamardovega vektorja simulira pravičen met kovanca,
    vrata \(Z\) pa rotirajo fazo, torej bomo naključno v polovici primerov kubitu \(a\) rotirali fazo.
    \[ \Qcircuit @C=1em @R=.7em {
            \lstick{a} & \qw & \qw & \qw & \qw & \gate{\g Z} & \rstick{a}\qw\\
            && \lstick{\ket{\mb0}} & \gate{\had} & \meter & \cctrl{-1}
        }
    \]
\end{example*}

Spomnimo se zdaj obeh definicij meritev:

% \begin{proposition}\label{th:is-eq-measurement}
%     V kvantnem računalništvu je meritev definirana s kolapsom ekvivalentna meritvi definirani z uničenjem kubita.
% \end{proposition}

% \begin{proof}
%     Očitno je prva definicija močnejša od druge, tako da moramo dokazati zgolj, da lahko iz druge definicije simuliramo obnašanje prve.
%     Ampak, če pogledamo prvi primer zgoraj, projekcijo na \(z\)-os, je to natanko to, kar potrebujemo. Izmerimo kubit \(a\) in dobimo klasični bit \(M{\p a}\), s katerim potem nov kubit nastavimo bodisi na \(\ket{\mb0}\) bodisi na \(\ket{\mb1}\), kar je pa natanko vrednost kubita \(a\) po meritvi v skladu s prvo definicijo.
% \end{proof}

\subsection{Opazljivke in stanja}
\label{sec:observables}
V našem trenutnem formalizmu smo omejeni na meritve glede na standardno bazo, vendar bi meritev lahko definirali bolj na splošno.
To se izkaže kot glavna ideja pri interpretacij vezij in programov, saj lahko namesto stanja spreminjamo kar bazo meritve.
To storimo z uporabo opazljivk, ki so poljubne fizikalne količine, ki jih lahko izmerimo.
Natančnejšo fizikalno obravnavo opazljivk najdete v~\cite[razdelek 3]{ramšak-qm} in~\cite[razdelek 9]{ess-qc}.

\begin{definition}
    \emph{Opazljivka} je sebi-adjungiran operator na prostoru \(\H[n]\) oziroma \(2ⁿ×2ⁿ\) hermitska matrika.
\end{definition}
\begin{definition}
    Naj bo \(A\) opazljivka in naj bodo \(λⱼ\) njene lastne vrednosti, \(P_{λⱼ}\) pa projekcije na pripadajoče lastne podprostore.
    Opazljivko \(A\) lahko potem zapišemo kot vsoto \(∑ⱼλⱼP_{λⱼ}\).
    Rezultat \emph{meritve opazljivke v stanju \(u\)} je ena od lastnih vrednosti \(λⱼ\) z verjetnostjo \(|P_{λⱼ}u|²\), stanje \(u\) pa se po meritvi spremeni v izmerjeno stanje, torej \(P_{λⱼ}u\).
    \emph{Pričakovana vrednost opazljivke v stanju \(u\)} je definirana kot \(\expval{A}{u} ≔ \matelt{u}{A}{u} = ∑ⱼ |P_{λⱼ}u|²λⱼ\).
\end{definition}

Opazljivka, ki jo merimo v definiciji meritve kubitov je enaka \(L ≔ D{\p{0,1}}\).
Omenimo še, da zaporedne meritve \(n\)-tih kubitov lahko izvedemo hkrati z opazljivko \(Lₙ ≔ D{\p{0, 1, 2, 3, …, 2ⁿ-1}}\). Opazljivke nam torej res posplošijo pojem meritve.

Lahko si predstavljamo, da nam opazljivke hranijo informacije o preostanku vezja na desni,
na primer~\ref{ex:proj-z} lahko s kubitom \(b\) še kaj dela.
Oglejmo si še primer, ki pripravi Bellovo stanje:
\[ \Qcircuit @C=1em @R=.7em {
        & \mbox{A} && \mbox{B} && \mbox{C} && \mbox{D}\\
        \lstick{\ket{\mb0}} & \qw & \gate{\had} & \qw & \qw & \qw & \multigate{1}{\ctl X} & \qw\\
        &&&&& \lstick{\ket{\mb0}} & \ghost{\ctl X} & \qw
    }
\]
Oznake \(A\), \(B\), \(C\), in \(D\) označujejo opazljivke, ki predstavljajo preostanek vezja desno od oznake.

Zanima nas, kako je opazljivka \(C\) povezana z opazljivko \(D\).
Izkaže se, da je \(C = \ctl{X}^* D \ctl{X}\).
To velja na splošno, namesto da izmerimo opazljivko \(Q\) v stanju \(Ua\), lahko izmerimo opazljivko \(U^*QU\) v stanju \(a\)~\cite[razdelek 3.5]{ramšak-qm}.
Podobno lahko izpeljemo \(B = \pmat{c₁₁&c₁₃\\c₃₁&c₃₃}\), kjer so \(cᵢⱼ\) elementi matrike \(C\), in \(A = \had^* B\, \had\).
Če se pomaknemo še za eno mesto v levo, pretvorimo \(A\) v \(a₁₁\), torej dobimo opazljivko, ki je konstantno \(a₁₁ = \frac{d₁₁ + d₁₄ + d₄₁ + d₄₄}{2}\).
Izračunajmo zdaj še pričakovano vrednost opazljivke \(D\) v Bellovem stanju \(a\):
\[\expval{D}{a} = \frac12\bmat{1&0&0&1}\bmat{d₁₁&d₁₂&d₁₃&d₁₄\\d₂₁&d₂₂&d₂₃&d₂₄\\d₃₁&d₃₂&d₃₃&d₃₄\\d₄₁&d₄₂&d₄₃&d₄₄}\vec{1,0,0,1} = \frac{d₁₁ + d₁₄ + d₄₁ + d₄₄}{2},\]
kar je natanko gornji rezultat!
\begin{remark}
    Temu pristopu pravimo tudi Heisenbergova slika, pri kateri stanja, ki so odvisna od časa,
    transformiramo v konstantne vektorje (v času \(t=0\)), hkrati pa transformiramo vse opazljivke, tako da so te odvisne od časa.
\end{remark}

\subsubsection{Čista in mešana stanja}
Še enkrat si oglejmo primer~\ref{ex:proj-z}, projekcijo na \(z\)-os.
Začetno stanje kubita \(a\) lahko predstavimo z našim formalizmom kot \(a₀\ket{\mb0} + a₁\ket{\mb1}\),
stanja kubita \(b\) na koncu pa ne.
To je problem, saj to pomeni, da opazljivk ne moremo meriti v stanju \(b\).
Preden se lahko lotimo tega problema, moramo najprej razviti poenoten pogled na taka stanja.

Mika nas, da bi kubit \(b\) predstavili kar z vektorjem \(a₀\ket{\mb0} + a₁\ket{\mb1}\),
saj je to skladno z našo intuicijo o meritvi, vendar kubit \(b\) ni v tem stanju, ampak je v enem izmed stanj \(\ket{\mb0}\) in \(\ket{\mb1}\), tako da ta interpretacija ni ustrezna.
Dejansko je \(b\) verjetnostna porazdelitev \(\pmat{|a₀|²&|a₁|²\\\ket{\mb0}&\ket{\mb1}}\),
ki jo imenujemo \emph{mešano stanje} in jo lahko zapišemo kot \(|a₀|²\state{\ket{\mb0}} + |a₁|²\state{\ket{\mb1}}\), kjer z \(\{a\}\) označimo ekvivalenčni razred vektorjev, ki predstavljajo enako stanje kot vektor \(a\).
Če stanje ni mešano, je \emph{čisto}.

\begin{definition}
    Naj je za vsak \(i ∈ \{1,…,m\}\) kvantni sistem z verjetnostjo \(λᵢ\) v stanju \(\state{uᵢ}\), kjer so \(uᵢ\) enotski vektorji. To je diskretna verjetnostna porazdelitev, ki jo bomo pisali kot \(λ₁\state{u₁} + ⋯ + λₘ\state{uₘ}\). Če je \(m = 1\), tako stanje imenujemo \emph{čisto stanje}, sicer pa \emph{mešano stanje}.
    Stanja \(\state{uᵢ}\) so \emph{čiste komponente mešanega stanja}.
\end{definition}
\begin{remark}
    Kvantni sistem je zgolj v enem stanju, tako da nam mešana stanja dejansko opisujejo znanje posameznega opazovalca sistema, ne pa stanja vektorja.
    Če ima sistem več kot enega opazovalca, so mešana stanja vsakega opazovalca lahko različna.
\end{remark}

Kvantna vrata lahko uporabimo tudi na mešanih stanjih, tako da jih uporabimo na vsaki čisti komponenti posebej.

Mešana stanja smo motivirali na primeru, ko je bil nek kubit verjetnostno porazdeljen med več stanji, vendar mešana stanja lahko nastanejo tudi drugače.
Naj bo \(a = a₀\ket{\mb{00}} + a₁\ket{\mb{11}}\), čisto stanje, sestavljeno iz dveh kubitov.
Označimo prvi kubit s \(b\).
Spet nas mika, da bi \(b\) predstavili s čistim stanjem \(a₀\ket{\mb0} + a₁\ket{\mb1}\), vendar
se ponovno izkaže, da je \(b\) v mešanem stanju \(a₀\state{\ket{\mb0}} + a₁\state{\ket{\mb1}}\). Kasneje bomo pokazali, da sta kubita \(b\) v obeh primerih res enaka.
% Tedaj je gostotna matrika prirejena \(b\) enaka \(\pmat{|a₀|²&0\\0&|a₁|²}\),
% ki predstavlja mešano stanje \(a₀\state{\ket{\mb0}} + a₁\state{\ket{\mb1}}\).
% \(\mixstate{a₀,a₁}{\ket{\mb0},\ket{\mb1}}\).
% \(\mixstate{a₀,\ket{\mb0},a₁,\ket{\mb1}}\)
% To stanje je enako stanju kubita \(b\) iz primera~\ref{ex:proj-z}.

Mešana stanja si lahko torej predstavljamo kot zapis informacije o nekem zunanjem vplivu.
V prvem primeru je bil zunanji vpliv klasičen (rezultat meritve), v tem primeru je pa vpliv kvantni (prepletenost).


\subsubsection{Gostotne matrike}
Zdaj imamo dve vrsti stanj, čista in mešana, vendar bi radi imeli nek poenoten način za njihovo obravnavo.
Izkaže se, da lahko poljubno stanje izrazimo z določenimi matrikami,
ki jih imenujemo gostotne matrike.

\begin{definition}
    \emph{Gostotna matrika} je neničelna pozitivno semi-definitna hermitska matrika, ki ima sled manjšo ali enako \(1\). Če je sled enaka \(1\), pravimo, da je gostotna matrika \emph{normirana}. Če ne piše drugače, naj bodo vse gostotne matrike normirane.
\end{definition}

Naj bo \(a\) nek (normiran) vektor.
Potem lahko stanju \(\state{a}\) dodelimo matriko \(\density{a}\).
Opazimo, da je \(\tr{\p{\density{a}}} = ∑ᵢ|aᵢ|² = 1\).
Hkrati pa je, če je \(γ\) neka kompleksna enota \(\p{|γ| = 1}\), matrika prirejena stanju \(\state{γa}\) enaka \(γ\bar{γ}\density{a} = \density{a}\), torej lahko vsakemu stanju določimo natanko eno gostotno matriko.

\begin{example}
    Naj bo \(a = \ket{-} = \frac{1}{\sqrt{2}}\p{\ket{\mb0} - \ket{\mb1}}\).
    Potem je gostotna matrika stanja \(\state{a}\) enaka
    \[ \density{a} = \frac12\pmat{1&-1\\-1&1}.\qedhere \]
\end{example}

Oglejmo si zdaj mešano stanje \(λ₁\state{a} + λ₂\state{b}\).
Če stanja \(\state{a}\) in \(\state{b}\) zamenjamo z njunimi gostotnimi matrikami dobimo matriko oblike \(λ₁\density{a} + λ₂\density{b}\).
Izkaže se, da je to spet gostotna matrika, saj je sled te matrike enaka \(λ₁ + λ₂ = 1\).
Stanju \(∑ᵢλᵢ\state{uᵢ}\) priredimo gostotno matriko \(∑ᵢλᵢ\density{uᵢ}\).

\begin{example}
    Izračunajmo gostotno matriko mešanega stanja \(\frac12\state{\ket{\mb0}}+\frac12\state{\ket{\mb1}}\):
    \[\frac12\pmat{1&0\\0&0} + \frac12\pmat{0&0\\0&1} = \frac12\pmat{1&0\\0&1}.\]
    Zdaj izračunajmo še gostotno matriko \(\frac12\state{\ket{+}} + \frac12\state{\ket{-}}\):
    \[\frac14\pmat{1&1\\1&1} + \frac14\pmat{1&-1\\-1&1} = \frac12\pmat{1&0\\0&1}.\]
    Čeprav sta mešani stanji različni, sta gostotni matriki enaki.
    Kot bomo videli malo kasneje, se izkaže, da opazovalec ne more ločiti med stanji, ki imajo enako gostotno matriko.
\end{example}

\begin{definition}
    Gostotna matrika je \emph{čista}, če je ranga \(1\).
\end{definition}

\begin{proposition}
    Vsaka čista gostotna matrika predstavlja neko čisto stanje.
\end{proposition}
\begin{proof}
    Naj bo \(A\) gostotna matrika. Ker je ranga \(1\) je oblike \(\density a\) za nek vektor \(a\), torej pripada čistemu stanju \(\state{a}\).
\end{proof}

\begin{proposition}
    Vsaki gostotni matriki pripada neko stanje.
\end{proposition}
\begin{proof}
    Naj bo \(A\) poljubna gostotna matrika.
    Ker je hermitska ima realne lastne vrednosti in ker je pozitivno semi-definitna so te lastne vrednosti nenegativne.
    Potem lahko \(A\) diagonaliziramo z \(UDU^*\), kjer je \(D\) diagonalna matrika lastnih vrednosti \(λᵢ\), \(U\) pa neka unitarna matrika.
    Sledi, da je \(A = ∑ᵢλᵢ\density{uᵢ}\), kjer so \(uᵢ = Ueᵢ\).
    Lastne vrednosti so nenegativne, njihova vsota je pa enaka \(\tr{A} = 1\), torej so ustrezne za koeficiente stanja \(∑ᵢλᵢ\state{uᵢ}\).
\end{proof}
\begin{remark}
    Gostotni matriki, ki se razlikujeta za nek skalarni večkratnik predstavljata isto stanje.
\end{remark}

Zdaj, ko imamo karakterizacijo stanj z gostotnimi matrikami, bi želeli na njih tudi neposredno izvajati kvantne operacije.
Poznamo dve, meritev in unitarne transformacije, in izkaže se, da lahko obe definiramo na preprost način.

Naj bo \(a\) nek vektor in \(U\) kvantna vrata primernega reda.
Potem je gostotna matrika stanja \(\state{Ua}\) enaka \(\density{Ua} = U\density{a}U^*\),
kar lahko linearno razširimo tudi na mešana stanja.
\begin{definition}
    Kvantna vrata lahko na gostotni matriki uporabimo s predpisom \(A ↦ UAU^*\).
\end{definition}

\begin{definition}
    Naj bo \(a = a₀\ket{\mb0} + a₁\ket{\mb1}\) nek kubit. Za preprostost definirajmo meritev le na enem kubitu, saj je v splošnem ideja enaka.
    % Potem je \(aᵢ ≔ ∑ⱼaᵢⱼ\ket{ij}\)
    Potem je gostotna matrika za \(a\) enaka \(\pmat{|a₀|²&a₀^*a₁\\a₁^*a₀&|a₁|²}\).
    Po meritvi na prvem mestu je \(a\) v stanju ene od polovic (dopolnjenih z ničlami na koncu ali začetku), gostotna matrika pa je enaka eni od \(\pmat{1&0\\0&0}\) ali \(\pmat{0&0\\0&1}\), odvisno od rezultata meritve.

    Če na rezultat meritve nato pozabimo, je matrika mešanega stanja, ki nastane potem enaka \(\pmat{\density{a₀}&0\\0&\density{a₁}}\).
\end{definition}

Ker sta meritev in uporaba kvantnih vrat edini operaciji, ki jih lahko izvajamo na kubitih, in jih lahko izvajamo neposredno na gostotnih matrikah, to natanko pomeni, da dveh stanj, če imata enako gostotno matriko, ne moremo ločiti.

\begin{example}
    Naj bodo \(a₁ = α\ket{\mb0} + β\ket{\mb1}\) in \(a₂ = α\ket{\mb{00}} + β\ket{\mb{11}}\)
    Naj bo \(b₁\) projekcija \(a₁\), kot v primeru~\ref{ex:proj-z}, \(b₂\) pa prvi kubit vektorja \(a₂\).
    Gostotna matrika, ki pripada vektorju \(a₂\) je \(ρ ≔ \pmat{|α|²&0&0&α^*β\\0&0&0&0\\0&0&0&0\\αβ^*&0&0&|β|²}\), gostotno matriko \(b₂\) pa lahko izračunamo kot delno sled \(\tr{₂ρ} = \pmat{|α|²&0\\0&|β|²}\)~\cite[stran 230]{ramšak-qm}, kar pa je enako gostotni matriki \(b₁\), torej je \(b₁ = b₂\), in nam mešana stanja res predstavljajo dele čistih stanj.
\end{example}

Vrnimo se zdaj na problem meritve opazljivk v mešanem stanju.
Naj bo \(ρ\) gostotna matrika  in \(A = ∑ⱼλⱼP_{λⱼ}\) neka opazljivka.
Potem je rezultat \emph{meritve opazljivke \(A\) v stanju predstavljenim z \(ρ\)} spet ena od lastnih vrednosti \(λ\) z verjetnostjo \(\tr{\p{ρP_λ}}\), gostotna matrika \(ρ\) pa se spremeni v \(P_λρP_λ\).
Izrazimo lahko tudi pričakovano vrednost opazljivke \(A\) kot
\(\expval{A}{ρ} = ∑ⱼ \tr{\p{ρP_{λⱼ}}}λⱼ = \tr{\p{ρ∑ⱼ λⱼP_{λⱼ}}} = \tr{\p{ρA}}\).

Lahko si torej mislimo, da so vse operacije izvedene na kubitih dejansko izvedene na primerni gostotni matriki, ki predstavlja njihovo (mešano) stanje.
