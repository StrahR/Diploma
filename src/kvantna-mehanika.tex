\section{Kvantna mehanika in kvantno računalništvo}
Ta del je povzet po \cite{ess-qc,selinger-qpl,ramšak-qm}.
Vsebuje osnovne definicije (in primere) matematičnih osnov kvantne mehanike,
ki jih potrebujemo za definicije želenih operacij nad kubiti.
% Te se nahajajo na prvih nekaj straneh.

\addtocontents{toc}{\protect\setcounter{tocdepth}{0}}  % skrij oznake
\subsection*{Oznake:}
Skozi ta del bomo uporabljali naslednje oznake:
\begin{itemize}
    \item \( ℕ = \{ 0, \dots \} \), \( ℕ⁺ = \{ 1, \dots \} \), \( ℕₙ = \{ 0, \dots, 2ⁿ-1 \} \),
    \item \(n, m ∈ ℕ⁺\), ki mu bomo pravili število kubitov,
    \item \(j, k, \dots ∈ ℕₙ\),
    \item \(aⱼ\) \(j\)-ta komponenta vektorja \(a\),
    \item \(j = j₁ \dots jₙ\) binarni zapis števila \(j\).
    % \item \( \mb 0ⁿ = 0\dots0, \mb 1ⁿ = 1\dots1 \).
    % \item \( \mb n = \{ 0, \dots, n-1 \} \), na primer \( \mb 2 = \{ 0, 1 \} \).
    % \item \( \mathbb n = \{ 0, \dots, n-1 \} \), na primer \( \mathbb 2 = \{ 0, 1 \} \).
\end{itemize}
\addtocontents{toc}{\protect\setcounter{tocdepth}{2}}

\subsection{Kvantni biti}

Klasični biti zavzamejo dve vrednosti: \(\mb{0}\) in \(\mb{1}\).
Te lahko posplošimo na kvantne bite (oziroma kubite) kot formalne kompleksne linearne kombinacije klasičnih stanj, t.j. vsak kubit \(q\) je oblike \(α\mb{0} + β\mb{1}\),
za \(α,β ∈ ℂ\), kjer nista oba 0. Številoma \(α\) in \(β\) pravimo amplitudi.
Tu \(\mb0\) in \(\mb1\) razumemo kot formalna simbola, ki predstavljata klasični stanji.
V splošnem stanje kubita ni odvisno od skalarnega večkratnika,
tako da kubita \(α\mb0 + β\mb1\) in \(α'\mb0 + β'\mb1\) predstavljata isto kvantno stanje
čim obstaja kak neničelen \(γ ∈ ℂ\), tako da je \(α = γα'\) in \(β = γβ'\).
Iz praktičnih razlogov se amplitude pogosto normalizira, torej jih pomnožimo s takim skalarjem,
da velja \(|α|² + |β|² = c ≤ 1\), kjer za \(c\) ponavadi vzamemo kar \(1\).
Ta normalizacija nam amplitude še vedno ne definira enolično;
\(α\) in \(β\) sta določeni zgolj do množenja s kompleksno enoto natančno.

Sledi, da klasični stanji \(\mb0\) in \(\mb1\) identificiramo z baznima vektorjema prostora \(ℂ²\), \(e₀\) in \(e₁\), te pa pogosto označujemo v t.i. \airquotes{ket} zapisu
kot \(\ket{0} ≔ e₀\) in \(\ket{1} ≔ e₁\).
Za kvantna stanja, ki niso klasična pravimo, da so kvantna superpozicija \(\ket0\) in \(\ket1\).

\begin{remark}
    \airquotes{ket} zapis je del \airquotes{braket} (izvira iz angl. \foreignlanguage{english}{bracket}) zapisa, kjer \airquotes{bra} igra vlogo dualnega vektorja in ga pišemo \(\bra{φ}\).
\end{remark}

\subsection{Kvantni vektorji}

Ko želimo kubite posplošiti na kvantne vektorje, je naš prvi instinkt, da vzamemo \(n\)-terice kubitov, vendar se izkaže, da to ne ustreza fizikalnim pojavom.
Ampak vseeno nismo daleč od resnice, le da moramo namesto \(n\)-kratnega kartezičnega produkta
vzeti \(n\)-kratni tenzorski produkt.

Začnimo s točnimi definicijami vseh gornjih pojmov.

\begin{definition}\label{binv}
    Binarni vektorji so elementi prostora \( \B[n] ≔ \{ 0, 1 \}ⁿ \) in jih pišemo kot nize v binarnem zapisu.
    % Za nas predstavljajo svet v katerem se odvijajo klasični programi.
\end{definition}

\begin{example}
    Ko nize naštevamo, jih ponavadi urejamo leksikografsko. V dveh dimenzijah so binarni vektorji sledeči: 
    \( \B[2] = \{\mb{00}, \mb{01}, \mb{10}, \mb{11}\} \).
\end{example}
\begin{remark}
    Niza \(\mb1\) in \(\mb{01}\) predstavljata različna vektorja.
\end{remark}

\begin{definition}[Hilbertov prostor]\label{hilb-sp}
    Elementom prostora \( \H[n] ≔ ℂ^{2ⁿ} \) pravimo kvantni vektorji, elementom \(\H ≔ \H[1]\) pa kubiti.  Prostoru \(\H[n]\) torej pravimo prostor kvantnih vektorjev reda \(n\), njegovo standardno bazo pa označimo z \(\{eⱼ\}ⱼ\).
    % Tu se izvajajo kvantni programi.
\end{definition}

\begin{definition}[Ket notacija]\label{ket}
    Naj bo \(j ∈ ℕₙ\), ter \( \hat{\jmath} ∈ \B[n] \) pripadajoč vektor v binarnem zapisu. Potem je \( \ket j = \ket{\hat{\jmath}} ≔ eⱼ \).
\end{definition}
\begin{remark}
    Po definiciji je torej \( \H[n] = \L[ℂ]{\set{\ket j}{j ∈ \B[n]}} \).
\end{remark}

\begin{definition}
    Stanje kubita \(q\) označimo z \(\state{q}\).
\end{definition}

Ker so kvantni vektorji zgolj vektorji, jih bomo pisali tudi kot vektorje.
Ekvivalenca med zapisoma je prikazana na naslednjih primerih:
\begin{example}[\( n = 1 \)]
    \[ a = \vec{a₀, a₁}
         = a₀\vec{1,0} + a₁\vec{0,1}
         = a₀\ket{\mb0} + a₁\ket{\mb1}.
    \]
\end{example}
\begin{example}[\( n = 2 \)]
    \[ a = \vec{a₀,  a₁,  a₂,  a₃}
         = \vec{a₀₀, a₀₁, a₁₀, a₁₁}
         = a₀₀\ket{\mb{00}} + a₀₁\ket{\mb{01}} + a₁₀\ket{\mb{10}} + a₁₁\ket{\mb{11}}.
    \]
\end{example}

\begin{example}[Hadamardov vektor]\label{had}
    Vektor, ki ima na vseh mestih enico (modulo skalarni večkratnik),
    se izkaže za zelo pomembnega. Kasneje bomo videli, kakšne raznolike uporabe ima, od vira naključnosti do bistva dokaza izreka o ne-kloniranju.
    \[ \hh ≔ ρ\vec{1,1} = ρ\p{\ket{\mb0} + \ket{\mb1}},\quad
        \hh[n] ≔ ρⁿ∑_{j ∈ \B[n]} \ket j,\quad
        ρ ≔ \frac{1}{\sqrt2}.
    \]
\end{example}

\subsection{Blochova sfera}
Spomnimo se, da so stanja kubitov določena zgolj do skalarja natančno, torej dejansko živijo v projektivnem prostoru.  Formalno to zajamemo v naslednji trditvi:

\begin{proposition}
    V fizičnem svetu sta dva kubita, ki se razlikujeta zgolj za (kompleksen) faktor, enaka.
    Matematično to pomeni, da stanja kubitov živijo v \( \mathbf{P}ℂ¹ ≅ 𝕊² \) in jih lahko zapišemo v naslednji obliki:
    % To pomeni, da lahko vsak kubit zapišemo kot točko v \( \S² \):
    \[ \state{a} = \state{\textstyle\cos\frac{θ}{2}\ket{\mb0}
                            + e^{iφ}\sin\frac{θ}{2}\ket{\mb1}},\quad
        φ ∈ [0, 2π), θ ∈ [0, π]. \]
    Temu prostoru pravimo Blochova sfera.
\end{proposition}
\begin{proof}
    Naj bo \( a = a₀\ket{\mb0} + a₁\ket{\mb1} = r₀e^{iφ₀}\ket{\mb0} + r₁e^{iφ₁}\ket{\mb1} \).
    Če označimo
    \[ r ≔ \sqrt{r₀² + r₁²}\text{, }φ ≔ φ₁ - φ₀\text{, }θ ≔ 2\arccos{\frac{r₀}{r}}, \]
    je potem \[ \hat a ≔ \frac{a}{re^{iφ₀}} = \frac{r₀}{r}\ket{\mb0} + \frac{r₁}{r}e^{iφ}\ket{\mb1} = \cos\frac{θ}{2}\ket{\mb0} + e^{iφ}\sin\frac{θ}{2}\ket{\mb1}\]
    in velja \(\{a\} = \{\hat a\}\) po konstrukciji.
\end{proof}

\begin{figure}[h]
\centering
\begin{tikzpicture}[baseline=(current bounding box.north)]
    % Define radius
    \def\r{7}
    
    % Bloch vector
    \draw  (   0,   0) node[circle, fill, inner sep=1                     ] (orig) {}
        -- (\r/6,\r/4) node[circle, fill, inner sep=0.7, label=above:\(a\)]    (a) {};
    \draw[dashed] (orig) -- (\r/6,-\r/10) node (phi) {} -- (a);
    
    % Sphere
    \draw                (orig) circle  (\r/2);
    \draw[dashed, gray]  (orig) ellipse (\r/2 and \r/6);
    
    % Axes
    \draw[->]            (orig) -- ++(-\r/10,-\r/6) node[below] (x) {\(x\)};
    \draw[->]            (orig) -- ++( \r/ 2,    0) node[right] (y) {\(y\)};
    \draw[->]            (orig) -- ++(     0, \r/2) node[above] (z) {\(z=\ket 0\)};
    \draw[->, draw=gray] (orig) -- ++(     0,-\r/2) node[below] (s) {\(\ket 1\)};
    
    %Angles
    % \draw[->, draw=gray] 
    % \draw[draw=gray,->] (orig) ellipse (\r/6 and \r/12) node {\(φ\)};
    \pic[draw=gray, ->, "\(φ\)", angle eccentricity=0.6] {angle=x--orig--phi};
    \pic[draw=gray, <-, "\(θ\)", angle eccentricity=1.4] {angle=a--orig--z};
\end{tikzpicture}
\caption{Blochova sfera}
\label{fig:bloch-sphere}
\end{figure}

\subsection{Tenzorski produkt}

\begin{definition}\label{tensorprod}
    Tenzorski produkt prostorov \(\H[n]\) in \(\H[m]\) je enak \(\H[n+m]\).
    Pišemo \(\H[n]⊗\H[m]\).
    % Če sta \(a ∈ \H[n]\) in \(b ∈ \H[m]\) je \(a⊗b ∈ \H[n]⊗\H[m]\).
\end{definition}
\begin{remark}
    Operator \(⊗\) je res tenzorski produkt.
\end{remark}

\begin{definition}
    Tenzorski produkt vektorjev \(a ∈ \H[n]\) in \(b ∈ \H[m]\) živi v prostoru \(\H[n]⊗\H[m]\)
    in je enak \(a⊗b = \displaystyle∑_{\substack{j ∈ \B[n],\\k ∈ \B[m]}} aⱼbₖ\ket{j}⊗\ket{k}.\)
\end{definition}
\begin{example}[Tenzorski produkt baznih vektorjev]
    Pišemo ga lahko na različne ekvivalentne načine:
    \[ \ket j ⊗ \ket k = \ket{j₁\dots jₙk₁\dots kₘ} = \ket j \ket k = \ket{jk}. \]
\end{example}
\begin{example}[\(n = m = 1\)]
    Naj bosta \(a, b ∈ \H\). Tedaj je \(a⊗b = a₀b₀\ket{\mb{00}} + a₀b₁\ket{\mb{01}} + a₁b₀\ket{\mb{10}} + a₁b₁\ket{\mb{11}}\). Kot vektor to napišemo tako:
    \[ \vec{a₀, a₁}⊗\vec{b₀, b₁} = \vec{a₀b₀, a₀b₁, a₁b₀, a₁b₁}\]
\end{example}
\begin{examples}[Tenzorski eksponent]
    \begin{align*}
        &\hh[n] = \hh^{⊗n}
        = ρⁿ\underbrace{(\ket{\mb0} + \ket{\mb1}) ⊗ ⋯ ⊗ \p{\ket{\mb0} + \ket{\mb1}}}_n,\\
        &\H[n] = \H^{⊗n} = \underbrace{\H ⊗ ⋯ ⊗ \H}_n.
    \end{align*}
\end{examples}

\begin{definition}
    Če lahko \(a ∈ \H[n]\) zapišemo kot \( ⨂_{j=1}^{n} aⱼ \) za neke \(aⱼ ∈ \H\) pravimo, da je enostaven ali separabilen, sicer je pa sestavljen oziroma kvantno prepleten.
\end{definition}

\begin{example}
    Vektor \(\ket{\mb{00}} + \ket{\mb{10}}\) je separabilen, saj je enak \(\p{\ket{\mb0} + \ket{\mb1}}⊗\ket{\mb0}\) in \(\ket{\mb{00}} + \ket{\mb{11}}\) je prepleten, saj ga ne moremo zapisati kot tenzorski produkt dveh kubitov. Res, če je \(a₀b₁ = 0\) mora biti vsaj en izmed \(a₀\) in \(b₁\) enak \(0\), vendar je to v nasprotju z eno od enačb \(a₀b₀ = a₁b₁ = 1\).
\end{example}

\subsection{Kvantne preslikave}

\begin{definition}%[Unitarna vrata]
    Prostor unitarnih vrat reda \(n\) je \( \U[n] ≔ \U{\p{2ⁿ}} \), prostor unitarnih \(2ⁿ×2ⁿ\) matrik.
\end{definition}
\begin{definition}
    Tenzorski produkt vrat \( U⊗V ≔ [u_{jk}V]_{j,k} \) je običajen Kroneckerjev produkt matrik.
    Produkt \(U⊗V\) uporabljen na \(a⊗b\) je enak \(Ua⊗Vb\).
\end{definition}

\begin{example}[Tenzorski produkt unitarnih vrat]
    Če je \(A\) reda \(1\), zgleda tenzorski produkt \(A⊗B\) tako:
    \[ \bmat{a₀₀ & a₀₁ \\ a₁₀ & a₁₁}⊗B = \bmat{a₀₀ B & a₀₁ B \\ a₁₀ B & a₁₁ B}. \]
\end{example}

\begin{definition}%[Bločno-diagonalna matrika]
    Za vrata \( U₁, …, Uₛ \) označimo njihovo bločno-diagnoalno matriko z \( D{\p{U₁,…,Uₛ}} \).
\end{definition}

\begin{theorem}[o nekloniranju]\label{no-cloning}
    Ne obstajajo vrata reda \(2\), ki vsak vektor \( a⊗\ket{\mb0} ∈ \H⊗\H \) slika v \(a⊗a\).
\end{theorem}

\begin{proof}
    Naj bo \(U\) tak, da za vsak \(a ∈ \H\) velja \( U{\p{a⊗\ket{\mb0}}} = a⊗a \).
    Potem za \( \hh⊗\ket{\mb0} = ρ(\ket{\mb{00}} + \ket{\mb{10}}) \) po definiciji \(U\) velja
    \( U{\p{\hh⊗\ket{\mb0}}} = \hh[2]. \)
    Vendar, če upoštevamo, da je \(U\) linearna lahko zapišemo tudi
    \[ U{\p{ρ\p{\ket{\mb{00}} + \ket{\mb{10}}}}}
       = ρ\p{U\ket{\mb{00}} + U\ket{\mb{10}}} = ρ\p{\ket{\mb{00}} + \ket{\mb{11}}},\]
    % \[
    %     U{\p{ρ\p{\ket{\mb{00}} + \ket{\mb{10}}}}} =
    %     \begin{cases}
    %         ρ²\p{\ket{\mb{00}} + \ket{\mb{01}} + \ket{\mb{10}} + \ket{\mb{11}}},\\
    %         ρ\p{U\ket{\mb{00}} + U\ket{\mb{10}}} = ρ\p{\ket{\mb{00}} + \ket{\mb{11}}},
    %     \end{cases}
    % \]
    kar je protislovje, saj \(ρ\p{\ket{\mb{00}} + \ket{\mb{11}}}\) ni separabilen.
\end{proof}

\begin{example}[Paulijeve matrike]
    To so matrike zrcaljenja okrog osi na Blochovi sferi:
    \[ I₂ = \bmat{1 &  0 \\ 0 &  1},\quad
       X  = \bmat{0 &  1 \\ 1 &  0},\quad
       Y  = \bmat{0 & -i \\ i &  0},\quad
       Z  = \bmat{1 &  0 \\ 0 & -1}.
    \]
    Velja \(X² = Y² = Z² = I₂\).
    Preslikavi \(X\) pravimo negacija,
    saj je \( X{\ket{\mb0}} = \ket{\mb1} \) in \( X{\ket{\mb1}} = \ket{\mb0} \).
\end{example}

\begin{example}[Hadamardova matrika]
    Namesto kanonične baze \(\ket{\mb0}\) in \(\ket{\mb1}\) bi lahko vzeli tudi kakšno drugo.
    Definirajmo Hadamardovo bazo \(\ket{+} ≔ ρ(\ket{\mb0} + \ket{\mb1})\) in \(\ket{-} ≔ ρ(\ket{\mb0} - \ket{\mb1})\).
    Potem prehodni matriki iz kanonične v Hadamardovo bazo pravimo Hadamardova matrika:
    \[ \had = ρ\bmat{1&1\\1&-1},\quad
       \had\ket{\mb{0}} = \hh = \ket{+},\quad
       \had\ket{\mb{1}} = \ket{-}.
    \]
\end{example}

% \[ \Qcircuit @C=1em @R=.7em {
%         & \gate{\had} & \gate{\g Z} & \gate{\had} & \qw\\
%         &             &      =      &             &    \\
%         & \qw         & \gate{\g X} & \qw         & \qw
%     }
% \]
% \begin{example}[Fazni zamik]
%     \begin{align*}
%         &S_α = \bmat{1&0\\0&e^{iα}}
%         \text{, posebej označimo } S ≔ S_{\newfrac{π}{2}}, T ≔ S_{\newfrac{π}{4}}, \\
%         &S_α\p{a₀\ket{\mb0} + a₁\ket{\mb1}} = a₀\ket{\mb0} + a₁e^{iα}\ket{\mb1}.
%     \end{align*}
% \end{example}

\subsection{Kvantna meritev}

V klasičnem računalništvu poznamo pogojne stavke. To lahko na kubite posplošimo na dva načina,
prvi z direktno meritvijo kubita (in uporabo klasičnih pogojnih stavkov),
drugi pa z uporabo kvantne prepletenosti.
Izkaže se, da če po vseh operacijah zmerimo kubite, se drugi način obnaša enako kot prvi.

\begin{definition}
    Meritev kubita \(a = a_0\ket{\mb0} + a_1\ket{\mb1}\) označimo \(M{\p{a}}\) in je \(0\) z verjetnostjo \(|a₀|^2\) ter \(1\) z verjetnostjo \(|a₁|^2\).
    Po izvedeni meritvi, je \(a\) bodisi v stanju \(a₀\ket{\mb0}\), bodisi v stanju \(a₁\ket{\mb1}\), odvisno od rezultata meritve.  Temu pravimo kolaps prepletenega stanja.
    V praksi to pomeni, da \(a\) po meritvi izgubi vse svoje kvantne lastnosti, tako da je ekvivalentno, če rečemo, da se \(a\) kar uniči.
    V fizikalnih in matematičnih delih bomo uporabljali prvotno definicijo, v računalniških delih pa definicijo z uničenjem.
    %To \airquotes{uniči} kubit \(a\).
    % \[ \Qcircuit @C=1em @R=.7em {
    %         & \meter & \cw %& & \ctrlo{1} & \ctrl{1} & \qw & & \ctrl{1} & \qw \\
    %         % &        &     & & \gate U & \gate V & \qw & & \targ & \qw
    %     }
    % \]
\end{definition}

\begin{proposition}
    V kvantnem računalništvu je meritev definirana s kolapsom ekvivalentna meritvi definirani z uničenjem kubita.
\end{proposition}

Dokaz prepustimo za malo kasneje, ko definiramo orodja za kvantno računalništvo.

\begin{definition}
    Naj \(αₖ\) označujejo tiste amplitude, ki se nahajajo pred keti, ki imajo na \(j\)-tem mestu \(\mb{0}\), in \(βₖ\) tiste, ki se nahajajo pred keti, ki imajo na \(j\)-tem mestu \(\mb{1}\).
    Meritev vektorja \(a\) na \(i\)-tem mestu označimo \(Mᵢ{\p{a}}\) in je \(0\) z verjetnostjo \(∑ᵢ|αᵢ|²\) ter \(1\) z verjetnostjo \(∑ᵢ|βᵢ|²\).
    Po meritvi bo \(a\) enak bodisi \(∑_{k = j₁⋯\mb0⋯jₙ} αₖ\ket{k}\) bodisi \(∑_{k = j₁⋯\mb1⋯jₙ} βₖ\ket{k}\).
\end{definition}

\begin{example}
    Naj bo \(a = α\ket{\mb{00}} + β\ket{\mb{01}} + γ\ket{\mb{10}} + δ\ket{\mb{11}}\), kjer so amplitude normalizirane na \(1\).
    Tedaj je rezultat \(M₁{\p a}\) lahko sledeč:
    \begin{itemize}
        \item \(0\) z verjetnostjo \(|α|² + |β|²\) in \(a = α\ket{\mb{00}} + β\ket{\mb{01}}\),
        \item \(1\) z verjetnostjo \(|γ|² + |δ|²\) in \(a = γ\ket{\mb{10}} + δ\ket{\mb{11}}\).
    \end{itemize}
    Če namesto po prvem kubitu merimo po drugem, je rezultat sledeč:
    \begin{itemize}
        \item \(0\) z verjetnostjo \(|α|² + |γ|²\) in \(a = α\ket{\mb{00}} + γ\ket{\mb{10}}\),
        \item \(1\) z verjetnostjo \(|β|² + |δ|²\) in \(a = β\ket{\mb{01}} + δ\ket{\mb{11}}\).
    \end{itemize}
    Če po meritvi po prvem kubitu izvedemo še meritev po drugem, je rezultat sledeč:
    \begin{itemize}
        \item Če je bil \(M₁{\p a} = 0\) označimo \(p₀ = |α|² + |β|²\)
        \begin{itemize}
            \item z verjetnostjo \(p₀₀ = \frac{|α|²}{|α|² + |β|²}\) je \(M₂{\p a} = 0\) in \(a = α\ket{\mb{00}}\),
            \item z verjetnostjo \(p₀₁ = \frac{|β|²}{|α|² + |β|²}\) je \(M₂{\p a} = 1\) in \(a = β\ket{\mb{01}}\).
        \end{itemize}
        \item Če je bil \(M₁{\p a} = 1\) označimo \(p₁ = |γ|² + |δ|²\)
        \begin{itemize}
            \item z verjetnostjo \(p₁₀ = \frac{|γ|²}{|γ|² + |δ|²}\) je \(M₂{\p a} = 0\) in \(a = γ\ket{\mb{10}}\),
            \item z verjetnostjo \(p₁₁ = \frac{|δ|²}{|γ|² + |δ|²}\) je \(M₂{\p a} = 1\) in \(a = δ\ket{\mb{11}}\).
        \end{itemize}
    \end{itemize}
    Skupaj je verjetnost, da na koncu izmerimo \(\ket{\mb{00}}\) enaka \(p₀p₀₀ = |α|²\), verjetnost, da izmerimo \(\ket{\mb{01}}\) je \(p₀p₀₁ = |β|²\), itd.
    Omenili smo, da ponavadi normaliziramo na \(1\), vendar se izkaže, da je bolj praktično, če normaliziramo na skupno verjetnost, da pristanemo v tem stanju.
    Ena prednost tega pristopa je, da po meritvi ne rabimo normalizirati kubita.
\end{example}

\begin{convention}
    Če ne piše drugače naj bodo vektorji normalizirani kot piše zgoraj.
\end{convention}

\begin{remark}
    Če meritev na vektorjih interpretiramo kot \airquotes{uničujočo}, potem preostanek vektorja po meritvi postane enak kot zgoraj, le da odstranimo izmerjeni \airquotes{kubit}.
    Na zgornjem primeru to zgleda kot \(a = α\ket{\mb{0}} + β\ket{\mb{1}}\) namesto \(α\ket{\mb{00}} + β\ket{\mb{01}}\), kjer \(a\) zdaj razumemo kot preostanek vektorja, ko mu odstranimo prvo komponento.
\end{remark}

\begin{definition}[Kontrola]
    Za \( r,s ∈ ℕ \) in \( U ∈ \U[1] \) definiramo \( C_{r,s}{\p U} \) in \( \overline{C}_{r,s}{\p U} \) s predpisoma
    \begin{align*}
        C_{r,s}{\p U}\ket j &= \begin{cases}
            \ket j &;\quad jᵣ = \mb0\\
            \ket{j₁\dots}\ket{U{jₛ}}\ket{\dots jₙ} &;\quad jᵣ = \mb1
        \end{cases}\\
        \overline{C}_{r,s}{\p U}\ket j &= \begin{cases}
            \ket{j₁\dots}\ket{U{jₛ}}\ket{\dots jₙ} &;\quad jᵣ = \mb0\\
            \ket j &;\quad jᵣ = \mb1
        \end{cases}
    \end{align*}
    Takim vratom pravimo kontrolirana (\airquotes{na ena} in \airquotes{na nič}).
    Posebej za \( U ∈ \U[1] \) označimo
    \[ \ctl U ≔ C_{1,2}{\p U} = \D{I₂, U},\quad
        \ctlo U ≔ \overline{C}_{1,2}{\p U} = \D{U, I₂}. \]
\end{definition}

\begin{example}
    Poglejmo si, kako \(\ctl X\) slika bazne vektorje:
    \begin{align*}
        \ctl{X}\ket{\mb{00}} &= \ket{\mb{00}}\\
        \ctl{X}\ket{\mb{01}} &= \ket{\mb{01}}\\
        \ctl{X}\ket{\mb{10}} &= \ket{\mb{11}}\\
        \ctl{X}\ket{\mb{11}} &= \ket{\mb{10}}
    \end{align*}
    Na prvi pogled zgleda, kot da vrata delujejo le na drugi kubit, vendar temu ni nujno tako!
    Oglejmo si isto preslikavo v Hadamardovi bazi.

    \begin{align*}
        \ctl{X}\ket{\mb{++}} &= \ctl{X}⋅\had^{⊗2}\ket{\mb{00}} = \ket{\mb{++}}\\
        \ctl{X}\ket{\mb{+-}} &= \ctl{X}⋅\had^{⊗2}\ket{\mb{01}} = \ket{\mb{--}}\\
        \ctl{X}\ket{\mb{-+}} &= \ctl{X}⋅\had^{⊗2}\ket{\mb{10}} = \ket{\mb{-+}}\\
        \ctl{X}\ket{\mb{--}} &= \ctl{X}⋅\had^{⊗2}\ket{\mb{11}} = \ket{\mb{+-}}
    \end{align*}
    V tem primeru pa zgleda, kot da je kontrolni kubit na drugem mestu, spremeni se pa zgolj prvi!
    Kar se dejansko zgodi je, da se oba kubita spremenita.
    Pravimo, da sta po operaciji kubita v prepletenem stanju.
\end{example}
% \begin{example}[Prepleteni pari kubitov]
%     Kontrolirana vrata prepletejo pare kubitov. Na primer
%     \( \eapply{\ctl X}{a, b} \) se obnaša kot
%     \qpl[breaklines]{if |\(\emeasure{a} = 0\)| then |\(\p{a, b}\)| else |\(\p{a, \lnot b}\)|},
%     vendar vemo, da drugi izraz ni veljaven (ker meritev uniči kubit a),
%     ampak je zato kontrola ravno tisto orodje, s katerim želimo nadomestiti pogojne stavke.
% \end{example}

\subsection{Kvantna vezja}

Kvantne programe lahko predstavimo kot diagrame vezja, ki jih beremo od leve proti desni.
Po enojnih žicah \airquotes{tečejo} kubiti, po dvojnih pa klasični biti.
Škatle (z imenom) predstavljajo uporabo unitarnih vrat na vhodnih kubitih.
Kontrolirana vrata imajo še piko in iz nje navpično črto;
če je pika polna kontrolira \airquotes{na ena}, če je pa prazna pa \airquotes{na nič}.
Meritev bomo predstavili s škatlo, ki ima v sebi \airquotes{merilec}.\cite{ess-qc}. %TODO: maybe expand

Spodaj sta dva primera kvantnih programov, opisana z besedami in diagrami, ki ju bomo srečali tudi še kasneje.

\begin{example}[Projekcija na \(z\)-os]\label{ex:proj-z}
    Najprej izmerimo \(a\) in nato glede na rezultat svež kubit bodisi negiramo bodisi ne.
    Na Blochovi sferi to zgleda približno kot projekcija na \(z\)-os (edina kubita na \(z\)-osi sta \( \ket{\mb0} \) in \( \ket{\mb1} = X \ket{\mb0} \)).
    \[ \Qcircuit @C=1em @R=.7em {
            & \lstick{\ket{\mb0}} & \gate{\g X} & \rstick{b} \qw\\
            \lstick{a} & \meter & \cctrl{-1}
        }
    \]
\end{example}

\begin{example}[Naključna rotacija faze]\label{ex:rand-ph-shift}
    Meritev Hadamardovega vektorja simulira pravičen met kovanca,
    vrata \(Z\) pa rotirajo fazo, torej bomo v polovici primerov kubitu \(a\) rotirali fazo.
    \[ \Qcircuit @C=1em @R=.7em {
            \lstick{a} & \qw & \qw & \qw & \qw & \gate{\g Z} & \rstick{a}\qw\\
            && \lstick{\ket{\mb0}} & \gate{\had} & \meter & \cctrl{-1}
        }
    \]
\end{example}

Spomnimo se zdaj trditve o ekvivalenci definicij meritev:

\begin{proposition}
    V kvantnem računalništvu je meritev definirana s kolapsom ekvivalentna meritvi definirani z uničenjem kubita.
\end{proposition}

\begin{proof}
    Očitno je prva definicija močnejša od druge, tako da moramo dokazati zgolj, da lahko iz druge definicije simuliramo obnašanje prve.
    Ampak če pogledamo prvi primer zgoraj, projekcijo na \(z\)-os, je to natanko to, kar potrebujemo. Izmerimo kubit \(a\) in dobimo klasični bit \(M{\p a}\), s katerim potem nov kubit nastavimo bodisi na \(\ket{\mb0}\) bodisi na \(\ket{\mb1}\), kar je pa natanko vrednost kubita \(a\) po meritvi v skladu s prvo definicijo.
\end{proof}

\subsection{Čista in mešana stanja}
Še enkrat si oglejmo prvi primer zgoraj, projekcijo na \(z\)-os.
Stanje kubita pred projekcijo znamo napisati: \(\state{a₀\ket{\mb0} + a₁\ket{\mb1}}\).
Ampak stanja kubita \(b\) na koncu pa ne. Kubit je namreč v tako imenovanem mešanem stanju med stanji \(\state{\ket{\mb0}}\) in \(\state{\ket{\mb1}}\). Zapišimo to stanje torej kot \(|a₀|²\state{\ket{\mb0}} + |a₁|²\state{\ket{\mb1}}\).

\begin{definition}
    Naj je za vsak \(i ∈ \{1,…,m\}\) kvantni sistem z verjetnostjo \(λᵢ\) v stanju \(\state{uᵢ}\), kjer so \(uᵢ\) enotski vektorji. To je diskretna verjetnostna porazdelitev, ki jo bomo pisali kot \(λ₁\state{u₁} + ⋯ + λₘ\state{uₘ}\). Če je \(m = 1\) takemu stanju pravimo čisto stanje, sicer pa mešano stanje.
    Stanjem \(\state{uᵢ}\) pravili tudi čiste komponente mešanega stanja.
\end{definition}
\begin{remark}
    Kvantni sistem je zgolj v enem stanju, tako da nam mešana stanja dejansko največ povejo o posameznem opazovalcu sistema in kaj on ve.
    Če ima sistem več kot enega opazovalca, so lahko mešana stanja vsakega opazovalca različna.
\end{remark}

\begin{definition}
    Kvantna vrata lahko uporabimo tudi na mešanih stanjih,
    tako da jih uporabimo na vsaki čisti komponenti posebej.
\end{definition}

\begin{example}
    Naj bo \(a = α\ket{\mb0} + β\ket{\mb1}\), ki ga izmerimo. Sedaj je \(a\) v enem od stanj \(α\ket{\mb0}\) in \(β\ket{\mb1}\). Sedaj \airquotes{pozabimo} koliko je rezultat meritve, je \(a\) (iz naše perspektive) v mešanem stanju \(\state{α\ket{\mb0}} + \state{β\ket{\mb1}} = |α|²\state{\ket{\mb0}} + |β|²\state{\ket{\mb1}}\).
\end{example}

\subsection{Gostotne matrike}

\begin{definition}
    Gostotna matrika je neničelna pozitivno semi-definitna hermitska matrika, ki ima sled manjšo ali enako \(1\).
\end{definition}

Izkaže se, da lahko stanje poljubnega vektorja izrazimo z gostotnimi matrikami. Res, naj bo \(a\) nek normaliziran vektor.
Potem \(\state{a}\) dodelimo gostotno matriko \(aa^*\).
Opazimo, da je \(\tr{\p{aa^*}} = ∑ᵢ|aᵢ|² = c\), hkrati pa, če je \(γ\) neka kompleksna enota \(\p{|γ| = 1}\) je gostotna matrika \(\state{γa}\) enaka \(γ\bar{γ}aa^* = aa^*\), torej nam normalizacija gostotnih matrik enolično določi stanje vektorja.

\begin{example}
    Naj bo \(a = \ket{-} = \frac{1}{\sqrt{2}}\p{\ket{\mb0} - \ket{\mb1}}\).
    Potem je gostotna matrika prirejena \(\state{a}\) enaka
    \[aa^* = \frac12\pmat{1&-1\\-1&1}\]
\end{example}

Oglejmo si sedaj mešano stanje \(∑ᵢλᵢ\state{uᵢ}\).
Če stanja \(\state{uᵢ}\) zamenjamo z njihovimi gostotnimi matrikami dobimo matriko oblike \(λ₁u₁u₁^* + ⋯ + λₘuₘuₘ^*\).
Izkaže se, da je to spet gostotna matrika, saj je sled te matrike konveksna kombinacija števil \(\tr{\p{uᵢuᵢ^*}} ≤ 1\).

\begin{example}
    Izračunajmo gostotno matriko mešanega stanja \(\frac12\state{\ket{\mb0}}+\frac12\state{\ket{\mb1}}\):
    \[\frac12\pmat{1&0\\0&0} + \frac12\pmat{0&0\\0&1} = \frac12\pmat{1&0\\0&1}.\]
    Sedaj pa izračunajmo še gostotno matriko \(\frac12\state{\ket{+}} + \frac12\state{\ket{-}}\):
    \[\frac14\pmat{1&1\\1&1} + \frac14\pmat{1&-1\\-1&1} = \frac12\pmat{1&0\\0&1}.\]
    Kljub temu, da sta mešani stanji različni, sta gostotni matriki enaki.
    Kot bomo videli malo kasneje, se izkaže, da opazovalec ne more ločiti med stanji, ki imajo enako gostotno matriko.
\end{example}

\begin{definition}
    Gostotna matrika je čista, če je ranga \(1\).
\end{definition}

\begin{proposition}
    Vsaka čista gostotna matrika predstavlja neko čisto stanje.
\end{proposition}
\begin{proof}
    Naj bo \(A\) gostotna matrika. Ker je ranga \(1\) je oblike \(aa^*\) za nek vektor \(a\), torej pripada stanju \(\state{a}\).
\end{proof}

\begin{proposition}
    Vsaki gostotni matriki pripada neko mešano stanje.
\end{proposition}
\begin{proof}
    Naj bo \(A\) poljubna gostotna matrika.
    Ker je hermitska ima realne lastne vrednosti in ker je pozitivno semi-definitna so te lastne vrednosti nenegativne.
    Potem lahko \(A\) diagonaliziramo z \(UDU^*\), kjer je diagonalna matrika lastnih vrednosti \(λᵢ\), \(U\) pa neka unitarna matrika.
    Sledi, da je \(A = ∑ᵢλᵢuᵢuᵢ^*\), kjer je \(uᵢ = Ueᵢ\).
    Lastne vrednosti so nenegativne, njihova vsota je pa enaka \(\tr{A} ≤ 1\), torej so ustrezne za koeficiente mešanega stanja \(∑ᵢλᵢ\state{uᵢ}\).
\end{proof}

Zdaj, ko imamo karakterizacijo stanj z gostotnimi matrikami, bi želeli na njih tudi neposredno izvajati kvantne operacije.
Poznamo dve, in izkaže se, da lahko obe definiramo na preprost način.

\begin{definition}
    Naj bo \(a\) nek vektor in \(U\) kvantna vrata.
    Potem je gostotna matrika stanja \(\state{Ua}\) enaka \(Uaa^*U^*\),
    kar lahko linearno razširimo tudi na mešana stanja, torej lahko kvantna vrata na gostotni matriki uporabimo s predpisom \(A ↦ UAU^*\).
\end{definition}

\begin{definition}
    Naj bo \(a\) nek vektor. Za preprostost definirajmo meritev le na prvem kubitu, saj je v splošnem ideja enaka.
    Razdelimo vektor na zgornjo in spodnjo polovico, \(a₀\) in \(a₁\).
    % Potem je \(aᵢ ≔ ∑ⱼaᵢⱼ\ket{ij}\)
    Potem je gostotna matrika za \(a\) enaka \(\pmat{a₀a₀^*&a₀a₁^*\\a₁a₀^*&a₁a₁^*}\).
    Po meritvi na prvem mestu je \(a\) v stanju ene od polovic (dopolnjenih z ničlami na koncu ali začetku), gostotna matrika pa je enaka eni od \(\pmat{a₀a₀^*&0\\0&0}\) ali \(\pmat{0&0\\0&a₁a₁^*}\), odvisno od rezultata meritve.
    Opazimo, da je sled teh gostotnih matrik enaka verjetnosti, da ta gostotna matrika ustreza kolapsiranemu vektorju, torej lahko dogovor o normalizaciji uporabljamo tudi za gostotne matrike.
\end{definition}

Ker sta meritev in uporaba kvantnih vrat edini operaciji, ki jih lahko izvajamo na kubitih, in jih lahko izvajamo neposredno na gostotnih matrikah, to natanko pomeni, da če imata dve stanji enako gostotno matriko, ju ne moremo ločiti.
