\section{Kvantna mehanika}

Ta del je povzet po \cite{ess-qc}.
Vsebuje osnovne definicije (in primere) matematičnih osnov kvantne mehanike,
ki jih potrebujemo za definicije želenih operacij nad kubiti.
% Te se nahajajo na prvih nekaj straneh.

\addtocontents{toc}{\protect\setcounter{tocdepth}{0}}  % skrij oznake
\subsection*{Oznake:}
Skozi ta del bomo uporabljali naslednje oznake:
\begin{itemize}
    \item \( ℕ = \{ 0, \dots \} \), \( ℕ⁺ = \{ 1, \dots \} \), \( ℕₙ = \{ 0, \dots, 2ⁿ-1 \} \),
    \item \(n, m ∈ ℕ⁺\), ki mu bomo pravili število kubitov,
    \item \(j, k, \dots ∈ ℕₙ\),
    \item \(aⱼ\) \(j\)-ta komponenta vektorja \(a\),
    \item \(j = j₁ \dots jₙ\) binarni zapis števila \(j\).
    % \item \( \mb 0ⁿ = 0\dots0, \mb 1ⁿ = 1\dots1 \).
    % \item \( \mb n = \{ 0, \dots, n-1 \} \), na primer \( \mb 2 = \{ 0, 1 \} \).
    % \item \( \mathbb n = \{ 0, \dots, n-1 \} \), na primer \( \mathbb 2 = \{ 0, 1 \} \).
\end{itemize}
\addtocontents{toc}{\protect\setcounter{tocdepth}{2}}

\subsection{Kvantni vektorji}

\begin{definition}\label{binv}
    Binarni vektorji so elementi prostora \( \B[n] ≔ \{ 0, 1 \}ⁿ \) in jih pišemo kot nize v binarnem zapisu.  Za nas predstavljajo svet v katerem se odvijajo klasični programi.
\end{definition}

\begin{example}  %TODO: dodaj opis
    \( \B[2] = \{00, 01, 10, 11\} \).
\end{example}
\begin{remark}
    Niza \(1\) in \(01\) predstavljata različna vektorja.
\end{remark}

\begin{definition}[Hilbertov prostor]\label{hilb-sp}
    Elementom prostora \( \H[n] ≔ ℂ^{2ⁿ} \) pravimo kvantni vektorji, elementom \(\H ≔ \H[1]\) pa kubiti.  Prostoru \(\H[n]\) torej pravimo prostor kvantnih vektorjev reda \(n\), njegovo standardno bazo pa označimo z \(\{eⱼ\}ⱼ\). Tu se izvajajo kvantni programi.
\end{definition}

\begin{definition}[Braket notacija]\label{braket}
    Naj bo \(j ∈ ℕₙ\), ter \( \hat{\jmath} ∈ \B[n] \) pripadajoč vektor v binarnem zapisu. Potem je \( \ket j = \ket{\hat{\jmath}} ≔ eⱼ \).
\end{definition}
\begin{remark}
    Po definiciji je torej \( \H[n] = \L[ℂ]{\set{\ket j}{j ∈ \B[n]}} \).
\end{remark}

\begin{example}[\( n = 1 \)]
    \[ a = \vec{a₀, a₁}
         = a₀\vec{1,0} + a₁\vec{0,1}
         = a₀\ket 0 + a₁\ket 1.
    \]
\end{example}
\begin{example}[\( n = 2 \)]
    \[ a = \vec{a₀,  a₁,  a₂,  a₃}
         = \vec{a₀₀, a₀₁, a₁₀, a₁₁}
         = a₀₀\ket{00} + a₀₁\ket{01} + a₁₀\ket{10} + a₁₁\ket{11}.
    \]
\end{example}

\begin{example}[Hadamardov vektor]\label{had}
    \[ \hh ≔ ρ \left(\ket 0 + \ket 1\right),\quad
        \hh[n] ≔ ρⁿ∑_{j ∈ \B[n]} \ket j,\quad
        ρ ≔ \frac{1}{\sqrt2}.
    \]
\end{example}

\subsection{Blochova sfera}

\begin{statement}  %TODO: elaborate
    V fizičnem svetu sta dva kubita, ki se razlikujeta zgolj za (kompleksen) faktor, enaka.
    Matematično to pomeni, da stanja kubitov (nadaljnje tudi kubiti) živijo v \( \mathbf{P}ℂ¹ ≅ 𝕊² \):
    % To pomeni, da lahko vsak kubit zapišemo kot točko v \( \S² \):
    \[ a = \cos{\textstyle\frac{θ}{2}}\ket0 + e^{iφ}\sin{\textstyle\frac{θ}{2}}\ket1,\quad
        φ ∈ [0, 2π), θ ∈ [0, π]. \]
\end{statement}
\begin{proof}
    Naj bo \( a = a₀\ket0 + a₁\ket1 = r₀e^{iφ₀}\ket0 + r₁e^{iφ₁}\ket1 \).
    Če označimo
    \[ r ≔ \sqrt{r₀² + r₁²}\text{, }φ ≔ φ₁ - φ₀\text{, }θ ≔ 2\arccos{\frac{r₀}{r}}, \]
    je potem \[ a = \hat a ≔ \frac{a}{re^{iφ₀}} = \frac{r₀}{r}\ket0 + \frac{r₁}{r}e^{iφ}\ket1 = \cos\frac{θ}{2}\ket0 + e^{iφ}\sin\frac{θ}{2}\ket1.\qedhere \]
\end{proof}

\begin{figure}[h]
\centering
\begin{tikzpicture}[baseline=(current bounding box.north)]
    % Define radius
    \def\r{7}
    
    % Bloch vector
    \draw  (   0,   0) node[circle, fill, inner sep=1                     ] (orig) {}
        -- (\r/6,\r/4) node[circle, fill, inner sep=0.7, label=above:\(a\)]    (a) {};
    \draw[dashed] (orig) -- (\r/6,-\r/10) node (phi) {} -- (a);
    
    % Sphere
    \draw                (orig) circle  (\r/2);
    \draw[dashed, gray]  (orig) ellipse (\r/2 and \r/6);
    
    % Axes
    \draw[->]            (orig) -- ++(-\r/10,-\r/6) node[below] (x) {\(x\)};
    \draw[->]            (orig) -- ++( \r/ 2,    0) node[right] (y) {\(y\)};
    \draw[->]            (orig) -- ++(     0, \r/2) node[above] (z) {\(z=\ket 0\)};
    \draw[->, draw=gray] (orig) -- ++(     0,-\r/2) node[below] (s) {\(\ket 1\)};
    
    %Angles
    % \draw[->, draw=gray] 
    % \draw[draw=gray,->] (orig) ellipse (\r/6 and \r/12) node {\(φ\)};
    \pic[draw=gray, ->, "\(φ\)", angle eccentricity=0.6] {angle=x--orig--phi};
    \pic[draw=gray, <-, "\(θ\)", angle eccentricity=1.4] {angle=a--orig--z};
\end{tikzpicture}
\caption{Blochova sfera}
\label{fig:bloch-sphere}
\end{figure}


\subsection{Tenzorski produkt}

\begin{definition}[Tenzorski produkt]\label{tensorprod}
    Tenzorski produkt prostorov \(\H[n]\) in \(\H[m]\) je enak \(\H[n+m]\).
    Pišemo \(\H[n]⊗\H[m]\).
    Če sta \(a ∈ \H[n]\) in \(b ∈ \H[m]\) je \(a⊗b ∈ \H[n]⊗\H[m]\).
\end{definition}
\begin{remark}
    Operator \(⊗\) je res tenzorski produkt.
\end{remark}
\begin{example}[Tenzorski produkt baznih vektorjev]  %TODO: elaborate
    \[ \ket j ⊗ \ket k = \ket{j₁\dots jₙk₁\dots kₘ} = \ket j \ket k = \ket{jk},
    \]
\end{example}
\begin{example}[Splošni tenzorski produkt]  %TODO: elaborate
    \[ \vec{a₀, a₁}⊗\vec{b₀, b₁} = \vec{a₀b₀, a₀b₁, a₁b₀, a₁b₁},\quad
       a⊗b = ∑_{\substack{j ∈ \B[n],\\k ∈ \B[m]}} aⱼbₖ\ket{jk}.
    \]
\end{example}
\begin{examples}[Tenzorski eksponent]
    \begin{align*}
        &\hh[n] = \hh^{⊗n}
        = ρⁿ\underbrace{(\ket 0 + \ket 1) ⊗ ⋯ ⊗ \p{\ket 0 + \ket 1}}_n,\\
        &\H[n] = \H^{⊗n} = \underbrace{\H ⊗ ⋯ ⊗ \H}_n.
    \end{align*}
\end{examples}

\begin{definition}
    Če lahko \(a ∈ \H[n]\) zapišemo kot \( ⨂_{j=1}^{n} aⱼ \) za neke \(aⱼ ∈ \H\) pravimo, da je enostaven ali separabilen, sicer je pa sestavljen oziroma kvantno prepleten.
\end{definition}

\subsection{Kvantne preslikave}

\begin{definition}%[Unitarna vrata]
    Prostor unitarnih vrat reda \(n\) je \( \U[n] ≔ \U{\p{2ⁿ}} \), prostor unitarnih \(2ⁿ×2ⁿ\) matrik.
    Tenzorski produkt vrat \( U⊗V ≔ [u_{jk}V]_{j,k} \) uporabljen na \(a⊗b\) je enak \(Ua⊗Vb\).
\end{definition}

\begin{example}[Tenzorski produkt unitarnih vrat]  %TODO: elaborate
    \[ \bmat{a₀₀ & a₀₁ \\ a₁₀ & a₁₁} = \bmat{a₀₀ B & a₀₁ B \\ a₁₀ B & a₁₁ B}.
    \]
\end{example}

\begin{definition}%[Bločno-diagonalna matrika]
    Za vrata \( U₀, …, Uₛ \) označimo njihovo bločno-diagnoalno matriko z \( D{\p{U₀,…,Uₛ}} \).
\end{definition}

\begin{theorem}[No cloning]\label{no-cloning}
    Ne obstajajo vrata reda \(2\), ki vsak vektor \( a⊗\ket{0} ∈ \H⊗\H \) slika v \(a⊗a\).
\end{theorem}

\begin{proof}
    Naj bo \(U\) tak, da za vsak \(a ∈ \H\) velja \( U{\p{a⊗\ket0}} = a⊗a \).\\
    Potem za \( \hh⊗\ket0 = ρ(\ket{00} + \ket{10}) \) velja:
    \[
        U{\p{ρ\p{\ket{00} + \ket{10}}}} =
        \begin{cases}
            ρ²\p{\ket{00} + \ket{01} + \ket{10} + \ket{11}},\\
            ρ U\ket{00} + U\ket{10} = ρ\p{\ket{00} + \ket{11}},
        \end{cases}
    \]
    kar je protislovje.
\end{proof}

\begin{example}[Paulijeve matrike]
    To so matrike zrcaljenja okrog osi na Blochovi sferi:
    \[ I₂ = \bmat{1 &  0 \\ 0 &  1},\quad
       X  = \bmat{0 &  1 \\ 1 &  0},\quad
       Y  = \bmat{0 & -i \\ i &  0},\quad
       Z  = \bmat{1 &  0 \\ 0 & -1}.
    \]
    Velja \(X² = Y² = Z² = I₂\).
    Preslikavi \(X\) pravimo negacija,
    saj je \( X{\ket0} = \ket1 \) in \( X{\ket1} = \ket0 \).
\end{example}

\begin{example}[Hadamardova matrika]  %TODO: elaborate
    \[ \had = ρ\bmat{1&1\\1&-1},\quad
       \had\ket{0} = \hh,\quad
       \had^{⊗n}\ket{\mathbf{0}ⁿ} = \hh[n].
    \]
\end{example}

% \[ \Qcircuit @C=1em @R=.7em {
%         & \gate{\had} & \gate{\g Z} & \gate{\had} & \qw\\
%         &             &      =      &             &    \\
%         & \qw         & \gate{\g X} & \qw         & \qw
%     }
% \]
\begin{example}[Fazni zamik]  %TODO: elaborate
    \begin{align*}
        &S_α = \bmat{1&0\\0&e^{iα}}
        \text{, posebej označimo } S ≔ S_{\newfrac{π}{2}}, T ≔ S_{\newfrac{π}{4}}, \\
        &S_α\p{a₀\ket 0 + a₁\ket1} = a₀\ket0 + a₁e^{iα}\ket1.
    \end{align*}
\end{example}

\subsection{Kvantna meritev}

V klasičnem računalništvu poznamo pogojne stavke. To lahko na kubite posplošimo na dva načina,
prvi z direktno meritvijo kubita (in uporabo klasičnih pogojnih stavkov),
drugi pa z uporabo kvantne prepletenosti.
Izkaže se, da če na koncu zmerimo kubite, se drugi način obnaša enako kot prvi.

\begin{definition}[Kvantna meritev]  %TODO: generalise
    Meritev kubita \(a = a_0\ket0 + a_1\ket1\) označimo \(M\p{a}\) in je \(0\) z verjetnostjo \(|a_0|^2\) in \(1\) z verjetnostjo \(|a_1|^2\). To \airquotes{uniči} kubit \(a\).
    % \[ \Qcircuit @C=1em @R=.7em {
    %         & \meter & \cw %& & \ctrlo{1} & \ctrl{1} & \qw & & \ctrl{1} & \qw \\
    %         % &        &     & & \gate U & \gate V & \qw & & \targ & \qw
    %     }
    % \]
\end{definition}

\begin{definition}[Kontrola]
    Za \( r,s ∈ ℕ \) in \( U ∈ \U[1] \) definiramo \( C_{r,s}{\p U} \) in \( \overline{C}_{r,s}{\p U} \) s predpisoma
    \begin{align*}
        C_{r,s}{\p U}\ket j &= \begin{cases}
            \ket j &;\quad jᵣ = 0\\
            \ket{j₁\dots}\ket{U{jₛ}}\ket{\dots jₙ} &;\quad jᵣ = 1
        \end{cases}\\
        \overline{C}_{r,s}{\p U}\ket j &= \begin{cases}
            \ket{j₁\dots}\ket{U{jₛ}}\ket{\dots jₙ} &;\quad jᵣ = 0\\
            \ket j &;\quad jᵣ = 1
        \end{cases}
    \end{align*}
    Takim vratom pravimo kontrolirana (\airquotes{na ena} in \airquotes{na nič}).
    Posebej za \( U ∈ \U[1] \) označimo
    \[ \ctl U ≔ C_{1,2}{\p U} = \D{I₂, U},\quad
        \ctlo U ≔ \overline{C}_{1,2}{\p U} = \D{U, I₂}. \]
\end{definition}

%TODO: give example

\begin{example}[Prepleteni pari kubitov]
    Kontrolirana vrata prepletejo pare kubitov. Na primer
    \( \eapply{\ctl X}{a, b} \) se obnaša kot
    \qpl[breaklines]{if |\(\emeasure{a} = 0\)| then |\(\p{a, b}\)| else |\(\p{a, \lnot b}\)|}.
    Seveda vemo, da drugi izraz ni veljaven (ker meritev uniči kubit a),
    ampak je zato kontrola ravno tisto orodje, s katerim želimo nadomestiti pogojne stavke.
\end{example}
