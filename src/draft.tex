\documentclass[a4paper]{article}
% Kodiranje in podpora slovenščini
\usepackage[T1]{fontenc}        % kodiranje znakov v .pdf
\usepackage[slovene]{babel}

\usepackage{fontspec}
\usepackage{lualatex-math}
\usepackage{unicode-math}

% \setmainfont{TeX Gyre Pagella}
% \setmathfont{TeX Gyre Pagella Math}
\setmathfont{Latin Modern Math}
\setmathfont{Asana Math}[range={scr}]
\setmathfont{STIX Two Math-Regular}[range={bb}]
% \setmathfont{Asana Math}[range={"007B,"007D}]  % {}
\setmathfont{Asana Math}[range={8709}]  % U+2205, emptyset

% \usepackage{mathtools}
\usepackage{amsthm}

\usepackage[braket]{qcircuit}
\usepackage{tikz}
\usetikzlibrary{babel}
\usetikzlibrary{angles,quotes}

\usepackage{minted}

\usepackage{stmaryrd}

\newtheorem{theorem}{Izrek}
\newtheorem{trditev}{Trditev}
\newtheorem*{corrolary}{Posledica}
\newtheorem{definition}{Definicija}
\newtheorem{zgled}{Zgled}
\newtheorem{lemma}{Lema}
\newtheorem*{example}{Primer}
\newtheorem*{opomba}{Opomba}

% \topmargin=-0.45in
% \evensidemargin=0in
% \oddsidemargin=0in
% \textwidth=6.5in
% \textheight=9.0in
% \headsep=0.25in

\linespread{1.1}

% \pagestyle{fancy}
% \lhead{\hmwkAuthorName}
% \chead{\hmwkClass:\ \hmwkTitle}
% \rhead{\firstxmark}
% \lfoot{\lastxmark}
% \cfoot{\thepage}

% \renewcommand\headrulewidth{0.4pt}
% \renewcommand\footrulewidth{0.4pt}

% \setlength\parindent{0pt}

\title{Kvantni algebrajski učinki}
% \title{Quantum Algebraic Effects}
\author{Strah}
% \mentor{doc.~dr.~Matija~Pretnar}
\date{\today}

\newcommand{\N}{\mathbb N}
\newcommand{\Z}{\mathbb Z}
\newcommand{\Q}{\mathbb Q}
\newcommand{\R}{\mathbb R}
\newcommand{\C}{\mathbb C}
\renewcommand{\d}{\;\mathrm d}
% \newcommand{\vphi}{\phi}
\renewcommand{\phi}{\varphi}
\newcommand{\eps}{\varepsilon}
\renewcommand{\hat}{\widehat}
\renewcommand{\tilde}{\widetilde}
\renewcommand{\bar}{\overline}
\newcommand{\subs}{\subseteq}
\newcommand{\nin}{\not\in}

\newcommand{\p}[1]{\left( {#1} \right)}
\newcommand{\set}[2]{\left\{ #1 \mid #2 \right\}}
\newcommand{\newfrac}[2]{{}^{#1}\!/_{\!#2}}
\newcommand{\im}[1]{\mathrm{im}{\p{#1}}}
\newcommand{\mb}[1]{\mathbold{#1}}
\newcommand{\mf}[1]{\mathfrak{#1}}
\newcommand{\mc}[1]{\mathcal{#1}}

\newcommand{\had}{\mathtt{Had}}
\newcommand{\hh}[1][]{\mathbf{h}_{#1}}
\newcommand{\B}[1][]{\mathbf{B}_{#1}}
\renewcommand{\H}[1][]{\mathbf{H}_{#1}}
\renewcommand{\S}{\mathbb{S}}
\renewcommand{\L}[2][]{\mathcal{L}_{#1}{\p{#2}}}
\newcommand{\g}[1]{\mathtt{#1}}
\newcommand{\D}[1]{D{\p{\g{#1}}}}
\newcommand{\ctl}[1]{\g{c#1}}
\newcommand{\ctlo}[1]{\g{\bar{c}#1}}

\newcommand{\cmd}[1]{\textnormal{\sffamily#1}}
\newcommand{\eff}[1]{\textnormal{\sffamily\ul{#1}}}
\newcommand{\tnew}[2]{\cmd{new}{\p{#1.#2}}}
\newcommand{\tapply}[3]{\cmd{apply}_{\g #1}{\p{#2.#3}}}
\newcommand{\tmeasure}[3]{\cmd{measure}{\p{#1; #2,#3}}}
\newcommand{\tdiscard}[2]{\cmd{discard}{\p{#1.#2}}}
\newcommand{\new}[2]{\nu{#1}.\,#2}
\newcommand{\apply}[3]{{\g #1}_{#2}{\p{#3}}}
\renewcommand{\measure}[3]{#2 {\;?_{#1}\,} #3}
\newcommand{\discard}[2]{\cmd{disc}_{#1}{\p{#2}}}

\newcommand{\sequent}[3]{#1 \mid #2 \vdash #3}
\newcommand{\seq}[1]{\sequent{\Gamma}{\Delta}{#1}}
\newcommand{\sem}[2][]{{\llbracket#2\rrbracket}_{#1}}


% \newcommand{\for}[2]{\forall#1.\;#2}
% \newcommand{\exist}[2]{\exists#1\smallni:\;#2}
% \newcommand{\existi}[2]{\exists!#1\smallni:\;#2}

\makeatletter
\newcommand{\oset}[3][0ex]{%
  \mathrel{\mathop{#3}\limits^{
    \vbox to#1{\kern-2\ex@
    \hbox{$\scriptstyle#2$}\vss}}}}
\makeatother

\begin{document}
\maketitle

\begin{abstract}
  test
\end{abstract}

\section{Kvantno računalništvo ter algebrajski učinki in diagrami}
\subsection{Kvantna mehanika}

V tem delu bomo uporabljali naslednje oznake:
\begin{itemize}
    \item \( \N = \{0, \dots \}, \N_+ = \{1, \dots\} \),
    \item \( n\in\N_+ \), ki mu bomo pravili število kubitov,
    \item \( j, k, \dots \in \{0, \dots, 2^n\} \),
    \item \( j = j_1\dots j_n \) binarni zapis števila \( j \).
\end{itemize}

\subsubsection{Kvantni vektorji}

\begin{definition}
    Binarni vektorji so elementi prostora \( \B[n] \coloneq 2^n \) in jih pišemo kot nize v binarnem zapisu.
\end{definition}

\begin{example}
    \(\B[2] = \{00, 01, 10, 11\}\).
\end{example}

\begin{definition}
    Elementom prostora \( \H[n] \coloneq \C^{2^n} \) pravimo kvantni vektorji, elementom \( \H \coloneq \H[1] \) pa kubiti.  Prostoru \( \H[n] \) pravimo prostor kvantnih vektorjev reda \( n \), njegovo standardno bazo pa označimo z \( \{e_j\} \).
\end{definition}

\begin{definition}%[Braket notacija]
    Naj bo \( j \in \{0, \dots, 2^n-1\} \), ter \( \hat{\jmath} \in \B[n] \) pripadajoč vektor v binarnem zapisu. Potem je \( \ket j = \ket{\hat{\jmath}} \coloneq e_j \).
\end{definition}
\begin{opomba}
    Po definiciji je torej \( \H[n] = \L[\C]{\set{\ket j}{j\in\B[n]}} \).
\end{opomba}

\begin{example}[\(n = 1\)]
    \[ 
        a = \begin{bmatrix}a_0 \\ a_1\end{bmatrix}
            = a_0\ket 0 + a_1\ket 1
            = a_0\begin{bmatrix}1 \\ 0\end{bmatrix} + a_1\begin{bmatrix}0 \\ 1\end{bmatrix}\text.\qedhere
    \]
\end{example}

\begin{example}[\(n = 2\)]
    \[
        a = \begin{bmatrix}a_0\\ a_1\\ a_2\\ a_3\end{bmatrix}
            = \begin{bmatrix}a_{00}\\ a_{01}\\ a_{10}\\ a_{11}\end{bmatrix}
            = a_{00}\ket{00} + a_{01}\ket{01} + a_{10}\ket{10} + a_{11}\ket{11}\text.\qedhere
    \]
\end{example}

\begin{example}%[Hadamardov vektor] % maybe skip this one?
    \[
        \hh \coloneq \rho \left(\ket 0 + \ket 1\right),\quad
        \hh[n] \coloneq \rho^n\sum_{j\in\B[n]} \ket j,\quad
        \rho \coloneq \frac{1}{\sqrt2}\text.\qedhere
        \]
\end{example}

\subsubsection{Tenzorski produkt}

\begin{definition}[Tenzorski produkt]
    Tenzorski produkt prostorov \( \H[n] \) in \( \H[m] \) je enak \( \H[n+m] \).
    Pišemo \( \H[n]\otimes\H[m] \). Če sta \( a\in\H[n] \) in \( b\in\H[m] \) je \( a\otimes b \in \H[n]\otimes\H[m] \).
\end{definition}
\begin{opomba}
    Operator \(\otimes\) je res tenzorski produkt.
\end{opomba}

\begin{example}[\(n=m=1\)]
    \[
        \begin{bmatrix}a_0 \\ a_1\end{bmatrix}\otimes \begin{bmatrix}b_0 \\ b_1\end{bmatrix}
        = \begin{bmatrix}a_0b_0\\ a_0b_1\\ a_1b_0\\ a_1b_1\end{bmatrix}\text.\qedhere
    \]
\end{example}

\begin{example}
    \[
        \ket j \otimes \ket k = \ket{j\#k} \eqcolon \ket{jk},\quad
        a\otimes b = \sum_{\substack{j\in\B[n],\\k\in\B[m]}} a_jb_k\ket{jk}.
    \]
\end{example}

\begin{example}
    \[
        \hh[n] = \hh^{\otimes n} = \rho^n\underbrace{(\ket 0 + \ket 1)\otimes\dots\otimes(\ket 0 + \ket 1)}_n\text.\qedhere
    \]
\end{example}
\begin{example}
    \[\H[n] = \H^{\otimes n}\text.\qedhere\]
\end{example}

\begin{definition}
    Če lahko \(a\in\H[n]\) zapišemo kot \(\bigotimes_{j=1}^{n} a_j; a_j\in\H\) pravimo, da je enostaven ali separabilen, sicer je pa sestavljen oziroma kvantno prepleten.
\end{definition}


\begin{definition}%[Unitarna vrata]
    Unitarna vrata reda \( n \) so unitarna matrika dimenzije \( 2^n \).
    Tenzorski produkt vrat \( U \otimes V = [u_{jk}V]_{j,k} \) uporabljen na \(a\otimes b\) je enak \( Ua \otimes Vb \).
\end{definition}
\begin{example}
    \[
        \begin{bmatrix}
            a_{00} & a_{01} \\ a_{10} & a_{11}
        \end{bmatrix}
        \otimes B
        % \begin{bmatrix}
        %     b_{00} & b_{01} \\ b_{10} & b_{11}
        % \end{bmatrix}
        =
        \begin{bmatrix}
            a_{00} B & a_{01} B \\ a_{10} B & a_{11} B
        \end{bmatrix}\text.\qedhere\]
        % \begin{bmatrix}
        %     a_{00}b_{00} & a_{01}b_{00} & a_{00}b_{01} & a_{01}b_{01} \\
        %     a_{10}b_{00} & a_{11}b_{00} & a_{10}b_{01} & a_{11}b_{01} \\
        %     a_{00}b_{10} & a_{01}b_{10} & a_{00}b_{11} & a_{01}b_{11} \\
        %     a_{10}b_{10} & a_{11}b_{10} & a_{10}b_{11} & a_{11}b_{11}
        % \end{bmatrix}
    % \]
\end{example}
\begin{definition}%[Bločno-diagonalna matrika]
    Za vrata \( U_0,\dots,U_n \) označimo njihovo bločno-diagnoalno matriko z \( D{\p{U_0,\dots,U_n}} \).
\end{definition}
% TODO: QPL, vezje

\begin{theorem}[No cloning]
    Ne obstaja unitarna matrika (vrata reda \(2\)), ki vsak vektor \(a\otimes\ket{0}\in \H\otimes\H\) slika v \(a\otimes a\).
\end{theorem}

\begin{proof}
    Naj bo \(U\) tak, da \(\forall a \in \H \) velja \( U{\p{a\otimes\ket0}} = a\otimes a \).
    Potem za \( \hh\otimes\ket0 = \rho(\ket{00} + \ket{10}) \) velja, da je \( U{\p{\rho(\ket{00} + \ket{10})}} = \rho^2(\ket{00} + \ket{01} + \ket{10} + \ket{11}) \), oziroma \(U\ket{00} + U\ket{10} = \rho(\ket{00} + \ket{01} + \ket{10} + \ket{11})\)
\end{proof}

\end{document}
