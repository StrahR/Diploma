\documentclass[a4paper,slovene]{article}
\usepackage[slovene]{babel}

% LTeX: enabled=false
\renewcommand{\d}{\;\mathrm d}
\renewcommand{\hat}{\widehat}
\renewcommand{\tilde}{\widetilde}
% \renewcommand{\bar}{\overline}
\newcommand{\subs}{\subseteq}
\newcommand{\nin}{\not\in}
\newcommand{\contradiction}{\,\,\lightning}

\newcommand{\p}[1]{\left( {#1} \right)}
\newcommand{\set}[2]{\left\{ #1 \mid #2 \right\}}
\newcommand{\newfrac}[2]{{}^{#1}\!/_{\!#2}}
\newcommand{\im}[1]{\mathrm{im}{\p{#1}}}
\newcommand{\mb}[1]{\mathbf{#1}}
\newcommand{\mf}[1]{\mathfrak{#1}}
\newcommand{\mc}[1]{\mathcal{#1}}
\newcommand{\id}{\mathrm{id}}
\newcommand{\cat}[1]{\mathbf{#1}}
\newcommand{\tr}[1]{\textrm{tr}{#1}}
\newcommand{\rang}[1]{\textrm{rang}{\p{#1}}}

\usepackage{xstring}
\newcommand{\mat}[1]{\begin{matrix} #1 \end{matrix}}
\newcommand{\pmat}[1]{\left(\mat{#1}\right)}
\newcommand{\bmat}[1]{\left[\mat{#1}\right]}
\renewcommand{\vec}[1]{\bmat{\StrSubstitute[0]{#1}{,}{\\}}}
\newcommand{\state}[1]{\left\{ {#1} \right\}}

\newcommand{\had}{\mathtt{Had}}
\newcommand{\swap}{\mathtt{swap}}
\newcommand{\hh}[1][]{\mathbf{h}_{#1}}
\newcommand{\B}[1][]{\mathbf{B}_{#1}}
\newcommand{\M}[1][]{\mathbf{M}_{#1}}
\renewcommand{\H}[1][]{\mathbf{H}_{#1}}
\renewcommand{\L}[2][]{\mathcal{L}_{#1}{\p{#2}}}
\newcommand{\g}[1]{\mathtt{#1}}
\newcommand{\D}[1]{D{\p{\g{#1}}}}
\newcommand{\U}[1][]{\mathbf{U}_{#1}}
\newcommand{\ctl}[1]{\g{c#1}}
\newcommand{\ctlo}[1]{\g{\bar c} \g #1}
\newcommand{\cstarcat}{\cat{Cstar}}
\newcommand{\cstarcpucat}{\cat{Cstar}_{\textrm{CPU}}}
\newcommand{\cstar}[2]{\cat{Cstar}{\arity{#1}{#2}}}
\newcommand{\brat}[2]{\cat{Brat}{\arity{#1}{#2}}}

\newcommand{\op}[1]{\textnormal{\sffamily#1}}
\newcommand{\mor}[1]{\textnormal{\sffamily#1}}
\newcommand{\eff}[1]{\textnormal{\sffamily\ul{#1}}}
\newcommand{\enew}{\eff{new}{\p{}}}
% \newcommand{\enew}[1]{\eff{new}{\p{#1}}}
\newcommand{\eapply}[2]{\eff{apply}_{\g{#1}}{\p{#2}}}
\newcommand{\emeasure}[1]{\eff{measure}{\p{#1}}}
\newcommand{\ediscard}[1]{\eff{discard}{\p{#1}}}
\newcommand{\tnew}[2]{\op{new}{\p{#1.\,#2}}}
\newcommand{\tapply}[3]{\op{apply}_{\g{#1}}{\p{#2;#2.\,#3}}}
\newcommand{\tapplyd}[3]{\op{apply}_{\D{#1}}{\p{#2;#2.\,#3}}}
\newcommand{\tmeasure}[3]{\op{measure}{\p{#1;\,#2,#3}}}
\newcommand{\tdiscard}[2]{\op{discard}{\p{#1;\,#2}}}
\newcommand{\new}[2]{\nu{#1}.\,#2}
\newcommand{\apply}[3]{{\g{#1}}_{#2}{\p{#3}}}
\newcommand{\applyd}[3]{{\D{#1}}_{#2}{\p{#3}}}
\renewcommand{\measure}[3]{\p{#2} {\;?_{#1}\,} \p{#3}}
\newcommand{\measureraw}[3]{#2 {\;?_{#1}\,} #3}
\newcommand{\discard}[2]{\op{disc}_{#1}{\p{#2}}}
\newcommand{\snew}[2]{\mor{new}#1{\p{#2}}}
\newcommand{\sapply}[2]{\mor{apply}_{\g{#1}}{\p{#2}}}
\newcommand{\smeasure}[3]{\mor{measure}#1{\p{#2,#3}}}

\newcommand{\type}[1]{\text{\ttfamily#1}}
\newcommand{\unit}{\type{I}}
\newcommand{\bit}{\type{bit}}
\newcommand{\qbit}{\type{qbit}}

\newcommand{\arity}[2]{\p{#1 \mid #2}}
% \newcommand{\arity}[2]{\p{#1 \mid \StrSubstitute[0]{#2}{,}{\ }}}

\newcommand{\sequent}[3]{#1 \mid #2 ⊢ #3}
\newcommand{\sseq}[2]{#1 ⊢ #2}
\newcommand{\seq}[1]{\sequent{x₁:m₁,…,xₖ:mₖ}{a₁,…,aₚ}{#1}}
\newcommand{\absseq}[1]{\sequent{\Gamma}{\Delta}{#1}}
\newcommand{\sem}[1]{\left⟦ #1 \right⟧}
% \newcommand{\brsem}[1]{⟬ #1 ⟭}
% \newcommand{\brsem}[1]{⦅ #1 ⦆}
\newcommand{\brsem}[1]{\left⦇ #1 \right⦈}
\usepackage[nomessages]{fp}
\newcommand{\mormap}[2]{% cf. https://tex.stackexchange.com/a/87423/64454
  \def\nextitem{\def\nextitem{⊕}}% Separator
  \forcsvlist\mormapitem{#2} → M_{\IfInteger{#1}{\FPeval\result{round(2^(#1):0)}\result}{2^{#1}}}{\p X}
}
\newcommand{\mormapitem}[1]{%
  \nextitem
  \def\param{\detokenize{#1}}%
  \if\param…%
    ⋯%
  \else%
    M_{\IfInteger{\param}{\FPeval\result{round(2^(\param):0)}\result}{2^{\param}}}{\p X}\!%
  \fi
}
\newcommand{\semmap}[2]{% cf. https://tex.stackexchange.com/a/87423/64454
  \def\nextitem{\def\nextitem{⊕}}% Separator
  \forcsvlist\semmapitem{#2} → \M[\IfInteger{#1}{\FPeval\result{round(2^(#1):0)}\result}{2^{#1}}]
}
\newcommand{\semmapitem}[1]{%
  \nextitem
  \def\param{\detokenize{#1}}%
  \if\param…%
    ⋯%
  \else%
    \M[\IfInteger{\param}{\FPeval\result{round(2^(\param):0)}\result}{2^{\param}}]\!%
  \fi
}
\newcommand{\abssemmap}[2]{% cf. https://tex.stackexchange.com/a/87423/64454
  \def\nextitem{\def\nextitem{×}}% Separator
  \forcsvlist\abssemmapitem{#2} → #1 • X
}
\newcommand{\abssemmapitem}[1]{%
  \nextitem
  \def\param{\detokenize{#1}}%
  \if\param…%
    ⋯%
  \else%
    (\param • X)\!%
  \fi
}
\newcommand{\semhom}[2]{
  \def\nextitem{\def\nextitem{⊕}}% Separator
  \cstar{\p{\forcsvlist\semhomitem{#2}, \M[\IfInteger{#1}{\FPeval\result{round(2^(#1):0)}\result}{2^{#1}}]}}
}
\newcommand{\semhomitem}[1]{%
  \nextitem
  \def\param{\detokenize{#1}}%
  \if\param…%
    ⋯%
  \else%
    \M[\IfInteger{\param}{\FPeval\result{round(2^(\param):0)}\result}{2^{\param}}]\!%
  \fi
}

\newcommand{\airquotes}[1]{\enquote{#1}}

\newminted[qplcode]{ocaml}{escapeinside=||, autogobble}
\newmintinline[qpl]{ocaml}{escapeinside=||, autogobble}

\makeatletter
\newcommand{\oset}[3][0ex]{%
  \mathrel{\mathop{#3}\limits^{
    \vbox to#1{\kern-2\ex@
    \hbox{$\scriptstyle#2$}\vss}}}}
\makeatother

\makeatletter
\newcommand{\listintertext}{\@ifstar\listintertext@\listintertext@@}
\newcommand{\listintertext@}[1]{% \listintertext*{#1}
  \hspace*{-\@totalleftmargin}#1}
\newcommand{\listintertext@@}[1]{% \listintertext{#1}
  \hspace{-\leftmargin}#1}
\makeatother

\renewcommand{\for}[2]{\forall#1.\;#2}
\newcommand{\exist}[2]{\exists#1\smallni:\;#2}
\newcommand{\existi}[2]{\exists!#1\smallni:\;#2}

\title{Kvantni algebrajski učinki}
% \title{Quantum Algebraic Effects}
\author{Strah}
% \mentor{doc.~dr.~Matija~Pretnar}
\date{\today}

\addbibresource{draft.bib}

\renewcommand{\thesection}{\Roman{section}}

\begin{document}
\maketitle

\begin{abstract}
    \subsubsection*{Motivacija} % TODO: Reword

    Kvantni programski jeziki predstavljajo nove probleme za teorijo programskih jezikov,
    kot so linearnost tipov, (kvantni) fizikalni pojavi, in še mnogi drugi.
    V tej nalogi se bomo posvetili tema dvema.
    % Dober način razumevanja je, da razumemo enakost programov.
    Naš cilj je razumeti, kako se kvantne programe,
    in dober način je razumevanje enakosti programov,
    tj. kdaj sta dva programa enaka?

    \subsubsection*{Pregled}

    Najprej bomo predstavili algebrajsko teorijo za kvantne programe;
    ta je zgrajena na unitarnih vratih in meritvah ter ima linearne parametre.
    Nato bomo dokazali, da lahko s to teorijo predstavimo vse programe (polnost)
    in nato iz nje izpeljali pravila za enakost kvantnih programov.

\end{abstract}

\section{Kvantno računalništvo ter algebrajski učinki in diagrami}

\subsection{Kvantna mehanika}
Ta del je povzet po \cite{ess-qc}.

\subsubsection*{Oznake}
Skozi ta del bomo uporabljali naslednje oznake:
\begin{itemize}
    \item \( ℕ = \{ 0, \dots \}, ℕ_+ = \{ 1, \dots \} \),
    \item \( n \in ℕ_+ \), ki mu bomo pravili število kubitov,
    \item \( j, k, \dots \in \{ 0, \dots, 2ⁿ - 1 \} \),
    \item \( j = j₁ \dots jₙ \) binarni zapis števila \( j \),
    \item \( \mb n = \{ 0, \dots, n-1 \} \), na primer \( \mb 2 = \{ 0, 1 \} \).
    % \item \( \mathbb n = \{ 0, \dots, n-1 \} \), na primer \( \mathbb 2 = \{ 0, 1 \} \).
\end{itemize}

\subsubsection{Kvantni vektorji}
\begin{definition}\label{binv}
    Binarni vektorji so elementi prostora \( \B[n] ≔ \mb 2ⁿ \) in jih pišemo kot nize v binarnem zapisu.
\end{definition}

\begin{example}
    \(\B[2] = \{00, 01, 10, 11\}\).
\end{example}
\begin{remark}
    \(1\) in \(01\) predstavljata različna vektorja.
\end{remark}

\begin{definition}[Hilbertov prostor]\label{hilb-sp}
    Elementom prostora \( \H[n] ≔ \C^{2ⁿ} \) pravimo kvantni vektorji, elementom \( \H ≔ \H[1] \) pa kubiti.  Prostoru \( \H[n] \) torej pravimo prostor kvantnih vektorjev reda \( n \), njegovo standardno bazo pa označimo z \( \{eⱼ\} \).
\end{definition}

\begin{definition}[Braket notacija]\label{braket}
    Naj bo \( j \in \{0, \dots, 2ⁿ-1\} \), ter \( \hat{\jmath} \in \B[n] \) pripadajoč vektor v binarnem zapisu. Potem je \( \ket j = \ket{\hat{\jmath}} ≔ eⱼ \).
\end{definition}
\begin{remark}
    Po definiciji je torej \( \H[n] = \L[\C]{\set{\ket j}{j\in\B[n]}} \).
\end{remark}

\begin{example}[\(n = 1\)]
    \[ 
        a = \begin{bmatrix}a₀\\a₁\end{bmatrix}
          = a₀\begin{bmatrix}1\\0\end{bmatrix} + a₁\begin{bmatrix}0\\1\end{bmatrix}
          = a₀\ket 0 + a₁\ket 1
        \text.
    \]
\end{example}

\begin{example}[\(n = 2\)]
    \[
        a = \begin{bmatrix}a₀\\a₁\\a₂\\a₃\end{bmatrix}
          = \begin{bmatrix}a₀₀\\ a₀₁\\ a₁₀\\ a₁₁\end{bmatrix}
          = a₀₀\ket{00} + a₀₁\ket{01} + a₁₀\ket{10} + a₁₁\ket{11}.
    \]
\end{example}

\begin{example}[Hadamardov vektor]\label{had}
    \[
        \hh ≔ ρ \left(\ket 0 + \ket 1\right),\quad
        \hh[n] ≔ ρⁿ\sum_{j\in\B[n]} \ket j,\quad
        ρ ≔ \frac{1}{\sqrt2}.
        \]
\end{example}

\subsubsection{Tenzorski produkt}

\begin{definition}[Tenzorski produkt]\label{tensorprod}
    Tenzorski produkt prostorov \( \H[n] \) in \( \H[m] \) je enak \( \H[n+m] \).
    Pišemo \( \H[n] ⊗\H[m] \).
    Če sta \( a\in\H[n] \) in \( b\in\H[m] \) je \( a⊗b \in \H[n]⊗\H[m] \).
\end{definition}
\begin{remark}
    Operator \(⊗\) je res tenzorski produkt.
\end{remark}

\begin{example}[\(n=m=1\)]
    \[
        \begin{bmatrix}a₀ \\ a₁\end{bmatrix}⊗\begin{bmatrix}b₀ \\ b₁\end{bmatrix}
        = \begin{bmatrix}a₀b₀\\ a₀b₁\\ a₁b₀\\ a₁b₁\end{bmatrix}.
    \]
\end{example}

\begin{example}
    \[
        \ket j ⊗ \ket k = \ket{j\#k} ≕ \ket j \ket k,\quad
        a⊗b = \sum_{\substack{j\in\B[n],\\k\in\B[m]}} aⱼbₖ\ket{jk}.
    \]
\end{example}

\begin{example}[Hadamardov vektor kot tenzorski produkt]
    \[
        \hh[n] = \hh^{⊗n} = ρⁿ\underbrace{(\ket 0 + \ket 1) ⊗ \dots ⊗ \p{\ket 0 + \ket 1}}_n.
    \]
\end{example}
\begin{example}[Hilbertov prostor kot tenzorski produkt]
    \[\H[n] = \H^{⊗n}.\]
\end{example}

\begin{definition}
    Če lahko \( a\in\H[n] \) zapišemo kot \( \bigotimes_{j=1}^{n} aⱼ \) za neke \( aⱼ\in\H \) pravimo, da je enostaven ali separabilen, sicer je pa sestavljen oziroma kvantno prepleten.
\end{definition}

\subsubsection{Kvantne preslikave/Unitarna vrata}%TODO: title

\begin{definition}%[Unitarna vrata]
    Unitarna vrata reda \( n \) so unitarna matrika dimenzije \( 2ⁿ \).
    Tenzorski produkt vrat \( U⊗V ≔ [u_{jk}V]_{j,k} \) uporabljen na \( a⊗b \) je enak \( Ua⊗Vb \).
\end{definition}

\begin{example}[Tenzorski produkt vrat]
    \[
        \begin{bmatrix}
            a₀₀ & a₀₁ \\ a₁₀ & a₁₁
        \end{bmatrix} ⊗ B
        % \begin{bmatrix}
        %     b₀₀ & b₀₁ \\ b₁₀ & b₁₁
        % \end{bmatrix}
        =
        \begin{bmatrix}
            a₀₀ B & a₀₁ B \\ a₁₀ B & a₁₁ B
        \end{bmatrix}.
    \]
        % \begin{bmatrix}
        %     a₀₀b₀₀ & a₀₁b₀₀ & a₀₀b₀₁ & a₀₁b₀₁ \\
        %     a₁₀b₀₀ & a₁₁b₀₀ & a₁₀b₀₁ & a₁₁b₀₁ \\
        %     a₀₀b₁₀ & a₀₁b₁₀ & a₀₀b₁₁ & a₀₁b₁₁ \\
        %     a₁₀b₁₀ & a₁₁b₁₀ & a₁₀b₁₁ & a₁₁b₁₁
        % \end{bmatrix}
    % \]
\end{example}

\begin{definition}%[Bločno-diagonalna matrika]
    Za vrata \( U₀,\dots,Uₙ \) označimo njihovo bločno-diagnoalno matriko z \( D{\p{U₀,\dots,Uₙ}} \).
\end{definition}
% TODO: QPL, vezje

\begin{theorem}[No cloning]\label{no-cloning}
    Ne obstaja unitarna matrika (vrata reda \(2\)), ki vsak vektor \(a ⊗\ket{0}\in \H ⊗\H\) slika v \(a⊗a\).
\end{theorem}

\begin{proof}
    Naj bo \(U\) tak, da \(\forall a \in \H \) velja \( U{\p{a⊗\ket0}} = a⊗a \).\\
    Potem za \( \hh ⊗\ket0 = ρ(\ket{00} + \ket{10}) \) velja:
    \[
        U{\p{ρ(\ket{00} + \ket{10})}} =
        \begin{cases}
            ρ²(\ket{00} + \ket{01} + \ket{10} + \ket{11}),\\
            ρ U\ket{00} + U\ket{10} = ρ(\ket{00} + \ket{11}),
        \end{cases}
    \]
    kar je protislovje.
\end{proof}

\begin{example}[Paulijeve matrike]
    To so matrike zrcaljenja okrog osi na Blochovi sferi:
    \[
        I₂ = \begin{bmatrix}1&0\\0&1\end{bmatrix}\text{,\quad}
        X   = \begin{bmatrix}0&1\\1&0\end{bmatrix}\text{,\quad}
        Y   = \begin{bmatrix}0&-i\\i&0\end{bmatrix}\text{,\quad}
        Z   = \begin{bmatrix}1&0\\0&-1\end{bmatrix}\text.
    \]
    Velja \(X² = Y² = Z² = I₂\).
\end{example}

\begin{example}[Hadamardova matrika]
    \[
        \had = ρ\begin{bmatrix}1&1\\1&-1\end{bmatrix}\text{,\quad}
        \had(\ket{0}) = \hh\text{,\quad}
        \had^{⊗n}(\ket{0ⁿ}) = \hh[n].
    \]
\end{example}

\[ \Qcircuit @C=1em @R=.7em {
        & \gate{\had} & \gate{\g Z} & \gate{\had} & \qw\\
        &             &      =      &             &    \\
        & \qw         & \gate{\g X} & \qw         & \qw
    }
\]
\begin{example}[Fazni zamik]
    \begin{align*}
        S_α = \begin{bmatrix}1&0\\0&e^{iα}\end{bmatrix}
        \text{, posebej označimo } S ≔ S_{\newfrac{π}{2}}\text{: }
        S² = X\\
        S_α\p{a₀\ket 0 + a₁\ket1} = a₀\ket0 + a₁e^{iα}\ket1.
    \end{align*}
\end{example}

\subsubsection{Kvantna meritev}

\begin{definition}[Kvantna meritev]
    Meritev kubita \(a = a_0\ket0 + a_1\ket1\) označimo \(M\p{a}\) in je \(0\) z verjetnostjo \(|a_0|^2\) in \(1\) z verjetnostjo \(|a_1|^2\). To "`uniči"' kubit \(a\).
    \[ \Qcircuit @C=1em @R=.7em {
            & \meter & \cw %& & \ctrlo{1} & \ctrl{1} & \qw & & \ctrl{1} & \qw \\
            % &        &     & & \gate U & \gate V & \qw & & \targ & \qw
        }
    \]
\end{definition}
% \begin{definition}[Kontrola]
%     Kontrolirana vrata so …
% \end{definition}

% TODO: QPL, vezje
\begin{example}[Projekcija na Z os]
    \[ \Qcircuit @C=1em @R=.7em {
            & \lstick{\ket0} & \gate{\g X} & \rstick{b} \qw\\
            \lstick{a} & \meter & \cctrl{-1}
        }
    \]
\end{example}

\begin{example}[Naključna rotacija faze]
    \[ \Qcircuit @C=1em @R=.7em {
            \lstick{a} & \qw & \qw & \qw & \qw & \gate{\g Z} & \rstick{a}\qw\\
            && \lstick{\ket0} & \gate{\had} & \meter & \cctrl{-1}
        }
    \]
\end{example}
% \[ \Qcircuit @C=1em @R=.7em {
%         & \lstick{\ket0} & \gate{\g X} & \rstick{b} \qw &&&&
%         \lstick{a} & \qw & \qw & \qw & \qw & \gate{\g Z} & \rstick{a}\qw\\
%         \lstick{a} & \meter & \cctrl{-1} &&&&
%         &&& \lstick{\ket0} & \gate{\had} & \meter & \cctrl{-1}
%     }
% \]

% \frame{
%     \frametitle[Vezja]{Grafični prikaz}

%     Grafično predstavimo kot kvantna vezja:
%     \[ \Qcircuit @C=1em @R=.7em {
%             & \gate{\g U} & \qw & & \meter & \cw \\
%             & \ctrl{1} & \qw & & \ctrlo{1} & \ctrl{1} & \qw \\
%             & \targ & \qw & & \gate{\g U} & \gate{\g V} & \qw
%         }
%     \]
% }

\subsection{Algebrajski učinki}

\begin{definition}[Računski učinki]
    Če ima funkcija ali operacija še kak navzven viden učinek poleg vrnjene vrednosti temu pravimo računski učinek (učinek računanja).
\end{definition}

\begin{definition}[Algebrajski učinki]
    So računski učinki, ki jih lahko predstavimo z algebrajsko teorijo.
\end{definition}

\subsection{Kvantno računalništvo}

Kaj sploh imamo?
\begin{itemize}
    \item Tip kubitov \texttt{qubit}. Funkcije dostopanja:
    \begin{itemize}
        \item \(\eff{new}\): dodeli nov kubit, z začetno vrednostjo \(\ket 0\).
        \item \(\eff{apply}_{\g U}\): Uporabi vrata \(U\) na danem kubitu.
        \item \(\eff{measure}\): izvede meritev na kubitu, vrne element tipa \texttt{bit}.
    \end{itemize}
    \item Za tipa \(A\) in \(B\) obstaja tip \(A\otimes B\) prepletenih parov.
\end{itemize}
\begin{example}[Prepleteni pari kubitov]  % TODO: Kontrolirana vrata
    Obstajajo ti. kontrolirana vrata, na primer \(\ctl X\).
    \(\eapply{\ctl X}{a, b}\) se obnaša kot
    \qpl{if |\(\emeasure{a} = 0\)| then |\( \p{a, b} \)| else |\( \p{a, ¬b} \)|}
\end{example}

\subsubsection{Pretvorba v algebrajske izraze}

\begin{table}[ht]
    % TODO: mogoče drug vrstni red?
    \centering
    \begin{tabular}{|l|l|l|}
        \hline
        Kvantni programski jezik  & Algebrajski izrazi       & Matematični simboli        \\
        \hline
        \qpl{let |\( a \leftarrow \enew \)| in |\( t \)|}
                                  & \( \tnew{a}{t} \)        & \( \new{a}{t} \)           \\
        \qpl{|\( \eapply{\g{U}}{a} \)|; |\( x\p{a} \)|}
                                  & \( \tapply{U}{a}{t} \)   & \( \apply{U}{a}{t} \)      \\
        \qpl{if |\( \emeasure{a} = 0 \)| then |\( t \)| else |\( u \)|}
                                  & \( \tmeasure{a}{t}{u} \) & \( \measureraw{a}{t}{u} \) \\
        \qpl{if |\( \emeasure{a} = 0 \)| then |\( t \)| else |\( t \)|}
        % \qpl{|\( \ediscard{a}; t \)|}
                                  & \( \tdiscard{a}{t} \)    & \( \discard{a}{t} \)       \\
        \hline
    \end{tabular}
\end{table}

\begin{example}[Projekcija na \( Z \)-os]\(\)
    \begin{enumerate}
        \item \qpl{if |\(\eff{measure}\p{a} = 0\)| then |\( \enew \)| else |\( \eapply{\g{X}}{\enew} \)|}
        \item \( \tmeasure{a}{\tnew{b}{x\p{b}}}{\tnew{b}{\tapply{X}{b}{x\p{b}}}} \)
        \item \(
            \measure{b}
                {\new{a}              {x\p{a}}}
                {\new{a} {\apply{X}{a}{x\p{a}}}
            }\)
    \end{enumerate}
\end{example}

\begin{example}[Naključna rotacija faze]\(\)
    \begin{enumerate}
        \item \qpl{if |\( \emeasure{\eapply{\had}{\enew}} = 0 \)| then |\( a \)| else |\( \eapply{\g{Z}}{a} \)|}
        \item \( \tnew{b}{\tapply{\had}{b}{\tmeasure{b}{x\p{a}}{\tapply{Z}{a}{x\p{a}}}}} \)
        \item \(
            \new{a}{ \apply{\had}{a}{
                \measureraw{a}
                    {x\p{b}}
                    {\apply{Z}{b}{x\p{b}}}
            }}\)
    \end{enumerate}
\end{example}

\subsubsection{Aksiomi}

Osnovni aksiomi na kratko:
\begin{gather}
\intertext{Kvantna negacija pred meritvijo je negacija po meritvi.}
    \apply{X}{a}{\measureraw{a}{x}{y}} = \measureraw{a}{y}{x}\label{ax-1}
\intertext{Kvantna kontrola je po meritvi kot klasična kontrola.}
    \measureraw{a}{\apply{U}{b}{x(b)}}{\apply{V}{b}{y(b)}}
        = \applyd{U, V}{a,b}{\measureraw{a}{x(b)}{y(b)}}\label{ax-2}
\intertext{Kvantna vrata uporabljena na zavrženih kubitih so odveč.}
    \apply{U}{a}{\discard{a}{t}} = \discard{a}{t}\label{ax-3}
\intertext{Meritve novih kubitov so vedno \(0\).}
    \new{a}{\measureraw{a}{x}{y}} = x\label{ax-4}
\intertext{Vrata kontrolirana z novimi kubiti se nikoli ne uporabijo.}
    \new{a}{\applyd{U,V}{a,b}{x{\p{a,b}}}}
        = \apply{U}{b}{\new{a}{x{\p{a,b}}}}\label{ax-5}
\intertext{Ostanejo še bolj "`administrativni"' aksiomi.}
\intertext{Spoštovanje simetrične grupe.}
    \apply{swap}{a,b}{x{\p{a,b}}} = x(b,a)\label{ax-6}\\
    \apply{I}{a}{x{\p a}} = x{\p a}\label{ax-7}\\
    \apply{UV}{a}{x{\p a}} = \apply{U}{a}{\apply{V}{a}{x{\p a}}}\label{ax-8}\\
    \apply{U⊗V}{a,b}{x{\p{a,b}}} = \apply{U}{a}{\apply{V}{b}{x{\p{a,b}}}}\label{ax-9}
\intertext{Komutativnost.}
    % \apply{U}{a}{\apply{V}{b}{x{\p{a,b}}}} = \apply{V}{b}{\apply{U}{a}{x{\p{a,b}}}}\\
    \measure{a}{\measureraw{b}{u}{v}}{\measureraw{b}{x}{y}}
        = \measure{b}{\measureraw{a}{u}{x}}{\measureraw{a}{v}{y}}\label{ax-10}\\
    \new{a}{\new{b}{x{\p{a,b}}}} = \new{b}{\new{a}{x{\p{a,b}}}}\label{ax-11}\\
    \new{a}{\measureraw{b}{x{\p a}}{y{\p a}}}
        = \measure{b}{\new{a}{x{\p a}}}{\new{a}{y{\p a}}}\label{ax-12}
\end{gather}

% \apply{X}{a}{\apply{\H}{b}{x{\p{a,b}}}} = \apply{\H}{b}{\apply{X}{a}{x{\p{a,b}}}}
\begin{example}
    \begin{align*}
        &\hspace{-3em}\measure{b}{\new{a}{x\p{a}}}{\new{a}{\apply{X}{a}{x\p{a}}}}\\
        =&\new{a}{\measureraw{b}{x\p{a}}{\apply{X}{a}{x\p{a}}}}&{(\ref{ax-12})}\\
        =&\new{a}{\apply{\ctl X}{b,a}{\measureraw{b}{x\p{a}}{x\p{a}}}}&{(\ref{ax-2})}\\
        =&\new{a}{\apply{\ctl X}{b,a}{\discard{b}{x\p{a}}}}&{(\text{def.})}\\
        =&\new{a}{\apply{\ctl X}{a,b}{\apply{\ctl X}{b,a}{\discard{a}{x\p{b}}}}}&\\
        =&\new{a}{\apply{\ctl X}{b,a}{\discard{a}{x\p{b}}}}&{(\ref{ax-5})}\\
        =&\new{a}{\apply{\had}{a}{\apply{\ctl Z}{a,b}{\apply{\had}{a}{\discard{a}{x\p{b}}}}}}&\\
        =&\new{a}{\apply{\had}{a}{\apply{\ctl Z}{a,b}{\discard{a}{x\p{b}}}}}&{(\ref{ax-3})}\\
        =&\new{a}{\apply{\had}{a}{\measureraw{a}{x\p{b}}{\apply{Z}{b}{x\p{b}}}}}.&(\ref{ax-2})
    \end{align*}
\end{example}

\printbibliography

\end{document}
