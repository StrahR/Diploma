\documentclass[mat1]{fmfdelo-muf}

% LTeX: enabled=false
\renewcommand{\d}{\;\mathrm d}
\renewcommand{\hat}{\widehat}
\renewcommand{\tilde}{\widetilde}
% \renewcommand{\bar}{\overline}
\newcommand{\subs}{\subseteq}
\newcommand{\nin}{\not\in}
\newcommand{\contradiction}{\,\,\lightning}

\newcommand{\p}[1]{\left( {#1} \right)}
\newcommand{\set}[2]{\left\{ #1 \mid #2 \right\}}
\newcommand{\newfrac}[2]{{}^{#1}\!/_{\!#2}}
\newcommand{\im}[1]{\mathrm{im}{\p{#1}}}
\newcommand{\mb}[1]{\mathbf{#1}}
\newcommand{\mf}[1]{\mathfrak{#1}}
\newcommand{\mc}[1]{\mathcal{#1}}
\newcommand{\id}{\mathrm{id}}
\newcommand{\cat}[1]{\mathbf{#1}}
\newcommand{\tr}[1]{\textrm{tr}{#1}}
\newcommand{\rang}[1]{\textrm{rang}{\p{#1}}}

\usepackage{xstring}
\newcommand{\mat}[1]{\begin{matrix} #1 \end{matrix}}
\newcommand{\pmat}[1]{\left(\mat{#1}\right)}
\newcommand{\bmat}[1]{\left[\mat{#1}\right]}
\renewcommand{\vec}[1]{\bmat{\StrSubstitute[0]{#1}{,}{\\}}}
\newcommand{\state}[1]{\left\{ {#1} \right\}}

\newcommand{\had}{\mathtt{Had}}
\newcommand{\swap}{\mathtt{swap}}
\newcommand{\hh}[1][]{\mathbf{h}_{#1}}
\newcommand{\B}[1][]{\mathbf{B}_{#1}}
\newcommand{\M}[1][]{\mathbf{M}_{#1}}
\renewcommand{\H}[1][]{\mathbf{H}_{#1}}
\renewcommand{\L}[2][]{\mathcal{L}_{#1}{\p{#2}}}
\newcommand{\g}[1]{\mathtt{#1}}
\newcommand{\D}[1]{D{\p{\g{#1}}}}
\newcommand{\U}[1][]{\mathbf{U}_{#1}}
\newcommand{\ctl}[1]{\g{c#1}}
\newcommand{\ctlo}[1]{\g{\bar c} \g #1}
\newcommand{\cstarcat}{\cat{Cstar}}
\newcommand{\cstarcpucat}{\cat{Cstar}_{\textrm{CPU}}}
\newcommand{\cstar}[2]{\cat{Cstar}{\arity{#1}{#2}}}
\newcommand{\brat}[2]{\cat{Brat}{\arity{#1}{#2}}}

\newcommand{\op}[1]{\textnormal{\sffamily#1}}
\newcommand{\mor}[1]{\textnormal{\sffamily#1}}
\newcommand{\eff}[1]{\textnormal{\sffamily\ul{#1}}}
\newcommand{\enew}{\eff{new}{\p{}}}
% \newcommand{\enew}[1]{\eff{new}{\p{#1}}}
\newcommand{\eapply}[2]{\eff{apply}_{\g{#1}}{\p{#2}}}
\newcommand{\emeasure}[1]{\eff{measure}{\p{#1}}}
\newcommand{\ediscard}[1]{\eff{discard}{\p{#1}}}
\newcommand{\tnew}[2]{\op{new}{\p{#1.\,#2}}}
\newcommand{\tapply}[3]{\op{apply}_{\g{#1}}{\p{#2;#2.\,#3}}}
\newcommand{\tapplyd}[3]{\op{apply}_{\D{#1}}{\p{#2;#2.\,#3}}}
\newcommand{\tmeasure}[3]{\op{measure}{\p{#1;\,#2,#3}}}
\newcommand{\tdiscard}[2]{\op{discard}{\p{#1;\,#2}}}
\newcommand{\new}[2]{\nu{#1}.\,#2}
\newcommand{\apply}[3]{{\g{#1}}_{#2}{\p{#3}}}
\newcommand{\applyd}[3]{{\D{#1}}_{#2}{\p{#3}}}
\renewcommand{\measure}[3]{\p{#2} {\;?_{#1}\,} \p{#3}}
\newcommand{\measureraw}[3]{#2 {\;?_{#1}\,} #3}
\newcommand{\discard}[2]{\op{disc}_{#1}{\p{#2}}}
\newcommand{\snew}[2]{\mor{new}#1{\p{#2}}}
\newcommand{\sapply}[2]{\mor{apply}_{\g{#1}}{\p{#2}}}
\newcommand{\smeasure}[3]{\mor{measure}#1{\p{#2,#3}}}

\newcommand{\type}[1]{\text{\ttfamily#1}}
\newcommand{\unit}{\type{I}}
\newcommand{\bit}{\type{bit}}
\newcommand{\qbit}{\type{qbit}}

\newcommand{\arity}[2]{\p{#1 \mid #2}}
% \newcommand{\arity}[2]{\p{#1 \mid \StrSubstitute[0]{#2}{,}{\ }}}

\newcommand{\sequent}[3]{#1 \mid #2 ⊢ #3}
\newcommand{\sseq}[2]{#1 ⊢ #2}
\newcommand{\seq}[1]{\sequent{x₁:m₁,…,xₖ:mₖ}{a₁,…,aₚ}{#1}}
\newcommand{\absseq}[1]{\sequent{\Gamma}{\Delta}{#1}}
\newcommand{\sem}[1]{\left⟦ #1 \right⟧}
% \newcommand{\brsem}[1]{⟬ #1 ⟭}
% \newcommand{\brsem}[1]{⦅ #1 ⦆}
\newcommand{\brsem}[1]{\left⦇ #1 \right⦈}
\usepackage[nomessages]{fp}
\newcommand{\mormap}[2]{% cf. https://tex.stackexchange.com/a/87423/64454
  \def\nextitem{\def\nextitem{⊕}}% Separator
  \forcsvlist\mormapitem{#2} → M_{\IfInteger{#1}{\FPeval\result{round(2^(#1):0)}\result}{2^{#1}}}{\p X}
}
\newcommand{\mormapitem}[1]{%
  \nextitem
  \def\param{\detokenize{#1}}%
  \if\param…%
    ⋯%
  \else%
    M_{\IfInteger{\param}{\FPeval\result{round(2^(\param):0)}\result}{2^{\param}}}{\p X}\!%
  \fi
}
\newcommand{\semmap}[2]{% cf. https://tex.stackexchange.com/a/87423/64454
  \def\nextitem{\def\nextitem{⊕}}% Separator
  \forcsvlist\semmapitem{#2} → \M[\IfInteger{#1}{\FPeval\result{round(2^(#1):0)}\result}{2^{#1}}]
}
\newcommand{\semmapitem}[1]{%
  \nextitem
  \def\param{\detokenize{#1}}%
  \if\param…%
    ⋯%
  \else%
    \M[\IfInteger{\param}{\FPeval\result{round(2^(\param):0)}\result}{2^{\param}}]\!%
  \fi
}
\newcommand{\abssemmap}[2]{% cf. https://tex.stackexchange.com/a/87423/64454
  \def\nextitem{\def\nextitem{×}}% Separator
  \forcsvlist\abssemmapitem{#2} → #1 • X
}
\newcommand{\abssemmapitem}[1]{%
  \nextitem
  \def\param{\detokenize{#1}}%
  \if\param…%
    ⋯%
  \else%
    (\param • X)\!%
  \fi
}
\newcommand{\semhom}[2]{
  \def\nextitem{\def\nextitem{⊕}}% Separator
  \cstar{\p{\forcsvlist\semhomitem{#2}, \M[\IfInteger{#1}{\FPeval\result{round(2^(#1):0)}\result}{2^{#1}}]}}
}
\newcommand{\semhomitem}[1]{%
  \nextitem
  \def\param{\detokenize{#1}}%
  \if\param…%
    ⋯%
  \else%
    \M[\IfInteger{\param}{\FPeval\result{round(2^(\param):0)}\result}{2^{\param}}]\!%
  \fi
}

\newcommand{\airquotes}[1]{\enquote{#1}}

\newminted[qplcode]{ocaml}{escapeinside=||, autogobble}
\newmintinline[qpl]{ocaml}{escapeinside=||, autogobble}

\makeatletter
\newcommand{\oset}[3][0ex]{%
  \mathrel{\mathop{#3}\limits^{
    \vbox to#1{\kern-2\ex@
    \hbox{$\scriptstyle#2$}\vss}}}}
\makeatother

\makeatletter
\newcommand{\listintertext}{\@ifstar\listintertext@\listintertext@@}
\newcommand{\listintertext@}[1]{% \listintertext*{#1}
  \hspace*{-\@totalleftmargin}#1}
\newcommand{\listintertext@@}[1]{% \listintertext{#1}
  \hspace{-\leftmargin}#1}
\makeatother

\renewcommand{\for}[2]{\forall#1.\;#2}
\newcommand{\exist}[2]{\exists#1\smallni:\;#2}
\newcommand{\existi}[2]{\exists!#1\smallni:\;#2}

\naslov{Kvantni algebrajski učinki}
\title{\foreignlanguage{en}{Quantum Algebraic Effects}}

\avtor{Strah}
\mentor{doc.~dr.~Matija~Pretnar}

\letnica{2022?}

\klasifikacija{81P68, 81P70}
\kljucnebesede{kvantno računalništvo, algebrajski učinki}
\keywords{quantum computing, algebraic effects}


\addbibresource{draft.bib}

\povzetek{
    % \subsubsection*{Motivacija} % TODO: Reword
    Kvantno računalništvo temelji na veliko modernih konceptih v teoriji programskih jezikov,
    % Kvantni programski jeziki predstavljajo nove probleme za teorijo programskih jezikov,
    kot na primer linearnost tipov, (kvantnimi) fizikalnimi pojavi, in še mnogo drugimi.
    V diplomski nalogi se bomo posvetili tema dvema, v tem članku pa zgolj drugemu.
    % V tej nalogi se bomo posvetili tema dvema.
    % Dober način razumevanja je, da razumemo enakost programov.
    Naš cilj je razumeti, kako se kvantni programi obnašajo,
    in dober način je razumevanje enakosti programov.
    % tj. kdaj sta dva programa enaka.

    % \subsubsection*{Pregled}

    % Najprej bomo predstavili algebrajsko teorijo za kvantne programe;
    % ta je zgrajena na unitarnih vratih in meritvah ter ima linearne parametre.
    % Nato bomo dokazali, da lahko s to teorijo predstavimo vse programe (polnost)
    % in nato iz nje izpeljali pravila za enakost kvantnih programov.
}

\abstract{
\begin{otherlanguage}{en}
    Test.
\end{otherlanguage}
}


\begin{document}

\section{Kvantna mehanika}

Ta del je povzet po \cite{ess-qc}.
Vsebuje osnovne definicije (in primere) matematičnih osnov kvantne mehanike,
ki jih potrebujemo za definicije želenih operacij nad kubiti.
% Te se nahajajo na prvih nekaj straneh.

\subsubsection*{Oznake}
Skozi ta del bomo uporabljali naslednje oznake:
\begin{itemize}
    \item \( ℕ = \{ 0, \dots \} \), \( ℕ_+ = \{ 1, \dots \} \), \( ℕₙ = \{ 0, \dots, 2ⁿ-1 \} \),
    \item \( n,m \in ℕ_+ \), ki mu bomo pravili število kubitov,
    \item \( j, k, \dots \in ℕₙ \),
    % \item \( j = jₙ \dots j₀ \) binarni zapis števila \( j \),
    \item \( aⱼ \) \( j \)-ta komponenta vektorja \( a \),
    \item \( j = j₁ \dots jₙ \) binarni zapis števila \( j \).
    % \item \( \mb 0ⁿ = 0\dots0, \mb 1ⁿ = 1\dots1 \).
    % \item \( \mb n = \{ 0, \dots, n-1 \} \), na primer \( \mb 2 = \{ 0, 1 \} \).
    % \item \( \mathbb n = \{ 0, \dots, n-1 \} \), na primer \( \mathbb 2 = \{ 0, 1 \} \).
\end{itemize}

\subsection{Kvantni vektorji}

\begin{definition}\label{binv}
    Binarni vektorji so elementi prostora \( \B[n] ≔ \{ 0, 1 \}ⁿ \) in jih pišemo kot nize v binarnem zapisu.  Za nas predstavljajo svet v katerem se odvijajo klasični programi.
\end{definition}

\begin{example}
    \(\B[2] = \{00, 01, 10, 11\}\).
\end{example}
\begin{remark}
    \(1\) in \(01\) predstavljata različna vektorja.
\end{remark}

\begin{definition}[Hilbertov prostor]\label{hilb-sp}
    Elementom prostora \( \H[n] ≔ \C^{2ⁿ} \) pravimo kvantni vektorji, elementom \( \H ≔ \H[1] \) pa kubiti.  Prostoru \( \H[n] \) torej pravimo prostor kvantnih vektorjev reda \( n \), njegovo standardno bazo pa označimo z \( \{eⱼ\} \). Tu se izvajajo kvantni programi.
\end{definition}

\begin{definition}[Braket notacija]\label{braket}
    Naj bo \( j \in ℕₙ \), ter \( \hat{\jmath} \in \B[n] \) pripadajoč vektor v binarnem zapisu. Potem je \( \ket j = \ket{\hat{\jmath}} ≔ eⱼ \).
\end{definition}
\begin{remark}
    Po definiciji je torej \( \H[n] = \L[\C]{\set{\ket j}{j\in\B[n]}} \).
\end{remark}

\begin{example}[\( n = 1 \) in \( n = 2 \)]
    \begin{align*}
        a &= \begin{bmatrix}a₀\\a₁\end{bmatrix}
        = a₀\begin{bmatrix}1\\0\end{bmatrix} + a₁\begin{bmatrix}0\\1\end{bmatrix}
        = a₀\ket 0 + a₁\ket 1,\\
        a &= \begin{bmatrix}a₀\\a₁\\a₂\\a₃\end{bmatrix}
        = \begin{bmatrix}a₀₀\\ a₀₁\\ a₁₀\\ a₁₁\end{bmatrix}
        = a₀₀\ket{00} + a₀₁\ket{01} + a₁₀\ket{10} + a₁₁\ket{11}.
    \end{align*}
\end{example}

\begin{example}[Hadamardov vektor]\label{had}
    \[ \hh ≔ ρ \left(\ket 0 + \ket 1\right),\quad
        \hh[n] ≔ ρⁿ\sum_{j\in\B[n]} \ket j,\quad
        ρ ≔ \frac{1}{\sqrt2}.
    \]
\end{example}

\subsection{Blochova sfera}

\begin{statement}
    V fizičnem svetu sta dva kubita, ki se razlikujeta zgolj za (kompleksen) faktor, enaka.
    Matematično to pomeni, da stanja kubitov (nadaljnje tudi kubiti) živijo v \( \mathbf{P}\C¹ \cong \S² \):
    % To pomeni, da lahko vsak kubit zapišemo kot točko v \( \S² \):
    \[ a = \cos\frac{θ}{2}\ket0 + e^{iφ}\sin\frac{θ}{2}\ket1,\quad
        φ \in [0, 2π), θ \in [0, π]. \]
\end{statement}
\begin{proof}
    Naj bo \( a = a₀\ket0 + a₁\ket1 = r₀e^{iφ₀}\ket0 + r₁e^{iφ₁}\ket1 \).
    Če označimo
    \[ r ≔ \sqrt{r₀² + r₁²}\text{, }φ ≔ φ₁ - φ₀\text{, }θ ≔ 2\arccos{\frac{r₀}{r}}, \]
    je potem \[ a = \hat a ≔ \frac{a}{re^{iφ₀}} = \frac{r₀}{r}\ket0 + \frac{r₁}{r}e^{iφ}\ket1 = \cos\frac{θ}{2}\ket0 + e^{iφ}\sin\frac{θ}{2}\ket1.\qedhere \]
\end{proof}

\begin{figure}[h]
\centering
\begin{tikzpicture}[baseline=(current bounding box.north)]
    % Define radius
    \def\r{7}
    
    % Bloch vector
    \draw (0,0) node[circle,fill,inner sep=1] (orig) {} -- (\r/6,\r/4) node[circle,fill,inner sep=0.7,label=above:\(a\)] (a) {};
    \draw[dashed] (orig) -- (\r/6,-\r/10) node (phi) {} -- (a);
    
    % Sphere
    \draw (orig) circle (\r/2);
    \draw[dashed, gray] (orig) ellipse (\r/2 and \r/6);
    
    % Axes
    \draw[->] (orig) -- ++(-\r/10,-\r/6) node[below] (x) {\(x\)};
    \draw[->] (orig) -- ++(\r/2,0) node[right] (y) {\(y\)};
    \draw[->] (orig) -- ++(0,\r/2) node[above] (z) {\(z=\ket 0\)};
    \draw[->, draw=gray] (orig) -- ++(0,-\r/2) node[below] (s) {\(\ket 1\)};
    
    %Angles
    % \draw[->, draw=gray] 
    % \draw[draw=gray,->] (orig) ellipse (\r/6 and \r/12) node {\(φ\)};
    \pic[draw=gray,->,"\(φ\)",angle eccentricity=0.6] {angle=x--orig--phi};
    \pic[draw=gray,<-,"\(θ\)",angle eccentricity=1.4] {angle=a--orig--z};
\end{tikzpicture}
\caption{Blochova sfera}
\label{fig:bloch-sphere}
\end{figure}


\subsection{Tenzorski produkt}

\begin{definition}[Tenzorski produkt]\label{tensorprod}
    Tenzorski produkt prostorov \( \H[n] \) in \( \H[m] \) je enak \( \H[n+m] \).
    Pišemo \( \H[n] ⊗\H[m] \).
    Če sta \( a\in\H[n] \) in \( b\in\H[m] \) je \( a⊗b \in \H[n]⊗\H[m] \).
\end{definition}
\begin{remark}
    Operator \(⊗\) je res tenzorski produkt.
\end{remark}
\begin{example}[Tenzorski produkt baznih vektorjev]
    \[ \ket j ⊗ \ket k = \ket{j₁\dots jₙk₁\dots kₘ} = \ket j \ket k = \ket{jk},
    \]
\end{example}
\begin{example}[Splošni tenzorski produkt]
    \[ 
        \begin{bmatrix}a₀ \\ a₁\end{bmatrix}⊗\begin{bmatrix}b₀ \\ b₁\end{bmatrix}
        = \begin{bmatrix}a₀b₀\\ a₀b₁\\ a₁b₀\\ a₁b₁\end{bmatrix},\quad
        a⊗b = \sum_{\substack{j\in\B[n],\\k\in\B[m]}} aⱼbₖ\ket{jk}.
    \]
\end{example}
\begin{examples}[Tenzorski eksponent]
    \begin{align*}
        &\hh[n] = \hh^{⊗n}
        = ρⁿ\underbrace{(\ket 0 + \ket 1) ⊗ \dots ⊗ \p{\ket 0 + \ket 1}}_n,\\
        &\H[n] = \H^{⊗n} = \underbrace{\H ⊗ \dots ⊗ \H}_n.
    \end{align*}
\end{examples}

\begin{definition}
    Če lahko \( a\in\H[n] \) zapišemo kot \( \bigotimes_{j=1}^{n} aⱼ \) za neke \( aⱼ\in\H \) pravimo, da je enostaven ali separabilen, sicer je pa sestavljen oziroma kvantno prepleten.
\end{definition}

\subsection{Kvantne preslikave}

\begin{definition}%[Unitarna vrata]
    Prostor unitarnih vrat reda \( n \) je \( \U[n] ≔ \U{\p{2ⁿ}} \), prostor unitarnih \( 2ⁿ \times 2ⁿ \) matrik.
    Tenzorski produkt vrat \( U⊗V ≔ [u_{jk}V]_{j,k} \) uporabljen na \( a⊗b \) je enak \( Ua⊗Vb \).
\end{definition}

\begin{example}[Tenzorski produkt unitarnih vrat]
    \[
        \begin{bmatrix}
            a₀₀ & a₀₁ \\ a₁₀ & a₁₁
        \end{bmatrix} ⊗ B
        % \begin{bmatrix}
        %     b₀₀ & b₀₁ \\ b₁₀ & b₁₁
        % \end{bmatrix}
        =
        \begin{bmatrix}
            a₀₀ B & a₀₁ B \\ a₁₀ B & a₁₁ B
        \end{bmatrix}.
    \]
        % \begin{bmatrix}
        %     a₀₀b₀₀ & a₀₁b₀₀ & a₀₀b₀₁ & a₀₁b₀₁ \\
        %     a₁₀b₀₀ & a₁₁b₀₀ & a₁₀b₀₁ & a₁₁b₀₁ \\
        %     a₀₀b₁₀ & a₀₁b₁₀ & a₀₀b₁₁ & a₀₁b₁₁ \\
        %     a₁₀b₁₀ & a₁₁b₁₀ & a₁₀b₁₁ & a₁₁b₁₁
        % \end{bmatrix}
    % \]
\end{example}

\begin{definition}%[Bločno-diagonalna matrika]
    Za vrata \( U₀,\dots,Uₛ \) označimo njihovo bločno-diagnoalno matriko z \( D{\p{U₀,\dots,Uₛ}} \).
\end{definition}

\begin{theorem}[No cloning]\label{no-cloning}
    Ne obstajajo vrata reda \( 2 \), ki vsak vektor \(a ⊗\ket{0}\in \H ⊗\H\) slika v \(a⊗a\).
\end{theorem}

\begin{proof}
    Naj bo \(U\) tak, da za vsak \( a \in \H \) velja \( U{\p{a⊗\ket0}} = a⊗a \).\\
    Potem za \( \hh ⊗\ket0 = ρ(\ket{00} + \ket{10}) \) velja:
    \[
        U{\p{ρ(\ket{00} + \ket{10})}} =
        \begin{cases}
            ρ²(\ket{00} + \ket{01} + \ket{10} + \ket{11}),\\
            ρ U\ket{00} + U\ket{10} = ρ(\ket{00} + \ket{11}),
        \end{cases}
    \]
    kar je protislovje.
\end{proof}

\begin{example}[Paulijeve matrike]
    To so matrike zrcaljenja okrog osi na Blochovi sferi:
    \[
        I₂ = \begin{bmatrix} 1 &  0 \\ 0 &  1 \end{bmatrix},\quad
        X  = \begin{bmatrix} 0 &  1 \\ 1 &  0 \end{bmatrix},\quad
        Y  = \begin{bmatrix} 0 & -i \\ i &  0 \end{bmatrix},\quad
        Z  = \begin{bmatrix} 1 &  0 \\ 0 & -1 \end{bmatrix}.
    \]
    Velja \(X² = Y² = Z² = I₂\).
    Preslikavi \( X \) pravimo negacija,
    saj je \( X{\ket0} = \ket1 \) in \( X{\ket1} = \ket0 \).
\end{example}

\begin{example}[Hadamardova matrika]
    \[
        \had = ρ\begin{bmatrix}1&1\\1&-1\end{bmatrix},\quad
        \had\ket{0} = \hh,\quad
        \had^{⊗n}\ket{\mathbf{0}ⁿ} = \hh[n].
    \]
\end{example}

% \[ \Qcircuit @C=1em @R=.7em {
%         & \gate{\had} & \gate{\g Z} & \gate{\had} & \qw\\
%         &             &      =      &             &    \\
%         & \qw         & \gate{\g X} & \qw         & \qw
%     }
% \]
\begin{example}[Fazni zamik]
    \begin{align*}
        &S_α = \begin{bmatrix}1&0\\0&e^{iα}\end{bmatrix}
        \text{, posebej označimo } S ≔ S_{\newfrac{π}{2}}, T ≔ S_{\newfrac{π}{4}}, \\
        &S_α\p{a₀\ket 0 + a₁\ket1} = a₀\ket0 + a₁e^{iα}\ket1.
    \end{align*}
\end{example}

\subsection{Kvantna meritev}

V klasičnem računalništvu poznamo pogojne stavke. To lahko na kubite posplošimo na dva načina,
prvi z direktno meritvijo kubita (in uporabo klasičnih pogojnih stavkov),
drugi pa z uporabo kvantne prepletenosti.
Izkaže se, da če na koncu zmerimo kubite, se drugi način obnaša enako kot prvi.

\begin{definition}[Kvantna meritev]
    Meritev kubita \(a = a_0\ket0 + a_1\ket1\) označimo \(M\p{a}\) in je \(0\) z verjetnostjo \(|a_0|^2\) in \(1\) z verjetnostjo \(|a_1|^2\). To \airquotes{uniči} kubit \(a\).
    % \[ \Qcircuit @C=1em @R=.7em {
    %         & \meter & \cw %& & \ctrlo{1} & \ctrl{1} & \qw & & \ctrl{1} & \qw \\
    %         % &        &     & & \gate U & \gate V & \qw & & \targ & \qw
    %     }
    % \]
\end{definition}

\begin{definition}[Kontrola]
    Za \( r,s\in\N \) in \( U \in \U[1] \) definiramo \( C_{r,s}{\p U} \) in \( \overline{C}_{r,s}{\p U} \) s predpisoma
    \begin{align*}
        C_{r,s}{\p U}\ket j &= \begin{cases}
            \ket j &;\quad jᵣ = 0\\
            \ket{j₁\dots}\ket{U{jₛ}}\ket{\dots jₙ} &;\quad jᵣ = 1
        \end{cases}\\
        \overline{C}_{r,s}{\p U}\ket j &= \begin{cases}
            \ket{j₁\dots}\ket{U{jₛ}}\ket{\dots jₙ} &;\quad jᵣ = 0\\
            \ket j &;\quad jᵣ = 1
        \end{cases}
    \end{align*}
    Takim vratom pravimo kontrolirana (\airquotes{na ena} in \airquotes{na nič}).
    Posebej za \( U \in \U[1] \) označimo
    \[ \ctl U ≔ C_{1,2}{\p U} = \D{I₂, U},\quad
        \ctlo U ≔ \overline{C}_{1,2}{\p U} = \D{U, I₂}. \]
\end{definition}

\begin{example}[Prepleteni pari kubitov]
    Kontrolirana vrata prepletejo pare kubitov. Na primer
    \( \eapply{\ctl X}{a, b} \) se obnaša kot
    \qpl[breaklines]{if |\(\emeasure{a} = 0\)| then |\( \p{a, b} \)| else |\( \p{a, \lnot b} \)|}.
    Seveda vemo, da drugi izraz ni veljaven (ker meritev uniči kubit a),
    ampak je zato kontrola ravno tisto orodje, s katerim želimo nadomestiti pogojne stavke.
\end{example}


\section{Kvantno računalništvo ter algebrajski učinki in diagrami}

\subsection{Kvantna vezja}

Kvantne programe lahko predstavimo kot diagrame vezja.
Škatle predstavljajo unitarna vrata, črte med njimi pa žice;
po enojnih žicah tečejo kubiti, po dvojnih pa klasični biti (\(0\) ali \(1\)).
Pike na žici (in potem navpična žica ven) pomenijo kontrolo;
prazna pika kontrolira \airquotes{na nič}, polna pa \airquotes{na ena}.
Taka vezja beremo od leve proti desni.
Natančnejši opis lahko najdete v \cite{ess-qc}.

Spodaj sta dva primera kvantnih programov, opisana z besedami in diagrami, ki ju bomo srečali tudi še kasneje.

\begin{example}[Projekcija na \(z\)-os]\label{ex:proj-z}
    Najprej zmerimo \( a \) in nato glede na rezultat svež kubit bodisi negiramo bodisi ne.
    Na Blochovi sferi to zgleda približno kot projekcija na \(z\)-os (edina kubita na \(z\)-osi sta \( \ket0 \) in \( \ket1 = X \ket0 \)).
    \[ \Qcircuit @C=1em @R=.7em {
            & \lstick{\ket0} & \gate{\g X} & \rstick{b} \qw\\
            \lstick{a} & \meter & \cctrl{-1}
        }
    \]
\end{example}

\begin{example}[Naključna rotacija faze]\label{ex:rand-ph-shift}
    Meritev Hadamardovega vektorja simulira pravičen met kovanca,
    vrata \( Z \) pa rotirajo fazo, torej bomo v polovici primerov kubitu \( a \) rotirali fazo.
    \[ \Qcircuit @C=1em @R=.7em {
            \lstick{a} & \qw & \qw & \qw & \qw & \gate{\g Z} & \rstick{a}\qw\\
            && \lstick{\ket0} & \gate{\had} & \meter & \cctrl{-1}
        }
    \]
\end{example}
% \[ \Qcircuit @C=1em @R=.7em {
%         & \lstick{\ket0} & \gate{\g X} & \rstick{b} \qw &&&&
%         \lstick{a} & \qw & \qw & \qw & \qw & \gate{\g Z} & \rstick{a}\qw\\
%         \lstick{a} & \meter & \cctrl{-1} &&&&
%         &&& \lstick{\ket0} & \gate{\had} & \meter & \cctrl{-1}
%     }
% \]

% \frame{
%     \frametitle[Vezja]{Grafični prikaz}

%     Grafično predstavimo kot kvantna vezja:
%     \[ \Qcircuit @C=1em @R=.7em {
%             & \gate{\g U} & \qw & & \meter & \cw \\
%             & \ctrl{1} & \qw & & \ctrlo{1} & \ctrl{1} & \qw \\
%             & \targ & \qw & & \gate{\g U} & \gate{\g V} & \qw
%         }
%     \]
% }

\subsection{Algebrajski učinki}

Z računskimi učinki se med programiranjem pogosto srečamo: globalno stanje spremenljivk, vhodno/izhodne naprave, naključnost, izjeme, nedeterminizem, ipd.

\begin{definition}[Računski učinki]
    Če ima funkcija ali operacija še kak navzven viden učinek poleg vrnjene vrednosti, slednjemu pravimo računski učinek (učinek računanja).
    % Programe lahko ločimo na dve vrsti:
    % \begin{itemize}
    %     \item na čiste funkcije, ki zgolj vrnejo vrednost,
    %     \item in operacije, ki imajo poleg vrnjene vrednosti še kak stranski učinek; temu pravimo računski učinek (učinek računanja).
    % \end{itemize}
\end{definition}

\begin{definition}[Algebrajski učinki]
    Računskim učinkom, ki jih lahko predstavimo s kašno algebrajsko teorijo, pravimo algebrajski učinki.
\end{definition}


\section{Programski jezik}

V našem jeziku\cite{algeff-lin-qpl} imamo navadne osnovne konstrukte,
npr. tipe, \texttt{let} ter \texttt{if} stavke, itd.
Poleg tega imamo pa še elemente kvantnega računalništva:
tip kubitov \texttt{qubit} in tip prepletenih parov \( A⊗B \) za vsaka dva tipa \(A\) in \(B\).
Zaradi narave kubitov ne moremo neposredno dostopati do notranjega stanja pomnilnika,
imamo pa naslednje funkcije dostopanja:
\begin{itemize}
    \item \(\eff{new}\): dodeli nov kubit, z začetno vrednostjo \(\ket 0\),
    \item \(\eff{apply}_{\g U}\): uporabi vrata \(U\) na danem vektorju,
    \item \(\eff{measure}\): izvede meritev na kubitu, vrne element tipa \texttt{bit}.
\end{itemize}

\subsection{Pretvorba v algebrajske izraze}

Konstruktom v programskem jeziku priredimo naslednje algebrajske izraze ter uvedemo še strnjeno obliko, za lažjo manipulacijo na papirju.

\begin{table}[ht]
    \centering
    \begin{tabular}{|l|l|l|}
        \hline
        Kvantni programski jezik 
            & Algebrajski izrazi              & Matematični simboli          \\
        \hline
        \qpl{let |\( a \leftarrow \enew \)| in |\( x{\p{a}} \)|}
            & \( \tnew{a}{x{\p a}} \)         & \( \new{a}{x{\p a}} \)       \\
        \qpl{|\( \eapply{\g{U}}{a} \)|; |\( x{\p{a}} \)|}
            & \( \tapply{U}{a}{x{\p a}} \)    & \(\apply{U}{a}{x{\p a}}\)    \\
        \qpl{if |\( \emeasure{a} = 0 \)| then |\( t \)| else |\( u \)|}
            & \( \tmeasure{a}{t}{u} \)        & \( \measureraw{a}{t}{u} \)   \\
        % \qpl{if |\( \emeasure{a} = 0 \)| then |\( t \)| else |\( t \)|}
        \qpl{|\( \ediscard{a} \)|; |\( t \)|}
            & \( \tdiscard{a}{t} \)           & \( \discard{a}{t} \)         \\
        \hline
    \end{tabular}
\end{table}

% \pagebreak

\begin{example}[Projekcija na \( z \)-os]\(\)
    \begin{enumerate}
        \item \qpl{if |\(\emeasure{a} = 0\)| then |\( \enew \)| else |\( \eapply{\g{X}}{\enew} \)|}
        \item \( \tmeasure{a}{\tnew{b}{x{\p b}}}{\tnew{b}{\tapply{X}{b}{x{\p b}}}} \)
        \item \(
            \measure{a}
                {\new{b}             {x{\p b}}}
                {\new{b}{\apply{X}{b}{x{\p b}}}
            }\)
    \end{enumerate}
\end{example}

\begin{example}[Naključna rotacija faze]\(\)
    \begin{enumerate}
        \item \qpl{if |\( \emeasure{\eapply{\had}{\enew}} = 0 \)| then |\( a \)| else |\( \eapply{\g{Z}}{a} \)|}
        \item \( \tnew{b}{\tapply{\had}{b}{\tmeasure{b}{x{\p a}}{\tapply{Z}{a}{x{\p a}}}}} \)
        \item \(
            \new{b}{ \apply{\had}{b}{
                \measureraw{b}
                    {x{\p a}}
                    {\apply{Z}{a}{x{\p a}}}
            }}\)
    \end{enumerate}
\end{example}

\subsection{Aksiomi}

% Aksiomi za enakost programov so (brez dokaza) sledeči:
Aksiome za enakost programov lahko delimo na dva dela;
prvih pet je glavnih, ostalih sedem pa bolj \airquotes{administrativnih} oziroma pomožnih.
Slednji nam povejo zgolj, da se \( \eff{apply} \) strinja s strukturo unitarnih matrik,
ter da stvari komutirajo, kolikor vezanje spremenljivk (in vrstni red uporabe matrik) dopušča.
Podrobnejši opis (z dokazom) najdete v \cite{algeff-lin-qpl}.

% \setlength\parindent{0pt}

\begin{axiom}{Kvantna negacija pred meritvijo je negacija po meritvi:}\label{ax:1}
    \( \apply{X}{a}{\measureraw{a}{x}{y}} = \measureraw{a}{y}{x} \).
\end{axiom}

\begin{axiom}{Kvantna kontrola je po meritvi kot klasična kontrola:}\label{ax:2}
    \( \applyd{U, V}{a,b}{\measureraw{a}{x(b)}{y(b)}}
        = \measureraw{a}{\apply{U}{b}{x(b)}}{\apply{V}{b}{y(b)}} \).
\end{axiom}

\begin{axiom}{Kvantna vrata uporabljena na zavrženih kubitih so odveč:}\label{ax:3}
    \( \apply{U}{a}{\discard{a}{t}} = \discard{a}{t} \).
\end{axiom}

% \begin{axiom}{Meritve novih kubitov so vedno \(0\):}\label{ax:4}
\begin{axiom}{Novi kubiti so \( \ket0 \) glede na meritev:}\label{ax:4}
    \( \new{a}{\measureraw{a}{x}{y}} = x \).
\end{axiom}

% \begin{axiom}{Vrata kontrolirana z novimi kubiti se nikoli ne uporabijo:}\label{ax:5}
\begin{axiom}{Novi kubiti so \( \ket0 \) glede na kontrolo:}\label{ax:5}
    \( \new{a}{\applyd{U,V}{a,b}{x{\p{a,b}}}} = \apply{U}{b}{\new{a}{x{\p{a,b}}}} \).
\end{axiom}

\begin{axiom}{Spoštovanje simetrične grupe \( \U[n] \):}\label{ax:6}
    \( \apply{swap}{a,b}{x{\p{a,b}}} = x{\p{b,a}} \),
\end{axiom}

\begin{axiom}{}\label{ax:7}
    \( \apply{I}{a}{x{\p a}} = x{\p a} \),
\end{axiom}

\begin{axiom}{}\label{ax:8}
    \( \apply{UV}{a}{x{\p a}} = \apply{V}{a}{\apply{U}{a}{x{\p a}}} \),
\end{axiom}

\begin{axiom}{}\label{ax:9}
    \( \apply{U⊗V}{a,b}{x{\p{a,b}}} = \apply{U}{a}{\apply{V}{b}{x{\p{a,b}}}} \).
\end{axiom}

\begin{axiom}{Komutativnost:}\label{ax:10}
    \( \measure{a}{\measureraw{b}{u}{v}}{\measureraw{b}{x}{y}}
        = \measure{b}{\measureraw{a}{u}{x}}{\measureraw{a}{v}{y}} \),
\end{axiom}

\begin{axiom}{}\label{ax:11}
    \( \new{a}{\new{b}{x{\p{a,b}}}} = \new{b}{\new{a}{x{\p{a,b}}}} \),
\end{axiom}

\begin{axiom}{}\label{ax:12}
    \( \new{a}{\measureraw{b}{x{\p a}}{y{\p a}}}
        = \measure{b}{\new{a}{x{\p a}}}{\new{a}{y{\p a}}} \).
\end{axiom}

% \begin{gather}
% \intertext{Kvantna negacija pred meritvijo je negacija po meritvi.}
%     \apply{X}{a}{\measureraw{a}{x}{y}} = \measureraw{a}{y}{x}\label{ax:1}
% \intertext{Kvantna kontrola je po meritvi kot klasična kontrola.}
%     \measureraw{a}{\apply{U}{b}{x(b)}}{\apply{V}{b}{y(b)}}
%         = \applyd{U, V}{a,b}{\measureraw{a}{x(b)}{y(b)}}\label{ax:2}
% \intertext{Kvantna vrata uporabljena na zavrženih kubitih so odveč.}
%     \apply{U}{a}{\discard{a}{t}} = \discard{a}{t}\label{ax:3}
% \intertext{Meritve novih kubitov so vedno \(0\).}
%     \new{a}{\measureraw{a}{x}{y}} = x\label{ax:4}
% \intertext{Vrata kontrolirana z novimi kubiti se nikoli ne uporabijo.}
%     \new{a}{\applyd{U,V}{a,b}{x{\p{a,b}}}}
%         = \apply{U}{b}{\new{a}{x{\p{a,b}}}}\label{ax:5}
% \intertext{Ostanejo še bolj \airquotes{administrativni} aksiomi:
% Spoštovanje simetrične grupe \( \U[n] \).}
%     \apply{swap}{a,b}{x{\p{a,b}}} = x(b,a)\label{ax:6}\\
%     \apply{I}{a}{x{\p a}} = x{\p a}\label{ax:7}\\
%     \apply{UV}{a}{x{\p a}} = \apply{U}{a}{\apply{V}{a}{x{\p a}}}\label{ax:8}\\
%     \apply{U⊗V}{a,b}{x{\p{a,b}}} = \apply{U}{a}{\apply{V}{b}{x{\p{a,b}}}}\label{ax:9}
% \intertext{Komutativnost.}
%     % \apply{U}{a}{\apply{V}{b}{x{\p{a,b}}}} = \apply{V}{b}{\apply{U}{a}{x{\p{a,b}}}}\\
%     \measure{a}{\measureraw{b}{u}{v}}{\measureraw{b}{x}{y}}
%         = \measure{b}{\measureraw{a}{u}{x}}{\measureraw{a}{v}{y}}\label{ax:10}\\
%     \new{a}{\new{b}{x{\p{a,b}}}} = \new{b}{\new{a}{x{\p{a,b}}}}\label{ax:11}\\
%     \new{a}{\measureraw{b}{x{\p a}}{y{\p a}}}
%         = \measure{b}{\new{a}{x{\p a}}}{\new{a}{y{\p a}}}\label{ax:12}
% \end{gather}

% \apply{X}{a}{\apply{\H}{b}{x{\p{a,b}}}} = \apply{\H}{b}{\apply{X}{a}{x{\p{a,b}}}}
\begin{example}[Izpeljava enakosti projekcije na \(z\)-os in naključne rotacije faze]
    Izpeljava se zanaša na identiteti 
    \( \ctl X . \swap . \ctl X \overset{\dagger}{=} \swap . \ctl X . \swap \) in 
    \( \swap . \ctl X . \swap \overset{\ddagger}{=} (\had⊗\g{I₂}) . \ctl Z . (\had⊗\g{I₂}) \).
    \begin{NoHyper}
    \begin{align*}
        &\hspace{-3em}\measure{a}{\new{b}{x{\p b}}}{\new{b}{\apply{X}{b}{x{\p b}}}}\\
        =&\new{b}{\measureraw{a}{x{\p b}}{\apply{X}{b}{x{\p b}}}}
            &{(\ref{ax:12})}\\
        =&\new{b}{\apply{\ctl X}{a,b}{\measureraw{a}{x{\p b}}{x{\p b}}}}
            &{(\ref{ax:2})}\\
        =&\new{b}{\apply{\ctl X}{a,b}{\discard{a}{x{\p b}}}}
            &{(\text{def.})}\\
        =&\new{b}{\apply{\ctl X}{b,a}{\apply{\ctl X}{a,b}{\discard{b}{x{\p a}}}}}
            &{(\dagger)}\\
        =&\new{b}{\apply{\ctl X}{a,b}{\discard{b}{x{\p a}}}}
            &{(\ref{ax:5})}\\
        =&\new{b}{\apply{\had}{b}{\apply{\ctl Z}{b,a}{\apply{\had}{b}{\discard{b}{x{\p a}}}}}}
            &{(\ddagger)}\\
        =&\new{b}{\apply{\had}{b}{\apply{\ctl Z}{b,a}{\discard{b}{x{\p a}}}}}
            &{(\ref{ax:3})}\\
        =&\new{b}{\apply{\had}{b}{\measureraw{b}{x{\p a}}{\apply{Z}{a}{x{\p a}}}}}.
            &(\ref{ax:2})
    \end{align*}
    \end{NoHyper}
    % Kot vidimo, smo definirali jezik (in orodja) s katerimi lahko relativno enostavno dokazujemo enakost med programi.
    % Opazimo, da ta dokaz ni bil popolnoma mehanski;
    % uporabiti smo morali dve identiteti v \( \U[2] \).
    % To nam namiguje, da enakost programov ni preprost problem,
    % ki bi ga lahko računalniki sami rešili.
\end{example}

\printbibliography
\end{document}
