\section{Semantika}

\subsection{Modeli algebrajskih teorij z linearnimi parametri}
% TODO: define models
% \subsubsection{Algebrajska teorija z linearnimi parametri}

\begin{definition}[Členost]
    Členost, \(\arity{p}{m₁, …, mₖ}\), je naravno število \(p\) skupaj s seznamom naravnih števil \(mᵢ\).
% \end{definition}
% \begin{definition}
    Signatura z linearnimi parametri je množica operacij s členostmi.
    Neformalno to pove, da operacija \(\op O\) sprejme \(p\) parametrov in \(k\) računskih spremenljivk, kjer \(i\)-ta veže \(mᵢ\) parametrov.
    Pišemo \(\op O : \arity{p}{m₁, …, mₖ}\).
\end{definition}

% TODO: define judgement

\begin{example}
    \begin{gather*}
        \op{new}          : \arity{0}{1}\qquad
        \op{measure}      : \arity{1}{0, 0}\qquad
        \op{apply}_{\g U} : \arity{n}{n}
    \end{gather*}
\end{example}

\begin{example}
    %TODO: elaborate
    \[\begin{prooftree}
        \hypo{\sequent{Γ}{Δ, a}{         t } }
      \infer1{\sequent{Γ}{Δ   }{\tnew{a}{t}} }
  \end{prooftree}\quad
  \begin{prooftree}
        \hypo{\sequent{Γ}{Δ   }{             t    } }
        \hypo{\sequent{Γ}{Δ   }{                u } }
      \infer2{\sequent{Γ}{Δ, a}{\tmeasure{a}{t}{u}} }
  \end{prooftree}\]
  \[\begin{prooftree}
        \hypo{\sequent{Γ}{Δ, a₁, …, aₙ}{                      t }}
      \infer1{\sequent{Γ}{Δ, a₁, …, aₙ}{\tapply{U}{a₁, …, aₙ}{t}}}
  \end{prooftree}\]
\end{example}

\begin{definition}[Aksiomi]
    %TODO: define term
    Aksiom v neki signaturi je par izrazov v istem kontekstu; pišemo \(\seq{t=u}\).
    Predstavitev teorije je signatura skupaj z množico aksiomov.
\end{definition}

\subsection{Semantika kvantnega programskega jezika}

\subsection{\texorpdfstring{\(C^*\)-algebre}{C*-algebre}}

%TODO: citations
\begin{definition} %TODO: cite vidav
    Za polje \(k\) je množica \(A\) \(k\)-algebra, če je
    \begin{enumerate}
        \item \(k\)-vektorski prostor,
        \item kolobar za seštevanje in množenje,
        \item velja \(\for{a,b ∈ A, λ ∈ k}{(λa)b = λ(ab) = a(λb)}\).
    \end{enumerate}
\end{definition}

\begin{definition}
    \(k\)-algebra \(A\) je normirana, če ima normo in velja \(\for{x, y ∈ A}{\|xy\| ≤ \|x\|\|y\|}\).
\end{definition}
\begin{definition} %TODO: cite vidav
    Involucija je operacija \(x ↦ x^*\) na elementih algebre, za katero velja:
    \begin{enumerate}
        \item \(\for{a,b ∈ A}{(a+b)^* = a^* + b^*}\),
        \item \(\for{λ ∈ ℂ, a ∈ A}{(λa)^* = \bar{λ}a^*}\),
        \item \(\for{a,b ∈ A}{(xy)^* = y^*x^*}\),
        \item \(\for{a ∈ A}{a^{**} = a}\).
    \end{enumerate}
\end{definition}
\begin{definition}
    Normirana \(k\)-algebra je Banachova, če je polna kot metrični prostor z metriko porojeno iz norme.
\end{definition}

\begin{definition}
    Množica \(A\) je \(C^*\)-algebra, če:
    \begin{enumerate}[(a)]
        \item je Banachova \(ℂ\)-algebra z enoto,
        \item ima involucijo \((-)^*\),
        \item za vsak \(x ∈ A\) velja \(\|x\|² = \|x^*x\|\).
    \end{enumerate}
\end{definition}
\begin{remark}
    \(C^*\) se prebere kot \airquotes{c zvezdica}.
\end{remark}
\begin{remark}
    V fiziki se iz zgodovinskih razlogov involucija pogosto označuje z \((-)^†\).
\end{remark}

\begin{definition}
    Za \(C^*\)-algebri \(A\) in \(B\) je preslikava \(f : A → B\) \(*\)-homomorfizem, če je linearna in ohranja množenje, enoto, ter involucijo.
\end{definition}
\begin{remark}
    \(*\)-homorfizem se prebere kot \airquotes{zvezdica homomorfizem}.
\end{remark}

\begin{example}
    Množica kompleksnih \(n×n\) matrik \( \M[n] ≔ Mₙ(ℂ) \), kjer za normo vzamemo determinanto in za involucijo hermitsko transponiranje, je \(C^*\)-algebra.
    Res, lastnosti (a) do (c) držijo (to vemo od prej), lastnost (d) je pa preprosto preveriti: \(\det{\p{A^{H}A}} = \p{\det{A^H}}\p{\det{A}} = \p{\det{A}}²\).
\end{example}

\begin{definition}
    %TODO: finish definition
    Direktna vsota \(C^*\)-algeber \(X⊕Y\) je kartezični produkt množic z operacijami definiranimi po komponentah.
\end{definition}

\begin{example}
    %TODO: finish definition
    Za vsako \(C^*\)-algebro \(X\) je tudi množica \(n×n\) matrik nad \(X\) \(C^*\)-algebra.
    Označimo jo z \(Mₙ(X)\).
\end{example}

\begin{proposition}
    \(C^*\)-algebre skupaj z \(*\)-homomorfizmi tvorijo kategorijo \(\cstar\).
\end{proposition}

\begin{proof}
    %TODO
\end{proof}

%TODO: definition of C*-algebras as models
%TODO: soundness (zdravost)

\subsection{Polnost za meritev}
%TODO: add motivation/explanation about why we don't model new

\begin{definition}
    Operacijo \(\op{measure}\) interpretiramo z \(*\)-homomorfizmom \( \op{measure} : \semmap{1}{0, 0} \) s predpisom \[\smeasure{α}{β} = \p{\mat{α & 0 \\ 0 & β}}.\]
\end{definition}
\begin{definition}
    %TODO: consider like 3×3 unitaries
    Za \(U ∈ \U[p]\) interpretiramo operacijo \(\op{apply}_{\g U}\) z \(*\)-homomorfizmom \( \op{apply}_{\g U} : \semmap{p}{p}\) s predpisom \(\sapply{U}{A} = U^*AU.\)
\end{definition}
\begin{remark}
    TODO: beseda o členosti operacij
\end{remark}

\begin{proposition}
    %TODO: subtheory?
    %TODO: reword?
    Kompleksna števila \(ℂ ∈ \cstar\) tvorijo model fragmenta kvantnega računalništva, ki vključuje operaciji \(\op{measure}\) in \(\op{apply}\) (ampak ne \(\op{new}\)) in vse relevantne aksiome ((A)–(C), (F)–(J)).
\end{proposition}

% TODO: bratteli diagrams

\begin{proof}
    TODO
\end{proof}

\begin{theorem}[Polnost za meritev]
    %TODO: prava polnost? univerzalnost/polnost?
    Kompleksna števila tvorijo poln model:
    \begin{enumerate}
        \item Za vsak \(*\)-homomorfizem \(f : \semmap{p}{m₁,…,mₖ}\) obstaja term v algebrajski teoriji, ki ne vsebuje \(\op{new}\), tako da je \(t : \arity{p}{m₁,…,mₖ}\) in \(\sem{t} = f\).
        \item Če \(\seq{t, u}\) ne vsebujeta \(\op{new}\) in \(\sem{t} = \sem{u}\) lahko izpeljemo \(\seq{t = u}\).
    \end{enumerate}
\end{theorem}

\begin{proof}
    TODO
\end{proof}

\subsection{Polnost v splošnem}

\begin{definition}
    Element \(x\) \(C^*\)-algebre je pozitiven, če obstaja kak element \(y\), da je \(x = y^*y\).
\end{definition}

\begin{definition}
    Preslikava \(f\) je popolnoma pozitivna, če za vsak \(k ∈ ℕ₊\) preslikava \(Mₖ{\p f}\) ohranja pozitivnost elementov.
\end{definition}

\begin{proposition}
    Za \(f : X → Y\), kjer množenje v enem od \(X\) in \(Y\) komutira, so naslednje trditve ekvivalentne:
    \begin{enumerate}
        \item \(f\) ohranja pozitivnost elementov,
        \item za vsak \(k ∈ ℕ₊\) preslikava \(Mₖ{\p f}\) ohranja pozitivnost elementov,
        \item \(f\) je popolnoma pozitivna.
    \end{enumerate}
\end{proposition}

\begin{proof}
    Trditvi \((2)\) in \((3)\) sta ekvivalentni po definiciji in očitno \((2) ⇒ (1)\).
    Dokazati moramo torej zgolj \((1) ⇒ (2)\).
    TODO
\end{proof}

\begin{definition}
    Preslikava \(f\) je eniška, če slika enoto v enoto.
\end{definition}

\begin{proposition}
    \(C^*\)-algebre skupaj s popolnoma pozitivnimi eniškimi preslikavami tvorijo kategorijo \(\cstarcpu\) (angl. \foreignlanguage{english}{Completely Positive Unital maps}).
\end{proposition}
\begin{proof}
    TODO
\end{proof}

\begin{definition}
    Operacijo \(\op{new}\) interpretiramo s preslikavo
    \(\op{new} : \semmap{0}{1}\) s predpisom \[\snew{\mat{α₁₁ & α₁₂ \\ α₂₁ & α₂₂}} = α₁₁.\]
\end{definition}

%NOTE: maybe different notation operations and the interpretation morphisms
\begin{proposition}
    Preslikava \(\op{new}\) ni \(*\)-homomorfizem.
\end{proposition}

\begin{proof}
    Denimo, da je. Potem je 
    \[0 = \op{new}{\pmat{0&1\\0&0}}\op{new}{\pmat{0&0\\1&0}}
        = \op{new}{\p{\bmat{0&1\\0&0}⋅\bmat{0&0\\1&0}}}
        = \op{new}\pmat{1&0\\0&0}
        = 1 \contradiction\qedhere
    \]
\end{proof}

\begin{proposition}
    Preslikava \(\op{new}\) je popolnoma pozitivna in eniška.
\end{proposition}

\begin{proof}
    Preverimo obe lastnosti:
    \begin{itemize}
        \item \(\op{new}{\pmat{1&0\\0&1}} = 1\),
        \item Naj bo \(x\) pozitiven element. Potem je \(x = y^*y\), za nek \(y ∈ \M[2]\).
        Če zapišemo \(y = \pmat{a&b\\c&d}\) je potem \(\op{new}{\p x} = a\bar a + c\bar c\).
        Število \(a\bar a + c\bar c\) je realno in nenegativno, torej lahko vzamemo \(y = \sqrt{a\bar a + c\bar c}\).\qedhere
    \end{itemize}
\end{proof}

\begin{proposition}
    Kompleksna števila \(ℂ ∈ \cstarcpu\) tvorijo model kvantnega računalništva.
\end{proposition}

\begin{proof}
    TODO
\end{proof}

\begin{theorem}[Polnost v splošnem]
    Kompleksna števila tvorijo poln model:
    \begin{enumerate}
        \item Za vsako popolnoma pozitivno eniško preslikavo \(f : \semmap{p}{m₁,…,mₖ}\) obstaja term v algebrajski teoriji, tako da je \(t : \arity{p}{m₁,…,mₖ}\) in \(\sem{t} = f\).
        \item Če \(\seq{t, u}\) in \(\sem{t} = \sem{u}\) lahko izpeljemo \(\seq{t = u}\).
    \end{enumerate}
\end{theorem}

\subsection{Dokaz polnosti}
Naštejmo najprej nekaj pomožnih izrekov.

\begin{theorem}[Stinespringov izrek o dilaciji]
    Naj bo \(f : \mc A → \M[p]\) CPU. Tedaj obstaja \(q ≥ p\) in \(*\)-homomorfizem \(g : \mc A → \M[q]\), tako da je \(f(A) = g(A)|ₚ\).
    \[\begin{tikzcd}
        A \\
        {\M[q]} & {\M[p]}
        \arrow[     from=2-1, to=2-2]
        \arrow["g", from=1-1, to=2-1]
        \arrow["f", from=1-1, to=2-2]
    \end{tikzcd}\]
\end{theorem}

\begin{theorem}[o minimalnosti dilacije]
    Lahko izberemo minimalno dilacijo: če je \(r ≥ p\) in \( h : \mc A → \M[p] \) \(*\)-homomorfizem tak, da je \( h(-)|ₚ = f(-) \) je \(r ≥ q\) in \(g(-) = Uh(-)U^*|_q\) za neko \(r×r\) unitarno matriko.
    % https://q.uiver.app/?q=WzAsNSxbMywwLCJBIl0sWzMsMSwiXFxNW3FdIl0sWzQsMSwiXFxNW3BdIl0sWzIsMSwiXFxNW3JdIl0sWzAsMSwiXFxNW3JdIl0sWzEsMl0sWzAsMSwiZyJdLFswLDIsImYiXSxbMywxXSxbNCwzLCJVKC0pVSoiLDFdLFswLDQsImgiLDJdLFs0LDIsIiIsMix7ImN1cnZlIjozfV1d
    \[\begin{tikzcd}
        &&& A \\
        {\M[r]} && {\M[r]} & {\M[q]} & {\M[p]}
        \arrow[from=2-4, to=2-5]
        \arrow["g", from=1-4, to=2-4]
        \arrow["f", from=1-4, to=2-5]
        \arrow[from=2-3, to=2-4]
        \arrow["{U(-)U*}"{description}, from=2-1, to=2-3]
        \arrow["h"', from=1-4, to=2-1]
        \arrow[curve={height=20pt}, from=2-1, to=2-5]
    \end{tikzcd}\]
\end{theorem}
