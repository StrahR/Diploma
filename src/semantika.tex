\section{Semantika}
Lotimo se sedaj modeliranja kvantnega računalništva. Najprej definirajmo matematične objekte, s katerimi bomo sploh interpretirali naše programe.

\subsection{\texorpdfstring{\(C^*\)-algebre}{C*-algebre}}
Kvantno računalništvo je standardno modelirati z operatorskimi algebrami.
V našem primeru ga bomo modelirali s \(C^*\)-algebrami.
% Standardne definicije in izreke povzamemo po
%TODO: cite standard literature

%TODO: citations
\begin{definition} %TODO: cite vidav
    Za polje \(k\) je množica \(A\) \emph{\(k\)-algebra}, če je
    \begin{enumerate}
        \item \(k\)-vektorski prostor,
        \item kolobar za seštevanje in množenje,
        \item velja \(\for{a,b ∈ A, λ ∈ k}{(λa)b = λ(ab) = a(λb)}\).
    \end{enumerate}
\end{definition}

\begin{definition}
    \(k\)-algebra \(A\) je \emph{normirana}, če ima normo in za poljubna \(a,b ∈ A\) velja \(\|ab\| ≤ \|a\|\|b\|\).
\end{definition}
\begin{definition} %TODO: cite vidav
    Involucija je operacija \(x ↦ x^*\) na elementih algebre, za katero velja:
    \begin{enumerate}
        \item \(\for{a,b ∈ A}{(a+b)^* = a^* + b^*}\),
        \item \(\for{λ ∈ ℂ, a ∈ A}{(λa)^* = \bar{λ}a^*}\),
        \item \(\for{a,b ∈ A}{(ab)^* = b^*a^*}\),
        \item \(\for{a ∈ A}{a^{**} = a}\).
    \end{enumerate}
\end{definition}
\begin{definition}
    Normirana \(k\)-algebra je \emph{Banachova}, če je polna kot metrični prostor z metriko porojeno iz norme.
\end{definition}

\begin{definition}
    Množica \(A\) je \emph{\(C^*\)-algebra}, če:
    \begin{enumerate}[(a)]
        \item je Banachova \(ℂ\)-algebra z enoto,
        \item ima involucijo \((-)^*\),
        \item za vsak \(a ∈ A\) velja \(\|a\|² = \|a^*a\|\).
    \end{enumerate}
\end{definition}
\begin{remark}
    \(C^*\) se prebere kot \airquotes{c zvezdica}.
\end{remark}
\begin{remark}
    V fiziki se iz zgodovinskih razlogov involucija pogosto označuje z \((-)^†\).
\end{remark}
\begin{remark}
    Velja \(\|a^*\| = \|a\|\), saj je \(\|a\|² = \|a^*a\| = \|a^*\|²\) in \(\|a^*a\| ≤ \|a^*\|\|a\|\).
\end{remark}

\begin{definition}
    Za \(C^*\)-algebri \(A\) in \(B\) je preslikava \(f : A → B\) \emph{\(*\)-homomorfizem}, če je linearna in ohranja množenje, enoto, ter involucijo.
\end{definition}
\begin{remark}
    \(*\)-homorfizem se prebere kot \airquotes{zvezdica homomorfizem}.
\end{remark}

\begin{example}
    Množica kompleksnih \(n×n\) matrik \( \M[n] ≔ Mₙ(ℂ) \), kjer za normo vzamemo spektralno normo in za involucijo hermitsko transponiranje, je \(C^*\)-algebra.
    Res, lastnosti (a) do (c) držijo (to vemo od prej), lastnost (d) je pa preprosto preveriti: \(σ₁{\p{A^{H}A}} = \sqrt{λ₁(A^{H}AA^{H}A)} = \sqrt{λ₁(A^{H}A)²} = λ₁(A^{H}A) = σ₁(A)²\), kjer \(λ₁(A)\) predstavlja največjo lastno vrednost matrike, \(σ₁(A)\) pa največjo singularno vrednost matrike.
\end{example}

\begin{definition}
    \emph{Direktna vsota \(C^*\)-algeber} \(X\) in \(Y\) je kartezični produkt množic z operacijami definiranimi po komponentah, za normo pa vzamemo \(\max\) normo.
    Označimo jo z \(X⊕Y\).
\end{definition}

\begin{example}
    Za vsako \(C^*\)-algebro \(X\) je tudi množica \(n×n\) matrik nad \(X\) \(C^*\)-algebra.
    Operacije definiramo kot navadne matrične, norma je pa spektralna norma.
    Označimo jo z \(Mₙ(X)\).
\end{example}

\begin{proposition}
    Velja \(Mₙ(\M[k]) ≅ \M[nk]\). Posebej je \(M_{2ᵐ}{\p{\M[2ⁿ]}} = \M[2ᵐ⁺ⁿ]\).
\end{proposition}

\begin{proposition}
    \(C^*\)-algebre skupaj z \(*\)-homomorfizmi tvorijo kategorijo \(\cstarcat\).
\end{proposition}

\subsection{Interpretacija programskega jezika}
Za nosilce našega modela kvantnega računalništva bomo vzeli kompleksna števila, operacijo \(n•ℂ\) pa definiramo kot \(\M[2ⁿ]\).
Sledi, da operacijo s členostjo \(\arity{p}{m₁,…,mₖ}\) interpretiramo z morfizmom iz \(\semmap{p}{m₁,…,mₖ}\).
Za naš jezik to pomeni, da za \(\op{new}\), \(\op{apply}\), in \(\op{measure}\) želimo morfizme iz \(\semmap{0}{1}\), \(\semmap{p}{p}\), in \(\semmap{1}{0,0}\).

Spomnimo se sedaj transformacij opazljivk, ki jih omenimo v razdelku~\ref{sec:observables}.
Dodelitev kubita in uporabo vrat že znamo izraziti z operacijami na opazljivkah, meritve kubita pa še ne.
Oglejmo si še enkrat (prirejen) primer~\ref{ex:proj-z}:
\[ \Qcircuit @C=1em @R=.7em {
    & \mbox{A} && \hspace{-2em}\mbox{B} & \mbox{C}\\
    &\lstick{a} & \meter & \cctrl{1}\\
    \lstick{b} & \qw & \qw & \gate{\g X} & \rstick{\hat{b}} \qw
}
\]
Predstavljamo si lahko, da se po meritvi univerza razveji, in smo v univerzi, kjer izmerimo \(1\), vrata \(X\) uporabili, v univerzi, kjer izmerimo \(0\) pa ne.
To tudi pomeni, da imamo glede na rezultat meritve dve različni opazljivki (v tem primeru \(B₀ ≔ C\) in \(B₁ ≔ X^*CX\)), ki ju moramo združiti, da se lahko premaknemo bolj levo.
Izkaže se, da je \(A\) enak \(\pmat{B₀&0\\0&B₁}\).

Natančno torej definiramo interpretacije takole:
\begin{itemize}
    \item \(\snew{}{\mat{A₁₁&A₁₂\\A₂₁&A₂₂}} ≔ \sem{\op{new}}{\p{\mat{A₁₁&A₁₂\\A₂₁&A₂₂}}} =  A₁₁\),
    \item \(\sapply{U}{A} ≔ \sem{\op{apply}_{\g U}}(A) = U^*AU\),
    \item \(\smeasure{}{A₁}{A₂} ≔ \sem{\op{measure}}{\p{A₁,A₂}} = \pmat{A₁ & 0 \\ 0 & A₂}\).
\end{itemize}

Če želimo interpretirati operacije, ki delujejo na poljubnih kubitih, lahko najprej kubite permutiramo na konec, tam izvedemo operacijo s pomočjo konstrukcije \(Mₚ(f)\), in nato spravimo kubite v prvotni vrstni red.
Pri meritvi na \(j\)-tem je to permutacija \((p,…,j)\) pred operacijo, in identiteta po njej,
pri uporabi vrat kubite, na katerih želimo uporabiti vrata, premaknemo na konec (v želenem vrstnem redu), uporabimo vrata \(I⊗U\) na vseh kubitih, in jih nato ponovno permutiramo kubite nazaj,
dodelitev \(j\)-tega kubita pa zgleda kot dodelitev kubita na zadnje mesto in nato permutacija oblike \((j,…,p+1)\).
% Če želimo interpretirati operacije, ki delujejo na poljubnih kubitih, lahko najprej kubite permutiramo na začetek, tam izvedemo operacijo, in nato spravimo kubite v prvotni vrstni red.
% Pri meritvi na \(j\)-tem je to permutacija \((1,…,j)\) pred operacijo, in identiteta po njej,
% pri uporabi vrat kubite, na katerih želimo uporabiti vrata, premaknemo na začetek (v želenem vrstnem redu), uporabimo vrata \(U⊗I\) na vseh kubitih, in jih nato ponovno permutiramo nazaj,
% dodelitev \(j\)-tega kubita pa zgleda kot dodelitev kubita na prvo mesto in nato permutacija oblike \((j,…,1)\).

Označimo z \(\mor{measure}ⱼ\) meritev na \(j\)-tem mestu in z \(\mor{new}ⱼ\) dodelitev \(j\)-tega kubita.

\begin{proposition}
    Preslikavi \(\mor{measure}\) in \(\mor{apply}\) sta \(*\)-homomorfizma.
\end{proposition}

\begin{proof}
    Preslikavi sta obe linearni. Preverimo še ostale lastnosti.\\
    \((\mor{measure})\):
    \nopagebreak
    \begin{itemize}
        \item \(\smeasure{}{A}{B}\smeasure{}{C}{D} = \pmat{AC&0\\0&BD} = \smeasure{}{AC}{BD}\),
        \item \(\smeasure{}{Iₖ}{Iₖ} = \pmat{Iₖ&0\\0&Iₖ} = I₂ₖ\),
        \item \(\smeasure{}{A}{B}^* = \pmat{A^*&0\\0&B^*} = \smeasure{}{A^*}{B^*}\).
    \end{itemize}
    \((\mor{apply})\):
    \begin{itemize}
        \item \(\sapply{U}{A}\sapply{U}{B} = U^*AUU^*BU = U^*ABU = \sapply{U}{AB}\),
        \item \(\sapply{U}{Iₖ} = U^*IₖU = U^*U = Iₖ\),
        \item \(\sapply{U}{A}^* = \p{U^*AU}^* = U^*A^*U = \sapply{U}{A^*}\).\qedhere
    \end{itemize}
\end{proof}

\begin{proposition}
    Preslikava \(\mor{new}\) ni \(*\)-homomorfizem.
\end{proposition}

\begin{proof}
    Denimo, da je \(\mor{new}\) \(*\)-homomorfizem. Potem je 
    \[0 = \snew{}{\mat{0&1\\0&0}}\snew{}{\mat{0&0\\1&0}}
        = \snew{}{\bmat{0&1\\0&0}⋅\bmat{0&0\\1&0}}
        = \snew{}{\mat{1&0\\0&0}}
        = 1 \contradiction\qedhere
    \]
\end{proof}

\begin{remark}
    Vredno je omeniti, da je standarden pristop k modeliranju kvantnega računalništva dejansko s preslikavami na gostotnih matrikah, ki pa delujejo v nasprotni smeri (oziroma kategoriji) od naše~\cite{selinger-qpl}.
    Standardni pristop tokrat zaobidemo, saj obliko morfizmov \(\semmap{p}{m₁,…,mₖ}\) nujno potrebujemo za dokaz polnosti, ki sledi kasneje.
    To pa žal pomeni, da se bodo matrike komponirale v obratni smeri, kot se izvajajo operacije, saj opazljivke preslikavamo od konca programa proti začetku.
\end{remark}

\subsubsection{Primeri interpretacij}
Oglejmo si sedaj interpretacije nekaj od naših \(6\)-ih primerov:
\begin{enumerate}
    \item \(\sem{\op{new}¹(a.\,x(a))}(A) = \snew₁{X^*AX} = a₂₂\),
    \item \(\sem{\op{rand}(x,y)}(A₁,A₂) = \snew₁{\had^*\bmat{A₁&0\\0&A₂}\had}
    % = \snew₁{\frac{1}{2}\mat{A₁+A₂&A₁-A₂\\A₁-A₂&A₁+A₂}}
    = \frac{1}{2}A₁ + \frac{1}{2}A₂\).
    \item \(\sem{\op{bell}(a,b.\,x(a,b))}(A) = \snew₁{\had^*\snew₂{\ctl{X}^*A\ctl{X}}\had} = \frac{a₁₁ + a₁₄ + a₄₁ + a₄₄}{2}\),
    \item \(\sem{\tdiscard{a}{x}}(A) = I₂⊗A\),
    \item \(\sem{\op{proj}(a;a.\,x(a))}(A) = \pmat{\snew₁{A}&0\\0&\snew₁{X^*AX}} = \pmat{a₁₁&0\\0&a₂₂}\),
    % \item \(\sem{\op{c-rot}(a;a.\,t)}(A) = \snew₂{(I⊗\had)^*\p{\smeasure₂{A}{Z^*AZ}}(I⊗\had)}\\
    \item \(\sem{\op{c-rot}(a;a.\,x(a))}(A) = \frac{1}{2}A + \frac{1}{2}Z^*AZ = \pmat{a₁₁&0\\0&a₂₂}\).
\end{enumerate}

Prej smo z aksiomi pokazali, da sta zadnja dva primera enaka, sedaj pa lahko opazimo, da sta tudi nuni interpretaciji enaki.
Kasneje bomo dokazali, da je to dejansko zadosten pogoj za enakost programov.

\subsection{Polnost za meritev}
Ker dodeljevanja kubitov ne moremo interpretirati zgolj z \(*\)-homomorfizmi, bomo najprej dokazali izrek le za meritev in uporabo vrat, nato pa bomo ta izrek uporabili v dokazu splošnega izreka.

\begin{proposition}
    Kompleksna števila \(ℂ ∈ \cstarcat\) tvorijo model fragmenta kvantnega računalništva, ki vključuje operaciji \(\op{measure}\) in \(\op{apply}\) (ampak ne \(\op{new}\)) in vse relevantne aksiome ((\ref{ax:A}) – (\ref{ax:B}), (\ref{ax:F}) – (\ref{ax:J})).
\end{proposition}

\begin{proof}
    Definirajmo \(n•ℂ ≔ \M[2ⁿ]\) in operaciji \(\op{measure}\) in \(\op{apply}\) interpretirajmo kot zgoraj.
    To je že struktura, torej moramo preveriti le še interpretacije aksiomov.
    \begin{enumerate}[(A)]
        \item \( \sequent{x:0,y:0}{a}{\apply{X}{a}{\measure{a}{x}{y}} = \measure{a}{y}{x}} \)
        \item \( \sequent{x:1,y:1}{a,b}{
                    \applyd{U, V}{a,b}{\measure{a}{x(b)}{y(b)}}
                    = \measure{a}{\apply{U}{b}{x(b)}}{\apply{V}{b}{y(b)}}} \)
        % \item \( \sequent{Γ}{a}{\apply{U}{a}{\tdiscard{a}{t}} = \tdiscard{a}{t}} \)
        \addtocounter{enumi}{3}
        \item \( \sequent{x:2}{a,b}{\apply{\swap}{a,b}{x{\p{a,b}}} = x{\p{b,a}}} \)
        \item \( \sequent{x:1}{a}{\apply{I}{a}{x{\p a}} = x{\p a}} \)
        \item \( \sequent{x:1}{a}{\apply{UV}{a}{x{\p a}} = \apply{V}{a}{\apply{U}{a}{x{\p a}}}} \)
        \item \( \sequent{x:2}{a,b}{
                      \apply{U⊗V}{a,b}{x{\p{a,b}}}
                    = \apply{U}{a}{\apply{V}{b}{x{\p{a,b}}}}} \)
        \item \( \sequent{x:0,y:0,u:0,v:0}{a,b}{
                      \measurebr{a}{\measure{b}{u}{v}}{\measure{b}{x}{y}}
                    = \measurebr{b}{\measure{a}{u}{x}}{\measure{a}{v}{y}}} \)
    \end{enumerate}
    Enakost bomo izpeljali zgolj za prve tri aksiome, saj so aksiomi (\ref{ax:F}) – (\ref{ax:J}) enostavne posledice definicij uporabe matrik.
    Omenimo le pri aksiomu (\ref{ax:I}), da \(a,b\) v zapisu spet interpretiramo kot kvantni vektor, kjer \(a\) naslavlja prvi kubit, \(b\) pa drugega.
    Intuitivno si lahko predstavljamo \(a,b\) kot \(a⊗b\), ampak je vredno omeniti, da sta tu \(a\) in \(b\) lahko prepletena, torej vektorja, ki ga \(a,b\) predstavlja ne moremo nujno zapisati kot tenzorski produkt kubitov.

    Interpretirajmo aksiom (\ref{ax:A}). Naj bosta \(A₁,A₂ ∈ \M[1] = ℂ\).
    \begin{align*}
        &\hspace{-3em}\sem{\apply{X}{a}{\measure{a}{x}{y}}}(A₁,A₂)
         = X^*\pmat{A₁&0\\0&A₂}X\\
        &= \pmat{A₂&0\\0&A₁}
         = \sem{\measure{a}{y}{x}}(A₁,A₂)
    \end{align*}

    Interpretirajmo še aksiom (\ref{ax:B}). Naj bosta \(A₁,A₂ ∈ \M[2]\).
    \begin{align*}
        &\hspace{-3em}\sem{\applyd{U, V}{a,b}{\measure{a}{x(b)}{y(b)}}}(A₁,A₂)\\
        &= \pmat{U^*&0\\0&V^*}\pmat{A₁&0\\0&A₂}\pmat{U&0\\0&V}
         = \pmat{U^*A₁U&0\\0&V^*A₂V}\\
        &\hspace{3em}= \sem{\measure{a}{\apply{U}{b}{x(b)}}{\apply{V}{b}{y(b)}}}(A₁,A₂)
    \end{align*}

    Podrobno si še oglejmo interpretacijo aksioma (\ref{ax:F}). Naj bo \(A ∈ \M[4]\).
    \begin{align*}
        &\hspace{-3em}\sem{\apply{swap}{a,b}{x{\p{a,b}}}}(A)
         = \swap^*⋅A⋅\swap\\
        &= \pmat{1&0&0&0\\0&0&1&0\\0&1&0&0\\0&0&0&1}\pmat{a₁₁&a₁₂&a₁₃&a₁₄\\a₂₁&a₂₂&a₂₃&a₂₄\\a₃₁&a₃₂&a₃₃&a₃₄\\a₄₁&a₄₂&a₄₃&a₄₄}\pmat{1&0&0&0\\0&0&1&0\\0&1&0&0\\0&0&0&1}\\
        &= \pmat{a₁₁&a₁₃&a₁₂&a₁₄\\a₃₁&a₃₃&a₃₂&a₃₄\\a₂₁&a₂₃&a₂₂&a₂₄\\a₄₁&a₄₃&a₄₂&a₄₄}
         = \sem{x{\p{b,a}}}(A)
    \end{align*}
    
    Ostale aksiome lahko preverimo podobno.
    %TODO: then do it?
\end{proof}

\begin{theorem}[Polnost za meritev]\label{th:partial}
    Model \(ℂ ∈ \cstarcat\) je univerzalno poln:
    \begin{enumerate}
        \item Za vsak \(*\)-homomorfizem \(f : \semmap{p}{m₁,…,mₖ}\) obstaja izraz v algebrajski teoriji, ki ne vsebuje \(\op{new}\), tako da je \(t : \arity{p}{m₁,…,mₖ}\) in \(\sem{t} = f\).
        \item Če \(\absseq{t, u}\) ne vsebujeta \(\op{new}\) in \(\sem{t} = \sem{u}\) lahko izpeljemo \(\absseq{t = u}\).
    \end{enumerate}
\end{theorem}

% \begin{proposition}
%     Neničelni \(*\)-homomorfizmi \(\M[k] → \M[n]\) obstajajo le, če je \(n ≥ k\).
% \end{proposition}

Navedimo najprej nekaj pomožnih izrekov in definicij, ki se uporabijo v dokazu.
\begin{proposition}
    Naj bo \(n ≥ k\) in \(f : \M[k] → \M[n]\) \(*\)-homomorfizem, ki ne (nujno) ohranja enote. Tedaj obstaja~\cite{pa-fillmore} unitarna preslikava \(U ∈ \M[n]\), tako da je \(f\) oblike \(A ↦ UD{(A,…,A,0)}U^*\), kjer se \(A\) na diagonali pojavi \(m\)-krat, \(0\) je \(r×r\) ničelna matrika, in velja \(n = m⋅k + r\). Številu \(m\) pravimo \emph{kratnost} \(*\)-homomorfizma.
    Če \(f\) ohranja enoto je \(r = 0\) in \(n = m⋅k\).
\end{proposition}

Dokaz v bistveno drugačnem jeziku najdemo v~\cite[izrek I.10.8]{cstar-by-example}.

Označimo \(\cat{C} ≔ \semmap{p}{m₁,…,mₖ}\).

\begin{definition}
    % Za preslikavo \(f : A → B\), kjer je \(A = \M[k₁]⊕⋯⊕\M[kᵣ]\) in \(B = \M[n₁]⊕⋯⊕\M[nₛ]\) je Brattelijev diagram multigraf z dvema vrsticama vozlišč označenih s \(kᵢ\) in \(nⱼ\), skupaj z \(mᵢⱼ\) povezavami med \(kᵢ\) in \(nⱼ\), kjer je \(mᵢⱼ\) kratnost preslikave \(πⱼ∘f∘ιᵢ\).
    Za \(m₁,…,mₖ,p ∈ ℕ₊\) so elementi množice \emph{Brattelijevih diagramov} \(\cat{B} ≔ \brat{2ᵖ}{2^{m₁},…,2^{mₖ}}\) \(k\)-terice naravnih števil \(\p{s₁,…,sₖ}\), tako da je \(∑ᵢsᵢ2^{mᵢ} = 2ᵖ\).
    Definiramo funkcijo \(μ : \cat{B} → \cat{C}\) s predpisom
    \[μ{\p{s₁,…,sₖ}}{\p{A₁,…,Aₖ}} = D{\p{A₁,…,A₁,A₂,…,Aₖ}},\] kjer se \(Aᵢ\) na diagonali pojavi \(sᵢ\)-krat. 
\end{definition}

\begin{lemma}
    Obstaja \(ρ : \cat{C} → \cat{B}\), da je \(ρ∘μ = \id\). Še več, \(ρ{\p f}\) in \(ρ{\p g}\) sta enaka natanko tedaj, ko obstaja \(2ᵖ×2ᵖ\) unitarna matrika \(U\), tako da je \(f = U^*(g(-))U\).
\end{lemma}

\begin{proof}
    Pokažimo lemo zgolj v obratni smeri. Celoten dokaz lahko bralec najde v~\cite{pa-fillmore,ola-bratteli}.

    Naj bosta \(f,g : \semmap{p}{m₁,…,mₖ}\) taka \(*\)-homomorfizma, da velja \(f = U^*g{(-)}U\).

    Definirajmo \(fᵢ ≔ f∘ιᵢ, gᵢ ≔ g∘ιᵢ : \M[2^{mᵢ}] → \M[2ᵖ]\), kjer je \(ιᵢ(A) = (0,…,A,…,0)\) (z \(A\)-jem na \(i\)-tem mestu).
    Preslikavo \(ρ\) definiramo tako, da \(f\) slika v \(k\)-terico kratnosti preslikav \(fᵢ\). Označimo \(ρ(f) = \p{t₁,…,tₖ}\).

    Zaradi tranzitivnosti lahko za \(g\) vzamemo kar \(μ(ρ(g)) = μ(s₁,…,sₖ)\).

    % \((⇐)\)
    % Naj bo \(f = U^*g{(-)}U\).
    Enakost komponiramo z \(ιᵢ\) in dobimo enakosti \(fᵢ = U^*gᵢ{(-)}U\).
    Po zgornji trditvi za vsak \(A\) velja \(fᵢ{(A)} = Vᵢ^*(D(A,…,A))Vᵢ ≕ Vᵢ^*D_{tᵢ}Vᵢ\), \(gᵢ{(A)}\) je pa enak \(D(0,A,…,A,0) ≕ D_{sᵢ}\), kjer se pri vsakem \(D\)-ju \(A\) na diagonali pojavi \(tᵢ\) in \(sᵢ\)-krat.
    Vstavimo te dve enačbi v predpostavko in nesemo unitarne matrike na eno stran, da dobimo \(D_{tᵢ} = (VᵢU)^*D_{sᵢ}(VᵢU)\).

    Naj bo sedaj \(A\) neka matrika ranga \(1\).
    Ker unitarne matrike ne vplivajo na rang, je \(tᵢ = \rang{D_{tᵢ}} = \rang{D_{sᵢ}} = sᵢ\).
    Ker je bil \(i\) poljuben, sta torej Brattelijeva diagrama, ki pripadata \(f\) in \(g\), enaka.
\end{proof}

\begin{proof}[Skica dokaza izreka \ref{th:partial}.]
    Povzamemo dokaz opisan v~\cite[trditev 9]{algeff-lin-qpl}.

    Najprej pokažemo direktno ekvivalenco med določenimi izrazi, zgrajenimi le iz meritev in Brattelijevimi diagrami.
    Bolj natančno, izraz \(t\) interpretiramo z Brattelijevim diagramom \(\brsem{t}\), tako da velja \(μ\brsem{t} = \sem{f}\).
    V obratni smeri pa definiramo \(\mor{Measure}{\p s}\) tako, da je \(\mor{Measure}{\brsem t} = t\) in \(\brsem{\mor{Measure}{\p s}} = s\) z rekurzijo.

    Potem naj bo pa \(f : \semmap{p}{m₁,…,mₖ}\) poljuben \(*\)-homomorfizem in \(s ≔ ρ(f)\).
    Po zgornji lemi je \(f\) oblike \(U^*(μ(s))U\), torej je \(f = \sem{\tapply{U}{a}{\mor{Measure}{\p s}}}\).

    S tem dokažemo prvi del, drugi del se pa dokaže v dveh korakih:
    \begin{enumerate}
        \item definicija preslikave \(f ↦ \tapply{U}{a}{\mor{Measure}{\p{ρ{\p f}}}}\) ni odvisna od \(U\) in
        \item preslikava je surjektivna.
    \end{enumerate}

    Dokaz prvega koraka je prezahteven za tu,
    drugi korak pa dokažemo takole:
    najprej uporabimo aksiom (\ref{ax:B}), da porinemo vse meritve v operacije \(\op{apply}\).
    Argumente meritvam uredimo kot zahteva ekvivalenca iz začetka dokaza z aksiomi (\ref{ax:A}), (\ref{ax:B}), (\ref{ax:F}), in (\ref{ax:J}),
    zaporedje operacij \(\op{apply}\) pa združimo v enega z aksiomi (\ref{ax:G}) – (\ref{ax:I}).
    Tako je vsak izraz ekvivalenten izrazu oblike \(\tapply{U}{a}{\mor{Measure}{\p{ρ{\p f}}}}\), tako da je preslikava res surjektivna.
\end{proof}

\subsection{Polnost v splošnem}
Za dokaz v splošnem bomo potrebovali bolj splošne morfizme.
Definirajmo najprej še nekaj pojmov glede pozitivnosti in navedimo nekaj standardnih rezultatov. Povzeto po~\cite{paulsen_2003}.

\begin{definition}
    Element \(x\) \(C^*\)-algebre je \emph{pozitiven}, če obstaja kak element \(y\), da je \(x = y^*y\).
\end{definition}

\begin{definition}
    Preslikava \(f\) je \emph{popolnoma pozitivna}, če za vsak \(k ∈ ℕ₊\) preslikava \(Mₖ{\p f}\) ohranja pozitivnost elementov.
\end{definition}

\begin{proposition}
    Za \(f : X → Y\), kjer množenje v enem od \(X\) in \(Y\) komutira, so naslednje trditve ekvivalentne:
    \begin{enumerate}
        \item \(f\) ohranja pozitivnost elementov,
        \item za vsak \(k ∈ ℕ₊\) preslikava \(Mₖ{\p f}\) ohranja pozitivnost elementov,
        \item \(f\) je popolnoma pozitivna.
    \end{enumerate}
\end{proposition}

Dokaz lahko najdete v~\cite[izreka 3.9 in 3.11]{paulsen_2003}.

% \begin{proof}
%     Trditvi \((2)\) in \((3)\) sta ekvivalentni po definiciji in očitno \((2) ⇒ (1)\).
%     Dokazati moramo torej zgolj \((1) ⇒ (2)\).
% \end{proof}

% \begin{definition}
%     Preslikava \(f\) je eniška, če slika enoto v enoto.
% \end{definition}

\begin{proposition}
    \(C^*\)-algebre skupaj s popolnoma pozitivnimi preslikavami, ki ohranjajo enoto, tvorijo kategorijo \(\cstarcpucat\) (angl. \foreignlanguage{english}{Completely Positive Unital maps}).
\end{proposition}

\begin{proposition}
    Preslikave \(\mor{new}\) so popolnoma pozitivne in ohranjajo enoto.
\end{proposition}

\begin{proof}
    Preverimo obe lastnosti za \(\mor{new}\):
    \begin{itemize}
        \item \(\snew₁{\mat{1&0\\0&1}} = 1\),
        \item Naj bo \(x\) pozitiven element. Potem je \(x = y^*y\), za nek \(y ∈ \M[2]\).
        Če zapišemo \(y = \pmat{a&b\\c&d}\) je potem \(\snew₁{x} = a\bar a + c\bar c\).
        Število \(a\bar a + c\bar c\) je realno in nenegativno, torej lahko vzamemo \(z = \sqrt{a\bar a + c\bar c}\) in je \(\snew₁{x} = z^*z\).\qedhere
    \end{itemize}
\end{proof}

\begin{proposition}
    Kompleksna števila \(ℂ ∈ \cstarcpucat\) tvorijo model kvantnega računalništva.
\end{proposition}

\begin{proof}
    Ker so \(*\)-homomorfizmi tudi popolnoma pozitivni in ohranjajo enoto, moramo pokazati zgolj enakost interpretacij preostalih aksiomov ((\ref{ax:D}), (\ref{ax:E}), (\ref{ax:K}), (\ref{ax:L})).
    \begin{enumerate}[(A)]
        \addtocounter{enumi}{3}
        \item \( \sequent{x:0,y:0}{-}{\new{a}{\measure{a}{x}{y}} = x} \),
        \item \( \sequent{x:2}{b}{
                    \new{a}{\applyd{U,V}{a,b}{x{\p{a,b}}}}
                    = \apply{U}{b}{\new{a}{x{\p{a,b}}}}} \),
        \addtocounter{enumi}{5}
        \item \( \sequent{x:2}{-}{\new{a}{\new{b}{x{\p{a,b}}}} = \new{b}{\new{a}{x{\p{a,b}}}}} \),
        \item \( \sequent{x:1,y:1}{b}{
                    \new{a}{\measure{b}{x{\p a}}{y{\p a}}}
                    = \measurebr{b}{\new{a}{x{\p a}}}{\new{a}{y{\p a}}}} \).
    \end{enumerate}

    Preverimo kar vse štiri aksiome po vrsti.
    Naj bosta \(A₁,A₂ ∈ \M[1]\).
    \begin{align*}
        \sem{\new{a}{\measure{a}{x}{y}}}(A₁,A₂)
         = \snew₁{\mat{A₁&0\\0&A₂}} = A₁ = \sem{x}(A₁,A₂).
    \end{align*}

    Naj bo \(A ∈ \M[4]\).
    \begin{align*}
        &\hspace{-3em}\sem{\new{a}{\applyd{U,V}{a,b}{x{\p{a,b}}}}}(A)
        %  = \snew₁{\sem{\applyd{U,V}{a,b}{x{\p{a,b}}}}(A)}\\
         = \snew₁{\bmat{U^*&0\\0&V^*}\bmat{A₁₁&A₁₂\\A₂₁&A₂₂}\bmat{U&0\\0&V}}\\
        &= \snew₁{\mat{U^*A₁₁U&U^*A₁₂V\\V^*A₂₁U&V^*A₂₂V}}
         = U^*A₁₁U\\
        &= U^*\p{\snew₁{A}}U
         = \sem{\apply{U}{b}{\new{a}{x{\p{a,b}}}}}(A).
    \end{align*}
    \begin{align*}
        &\hspace{-3em}\sem{\new{a}{\new{b}{x{\p{a,b}}}}}(A)
         = \snew₁{\snew₂{A}}
         = a₁₁\\
        &= \snew₂{\snew₁{A}}
         = \sem{\new{b}{\new{a}{x{\p{a,b}}}}}(A).
    \end{align*}

    Naj bosta \(A,B ∈ \M[2]\).
    \begin{align*}
        &\hspace{-3em}\sem{\new{a}{\measure{b}{x{\p a}}{y{\p a}}}}(A,B)
         = \snew₁{\smeasure₂{A}{B}}\\
        &= \pmat{a₁₁&0\\0&b₁₁}
         = \smeasure₁{\snew₁{A}}{\snew₁{B}}\\
        &= \sem{\measurebr{b}{\new{a}{x{\p a}}}{\new{a}{y{\p a}}}}(A,B).\qedhere
    \end{align*}
\end{proof}

\begin{theorem}[Polnost v splošnem]
    Model \(ℂ ∈ \cstarcpucat\) je univerzalno poln:
    \begin{enumerate}
        \item Za vsako popolnoma pozitivno preslikavo \(f : \semmap{p}{m₁,…,mₖ}\), ki ohranja enoto, obstaja izraz v algebrajski teoriji, tako da je \(t : \arity{p}{m₁,…,mₖ}\) in \(\sem{t} = f\).
        \item Če \(\absseq{t, u}\) in \(\sem{t} = \sem{u}\) lahko izpeljemo \(\absseq{t = u}\).
    \end{enumerate}
\end{theorem}

Navedimo najprej ključni izrek, ki ga uporabimo v dokazu.

\begin{theorem}[Stinespringov izrek o dilaciji]
    Naj bo \(\mc A\) \(C^*\)-algebra in \(f : \mc A → \M[p]\) popolnoma pozitivna preslikava, ki ohranja enoto. Tedaj obstaja \(q ≥ p\) in \(*\)-homomorfizem \(g : \mc A → \M[q]\), tako da je \(f(A) = g(A)|ₚ\).
    \[\begin{tikzcd}
        \mc{A} \\
        {\M[q]} & {\M[p]}
        \arrow[     from=2-1, to=2-2]
        \arrow["g", from=1-1, to=2-1]
        \arrow["f", from=1-1, to=2-2]
    \end{tikzcd}\]
\end{theorem}

\begin{theorem}[o minimalnosti dilacije]
    Lahko izberemo minimalno dilacijo: če je \(r ≥ p\) in \( h : \mc A → \M[p] \) \(*\)-homomorfizem tak, da je \( h(-)|ₚ = f(-) \) je \(r ≥ q\) in \(g(-) = Uh(-)U^*|_q\) za neko \(r×r\) unitarno matriko.
    % https://q.uiver.app/?q=WzAsNSxbMywwLCJBIl0sWzMsMSwiXFxNW3FdIl0sWzQsMSwiXFxNW3BdIl0sWzIsMSwiXFxNW3JdIl0sWzAsMSwiXFxNW3JdIl0sWzEsMl0sWzAsMSwiZyJdLFswLDIsImYiXSxbMywxXSxbNCwzLCJVKC0pVSoiLDFdLFswLDQsImgiLDJdLFs0LDIsIiIsMix7ImN1cnZlIjozfV1d
    \[\begin{tikzcd}
        &&& \mc{A} \\
        {\M[r]} && {\M[r]} & {\M[q]} & {\M[p]}
        \arrow[                          from=2-4, to=2-5]
        \arrow["g",                      from=1-4, to=2-4]
        \arrow["f",                      from=1-4, to=2-5]
        \arrow[                          from=2-3, to=2-4]
        \arrow["{U(-)U^*}"{description}, from=2-1, to=2-3]
        \arrow["h"',                     from=1-4, to=2-1]
        \arrow[curve={height=20pt},      from=2-1, to=2-5]
    \end{tikzcd}\]
\end{theorem}

Dokaza teh izrekov izpustimo, saj sta izven obsega tega dela, najdete pa ju v~\cite[poglavje 4]{paulsen_2003}.

\begin{proof}[Dokaz izreka o polnosti.]
    Predstavimo dokaz iz~\cite[trditev 11]{algeff-lin-qpl}.

    Naj bo \(f : \semmap{p}{m₁,…,mₖ}\) kot v izreku. Potem lahko uporabimo Stinespringov izrek o dilaciji, skupaj z minimalnostjo, da \(f\) razcepimo na \(g : \mc A → \M[q]\), za nek \(q ≥ 2ᵖ\), kjer je \(\mc A = \M[2^{m₁}]⊗⋯⊗\M[2^{mₖ}]\).
    Če je \(q\) potenca \(2\), je izrek praktično dokazan; \(g\) lahko definiramo po izreku \ref{th:partial}, skrčitev pa z zaporednimi \(\mor{new}\) in premešavanjem koordinat.
    % Drugi del sledi iz minimalnosti dilacije. % source?
    Vendar v splošnem \(q\) ni potenca \(2\). Kljub temu, pa lahko tudi \(g\) ustrezno faktoriziramo:
    % https://q.uiver.app/?q=WzAsNSxbMCwwLCJBIl0sWzMsMCwiMl57beG1on1cXGNkb3QgXFxNW3Fd4oqVKDJecS1xKVxcY2RvdFxcTVsyXntt4bWifV0iXSxbNCwwLCJcXE1bMl57beG1oitxfV0iXSxbNCwxLCJcXE1bcV0iXSxbNCwyLCJcXE1bMl5wXSJdLFswLDQsImYiLDJdLFswLDMsImciLDJdLFswLDEsIigyXntt4bWifVxcY2RvdCBnLCgyXmwtcSlcXGNkb3Qgz4DhtaIpIl0sWzEsMl0sWzIsM10sWzMsNF0sWzEsMywiz4DigoEiXV0=
    \[\begin{tikzcd}
        \mc{A} &&& {2^{mᵢ}\cdot \M[q]⊕(2^q-q)\cdot\M[2^{mᵢ}]} & {\M[2^{mᵢ+q}]} \\
        &&&& {\M[q]} \\
        &&&& {\M[2^p]},
        \arrow["f"',                                from=1-1, to=3-5]
        \arrow["g"',                                from=1-1, to=2-5]
        \arrow["{(2^{mᵢ}\cdot g,(2^q-q)\cdot πᵢ)}", from=1-1, to=1-4]
        \arrow["h",                                 from=1-4, to=1-5]
        \arrow[                                     from=1-5, to=2-5]
        \arrow[                                     from=2-5, to=3-5]
        \arrow["{π₁}",                              from=1-4, to=2-5]
    \end{tikzcd}\]
    kjer zapis \(n⋅\M[l]\) predstavlja \(n\)-kratno direktno vsoto \(C^*\)-algeber \(\M[l]\).

    Navpične preslikave so zožitve, vodoravne pa \(*\)-homomorfizmi, torej lahko kot prej definiramo izraz \(t\), da je \(\sem{t} = f\).
    Ostane nam še drugi del izreka, naj bosta \(t\) in \(u\) taka, da je \(\sem{t} = \sem{u}\).
    Po aksiomih o komutativnosti lahko zapišemo \(t\) kot \(\tnew{a₁,…,aₗ}{t'}\) (in podobno za \(u\)). Iz aksioma \ref{ax:D} sledi \(\tnew{a}{\tdiscard{a}{x}} = x\), torej je BŠS število \(\mor{new}\) na začetku \(t\) in \(u\) enako.
    Po izreku \ref{th:partial} imamo \(*\)-homomorfizma \(\sem{t'}, \sem{u'}: \mc A → \M[2ˡ]\), tako da velja \(\sem{t} = \mc A \xrightarrow{\sem{t'}} \M[2ˡ] → \M[2ᵖ]\) in \(\sem{u} = \mc A \xrightarrow{\sem{u'}} \M[2ˡ] → \M[2ᵖ]\).
    Po Stinespringovem izreku in minimalnosti spet sledi, da lahko \(\sem{t} = \sem{u}\) faktoriziramo skozi nek \(\M[q]\). Po minimalnosti (kjer za \(h\) vzamemo \(\sem{t'}\) in \(\sem{u'}\) ter \(r = 2ˡ\)) tudi obstajata \(2ˡ×2ˡ\) unitarni matriki \(U\) in \(V\), da spodnji diagram komutira.
    % https://q.uiver.app/?q=WzAsNyxbMCwyLCJBIl0sWzMsMCwiXFxNWzLLoV0iXSxbNCwxLCJcXE1bMsuhXSJdLFs1LDIsIlxcTVsyXnFdIl0sWzMsNCwiXFxNWzLLoV0iXSxbNCwzLCJcXE1bMsuhXSJdLFs3LDIsIlxcTVsy4bWWXSJdLFswLDEsIlxcc2Vte3QnfSJdLFsxLDIsIlUoLSlVXioiLDJdLFsyLDNdLFswLDQsIlxcc2Vte3UnfSJdLFs0LDUsIlYoLSlWXioiXSxbNSwzXSxbNCw2LCIiLDEseyJjdXJ2ZSI6M31dLFsxLDYsIiIsMSx7ImN1cnZlIjotM31dLFszLDZdXQ==
    \[\begin{tikzcd}
        &&& {\M[2ˡ]} \\
        &&&& {\M[2ˡ]} \\
        \mc{A} &&&&& {\M[2^q]} && {\M[2ᵖ]} \\
        &&&& {\M[2ˡ]} \\
        &&& {\M[2ˡ]}
        \arrow["{\sem{t'}}",         from=3-1, to=1-4]
        \arrow["{U(-)U^*}"',         from=1-4, to=2-5]
        \arrow[                      from=2-5, to=3-6]
        \arrow["{\sem{u'}}",         from=3-1, to=5-4]
        \arrow["{V(-)V^*}",          from=5-4, to=4-5]
        \arrow[                      from=4-5, to=3-6]
        \arrow[curve={height=18pt},  from=5-4, to=3-8]
        \arrow[curve={height=-18pt}, from=1-4, to=3-8]
        \arrow[                      from=3-6, to=3-8]
    \end{tikzcd}\]
    Kompozituma preslikav na notranjem diamantu sta \(*\)-homomorfizma.
    Ker desna trikotnika komutirata, matriki \(U\) in \(V\) ne spreminjata prvih \(p\) kubitov
    in lahko po spodnji lemi BŠS vzamemo \(U = V = I\).

    \begin{lemma}\label{lemma:fix}
        Naj bo \(m < n\) in \(U\) vrata reda \(n\), ki fiksirajo prvih \(m\) kubitov.
        Tedaj lahko izpeljemo
        \[ \sequent{x:n}{a₁,…,aₘ}{\tnew{a_{m+1},…,aₙ}{\tapply{U}{a}{x(a)}}
                                  = \tnew{a_{m+1},…,aₙ}{x(a)}}.\]
    \end{lemma}
    
    \begin{proof}[Dokaz leme]
        Ker je \(U = D{\p{I, U'}}\) za neka vrata \(U'\) reda \(n-m\), je po aksiomu \ref{ax:E} leva stran enaka \(\tapply{I}{a₁,…,aₘ}{\tnew{a_{m+1},…,aₙ}{x(a)}}\), kar je pa enako desni strani po aksiomu \ref{ax:G}.
    \end{proof}

    Zgornji diagram lahko torej poenostavimo na
    % https://q.uiver.app/?q=WzAsNCxbMCwxLCJBIl0sWzEsMCwiXFxNWzLLoV0iXSxbMiwxLCJcXE1bMl5xXSJdLFsxLDIsIlxcTVsyy6FdIl0sWzAsMSwiXFxzZW17dCd9Il0sWzAsMywiXFxzZW17dSd9IiwyXSxbMywyXSxbMSwyXV0=
    \[\begin{tikzcd}
        & {\M[2ˡ]} \\
        \mc{A} && {\M[2^q]} \\
        & {\M[2ˡ]}
        \arrow["{\sem{t'}}",  from=2-1, to=1-2]
        \arrow["{\sem{u'}}"', from=2-1, to=3-2]
        \arrow[               from=3-2, to=2-3]
        \arrow[               from=1-2, to=2-3]
    \end{tikzcd}\]
    in uporabimo sledečo lemo:

    \begin{lemma}
        Če sta \(f\) in \(g\) taka \(*\)-homomorfizma, da spodnji diagram komutira
        \[\begin{tikzcd}
            {\mc{A}} \\
            {\M[p+q]} & {\M[p]},
            \arrow[      from=2-1, to=2-2]
            \arrow["f"', from=1-1, to=2-1]
            \arrow["g",  from=1-1, to=2-2]
        \end{tikzcd}\]
        kjer je vodoravna preslikava zožitev, lahko preslikavo \(f\) faktoriziramo skozi bločno-diagonalno preslikavo \(\M[p]⊕\M[q] → \M[p+q]\).
    \end{lemma}

    Iz te leme dobimo preslikave \(f\), \(f'\), \(g\), \(h\) in lahko z njimi razširimo naš diagram.
    % https://q.uiver.app/?q=WzAsNixbMCwxLCJBIl0sWzMsMCwiXFxNWzLLoV0iXSxbNCwxLCJcXE1bMl5xXSJdLFszLDIsIlxcTVsyy6FdIl0sWzEsMCwiXFxNW3Fd4oqVXFxNWzLLoS1xXSJdLFsxLDIsIlxcTVtxXeKKlVxcTVsyy6EtcV0iXSxbMCwxLCJcXHNlbXt0J30iLDIseyJsYWJlbF9wb3NpdGlvbiI6ODB9XSxbMCwzLCJcXHNlbXt1J30iLDAseyJsYWJlbF9wb3NpdGlvbiI6ODB9XSxbMywyXSxbMSwyXSxbMCw1LCIoZicsIGgpIiwyXSxbMCw0LCIoZiwgZykiXSxbNSwzXSxbNCwxXV0=
    \[\begin{tikzcd}
        & {\M[q]⊕\M[2ˡ-q]} && {\M[2ˡ]} \\
        \mc{A} &&&& {\M[2^q]} \\
        & {\M[q]⊕\M[2ˡ-q]} && {\M[2ˡ]}
        \arrow["{\sem{t'}}"'{pos=0.6}, from=2-1, to=1-4]
        \arrow["{\sem{u'}}"{pos=0.6},  from=2-1, to=3-4]
        \arrow[                        from=3-4, to=2-5]
        \arrow[                        from=1-4, to=2-5]
        \arrow["{(f', h)}"',           from=2-1, to=3-2]
        \arrow["{(f, g)}",             from=2-1, to=1-2]
        \arrow[                        from=3-2, to=3-4]
        \arrow[                        from=1-2, to=1-4]
    \end{tikzcd}\]
    Ker diagram komutira, takoj vidimo, da je \(f = f'\).

    Vemo, da je \(t = \tnew{a}{\tnew{b}{\tmeasure{b}{t'}{u'}}}\) in po lemi \ref{lemma:fix}
    je to enako tudi \(\tnew{a}{\tnew{b}{\tapply{U}{a,b}{\tmeasure{b}{t'}{u'}}}}\),
    kjer je \(U\) unitarna matrika, ki zamenja pojavitvi \((2ˡ-q)\) v \(\M[q + (2ˡ-q) + q + (2ˡ-q)] = \M[2ˡ]⊗\M[2]\). Matrika je res prave dimenzije (saj je \(a\) vektor \(l\)-tih kubitov, ki mu priključimo še en kubit \(b\)), in fiksira prvih \(q > 2ᵖ\) vrstic, torej prvih \(p\) kubitov.

    Sledi, da je \(t = u\) čim je \(\tapply{U}{a,b}{\tmeasure{b}{t'}{u'}} = \tmeasure{b}{u'}{t'}\).
    Po izreku \ref{th:partial} lahko enakost izpeljemo natanko tedaj, ko sta pripadajoča \(*\)-homo\-morfizma enaka. Če ju izračunamo res vidimo, da sta enaka:
    \begin{align*}
        &\hspace{-3em}\sem{\tapply{U}{a,b}{\tmeasure{b}{t'}{u'}}}(x)
         = U^*\pmat{\sem{t'}(x)&0\\0&\sem{u'}(x)}U\\
        &= U^*\pmat{f(x)&0&0&0\\0&g(x)&0&0\\0&0&f(x)&0\\0&0&0&h(x)}U
         = \pmat{f(x)&0&0&0\\0&h(x)&0&0\\0&0&f(x)&0\\0&0&0&g(x)}\\
        &= \pmat{\sem{u'}(x)&0\\0&\sem{t'}(x)}
         = \sem{\tmeasure{b}{u'}{t'}},
    \end{align*}
    torej je dokaz končan.
\end{proof}
