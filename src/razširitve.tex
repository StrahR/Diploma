\section{Razširitve}

Prednost uporabe algebrajskih teorij kot modele računanja je razširljivost;
naš jezik lahko zelo preprosto razširimo in dokaz polnosti za razširitev bo ponavadi zgolj metoda, kako razširitev razcepiti prevesti na osnovni model.

Za primer si poglejte prejšnji razdelek, kjer lahko \(\op{new}\) smatramo kot razširitev.
Dokaz res poteka kot opisano: vsak morfizem faktoriziramo na \(*\)-homomorfizem in nek morfizem posebne oblike, ki ga lahko obravnavamo posebej.

Brez dokaza navedimo še nekaj razširitev:

\subsection{Nedefiniranost in rekurzija}

Uvedemo lahko operacijo \(\op{undef} : \arity{0}{-}\), ki predstavlja nekakšno neizračunljivo stanje (npr. \(1/0\) ali neskončna zanka).

Opisati moramo, kako nova operacija komutira z ostalimi, kar nam da nove aksiome, na primer \(\sequent{-}{a}{\tdiscard{a}{\op{undef}}} = \op{undef}\) in podobno za \(\op{new}\) in \(\op{apply}\).
%TODO: measure?

Operacijo interpretiramo z ničelno preslikavo \(\M[0] → \M[1]\), kjer je \(\M[0]\) množica \(0×0\) matrik, torej množica z enim elementom.
Podobno kot za \(\op{measure}\) in \(\op{new}\) jo razširimo, tako da lahko veže poljubno mnogo kubitov (recimo v \(\tapply{U}{a}{undef} = undef\)).

Ta preslikava je popolnoma pozitivna, ampak ne ohranja enote (ker vse slika v \(0\)), ampak pripada razredu preslikav, za katere je \(1 - f(1)\) pozitivno.
Recimo kategoriji \(C*\)-algeber in popolnoma pozitivnih preslikav, za katere je \(1-f(1)\) pozitivno, \(\cstarcat_{\textrm{CPSU}}\) (angl. \foreignlanguage{english}{Completely Positive Sub-Unital maps}).

\begin{proposition}
    V kategoriji \(C*\)-algeber in popolnoma pozitivnih preslikav, za katere je \(1-f(1)\) pozitivno,
    tvorijo kompleksna števila poln model kvantnega računalništva, razširjenega z \(\op{undef}\).
\end{proposition}

Dokaz sloni na dejstvu, da so te preslikave oblike \(X → Y\) natanko določene s popolnoma pozitivnimi preslikavami, ki ohranjajo enoto, oblike \(X⊕ℂ → Y\).

Isto teorijo je mogoče uporabiti tudi za modeliranje rekurzije\cite{selinger-qpl}.

\subsection{Nedeterminizem}
% ⦶
Algebrajski teoriji dodajmo operacijo \(⦶ : \arity{0}{0,0}\), za katero veljajo enačbe
\[(x⦶y)⦶z = x⦶(y⦶z) \qquad x⦶y = y⦶x \qquad \op{undef}⦶x = x.\]

Naj še komutira z ostalimi operacijami (\(\op{new}\), \(\op{apply}\), in \(\op{measure}\)).

Operacija \(⦶\) je zelo podobna nedeterminizmu, razen, da ne zahteva idempotence \(x⦶x = x\).
Izkaže se, da ne obstaja nobena komutativna algebrajska teorija, ki kvantnemu računalništvu doda nedeterminizem.

Interpretacijski morfizem \(⦶ : \semmap{0}{0,0}\) naj bo \(⦶(α,β) = α + β\) (posplošen kot ponavadi).
Ta morfizem je popolnoma pozitiven, ampak ne ohranja enote.

\begin{proposition}
    V kategoriji \(C*\)-algeber in popolnoma pozitivnih preslikav, tvorijo kompleksna števila poln model kvantnega računalništva, razširjenega z \(\op{undef}\) in \(⦶\).
\end{proposition}

\begin{proof}[Skica dokaza.]
    Definiramo operacijo enostranske meritve \(\op{m}₀(a;x) : \arity{1}{0}\) s predpisom \(\op{m}₀(a;x) ≔ \tmeasure{a}{x}{\op{undef}}\).
    Potem je \[\tmeasure{a}{x}{y} = \op{m}₀(a;x)⦶\sapply{a}{\op{m}₀(a;y)},\]
    torej bo interpretacijski morfizem vsota operatorjev, kot zahteva Choiev izrek, tako da je popolnoma pozitiven.
\end{proof}

\subsection{Klasično računalništvo}
Naša definicija vrat se ne zanaša preveč na njihovo strukturo, torej jih lahko zamenjamo s kakšno drugo skupino matrik.
Če vzamemo za matrike zgolj matrike z vrednostmi v \(\{0,1\}\) in dodamo nov aksiom \(x(a) = \smeasure{a}{\snew⁰{a}{x(a)}}{\snew¹{a}{x(a)}}\) (kolaps)
dobimo model klasičnega računalništva:

\begin{proposition}
    Množica \(\{0,1\}\) je poln model klasičnega računalništva v kategoriji \(\cat{Set}^{\textrm{op}}\).
\end{proposition}

Omenimo še, da je ta aksiom nekonsistenten s kvantnim računalništvom:
\begin{align*}
    x &= \new{a}{\measure{a}{x}{y}}\\
      &= \new{a}{\apply{\had}{a}{\apply{\had}{a}{\measure{a}{x}{y}}}}\\
      &= \new{a}{\apply{\had}{a}{\measure{a}{
            \new{a}{\apply{\had}{a}{             \measure{a}{x}{y}}}}{
            \new{a}{\apply{\had}{a}{\apply{X}{A}{\measure{a}{x}{y}}}}
        }}}\\
      &= \new{a}{\apply{\had}{a}{\measure{a}{
            \new{a}{\apply{\had}{a}{             \measure{a}{y}{x}}}}{
            \new{a}{\apply{\had}{a}{\apply{X}{A}{\measure{a}{y}{x}}}}
        }}}\\
      &= \new{a}{\measure{a}{y}{x}}\\
      &= y \contradiction
\end{align*}
\begin{remark}
    Zgornji rezultat pokaže, da je ta aksiom nekonsistenten z \emph{vsemi} modeli kvantnega računalništva, ne le našim, saj protislovje sledi neposredno iz aksiomov.
\end{remark}

