\section{Kvantno računalništvo ter algebrajski učinki in diagrami}
% TODO: move this to when I need it

\subsection{Algebrajski učinki}

Diagrami so v rabi že dolgo časa, vendar pa so manipulacije teh diagramov nerodne;
na naših programih raje izvajamo bolj matematične pristope, in zato na programe raje poglejmo iz bolj matematičnega vidika. Izkaže se, da lahko zelo dobro predstavimo naše programe z algebrajskimi izrazi.
Vse naše operacije na kubitih lahko interpretiramo kot računske učinke nekih funkcij, katere pa lahko predstavimo z algebrajsko teorijo, ki jo nato lahko razvijamo povsem matematično.

Z računskimi učinki se med programiranjem pogosto srečamo: globalno stanje spremenljivk, vhodno/izhodne naprave, naključnost, izjeme, nedeterminizem, ipd.

\begin{definition}[Računski učinki]
    Če ima funkcija ali operacija še kak navzven viden učinek poleg vrnjene vrednosti, slednjemu pravimo računski učinek (učinek računanja).
\end{definition}

\begin{definition}[Algebrajski učinki]
    Računskim učinkom, ki jih lahko predstavimo s kašno algebrajsko teorijo, pravimo algebrajski učinki.
\end{definition}

Večina zanimivih (in vseh zgoraj naštetih) učinkov je algebrajskih, tako da je pristop v tem delu uporaben tudi bolj v splošnem.

% TODO: fill out with some theory on algebraic effects and how they work in our particular example
