\section{Kvantno računalništvo ter algebrajski učinki in diagrami}

\subsection{Kvantna vezja}

Kvantne programe lahko predstavimo kot diagrame vezja.
Škatle predstavljajo unitarna vrata, črte med njimi pa žice;
po enojnih žicah tečejo kubiti, po dvojnih pa klasični biti (\(0\) ali \(1\)).
Pike na žici (in potem navpična žica ven) pomenijo kontrolo;
prazna pika kontrolira \airquotes{na nič}, polna pa \airquotes{na ena}.
Taka vezja beremo od leve proti desni.
Natančnejši opis lahko najdete v \cite{ess-qc}.

Spodaj sta dva primera kvantnih programov, opisana z besedami in diagrami, ki ju bomo srečali tudi še kasneje.

\begin{example}[Projekcija na \(z\)-os]\label{ex:proj-z}
    Najprej zmerimo \(a\) in nato glede na rezultat svež kubit bodisi negiramo bodisi ne.
    Na Blochovi sferi to zgleda približno kot projekcija na \(z\)-os (edina kubita na \(z\)-osi sta \( \ket0 \) in \( \ket1 = X \ket0 \)).
    \[ \Qcircuit @C=1em @R=.7em {
            & \lstick{\ket0} & \gate{\g X} & \rstick{b} \qw\\
            \lstick{a} & \meter & \cctrl{-1}
        }
    \]
\end{example}

\begin{example}[Naključna rotacija faze]\label{ex:rand-ph-shift}
    Meritev Hadamardovega vektorja simulira pravičen met kovanca,
    vrata \(Z\) pa rotirajo fazo, torej bomo v polovici primerov kubitu \(a\) rotirali fazo.
    \[ \Qcircuit @C=1em @R=.7em {
            \lstick{a} & \qw & \qw & \qw & \qw & \gate{\g Z} & \rstick{a}\qw\\
            && \lstick{\ket0} & \gate{\had} & \meter & \cctrl{-1}
        }
    \]
\end{example}

\subsection{Algebrajski učinki}

Z računskimi učinki se med programiranjem pogosto srečamo: globalno stanje spremenljivk, vhodno/izhodne naprave, naključnost, izjeme, nedeterminizem, ipd.

\begin{definition}[Računski učinki]
    Če ima funkcija ali operacija še kak navzven viden učinek poleg vrnjene vrednosti, slednjemu pravimo računski učinek (učinek računanja).
\end{definition}

\begin{definition}[Algebrajski učinki]
    Računskim učinkom, ki jih lahko predstavimo s kašno algebrajsko teorijo, pravimo algebrajski učinki.
\end{definition}