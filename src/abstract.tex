\povzetek{
    % \subsubsection*{Motivacija} % TODO: Reword
    Kvantno računalništvo temelji na veliko modernih konceptih v teoriji programskih jezikov,
    % Kvantni programski jeziki predstavljajo nove probleme za teorijo programskih jezikov,
    kot na primer linearnost tipov, (kvantnimi) fizikalnimi pojavi, in še mnogo drugimi.
    V diplomski nalogi se bomo posvetili tema dvema, v tem članku pa zgolj drugemu.
    % V tej nalogi se bomo posvetili tema dvema.
    % Dober način razumevanja je, da razumemo enakost programov.
    Naš cilj je razumeti, kako se kvantni programi obnašajo,
    in dober način je razumevanje enakosti programov.
    % tj. kdaj sta dva programa enaka.

    % \subsubsection*{Pregled}

    % Najprej bomo predstavili algebrajsko teorijo za kvantne programe;
    % ta je zgrajena na unitarnih vratih in meritvah ter ima linearne parametre.
    % Nato bomo dokazali, da lahko s to teorijo predstavimo vse programe (polnost)
    % in nato iz nje izpeljali pravila za enakost kvantnih programov.
}

\abstract{
\begin{otherlanguage}{en}
    Test.
\end{otherlanguage}
}
