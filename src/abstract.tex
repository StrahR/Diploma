\povzetek{
    V tem delu predstavimo nov pristop do formalizacije kvantnega računalništva, zasnovanem na linearnosti in algebrajskih učinkih.

    Najprej predstavimo, kaj kvantno računalništvo sploh je in kako ga lahko razumemo iz matematičnega vidika, in definiramo jezik za kvantne programe.
    Nato razvijemo algebrajsko teorijo z linearnimi parametri
    in dokažemo, da lahko s to teorijo predstavimo vse kvantne programe (polnost) in da iz nje lahko izpeljemo pravila za enakost kvantnih programov.

    Na koncu predstavimo še nekaj razširitev jezika in teorije ter demonstriramo, da je mogoče to algebrajsko teorijo uporabiti tudi bolj v splošnem.
}

\abstract{
\begin{otherlanguage}{english}
    In this thesis we present a new approach to formalizing quantum computation based on linearity and algebraic effects.

    First we present what quantum computation is from a mathematical point of view. Then we define a quantum programming language. We develop an algebraic theory with linear parameters. We show that we can use it to represent all quantum programs and that we can extract rules for program equality from them.

    Finally, we present some extensions to our language and theory. We also demonstrate how to use this algebraic framework in general.
\end{otherlanguage}
}
