\povzetek{
    V tem delu predstavimo nov pristop do formalizacije kvantnega računalništva, zasnovanem na linearnosti in algebrajskih učinkih.

    Najprej predstavimo, kaj kvantno računalništvo sploh je, in kako ga lahko razumemo iz matematičnega vidika, ter definiramo jezik za kvantne programe.
    Nato razvijemo algebrajsko teorijo z linearnimi parametri
    in dokažemo, da lahko s to teorijo predstavimo vse kvantne programe (polnost),
    ter da lahko iz nje izpeljemo pravila za enakost kvantnih programov.

    Na koncu predstavimo še nekaj razširitev jezika in teorije ter demonstriramo, da se to algebrajsko teorijo lahko uporabi tudi bolj v splošnem.
}

\abstract{
\begin{otherlanguage}{english}
    TODO: Translate above
    Quantum computation is based on many modern concepts in programming language theory,
    such as linear types, (quantum) physical phenomena, algebraic effects, and many others.
    In this thesis we focus just on these
    and show how to use them together with \(C^*\)-algebras
    to better understand how quantum programs behave.
    In particular, we develop a simple axiom system for determining the equality of programs.

    We first present what quantum computation even is from a mathematical standpoint,
    then we present a theory of quantum computation based on linear types and algebraic effects.
    Later we develop a corresponding algebraic theory with linear parameters from which we extract an equational theory for our quantum programming language.
\end{otherlanguage}
}
