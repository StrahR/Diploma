\povzetek{
    Kvantno računalništvo temelji na veliko modernih konceptih v teoriji programskih jezikov,
    kot na primer linearnost tipov, (kvantnimi) fizikalnimi pojavi, in še mnogo drugih.
    V diplomski nalogi se nameravam posvetiti zgolj tema dvema in pokazati, kako ju lahko uporabimo skupaj s teorijo \(C^*\)-algeber, da bolje razumemo kvantne programe.

    Najprej bomo predstavili kaj kvantno računalništvo sploh je in kako ga lahko razumemo iz matematičnega vidika,
    nato bomo pa razvili teorijo za kvantno računalništvo osnovano na linearnih tipih in algebrajskih učinkih.
    Kasneje bomo iz te teorije razvili algebrajsko teorijo z linearnimi parametri za kvantne programe.
    Nato bomo dokazali, da lahko s to teorijo predstavimo vse programe (polnost)
    in nato bomo iz nje izpeljali pravila za enakost kvantnih programov.
}

\abstract{
\begin{otherlanguage}{english}
    Quantum computation is based on many modern concepts in programming language theory,
    such as linear types, (quantum) physical phenomena, algebraic effects, and many others.
    In this thesis we focus just on these
    and show how to use them together with \(C^*\)-algebras
    to better understand how quantum programs behave.
    In particular, we develop a simple axiom system for determining the equality of programs.

    We first present what quantum computation even is from a mathematical standpoint,
    then we present a theory of quantum computation based on linear types and algebraic effects.
    Later we develop a corresponding algebraic theory with linear parameters from which we extract an equational theory for our quantum programming language.
\end{otherlanguage}
}
