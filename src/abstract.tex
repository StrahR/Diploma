\povzetek{
    % \subsubsection*{Motivacija} % TODO: Reword
    Kvantno računalništvo temelji na veliko modernih konceptih v teoriji programskih jezikov,
    % Kvantni programski jeziki predstavljajo nove probleme za teorijo programskih jezikov,
    kot na primer linearnost tipov, (kvantnimi) fizikalnimi pojavi, in še mnogo drugimi.
    V diplomski nalogi se bomo posvetili tema dvema.
    % V tej nalogi se bomo posvetili tema dvema.
    % Dober način razumevanja je, da razumemo enakost programov.
    Naš cilj je razumeti, kako se kvantni programi obnašajo,
    in dober način je razumevanje enakosti programov.
    % tj. kdaj sta dva programa enaka.

    % \subsubsection*{Pregled}

    Najprej bomo predstavili algebrajsko teorijo za kvantne programe;
    ta je zgrajena na unitarnih vratih in meritvah ter ima linearne parametre.
    Nato bomo dokazali, da lahko s to teorijo predstavimo vse programe (polnost)
    in nato iz nje izpeljali pravila za enakost kvantnih programov.
}

\abstract{
\begin{otherlanguage}{english}
    Quantum computation is based on many modern concepts in programming language theory,
    such as linear types, (quantum) physical phenomena, and many others.
    In this thesis we will be focusing just on these two.
    Our goal is to understand how quantum programs behave
    and a good way of understanding is understanding program equality.

    We will first present a theory of quantum computation based on linear types and algebraic effects, then develop an algebraic theory as the framework, from which we extract an equational theory for our quantum programming language.
\end{otherlanguage}
}
