\section{Programski jezik}

\subsection{Signatura}

\subsection{Teorija tipov}
Kvantno računalništvo se najbolje modelira z linearno logiko. Spomnimo se, da kubitov ne moremo klonirati, niti jih uničiti (z unitarnimi vrati, uničimo jih lahko z meritvijo).
Linearna logika ta koncept formalizira: vsako hipotezo moramo uporabiti natanko enkrat.
Če nato iz linearne logike zgradimo teorijo tipov, se ta pogoj naravno spremeni v pogoj, da moramo vsak parameter funkciji (in na splošno vsak kubit v programu) uporabiti natanko enkrat,
kot to fizika tudi zahteva.

Definiramo fragment linearne logike s sledečimi simboli: \(⊗, +, 0, 1, ⊸, !\).
%TODO: add prooftrees

Na podoben način kot zgoraj definiramo teorijo tipov, kjer funkcijski tip razumemo kot linearnega le v kvantnih argumentih.
\begin{align*}%TODO: linear type theory
    A,B ::= \type{0} \mid \unit \mid \qbit \mid A⊗B \mid A+B \mid A → B
\end{align*}
Definiramo \(\bit ≔ \unit + \unit\).

\subsection{Sintaksa}

Sintaktične elemente delimo na dvoje, vrednosti in izračune:\\
\begin{tabular}{r l l}
    vrednost \(v\) \(::=\)& \(x\)                                & spremenljivka\\
                %   \(\mid\)& \qpl{true} \(\mid\) \qpl{false} &\\
                  \(\mid\)& \bf{\texttt{fn}} \(x : A ⇒ c\)       & funkcija\\
    izračun \(c\)  \(::=\)& \qpl{return|\((v)\)|}                &\\
                  \(\mid\)& \(\eff{op}{\p c}\)                   & operacija\\
                  \(\mid\)& \qpl{if |\(\emeasure{c} = 0\)| then |\(c₁\)| else |\(c₂\)|}
                                                                 & pogojni stavek\\
                  \(\mid\)& \qpl{let |\((x,y) = z\)| in |\(c\)|} & razcep vektorja\\
                  \(\mid\)& \(v₁{\p{v₂}}\)                       & uporaba funkcije\\
                  \(\mid\)& \qpl{|\(c₁\)|;|\(c₂\)|}             & veriženje
\end{tabular}

Naš jezik definira naslednje operacije oziroma algebrajske učinke:
\begin{itemize}
    \item \(\eff{new} : \unit → \qbit\): Dodeli nov kubit.
    \item Za vsaka vrata \(U\) reda \(n\) operacijo \(\eff{apply}_{\g U} : \qbit^{⊗n} → \qbit^{⊗n}\): Uporabi vrata \(U\) na danem kubitu.
    \item \(\eff{measure} : \qbit → \bit\): Izmeri kubit in vrne rezultat meritve (\(0\) ali \(1\)).
    \item \(\eff{discard} : \qbit → \unit\): Uniči kubit.
\end{itemize}



\subsection{Aksiomi}

Aksiome za enakost programov\cite{algeff-lin-qpl} lahko delimo na dva dela: prvih pet je glavnih, ostalih sedem pa bolj \airquotes{administrativnih} oziroma pomožnih.
Slednji nam povejo zgolj, da se \(\eff{apply}\) strinja s strukturo unitarnih matrik,
ter da stvari komutirajo, kolikor vezanje spremenljivk (in vrstni red uporabe matrik) dopušča.

\begin{axiom}{Kvantna negacija pred meritvijo je negacija po meritvi:}\label{ax:1}
    \( \apply{X}{a}{\measureraw{a}{x}{y}} = \measureraw{a}{y}{x} \).
\end{axiom}

\begin{axiom}{Kvantna kontrola je po meritvi kot klasična kontrola:}\label{ax:2}
    \( \applyd{U, V}{a,b}{\measureraw{a}{x(b)}{y(b)}}
        = \measureraw{a}{\apply{U}{b}{x(b)}}{\apply{V}{b}{y(b)}} \).
\end{axiom}

\begin{axiom}{Kvantna vrata uporabljena na zavrženih kubitih so odveč:}\label{ax:3}
    \( \apply{U}{a}{\discard{a}{t}} = \discard{a}{t} \).
\end{axiom}

% \begin{axiom}{Meritve novih kubitov so vedno \(0\):}\label{ax:4}
\begin{axiom}{Novi kubiti so \( \ket{\mb0} \) glede na meritev:}\label{ax:4}
    \( \new{a}{\measureraw{a}{x}{y}} = x \).
\end{axiom}

% \begin{axiom}{Vrata kontrolirana z novimi kubiti se nikoli ne uporabijo:}\label{ax:5}
\begin{axiom}{Novi kubiti so \( \ket{\mb0} \) glede na kontrolo:}\label{ax:5}
    \( \new{a}{\applyd{U,V}{a,b}{x{\p{a,b}}}} = \apply{U}{b}{\new{a}{x{\p{a,b}}}} \).
\end{axiom}

\begin{axiom}{Spoštovanje simetrične grupe \( \U[n] \):}\label{ax:6}
    \( \apply{swap}{a,b}{x{\p{a,b}}} = x{\p{b,a}} \),
\end{axiom}

\begin{axiom}{}\label{ax:7}
    \( \apply{I}{a}{x{\p a}} = x{\p a} \),
\end{axiom}

\begin{axiom}{}\label{ax:8}
    \( \apply{UV}{a}{x{\p a}} = \apply{V}{a}{\apply{U}{a}{x{\p a}}} \),
\end{axiom}

\begin{axiom}{}\label{ax:9}
    \( \apply{U⊗V}{a,b}{x{\p{a,b}}} = \apply{U}{a}{\apply{V}{b}{x{\p{a,b}}}} \).
\end{axiom}

\begin{axiom}{Komutativnost:}\label{ax:10}
    \( \measure{a}{\measureraw{b}{u}{v}}{\measureraw{b}{x}{y}}
        = \measure{b}{\measureraw{a}{u}{x}}{\measureraw{a}{v}{y}} \),
\end{axiom}

\begin{axiom}{}\label{ax:11}
    \( \new{a}{\new{b}{x{\p{a,b}}}} = \new{b}{\new{a}{x{\p{a,b}}}} \),
\end{axiom}

\begin{axiom}{}\label{ax:12}
    \( \new{a}{\measureraw{b}{x{\p a}}{y{\p a}}}
        = \measure{b}{\new{a}{x{\p a}}}{\new{a}{y{\p a}}} \).
\end{axiom}

\subsection{Primeri}

% \section{Razširitve}
